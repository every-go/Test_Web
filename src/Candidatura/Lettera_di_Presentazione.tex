% Configurazione
\documentclass{article}

\usepackage{titling} % Required for inserting the subtitle
\usepackage{graphicx} % Required for inserting images
\usepackage{tabularx} % Per l'ambiente tabularx (tabelle)
\usepackage{calc} % Sempre per le tabelle
\usepackage[hidelinks]{hyperref} % Per i collegamenti ipertestuali, ad esempio sulla table of contents
\usepackage{xcolor} % Per colorare il testo
\usepackage{colortbl} % Per colorare le celle delle tabelle
\usepackage{lipsum} % Per generare lorem ipsum
\usepackage[normalem]{ulem} % Per sottolineare il testo
\usepackage{array} % Per la visualizzazione fluttuante di array di domande e risposte
\usepackage{ragged2e} % Pacchetto necessario per \justifying che giustifica il testo di tabelle
\usepackage{placeins}
\usepackage{hyperref}
\usepackage{xcolor}


\newcommand{\ulhref}[2]{\href{#1}{\uline{#2}}} % Nuovo comando per sottolineare i link
\newcommand{\ulref}[1]{\uline{\ref{#1}}} % Nuovo comando per sottolineare i collegamenti a immagini
\setlength{\parindent}{0pt} % Rimuove il rientro automatico dei paragrafi

\graphicspath{ {immagini/} {../shared/images} }

% Struttura
\begin{document}
	\pagestyle{empty}
	\begin{minipage}{0.4\textwidth}
		\includegraphics[width=0.6\textwidth]{logo_gruppo}
	\end{minipage}
	\begin{minipage}{0.55\textwidth}
		\textbf{NullPointers Group} \\
		\textsf{nullpointersg@gmail.com}
	\end{minipage}
	
	\vspace{2cm}
	
	{
		\centering
		\Huge\bfseries Candidatura Capitolato C8\par
		\vspace{0.5cm}
	}
	
	\begin{center}
		\begin {minipage}{1.0\textwidth}
			Destinatari:\\
			Prof. Vardanega Tullio,\\
			Prof. Cardin Riccardo.\\
		\end{minipage}
	\end{center}
	Egregi Professori Vardanega e Cardin,\\
	con il presente documento il team \textbf{"NullPointers Group"} desidera formalizzare la propria candidatura per la realizzazione del  capitolato del progetto \textbf{C8}:\\
	\begin{center}
		\textbf{SmartOrder}
	\end{center}
	proposto dall'azienda \textbf{Ergon Informatica Srl}.\\
	Le principali motivazioni che ci hanno portato a scegliere il presente capitolato sono le seguenti:
	\begin {itemize}
		\item Riscontro positivo dell'incontro con l'azienda Ergon, che si è resa disponibile a rispondere ai quesiti del gruppo e ha dimostrato interesse nell'affiancarci nelle fasi progettuali più complesse.
		\item Il capitolato si è distinto per l'uso più vasto degli LLM portando interesse spontaneo da parte del gruppo.
		\item Apprezzamento della flessibilità tecnologica consentendo la scelta di linguaggi più familiari per il gruppo.
	\end {itemize} 
	Per maggiori dettagli si rimanda al documento di \underline{\href{https://nullpointersgroup.github.io/Documentazione/output/Candidatura/Valutazione_Capitolati.pdf}{Valutazione dei Capitolati}}.\\

	Come discusso nel documento di \underline{\href{https://nullpointersgroup.github.io/Documentazione/output/Candidatura/Preventivo_Costi.pdf}{Preventivo Costi}} abbiamo preventivato un costo complessivo per la realizzazione del progetto di \textbf{11.440€}, fissando come termine ultimo per il completamento il \textbf{30 Aprile 2026}.\\
	
	Tutta la documentazione redatta dal gruppo è pubblicamente disponibile nel seguente \underline{\href{https://nullpointersgroup.github.io/Documentazione}{sito ufficiale}} e nella \underline{\href{https://github.com/NullPointersGroup/Documentazione}{repository}} dedicata.\\
	NullPointers Group è composto dai seguenti membri:\newline
	
	\begin{tabular}{|p{0.33\textwidth}|p{0.33\textwidth}|p{0.33\textwidth}|}
		\hline
		\rowcolor{red!10!blue!20}
		\textbf{Nome} & \textbf{Cognome} & \textbf{Matricola} \\
		\hline
		Laura & Pieripolli & 2048057 \\ \hline
		Lisa & Casagrande & 2116440 \\ \hline
		Luca & Marcuzzo & 2113198 \\ \hline
		Marco & Brunello & 2110997 \\ \hline
		Matteo & Mazzaretto & 2111005 \\\hline
		Tommaso & Ceron & 2101045 \\ \hline
	\end{tabular} \\
	
	\vskip 0.2in
	Cordiali Saluti, \\
	NullPointers Group.
\end{document}