% Configurazione
\documentclass{article}

\usepackage{titling} % Required for inserting the subtitle
\usepackage{graphicx} % Required for inserting images
\usepackage{tabularx} % Per l'ambiente tabularx (tabelle)
\usepackage{calc} % Sempre per le tabelle
\usepackage[hidelinks]{hyperref} % Per i collegamenti ipertestuali, ad esempio sulla table of contents
\usepackage{xcolor} % Per colorare il testo
\usepackage{colortbl} % Per colorare le celle delle tabelle
\usepackage{lipsum} % Per generare lorem ipsum
\usepackage[normalem]{ulem} % Per sottolineare il testo
\usepackage{array} % Per la visualizzazione fluttuante di array di domande e risposte
\usepackage{ragged2e} % Pacchetto necessario per \justifying che giustifica il testo di tabelle
\usepackage{placeins}

\newcommand{\ulhref}[2]{\href{#1}{\uline{#2}}} % Nuovo comando per sottolineare i link
\newcommand{\ulref}[1]{\uline{\ref{#1}}} % Nuovo comando per sottolineare i collegamenti a immagini
\setlength{\parindent}{0pt} % Rimuove il rientro automatico dei paragrafi

\graphicspath{{immagini/} {../shared/images/}}

% Struttura
\begin{document}
	\begin{minipage}{0.4\textwidth}
		\includegraphics[width=0.6\textwidth]{logo_unipd}
	\end{minipage}
	\begin{minipage}{0.55\textwidth}
		\textcolor{red}{\textbf{Università degli Studi di Padova}} \\
		\textcolor{red}{Laurea: Informatica} \\
		\textcolor{red}{Corso: Ingegneria del Software} \\
		\textcolor{red}{Anno Accademico: 2025/2026}
	\end{minipage}
	
	\begin{minipage}{0.4\textwidth}
		\includegraphics[width=0.6\textwidth]{logo_gruppo}
	\end{minipage}
	\begin{minipage}{0.55\textwidth}
		\textbf{Gruppo: NullPointers Group} \\
		Email: \textsf{nullpointersg@gmail.com}
	\end{minipage}
	
	\vspace{2cm}
	
	{
		\centering
		\Huge\bfseries Preventivo Costi\par
		\vspace{0.5cm}
	}
	
	\begin{center}
		\begin{tabular}{r|l}
			Stato & Approvato \\
			Versione & 2.0 \\
			Data ultima modifica & 04/11/2025 \\
			Destinatari & NullPointers Group \\
			& Prof. Tullio Vardanega \\
			& Prof. Riccardo Cardin \\
		\end{tabular}
	\end{center}
	
	\newpage
	
	\section*{Registro delle modifiche}
	
	\begin{table}[htbp]
		\begin{tabular}{|p{0.1\textwidth}|p{0.1\textwidth}|p{0.25\textwidth}|p{0.25\textwidth}|p{0.3\textwidth}|}
			\hline
			\rowcolor[gray]{0.9}
			Versione & Data & Autore & Ruolo & Descrizione \\
			\hline
			2.0 & 04-11 & Laura Pieripolli & - & Revisione e approvazione del documento\\
			\hline
		   2.0 & 04-11 & Lisa Casagrande & - & Aggiunte specifiche sul metodo di rotazione dei ruoli \\
			\hline
			1.0 & 30-10 & Luca Marcuzzo & - & Revisione e approvazione del documento \\
			\hline
			0.5 & 30-10 & Lisa Casagrande & - & Aggiunta indice \\ 
			\hline
			0.4 & 27-10 & Matteo Mazzaretto & - & Modifica preventivo \\ 
			\hline
			0.31 & 26-10 & Matteo Mazzaretto & - & Miglioramento definizione amministratore \\ 
			\hline
			0.3 & 26-10 & Tommaso Ceron & - & Aggiunta grafico impegni orari e suddivisione ruoli \\ 
			\hline
			0.2 & 24-10 & Laura Pieripolli & - & Continuazione stesura \\ 
			\hline
			0.1 & 24-10 & Matteo Mazzaretto & - & Creazione e stesura documento \\ 
			\hline
		\end{tabular}
	\end{table}
	
	\newpage

	\newpage
	\tableofcontents
         \newpage
	
	
	\section{Scopo del documento}
	Questo documento ha lo scopo di illustrare e identificare il nostro impegno per il progetto, basandoci sulla nostra intensità e volontà.\\
	Elencheremo i nostri impegni orari, indicando un'analisi dei ruoli e la nostra strategia di rotazione, concludendo con il preventivo del progetto e la data di consegna.
	
	\section{Impegni orari}
	Dopo aver discusso tra i membri del gruppo, abbiamo definito le ore che ogni membro può dedicare al progetto.\\
	Ogni membro del gruppo si impegna a dedicare al progetto un totale di 92 ore produttive, ripartite equamente tra i ruoli di \textbf{Responsabile, Amministratore, Analista, Progettista, Programmatore} e \textbf{Verificatore}.\\
	Di seguito è riportata una tabella che riassume gli impegni orari di ogni ruolo:	
	
	\begin{table}[h!]
		\centering
		\begin{tabular}{|c|c|}
			\hline
			\rowcolor{gray!25}
			Ruolo & Ore \\ \hline
			Responsabile & 50 \\ \hline
			Amministratore & 42 \\ \hline
			Analista & 95 \\ \hline
			Progettista & 125 \\ \hline
			Verificatore & 115 \\ \hline
			Programmatore & 125 \\ \hline
		\end{tabular}
		\caption{Ripartizione delle ore di lavoro tra i ruoli}
	\end{table}
	
	\section{Analisi dei ruoli}
	\subsection{Responsabile}
	Il Responsabile riveste un ruolo centrale nelle prime fasi del progetto, coordinando le attività del gruppo e garantendo una pianificazione efficace.
	Con l’avanzare del lavoro, il team acquisirà maggiore autonomia, mentre il Responsabile continuerà a svolgere funzioni di supervisione e controllo, assicurando il rispetto delle scadenze e la coerenza dei risultati con gli obiettivi prefissati.
	Sarà inoltre il principale punto di riferimento nelle comunicazioni tra il gruppo, i proponenti e i committenti.
	Trattandosi di un ruolo principalmente di rappresentanza, abbiamo previsto di destinargli circa il 9\% delle ore totali.
	
	\subsection{Amministratore}
	L'Amministratore si occupa della configurazione e gestione dell'infrastruttura IT di supporto al progetto.
	Il suo ruolo è particolarmente importante nelle fasi iniziali e durante il deployment, dove raggiunge il picco di impegno per garantire un deploy corretto.
	Con il progredire del progetto, il suo contributo diminuisce man mano che i membri del gruppo diventano autonomi nell'uso degli strumenti predisposti.
	
	\subsection{Analista}
	L'Analista è cruciale durante le fasi iniziali del progetto; si occupa di identificare e chiarire i requisiti, interpretando le esigenze degli utilizzatori finali per garantire una corretta definizione delle funzionalità.\\
	L'analisi viene svolta in collaborazione con l'azienda proponente e successivamente rielaborata dal gruppo, che redigerà il documento dei requisiti.\\
	Con l’avanzare del progetto, il monte ore dedicato a questo ruolo diminuirà, pur restando attivo per eventuali aggiornamenti o adattamenti dei requisiti in base al confronto con il proponente.
	
	\subsection{Progettista}
	Il Progettista traduce i requisiti del sistema in un'architettura software dettagliata, definendo moduli, interfacce, flussi dati e vincoli tecnici. Supervisiona l'implementazione del codice, verifica la conformità all'architettura e supporta l'integrazione dei componenti, senza occuparsi della manutenzione successiva.
	
	\subsection{Programmatore}
	Il Programmatore è responsabile dello sviluppo del codice sorgente del progetto, traducendo il design in codice funzionante testabile dal proponente.\\
	Collabora attivamente con il Progettista per assicurarsi che tutte le funzionalità siano implementate correttamente.\\
	Si stima un monte ore superiore alla media dovuto alla necessità di integrare le numerose tecnologie che si andranno a utilizzare.
	
	\subsection{Verificatore}
	Il Verificatore si occupa di assicurare la qualità dei prodotti e dei processi adottati, effettuando revisioni e test. 
	Questo ruolo richiede una presenza costante per tutta la durata del progetto, in modo da poter seguire efficacemente tutte le fasi previste.

    \section{Rotazione dei Ruoli}
	È stato stabilito di adottare un sistema di rotazione con cadenza bisettimanale, allineato alla durata degli sprint di progetto.
	Tale scelta consente a ciascun membro del gruppo di disporre di un periodo di impegno sufficiente per assimilare le responsabilità associate al ruolo, garantendo al contempo un contributo sostanziale e continuativo al progresso delle attività.
	
	\section{Preventivo costi}
	Di seguito è riportata una tabella contenente la ripartizione individuale nei vari ruoli, comprensiva del preventivo dei costi orari e totali.
	\begin{table}[h!]
		\centering
		\begin{tabular}{|c|c|c|c|}
			\hline
			\rowcolor{gray!25}
			Ruolo & Ore totali & Costo (€/h) & Costo totale (€) \\ \hline
			Responsabile & 50 & 30 & 1500 \\ \hline
			Amministratore & 42 & 20 & 840 \\ \hline
			Analista & 95 & 25 & 2375 \\ \hline
			Progettista & 125 & 25 & 3125 \\ \hline
			Verificatore & 115 & 15 & 1725 \\ \hline
			Programmatore & 125 & 15 & 1875 \\ \hline
		\end{tabular}
		\caption{Distribuzione delle ore e calcolo dei costi totali per ciascun ruolo}
	\end{table}
	
	\FloatBarrier
	\includegraphics[width=1.0\textwidth]{grafico_costi.jpeg}
	Il gruppo prevede un impegno complessivo di 552 ore produttive, per un costo totale preventivato di 11.440 €

	\section{Data di consegna}
	Dopo le stime orarie e dei costi sopra riportati, il gruppo si impegna a terminare il progetto didattico entro il giorno 30/04/2026.

	
	
	
\end{document}