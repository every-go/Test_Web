\documentclass[11pt]{beamer}

% --- Codifica e font ---
\usepackage[T1]{fontenc}
\usepackage[utf8]{inputenc}
\usepackage{lmodern}

% --- Colori ---
\usepackage{xcolor}
\definecolor{Teal}{HTML}{2EAFC0}
\definecolor{Green}{HTML}{88C084}
\definecolor{Purple}{HTML}{6A2E7A}
\definecolor{Pink}{HTML}{CF8AA6}
\definecolor{Yellow}{HTML}{F0D78A}
\definecolor{Cream}{HTML}{FFF8F0}
\definecolor{LightGray}{HTML}{F5F5F5}

% --- TikZ per sfondi ---
\usepackage{tikz}
\usetikzlibrary{calc,backgrounds,positioning}
\usepackage{graphicx}
\graphicspath{{./}{../../shared/images/}}

% --- Pacchetto per link cliccabili ---
\usepackage{hyperref}

% --- Tema base Beamer ---
\setbeamertemplate{navigation symbols}{}
\setbeamercolor{normal text}{fg=black,bg=white}
\setbeamercolor{frametitle}{fg=black}

% Template del frametitle semplificato
\setbeamertemplate{frametitle}{%
  \vspace{1.2cm}
  \centering
  \insertframetitle%
}

% --- Template ---
\newcommand{\BaseBackground}[1]{%
  \setbeamertemplate{background}{%
    \begin{tikzpicture}[remember picture,overlay]
      % Sfondo principale bianco con nuance
      \fill[#1!5!white] (current page.south west) rectangle (current page.north east);
      
      % Pattern geometrico sottile
      \foreach \x in {0,2,...,12} {
        \foreach \y in {0,2,...,8} {
          \fill[#1!8!white, rotate=45] 
            ($(current page.south west)+(\x cm + 0.5cm, \y cm + 0.5cm)$) 
            rectangle +($(0.3cm,0.3cm)$);
        }
      }
      
      % Header 
      \fill[#1!15!white] 
        (current page.north west) rectangle ($(current page.north east)+(0,-2.5cm)$);
      
      % Barra colorata nell'header
      \fill[#1] 
        ($(current page.north west)+(0,-2.3cm)$) rectangle 
        ($(current page.north west)+(4cm,-2.5cm)$);
      
      % Logo gruppo
      \coordinate (logocenter) at ($(current page.north east)+(-1.5cm,-1.2cm)$);
      \node at (logocenter) {\includegraphics[height=1.5cm]{logo_gruppo_.png}};
      
      % Footer
      \fill[#1!10!white] 
        (current page.south west) rectangle ($(current page.south east)+(0,1.2cm)$);
      
      \node[anchor=west] (unipdlogo) at ($(current page.south west)+(0.8cm,0.6cm)$)
        {\includegraphics[height=0.9cm]{logo_unipd_.png}};
      
      % Numero slide
      \node[anchor=south east, text=#1!70!black, font=\bfseries, inner sep=3pt] 
        at ($(current page.south east)+(-0.5cm,0.4cm)$) 
        {\insertframenumber/\inserttotalframenumber};
    \end{tikzpicture}%
  }%
  % Imposta anche i colori del testo per questa sezione
  \setbeamercolor{normal text}{fg=black,bg=white}
  \setbeamercolor{frametitle}{fg=#1!80!black}
}

% --- Background ---
\newcommand{\SetTitleBackground}{%
  \setbeamertemplate{background}{%
    \begin{tikzpicture}[remember picture,overlay]
      % Sfondo titolo con gradiente
      \path[shade, left color=Purple!30!white, right color=Teal!30!white]
        (current page.south west) rectangle (current page.north east);
      
      % Logo gruppo
      \coordinate (logocenter) at ($(current page.north east)+(-1.3cm,-1.3cm)$);
      \node at (logocenter) {\includegraphics[height=2.0cm]{logo_gruppo_.png}};
      
      % Fondo con logo UniPD
      \fill[color=Cream!50, opacity=0.25] 
        ($(current page.south west)$) rectangle ($(current page.south east)+(0,1.5cm)$);
      \node[anchor=west] (unipdlogo) at ($(current page.south west)+(0.5cm,0.75cm)$)
        {\includegraphics[height=1.3cm]{logo_unipd_.png}};
      
      % Numero slide
      \node[anchor=south east, text=gray, font=\bfseries, inner sep=3pt] 
        at ($(current page.south east)+(-0.4cm,0.3cm)$) 
        {\insertframenumber/\inserttotalframenumber};
    \end{tikzpicture}%
  }%
  \setbeamercolor{normal text}{fg=black,bg=white}
  \setbeamercolor{frametitle}{fg=black}
}

\newcommand{\ResetToNormalBackground}{\BaseBackground{Teal}}
\newcommand{\SetSectionBackground}[1]{\BaseBackground{#1}}

% --- Colori liste ---
\newcommand{\SetListItemColor}[1]{%
  \setbeamertemplate{itemize item}{\color{#1!70!black}$\bullet$}%
  \setbeamertemplate{itemize subitem}{\color{#1!60!black}$\circ$}%
}

% --- Stili testo ---
\setbeamerfont{title}{size=\Huge, series=\bfseries}
\setbeamerfont{frametitle}{size=\Large, series=\bfseries}
\setlength{\leftmargini}{1.5em}

% --- Rimuove il piedipagina  ---
\setbeamertemplate{footline}{}

% --- Info presentazione ---
\title{Diario di Bordo}
\author{NullPointers Group}
\subtitle{Ingegneria del Software}
\institute{Università degli Studi di Padova}
\date{17 Novembre 2025}

% === DOCUMENTO ===
\begin{document}

% --- SLIDE TITOLO ---
\SetTitleBackground
\begin{frame}[plain]
  \vspace{2cm}
  \centering
  {\usebeamerfont{title}\color{black}\inserttitle\par}
  \vspace{0.8cm}
  {\Large\color{black}\insertauthor\par}
  \vspace{0.5cm}
  {\large\color{black}\insertsubtitle\par} 
  \vspace{0.5cm}
  {\color{black}\insertdate\par}
\end{frame}

% --- SLIDE NORMALI ---
\ResetToNormalBackground

%-------------------------------------------------------- MODIFICA DA QUI -------------------------------------------------------

% --- SEZIONE 1: INTRODUZIONE ---
\section{Decisioni}
\SetSectionBackground{Teal}
\SetListItemColor{Teal}
\begin{frame}{Decisioni}
\vspace*{-0.5cm}
  \begin{itemize}
    \setlength\itemsep{1.0em}
    \item Continuazione stesura del Glossario;
    \item Inizio scrittura documento Norme di Progetto;
    \item Selezione delle domande da porre all’azienda nel primo incontro.
  \end{itemize}
\end{frame}

% --- SEZIONE 2: SVILUPPO ---
\section{Problematiche riscontrate}
\SetSectionBackground{Purple}
\SetListItemColor{Purple}
\begin{frame}{Problematiche riscontrate}
  \begin{itemize}
    \setlength\itemsep{1.0em}
    \item Difficoltà nello sviluppo dello script per la conversione da PDF a HTML;
    \item Dubbio sulla necessità di versionare il glossario.

  \end{itemize}
\end{frame}

%-------------------------------------------------------- FINO A QUI -------------------------------------------------------

% --- SLIDE FINALE ---
\SetSectionBackground{Teal}
\SetListItemColor{Teal}
\begin{frame}[plain]
  \vspace{1.5cm}
  \centering
  {\Huge\textbf{Grazie per l'attenzione!}\par}
  \vspace{1.0cm}
  
  % Contatti e repository
  \begin{tabular}{c}
    \href{mailto:nullpointersg@gmail.com}{\large nullpointersg@gmail.com} \\
    \vspace{0.3cm} \\
    \href{https://github.com/NullPointersGroup/Documentazione.git}{
      \begin{tabular}{@{}c@{}}
        \includegraphics[height=1.0cm]{logo_github_.png}
        \hspace{0.3cm}
        \raisebox{0.3cm}{Visita il nostro repository}
      \end{tabular}
    }
  \end{tabular}
\end{frame}

\end{document}