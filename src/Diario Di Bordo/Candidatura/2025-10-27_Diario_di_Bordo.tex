\documentclass[11pt]{beamer}
\usetheme{CambridgeUS}

%-----------------------------------------------------------------------
% Packages
\usepackage[utf8]{inputenc}
\usepackage{amsmath}
\usepackage{amsfonts}
\usepackage{amssymb}
\usepackage{graphicx}

%-----------------------------------------------------------------------
% Package customization
\definecolor{links}{HTML}{FC0D30}
\hypersetup{colorlinks,linkcolor=,urlcolor=links}

%-----------------------------------------------------------------------
% Commands
\renewcommand{\emph}[1]{\textbf{#1}}
\graphicspath{ {immagini/} {../../shared/images/} }

%-----------------------------------------------------------------------
% Headings
\author{NullPointers Group}
\title{Diario di Bordo}
\setbeamercovered{transparent} 
\setbeamertemplate{navigation symbols}{}
\logo{%
	\makebox[0.95\paperwidth]{%  <-- Larghezza totale quasi quanto la slide
		\includegraphics[width=0.05\textwidth]{logo_unipd}\hfill
		\includegraphics[width=0.05\textwidth]{logo_gruppo}% ← secondo logo
	}%
}
\institute{Università di Padova} 
\date{27 Ottobre 2025}

%-----------------------------------------------------------------------
% Document
\begin{document}

\begin{frame}
\titlepage
\end{frame}

\begin{frame}{Problemi principali}
	Nel corso del periodo compreso tra l’inizio della composizione dei gruppi del progetto del primo lotto e oggi abbiamo affrontato diversi problemi:
	\begin{itemize}
		\item Gestione repository e automatizzazione di essa
		\item Creazione sito web, compreso design
		\item Gestione indirizzo mail
		\item Scrittura verbali
	\end{itemize}
\end{frame}

\begin{frame}{Repository}
	Abbiamo deciso di usare GitHub per mantenere la repository. \\
	Poiché lavoriamo come team, abbiamo utilizzato la nostra mail per creare un’organizzazione, una funzionalità di GitHub che consente di gestire più collaboratori su diverse repository senza dover assegnare manualmente i permessi.\\
	Abbiamo avuto difficoltà sulla scelta di come gestire le cartelle, decidendo di tenere una cartella \emph{src} che contiene i file sorgenti \textit{.tex} per la scrittura di documenti ed una cartella \emph{website} con il file \textit{.css} e il file \textit{.js} necessari per il sito hostato grazie a GitHub Pages, un servizio messo a disposizione da GitHub per permettere di mostrare un sito web partendo dai file \textit{.html} mantenuti nella repository. \\
	Il nostro sito si trova al seguente link: \url{https://nullpointersgroup.github.io/Documentazione/}
\end{frame}

\begin{frame}{Automatizzazione}
	Oltre a quanto menzionato prima, c'è stata molta difficoltà nel costruire un'automatizzazione del nostro sito.\\
	Questo ci è stato possibile collocando nella cartella \emph{.github/workflows} due file: 
	\begin{itemize}
		\item build.yml: i file con quell'estensione permettono di costruire un workflow automatico su GitHub. Il nostro workflow scarica le dipendenze necessarie per compilare i file LaTeX e avvia lo script Python
		\item main.py: lo script compila prima tutti i file \textit{.tex} nella cartella src inserendo i \textit{.pdf} nella cartella output copiando le sottocartelle fino a una profondità massima di due livelli. Infine riscrive l'HTML per mantenere tutti i collegamenti ai documenti, sia appena inseriti che quelli precedenti
	\end{itemize}
\end{frame}

\begin{frame}{Sito web}
	La difficoltà del sito web è stata quella di realizzare una struttura \textit{.html} con dei file \textit{.css} e \textit{.js} adatti a costruire un sito web esteticamente gradevole e compatibile con lo script.\\
	Abbiamo privilegiato la funzionalità rispetto all’estetica.
\end{frame}

\begin{frame}{Indirizzo mail}
	Abbiamo avuto un grosso problema ben due volte con l'indirizzo mail.\\
	Google ha bloccato la nostra casella di posta perché risultava utilizzata su troppi dispositivi con IP diversi, in violazione delle loro policy.\\
	Questo problema ci ha costretto a creare una nuova mail (con la creazione di un nuovo account GitHub associato) e riscrivere alcuni file scritti fino a quel momento.\\
	Fortunatamente non abbiamo perso la repository GitHub e ci è stato facile ripristinarla.
\end{frame}

\begin{frame}{Verbali}
	L'ultima grande difficoltà che abbiamo avuto è stata la scelta della struttura per redigere i verbali.\\
	Abbiamo optato per una struttura fissa sia per i verbali interni che esterni, che comprende fra gli altri elementi il titolo, il registro delle modifiche, il diario di bordo, le decisioni e le task future da affrontare (questi ultimi identificati tramite codice univoco descritto nel primo verbale interno).\\
	Infine, abbiamo deciso di salvare i nomi del file col formato YYYY-MM-DD iniziale per permettere un corretto ordinamento da parte di tutti i sistemi operativi.
\end{frame}

\begin{frame}{Conclusione}
 Per ogni domanda:
 \begin{itemize}
  \item email: \href{mailto:nullpointersg@gmail.com}{nullpointersg@gmail.com}
 \end{itemize}
 
 \begin{columns}
  \begin{column}{0.15\textwidth}
   \includegraphics[width=0.99\textwidth]{logo_github}
  \end{column}
  \begin{column}{0.85\textwidth}
   La nostra repository è disponibile al seguente link: \url{https://github.com/NullPointersGroup/Documentazione}
  \end{column}
 \end{columns}
\end{frame}

\end{document}
