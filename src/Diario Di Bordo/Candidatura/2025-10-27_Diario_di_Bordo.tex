\documentclass[11pt]{beamer}
\usetheme{CambridgeUS}

%-----------------------------------------------------------------------
% Packages
\usepackage[utf8]{inputenc}
\usepackage{amsmath}
\usepackage{amsfonts}
\usepackage{amssymb}
\usepackage{graphicx}

%-----------------------------------------------------------------------
% Package customization
\definecolor{links}{HTML}{FC0D30}
\hypersetup{colorlinks,linkcolor=,urlcolor=links}

%-----------------------------------------------------------------------
% Commands
\renewcommand{\emph}[1]{\textbf{#1}}
\graphicspath{ {immagini/} {../../shared/images/} }

%-----------------------------------------------------------------------
% Headings
\author{NullPointers Group}
\title{Diario di Bordo}
\setbeamercovered{transparent} 
\setbeamertemplate{navigation symbols}{}
\logo{%
	\makebox[0.95\paperwidth]{%  <-- Larghezza totale quasi quanto la slide
		\includegraphics[width=0.05\textwidth]{logo_unipd}\hfill
		\includegraphics[width=0.05\textwidth]{logo_gruppo}% ← secondo logo
	}%
}
\institute{Università di Padova} 
\date{27 Ottobre 2025}

%-----------------------------------------------------------------------
% Document
\begin{document}

\begin{frame}
\titlepage
\end{frame}

\begin{frame}{Decisioni}
	Nel corso del periodo compreso abbiamo preso le seguenti decisioni:
	\begin{itemize}
		\item Nome gruppo e logo
		\item GitHub
		\item Gestione cartelle
		\item Sito web
	\end{itemize}
\end{frame}

\begin{frame}{Problemi}
	Invece, i nostri problemi principali sono stati:
	\begin{itemize}
		\item Automazione
		\item Scrittura verbali
		\item Calcolo ore a lungo termine
		\item Riunioni con azienda
	\end{itemize}
\end{frame}

\begin{frame}{Attività future}
	Le nostre attività per il futuro sono:
	\begin{itemize}
		\item Terminazione domanda di candidatura
		\item Perfezionare documenti per candidatura
		\item Avvio della stesura di proposte progettuali
		\item Colloquio con azienda VarGroup per approfondimento capitolato C2
	\end{itemize}
\end{frame}

\begin{frame}{Conclusione}
 Per ogni domanda:
 \begin{itemize}
  \item email: \href{mailto:nullpointersg@gmail.com}{nullpointersg@gmail.com}
 \end{itemize}
 
 \begin{columns}
  \begin{column}{0.15\textwidth}
   \includegraphics[width=0.99\textwidth]{logo_github}
  \end{column}
  \begin{column}{0.85\textwidth}
   La nostra repository è disponibile al seguente link: \url{https://github.com/NullPointersGroup/Documentazione}
  \end{column}
 \end{columns}
\end{frame}

\end{document}
