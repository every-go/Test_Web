\section{Introduzione}
	\subsection{Scopo del documento}
	Il presente documento nasce per descrivere il Way of Working$^G$ adottato da \textbf{\vr{NullPointers Group}} durante lo svolgimento del progetto$^G$ SmartOrder.\\
	Lo standard di riferimento è l'ISO/IEC 12207:1995, il quale prevede tre tipologie di processi.
	\begin{itemize}
	    \setlength\itemsep{-0.1em}
	    \item \textbf{Processi primari:} processi fondamentali senza i quali un progetto$^G$ non può definirsi tale;
	    \item \textbf{Processi di Supporto:} processi che coadiuvano i processi primari nello svolgimento delle rispettive azioni;
	    \item \textbf{Processi organizzativi:} processi di carattere più generale che aiutano la realizzazione del progetto$^G$.
	\end{itemize}

	La stesura di questo documento è incrementale, cioè una stesura passo passo con modifiche, aggiunte e cancellazioni a seguito di miglioramenti del metodo di lavoro. I membri del gruppo si impegnano a visionare costantemente questo documento e a rispettare rigorosamente le regole definite in esso, per svolgere il progetto$^G$ in modo professionale, coerente ed uniforme.

\subsection{Scopo del prodotto}
	La gestione automatizzata degli ordini di acquisto in contesti multicanale rappresenta una sfida complessa per le aziende moderne, che devono affrontare la necessità di interpretare richieste provenienti da fonti eterogenee come email, chat, messaggi vocali e immagini.\\
	Il capitolato$^G$ numero C8 di Ergon Informatica propone di sviluppare una piattaforma intelligente in grado di analizzare input multimodali e convertirli automaticamente in ordini strutturati, pronti per l'inserimento nei sistemi gestionali aziendali.

	L'obiettivo che si è posto questo gruppo è realizzare un sistema basato su architettura a microservizi che integri tecniche avanzate di intelligenza artificiale$^G$, machine learning$^G$ e natural language processing$^G$, in grado di riconoscere le intenzioni del cliente, estrarre le informazioni rilevanti e validarle in maniera coerente con il catalogo prodotti aziendale. Questo approccio consentirà di ridurre drasticamente l'intervento umano nelle fasi ripetitive, migliorando al contempo l'efficienza$^G$ complessiva e la soddisfazione del cliente finale.

	Il progetto$^G$ SmartOrder si propone quindi di dimostrare come le tecnologie di intelligenza artificiale$^G$ possano essere applicate con successo a processi reali di business, trasformando un compito complesso e frammentato in un flusso lineare, automatizzato e scalabile. L'obiettivo è realizzare questo progetto$^G$ entro il 30 Aprile 2026 con un budget a disposizione di: Euro 11.440.
	
\subsection{Glossario}
    La realizzazione di un sistema software complesso come SmartOrder richiede, ancora prima della scrittura del codice, un'importante operazione di confronto, analisi e progettazione$^G$. Per supportare e facilitare il lavoro asincrono tra i membri del gruppo e garantire una comunicazione efficace con il committente$^G$, tutte le informazioni derivanti da questa attività saranno appositamente documentate in un glossario condiviso, utile per evitare ambiguità o incomprensioni riguardanti la nomenclatura adottata in tutti i documenti visionabili.

    In accordo con quanto stabilito nel \href{https://nullpointersgroup.github.io/Documentazione/output/RTB/Verbali\%20Interni/2025-11-06\_verbale\_interno.pdf}{verbale interno del 6 novembre 2025}, si è deciso 
	di adottare il glossario come strumento ufficiale per la standardizzazione della terminologia di progetto$^G$ e di assegnare la responsabilità della sua manutenzione$^G$ alla figura dell'Analista$^G$.

    La nomenclatura utilizzata per segnalare che la definizione di una parola è contenuta nel glossario è la seguente: 
    \begin{center}
        termine$^{G}$
    \end{center}

    I termini sono ordinati alfabeticamente per facilitarne la consultazione e vengono inclusi sia termini tecnici che acronimi significativi.

    Il gruppo si impegna a visionare il Glossario periodicamente, per permettere la più completa comprensione di ogni tipo di documento pubblicato e per mantenere un allineamento semantico costante tra tutti i partecipanti al progetto$^G$. 

    \subsection{Riferimenti}
	\subsubsection{Riferimenti normativi}

	\begin{itemize}[itemsep=5pt, parsep=5pt, label=$\scriptstyle\bullet$]

    \item \textbf{Capitolato$^G$ C8 - Ergon Informatica Srl - SmartOrder}\\
    \url{https://www.math.unipd.it/~tullio/IS-1/2025/Progetto/C8.pdf}\\[3pt]
    \textbf{Ultima consultazione: 30 Novembre 2025}

	\end{itemize}

	\subsubsection{Riferimenti informativi}
	\begin{itemize}[itemsep=5pt, parsep=5pt, label=$\scriptstyle\bullet$]

    \item \textbf{Standard ISO/IEC 9126}\\
     \url{https://en.wikipedia.org/wiki/ISO/IEC\_9126}\\[3pt]
    \textbf{Ultima consultazione: 13 Dicembre 2025}

    \item \textbf{Standard ISO/IEC/IEEE$^G$ 12207:1995}\\
     \url{https://www.math.unipd.it/~tullio/IS-1/2009/Approfondimenti/ISO\_12207-1995.pdf}\\[3pt]
    \textbf{Ultima consultazione: 12 Dicembre 2025}

	\item \textbf{Glossario, versione 1.0.0}\\
     \url{https://nullpointersgroup.github.io/Documentazione/output/RTB/Documenti\%20Interni/Glossario.pdf}\\[3pt]
    \textbf{Ultima consultazione: 13 Dicembre 2025}

	\end{itemize}
