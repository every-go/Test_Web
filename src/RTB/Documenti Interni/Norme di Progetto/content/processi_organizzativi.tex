\section{Processi Organizzativi}
I processi organizzativi definiscono un insieme di operazioni di supporto per lo sviluppo$^G$ software che operano trasversalmente rispetto al ciclo di vita del software garantendo che il gruppo possieda l’organizzazione, le infrastrutture e le competenze necessarie per sostenere i processi primari.\\
Assicurano la buona esecuzione di tutti i processi adottati e eventuali miglioramenti.\\
Si individuano i seguenti processi:
\begin{itemize}[itemsep=3pt, parsep=1pt, label=$\scriptstyle\bullet$]
    \item Gestione dei Processi;
    \item Gestione dell'Infrastruttura;
    \item Processo$^G$ di Miglioramento;
    \item Processo$^G$ di Formazione.
\end{itemize}

\subsection{Gestione dei Processi}
Secondo lo standard ISO$^G$ 12207:1995, \vr{La gestione dei processi comprende le attività e i compiti che possono essere svolti da qualsiasi soggetto che debba gestire i propri processi}.\\
Sulla base di questo principio, il suo scopo principale è stabilire come un processo$^G$ deve essere pianificato e monitorato secondo le relative responsabilità dei membri del gruppo.\\
Un altro obiettivo fondamentale è garantire un flusso comunicativo efficace, sia interno che esterno assicurando: coerenza, controllo e miglioramento continuo.
 \subsubsection{Attività previste}
 Per assicurare il raggiungimento degli obiettivi nel rispetto di tempi e qualità, il processo$^G$ è strutturato nelle seguenti attività, che definiscono un flusso di lavoro chiaro e responsabilità precise:
    \subsubsubsection{Inizializzazione}
    L’avvio del processo$^G$ avviene tramite la selezione dei Requisiti, presenti nel documento \vr{Analisi dei Requisiti$^G$}, da portare a termine tramite le attività di tale processo$^G$.\\
    Il Responsabile$^G$ valuta preliminarmente la fattibilità del processo$^G$: se alcuni requisiti risultano irrealizzabili per vincoli di tempo, risorse o competenze, e previo accordo di tutto il gruppo, i requisiti del processo$^G$ possono essere modificati in questa fase per garantire il raggiungimento dei criteri di completamento.
    \subsubsubsection{Pianificazione}
    L’attività di pianificazione$^G$, portata a termine dal Responsabile$^G$, ha lo scopo di preparare il piano di esecuzione delle attività del processo$^G$. 
    In particolare deve verificare la disponibilità delle risorse necessarie (budget residuo, della disponibilità dei componenti del gruppo, competenza, etc.) per completare il processo$^G$ entro i tempi stabiliti.\\
    Infine assegna le attività del processo$^G$ ai membri del team in base ai loro ruoli.
    \subsubsubsection{Esecuzione e controllo}
    Durante l’esecuzione: i membri del team portano a termine le attività assegnategli, mentre il Responsabile$^G$ ha il compito di monitorare l’andamento delle attività.\\
    Qualora si presentassero problemi, il Responsabile$^G$ deve essere immediatamente notificato e in caso di stallo, contattare la proponente$^G$ o il committente$^G$ per un chiarimento.
    \subsubsubsection{Verifica}
    La Verifica$^G$ del prodotto realizzato dai membri del team è di competenza del Verificatore$^G$, che assicura la corretta realizzazione del singolo requisito tramite la procedura di Verifica$^G$. 
    \subsubsubsection{Chiusura}
    La chiusura del processo$^G$ consegue la terminazione di tutte le attività che ne hanno preso parte.\\
    È compito del Responsabile$^G$ o del Verificatore$^G$ approvare il merge$^G$ della Pull Request$^G$ nel branch$^G$ main; una volta fatto il merge$^G$, il processo$^G$ è da definirsi chiuso. %da confermare se è il responsabile$^G$ 

\subsubsection{Tracciamento delle ore}
Per monitorare il tempo dedicato ai diversi ruoli durante il progetto$^G$, viene utilizzato uno spreadsheet dedicato, accessibile al gruppo su Google Drive.\\
Al termine di ogni sprint$^G$, viene generato il consuntivo$^G$ dello sprint$^G$ appena concluso e il preventivo di quello successivo. \\
Il membro che ricopre il ruolo di Responsabile$^G$ ha il compito di inserire tali informazioni nel documento \vr{Piano di Progetto$^G$}.
\subsubsubsection{Preventivo}
Il preventivo delle ore è presentato in forma tabellare e riporta le ore stimate per ciascun membro del gruppo, suddivise per ruolo e riferite al singolo sprint$^G$.\\
Viene inoltre creato un diagramma circolare che illustra la distribuzione$^G$ delle ore previste per ogni ruolo, fornendo una rappresentazione immediata e intuitiva delle risorse che si prevede di allocare per quello sprint$^G$.
\subsubsubsection{Consuntivo}
Il consuntivo$^G$ delle ore è anch’esso presentato tramite una tabella che riporta le ore effettivamente registrate da ciascun membro, suddivise per ruolo e relative allo sprint$^G$.\\ 
Anche in questo caso viene incluso un diagramma circolare che visualizza la ripartizione delle ore effettive per ruolo, fornendo una visione immediata delle risorse realmente impiegate durante lo sprint$^G$.

\subsubsection{Ruoli}
\begin{table}[H]
\centering
\renewcommand{\arraystretch}{1.15}
\begin{tabular}{ |p{0.25\textwidth}|p{0.65\textwidth}|}
    \hline
    \textbf{Ruolo} & \textbf{Compiti} \\
    \hline
    Responsabile$^G$ & Il Responsabile$^G$ coordina le attività del gruppo garantendo una pianificazione$^G$ efficace. \\
    \hline
    Amministratore$^G$ & L’Amministratore$^G$ si occupa della configurazione e gestione dell’infrastruttura IT di supporto al progetto$^G$. \\
    \hline
    Analista$^G$ & L’Analista$^G$ si occupa di identificare e chiarire i requisiti, interpretando le esigenze degli utilizzatori finali per garantire una corretta definizione delle funzionalità$^G$. \\
    \hline
    Verificatore$^G$ & Il Verificatore$^G$ si occupa di assicurare la qualità dei prodotti e dei processi adottati, effettuando revisioni e test$^G$. \\
    \hline
    Programmatore$^G$ & Il Programmatore$^G$ è responsabile$^G$ dello sviluppo$^G$ del codice sorgente del progetto$^G$, traducendo il design in codice funzionante e testabile dal proponente$^G$. \\
    \hline
    Progettista$^G$ & Il Progettista$^G$ traduce i requisiti del sistema in un’architettura software dettagliata, definendo moduli, interfacce, flussi dati e vincoli tecnici. \\
    \hline
\end{tabular}
\caption{Ruoli}
\end{table}

\subsubsubsection{Rotazione dei ruoli}
La rotazione dei ruoli avviene ogni due settimane, in modo da garantire una distribuzione$^G$ equilibrata delle competenze e favorire l’apprendimento trasversale all’interno del gruppo.\\
Ci riserviamo tuttavia la possibilità di effettuare modifiche ai ruoli anche a metà sprint$^G$, qualora necessario.\\
Tale decisione$^G$ può derivare dall’esito dell’incontro intermedio, durante il quale viene valutato l’avanzamento delle attività e individuati eventuali ritardi, blocchi o membri sottoutilizzati.\\
In questi casi, l’assegnazione di nuovi incarichi o la riorganizzazione dei ruoli permette una migliore suddivisione dei task$^G$ in unità più piccole e gestibili, favorendo il completamento efficace delle attività pianificate.

\subsubsection{Coordinamento}
Il coordinamento è un fattore fondamentale per il buon andamento del progetto$^G$, perché permette di gestire in modo efficace le attività del team e le relazioni con la proponente$^G$ e i committenti.\\
Per assicurare un flusso informativo continuo e aggiornato, sono pianificate riunioni regolari e l’impiego di canali di comunicazione adeguati.

\subsubsubsection{Riunioni}
Al fine di garantire un flusso di comunicazione efficace, un costante allineamento del gruppo e un confronto attivo con la proponente$^G$, sono previsti due tipi distinti di incontri:
\subsubsubsubsection{Riunioni interne}
Le riunioni interne vengono organizzate regolarmente, in genere ogni lunedì della settimana.\\
Durante questi incontri il gruppo effettua un punto della situazione: vengono analizzate le attività svolte, quelle ancora in corso e le eventuali criticità emerse.\\
Il Responsabile$^G$, grazie al confronto tra i membri, può così ottenere una visione aggiornata dell’avanzamento del progetto$^G$ e pianificare al meglio le attività successive.
\subsubsubsubsection{Riunioni esterne}
Le riunioni esterne coinvolgono i membri del gruppo e la proponente$^G$.\\
Tali incontri non seguono una periodicità fissa: vengono programmati secondo necessità tramite richiesta via canale concordato con la proponente$^G$ (come specificato al punto \ref{sec:strumenti-supporto}).\\
Durante le riunioni esterne il gruppo presenta lo stato di avanzamento del lavoro, chiarisce eventuali dubbi e riceve indicazioni utili per proseguire nelle attività di sviluppo$^G$.\\
Al termine di ogni riunione, interna o esterna, viene redatto un verbale (specifiche a sezioni \ref{sec:verbali-interni} e \ref{sec:verbali-esterni}) che documenta gli argomenti trattati e le decisioni prese, garantendo tracciabilità e condivisione delle informazioni.

\subsubsubsection{Comunicazioni}
Le comunicazioni costituiscono un elemento fondamentale per garantire coordinamento, continuità informativa e tempestività nelle attività del gruppo. Gli strumenti utilizzati per la comunicazione sono descritti nella sezione \ref{sec:strumenti-supporto}; in questa sezione viene invece specificato come tali strumenti vengono impiegati all’interno dei processi organizzativi.
\subsubsubsubsection{Comunicazioni interne}
Le comunicazioni interne hanno lo scopo di supportare il lavoro quotidiano del team, favorire la condivisione di informazioni e consentire un rapido confronto operativo.
\begin{itemize}[itemsep=3pt, parsep=1pt, label=$\scriptstyle\bullet$]
    \item Le comunicazioni rapide o di servizio avvengono mediante il canale di messaggistica adottato dal gruppo (WhatsApp$^G$);
    \item Le discussioni che richiedono un confronto più approfondito o decisioni condivise vengono svolte tramite incontri a distanza (Discord$^G$);
    \item La documentazione dei task$^G$, degli avanzamenti e delle attività pianificate è gestita attraverso lo strumento di versionamento$^G$ e ticketing adottato dal team (GitHub$^G$).
\end{itemize}
Questa organizzazione consente di distinguere in modo chiaro tra comunicazioni informali, operative e momenti di coordinamento strutturato.
\subsubsubsubsection{Comunicazioni esterne}
Le comunicazioni esterne garantiscono il mantenimento di un flusso regolare e trasparente con la proponente$^G$ e con i committenti.
\begin{itemize}[itemsep=3pt, parsep=1pt, label=$\scriptstyle\bullet$]
    \item Le comunicazioni formali, come approvazioni di verbali esterni, avvengono tramite l’indirizzo e-mail istituzionale del gruppo (Google Mail);
    \item Le riunioni esterne sono organizzate attraverso la piattaforma concordata per gli incontri a distanza (Google Meet$^G$);
    \item Per eventuali comunicazioni asincrone o rapide è previsto anche un canale condiviso con la proponente$^G$ (WhatsApp$^G$), come stabilito in fase iniziale.
\end{itemize}
Tutte le comunicazioni esterne rilevanti ai fini del progetto$^G$ vengono tracciate mediante la produzione$^G$ dei verbali esterni.

\subsection{Gestione dell'Infrastruttura}
Il processo$^G$ di infrastruttura ha lo scopo di fornire, configurare e mantenere l’ambiente di lavoro necessario all'esecuzione di tutti i processi di sviluppo$^G$ e documentazione.\\
Esso comprende la gestione delle risorse, siano esse hardware o software, garantendone la disponibilità e l'efficienza$^G$ per l'intera durata del progetto$^G$.
\subsubsection{Attività previste}
\subsubsubsection{Implementazione}
Per supportare il lavoro asincrono, la tracciabilità e la qualità dei prodotti NullPointers Group adotta i seguenti strumenti che costituiscono l’infrastruttura del progetto$^G$:
\begin{itemize}[itemsep=3pt, parsep=1pt, label=$\scriptstyle\bullet$]
    \item Gestione del versionamento$^G$: Git$^G$
    \item Piattaforma: GitHub$^G$
    \item Automazione$^G$: GitHub$^G$ Action$^G$, script Python$^G$ e Lua$^G$
    \item Comunicazione: Discord$^G$ e Whatsapp$^G$
\end{itemize}

\subsubsubsection{Predisposizione}
L’attività di predisposizione stabilisce le regole di interazione tra i membri del gruppo e l’ambiente di lavoro, inoltre definisce la natura dell’infrastruttura utilizzata.\\
L’infrastruttura adottata è finalizzata a minimizzare gli errori e a garantire la coerenza del prodotto.\\
Vengono riportati gli strumenti principali:

\begin{table}[!h]
\centering
\renewcommand{\arraystretch}{1.15}
\begin{tabular}{|p{0.23\textwidth}|p{0.7\textwidth}|}
    \hline
    \textbf{Strumento} & \textbf{Predisposizione} \\
    \hline
    Git$^G$ & Definizione di un file .gitignore condiviso per escludere i file temporanei e di build$^G$ garantendo che la repository$^G$ contenga solamente i file sorgente. \\
    \hline
    GitHub$^G$ & È stata creata una repository$^G$ dedicata alla documentazione del progetto$^G$. È stata applicata una  branch$^G$ protection rule sul ramo main: ogni modifica$^G$ deve provenire da una pull request$^G$ e richiede l’approvazione di un Verificatore$^G$ per il suo merge$^G$.\\
    \hline
    Discord$^G$ e \newline Whatsapp$^G$ & Per consentire al gruppo di riunirsi settimanalmente, e venire incontro al fatto che ci sono significative distanze tra i membri, è stato creato un server sulla piattaforma Discord$^G$, un’applicazione che consente videochiamate e scambio di messaggi; ideale per il nostro scopo. È stato inoltre creato un gruppo Whatsapp$^G$ per questioni minori che non richiedono una videochiamata. \\
    \hline
    Labels GitHub$^G$ & Sono state implementate delle Labels per categorizzare le attività e Milestones per tracciare l’avanzamento del progetto$^G$.\\
    \hline
    GitHub$^G$ Action$^G$, script Python$^G$ e script Lua$^G$ & Sono state configurate le GitHub$^G$ Action$^G$ per l’esecuzione automatica degli script che compilano i file sorgente LaTeX$^G$ ad ogni push garantendo che la versione PDF visibile sul sito sia sempre sincronizzata con l’ultima versione dei documenti. \\
    \hline
\end{tabular}
\caption{Strumenti di predisposizione}
\end{table}

\newpage
\subsubsubsection{Manutenzione}
Data la complessità del progetto$^G$ è probabile che l’infrastruttura subisca dei cambiamenti nel corso del tempo per l’aggiornamento o il miglioramento delle sue funzionalità$^G$.\\
È compito dell’Amministratore$^G$ la manutenzione$^G$ dell’infrastruttura ovvero le attività di controllo delle funzionalità$^G$ ed aggiornamento/creazione degli script di automazione$^G$.\\
Successivamente verranno illustrate le norme da seguire per mantenere e aggiornare l’infrastruttura affinché il flusso di lavoro non venga spezzato.

\subsubsubsubsection{Git}
Git$^G$ non ha bisogno di particolari configurazioni, è sufficiente accedere localmente con le credenziali che il membro usa per accedere a GitHub$^G$.

\subsubsubsubsection{GitHub}
Su GitHub$^G$, l’account di NullPointers Group è gestito come organizzazione, ovvero un account che serve da contenitore per il lavoro condiviso tra membri di un team.\\
Sono state create 3 repository$^G$ dentro l’organizzazione:
\begin{itemize}[itemsep=3pt, parsep=1pt, label=$\scriptstyle\bullet$]
    \item \textbf{Documentazione}: repository$^G$ dove viene salvata e versionata tutta la documentazione in merito al capitolato$^G$ SmartOrder e non solo.
    \item \textbf{PoC$^G$}: repository$^G$ dove viene salvato e versionato il codice sorgente per il Proof of Concept$^G$.
    \item \textbf{SmartOrder}: repository$^G$ dove viene salvato e versionato il codice sorgente dell’applicativo SmartOrder.
    \item \textbf{.github$^G$}: repository$^G$ che serve a creare il README.md del gruppo, gestire i workflow condivisi fra le varie repository$^G$. In generale, serve a centralizzare configurazioni e contenuti condivisi.
\end{itemize}
Affinché le nuove impostazioni vengano effettivamente applicate nelle repository$^G$ di documentazione o di codice sorgente, il verificatore$^G$ dovrà assicurarsi che la nuova infrastruttura proposta superi le metriche di qualità di processo$^G$ definite in seguito.

\subsubsubsubsection{Action e script ausiliari}
Nelle repository$^G$ di Documentazione e SmartOrder, si impiegano strumenti di CI messi a disposizione da GitHub$^G$: le GitHub$^G$ Action$^G$, le quali vengono definite in un file “.yml” dentro la cartella “.github/workflows”.\\
Affinché le GitHub$^G$ Action$^G$ potessero assolvere allo scopo per cui sono state configurate, sono stati sviluppati script ausiliari, in Python$^G$ e Lua$^G$, da far eseguire a quest'ultime.\\
La configurazione di nuove GitHub$^G$ o la modifica$^G$ di GitHub$^G$ Action$^G$ esistenti spetta agli Amministratori.\\
La creazione di nuove GitHub$^G$ Action$^G$ viene richiesta agli Amministratori dal Responsabile$^G$, sotto comune accordo dai membri del gruppo.\\

\subsubsubsubsection{Infrastruttura di Tracking}
La configurazione e la manutenzione$^G$ ordinaria degli strumenti di GitHub$^G$ (la cui organizzazione è descritta nella sezione \ref{sec:ticketing}) seguono le procedure definite successivamente.\\[1em]
\textbf{Sistema di Label}\\
Viene definita una palette standard di label e le convenzioni di nomenclatura. Periodicamente si esegue una pulizia per rimuovere quelle obsolete o unificarne di duplicate.\\
Qualsiasi modifica$^G$ allo standard viene discussa e approvata dal gruppo.\\[1em]
\textbf{Project Board}\\
Le board per le milestone$^G$ (RTB$^G$, PB$^G$) vengono create da un template predefinito. Al termine di una milestone$^G$, le board vengono archiviate dopo aver chiuso le issue$^G$ residue.\\
Mensilmente si verifica$^G$ che la posizione delle issue$^G$ nelle colonne corrisponda al loro stato effettivo.\\[1em]
\textbf{Issue$^G$ e Automazioni}\\
Si mantengono aggiornati i template predefiniti per la creazione di nuove issue$^G$. Vengono identificate e chiuse periodicamente le issue$^G$ inattive o completate. \\

\subsubsubsubsection{Discord}
Discord$^G$ è la piattaforma principale per la comunicazione interna del gruppo.\\
Il server è organizzato in canali testuali dedicati a specifiche aree, come la documentazione e il codice sorgente, per mantenere le discussioni focalizzate.\\
La creazione di nuovi canali avviene su richiesta di qualsiasi membro, ma è riservata all'Amministratore$^G$ del server, che provvede a configurarli, garantire l'accesso a tutti e mantenere la moderazione.

\subsection{Processo di Miglioramento}
Il miglioramento è un processo$^G$ finalizzato a valutare, misurare, controllare e ottimizzare il ciclo di vita del software.\\
Il suo obiettivo è garantire che il prodotto non solo risponda alle aspettative, ma raggiunga e mantenga standard elevati di qualità ed efficienza$^G$, attraverso revisioni periodiche e perfezionamenti incrementali.\\
L'approccio è ciclico e, secondo il modello Plan-Do-Check-Act (PDCA), si compone di quattro fasi principali, applicate periodicamente per garantire un progresso continuo.

\subsubsection{Ciclo PDCA}
Il miglioramento continuo si articola nelle seguenti quattro fasi, iterabili su base periodica o al verificarsi di eventi significativi:
\begin{itemize}[itemsep=3pt, parsep=1pt, label=$\scriptstyle\bullet$]
    \item \textbf{Plan} (Pianificazione$^G$): identificazione e documentazione dei processi organizzativi da applicare al ciclo di vita del software.
    \item \textbf{Do} (Implementazione): integrazione dei miglioramenti nei processi esistenti, con aggiornamento della documentazione e raccolta di dati storici e tecnici.
    \item \textbf{Check} (Valutazione): analisi dei processi rispetto agli obiettivi prefissati e alle metriche adottate, mediante i dati raccolti e revisioni periodiche.
    \item \textbf{Act} (Standardizzazione): consolidamento dei miglioramenti efficaci nei processi, per evitarne la regressione e assicurarne l'evoluzione costante.
\end{itemize}

\subsection{Processo di Formazione}
Il processo$^G$ di formazione è un'attività di supporto volta a garantire che tutti i membri del gruppo possiedano le competenze necessarie per svolgere i compiti assegnati e gestire le tecnologie richieste dal progetto$^G$.

\subsubsection{Attività}
La formazione del team si articola in un percorso strutturato che comprende:
\begin{itemize}[itemsep=3pt, parsep=1pt, label=$\scriptstyle\bullet$]
    \item \textbf{Analisi dei bisogni}: identificare le competenze richieste dal progetto$^G$ e le eventuali lacune del gruppo, partendo dai requisiti della proponente$^G$ e dalle tecnologie scelte.
    \item \textbf{Formazione con la Proponente$^G$}: partecipazione a incontri tecnici dedicati, organizzati dalla proponente$^G$, sulle tecnologie cardine del progetto$^G$ (Docker$^G$, FastAPI, LangChain, React$^G$). Queste sessioni accelerano la curva di apprendimento e allineano il gruppo con gli standard tecnici e le metodologie di sviluppo$^G$ adottate dall'azienda.
    \item \textbf{Apprendimento individuale}: studio autonomo, da parte di ciascun membro, degli strumenti, linguaggi e framework$^G$ necessari per il proprio ruolo e per le attività comuni.
    \item \textbf{Condivisione delle conoscenze} (Knowledge Sharing): diffusione interna delle competenze, facilitata dai membri più esperti in un dato ambito attraverso discussioni tecniche, sessioni di pair programming o documentazione informale.
    \item \textbf{Consolidamento e uniformità}: l'obiettivo è garantire un livello di preparazione omogeneo e adeguato agli obiettivi di progetto$^G$, promuovendo la crescita professionale di tutti i componenti.
\end{itemize}

\subsubsection{Implementazione del processo}
In base a quanto previsto dallo standard ISO/IEC 12207:1995, è necessario realizzare una revisione dei requisiti del progetto$^G$.\\
Questo passaggio serve a comprendere le competenze che i membri del NullPointers Group dovranno sviluppare per completare il progetto$^G$ didattico.\\
Come gruppo abbiamo quindi definito quali sono le tecnologie necessarie da approfondire ed utilizzare per la corretta realizzazione del progetto$^G$.\\
Queste tecnologie sono:
\begin{enumerate}
    \item \textbf{React$^G$}: framework$^G$ JavaScript/TypeScript per la realizzazione dell'interfaccia web.
    \item \textbf{Python$^G$}: linguaggio di programmazione versatile, utilizzato per implementare API$^G$ REST$^G$, gestire le comunicazioni con modelli di machine learning$^G$ e con il database$^G$, e realizzare il back-end.
    \item \textbf{PostgreSQL$^G$}: sistema di gestione di database$^G$ relazionale open source, utilizzato per memorizzare e gestire i dati dell’applicazione.
    \item \textbf{Docker$^G$}: piattaforma per la containerizzazione, che consente di distribuire e eseguire l’applicazione in ambienti isolati e replicabili.
    \item \textbf{LaTeX$^G$}: linguaggio di markup utilizzato per produrre la documentazione
\end{enumerate}