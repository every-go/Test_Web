\section{Processi di Supporto}
\subsection{Documentazione}
    Il processo di documentazione è un elemento cardine di tutti i processi primari. Il suo output è fondamentale per tracciare le decisioni$^{G}$ prese e per facilitare il lavoro asincrono, 
    che nel nostro contesto si rivela notevolmente più produttivo di quello sincrono. \\
    Nello specifico, questo processo si occupa di registrare le informazioni generate da ciascuna attività o processo del ciclo di vita del prodotto; comprende quindi tutte le operazioni di pianificazione$^{G}$, 
    progettazione$^{G}$, sviluppo$^{G}$, produzione$^{G}$, modifica$^{G}$, distribuzione$^{G}$ e manutenzione$^{G}$ dei documenti destinati a tutti i soggetti coinvolti.

\subsubsection{Linguaggio di Markup}
    Per la redazione dei documenti il gruppo ha deciso di utilizzare \textbf{LaTeX}$^{G}$ ovvero un linguaggio di markup consolidato e ampiamente utilizzato per la stesura di documenti tecnici e scientifici. LaTeX$^G$ consente 
    di mantenere un’elevata qualità tipografica e di gestire in modo efficiente la struttura del documento.\\
    NullPointers Group utilizza LaTeX$^G$ per la produzione$^G$ di tutta la documentazione, facendo uso di pacchetti e template appositamente sviluppati dai membri del gruppo.

\subsubsection{Versionamento}
    Il gruppo utilizza \textbf{GitHub}$^{G}$ come piattaforma principale per la gestione del versionamento$^{G}$ e della collaborazione nella redazione dei documenti.
    Attraverso il sistema di controllo di versione Git, è possibile mantenere uno storico completo di tutte le modifiche, garantendo tracciabilità, ordine e coordinamento tra i membri del gruppo.

\subsubsection{Caricamento in Repository}
    Ogni volta che si inserisce un nuovo documento o si effettua una modifica$^G$ nel Repository$^{G}$ si utilizza un branch$^G$ feature/* personalizzato.\\
    Una volta creato un commit nel branch$^{G}$ una GitHub$^G$ Action$^{G}$ si occupa di creare automaticamente una pull request$^{G}$, la quale deve essere approvata dai Verificatori$^{G}$.

\subsubsection{Struttura base dei documenti}
\begin{center}
\textbf{Intestazione}
\end{center}
\vspace{-0.5em}
La prima pagina funge da intestazione del documento e contiene i seguenti elementi:
\begin{itemize}[itemsep=0pt, parsep=1pt, label=$\scriptstyle\bullet$]
        \item Logo dell'Università degli Studi di Padova;
        \item Logo NullPointers Group;
        \item \textbf{Nome del documento};
        \item \textbf{Stato}: se il documento è stato Approvato o se è ancora In Approvazione;
        \item \textbf{Versione}: ultima versione verificata o approvata del documento;
        \item \textbf{Data ultima modifica}: ultima data in cui è stato modificato il documento (se ritenuta necessaria);
        \item \textbf{Redattori}: coloro che hanno partecipato alla redazione del documento;
        \item \textbf{Verificatori}: coloro che hanno partecipato alla verifica$^{G}$ di parti del documento (presente in documenti diversi da Verbali);
        \item \textbf{Partecipanti}: coloro che partecipano alla riunione, interna o esterna che sia (presente solo nei Verbali);
        \item \textbf{Destinatari} del documento, ovvero a chi è rivolto.
    \end{itemize}

\begin{center}
\textbf{Registro delle modifiche}
\end{center}
\vspace{-0.5em}
Segue il Registro delle modifiche in forma tabellare che consente la tracciabilità delle modifiche apportate al documento, nel quale viene specificato: 
    \begin{itemize}[itemsep=0pt, parsep=1pt, label=$\scriptstyle\bullet$]
        \item \textbf{Versione}: numero della versione del documento (identificativo unico);
        \item \textbf{Data}: data di approvazione della versione del documento;
        \item \textbf{Autore}: persona che ha apportato modifiche;
        \item \textbf{Verificatore}: persona che ha approvato le modifiche;
        \item \textbf{Descrizione}: breve descrizione delle modifiche apportate al documento.
    \end{itemize}

\begin{center}
\textbf{Contents}
\end{center}
\vspace{-0.5em}
Nella pagina successiva al registro delle modifiche è presente l'indice generale, nel quale vengono elencate tutte le sezioni che compongono il documento.

\begin{center}
\textbf{Indice delle tabelle e delle immagini}
\end{center}
\vspace{-0.5em}
Successivamente all'indice, qualora il documento contenga elementi grafici o tabellari, vengono riportati l'indice delle tabelle e l'indice delle immagini. 
Tali indici descrivono il contenuto di ciascun elemento e ne specificano la collocazione all'interno del documento.

\begin{center}
\textbf{Contenuto principale}
\end{center}
\vspace{-0.5em}
Il contenuto del documento è strutturato in modo gerarchico per organizzare al meglio i contenuti:
\begin{itemize}[itemsep=0pt, parsep=1pt, label=$\scriptstyle\bullet$]
    \item \textbf{Capitoli}: rappresentano le macro-aree tematiche;
    \item \textbf{Sezioni}: suddividono i capitoli in argomenti specifici;
    \item \textbf{Sottosezioni}: se necessarie, approfondiscono i dettagli di ogni argomento.
\end{itemize}
\vspace{0.1em}

\subsubsubsection{Verbali Interni}
Il contenuto principale dei verbali interni segue una struttura standardizzata composta dai seguenti elementi: \vspace{-0.5em}

    \begin{enumerate}[label=\arabic*., itemsep=0pt, parsep=1pt, leftmargin=*]
        \item \textbf{Informazioni generali}
        \begin{itemize}[label=--, itemsep=0pt, parsep=1pt, leftmargin=1.5em]
            \item Tipo di riunione: Interna;
            \item Luogo della riunione: in presenza o sulla piattaforma Discord;
            \item Data della riunione;
            \item Orario di inizio; 
            \item Orario di fine;
            \item Scriba, ovvero il nome di chi si occupa di redigere il Verbale.
        \end{itemize}
        \item \textbf{Ordine del giorno} 
    \end{enumerate}

    \vspace{-0.5em}
    Ha lo scopo di delineare in modo strutturato e puntuale gli argomenti che verranno discussi durante la riunione.

    \begin{enumerate}[resume*, itemsep=0pt, parsep=1pt, leftmargin=*, label=\arabic*.]
        \item \textbf{Diario della riunione}
    \end{enumerate}

    \vspace{-0.5em}
    Documenta in modo sintetico ma completo lo svolgimento della riunione, registrando le discussioni principali, le decisioni prese e le attività concordate.

    \begin{enumerate}[resume*, itemsep=0pt, parsep=1pt, leftmargin=*, label=\arabic*.]
        \item \textbf{Decisioni e Azioni}
    \end{enumerate}

    \vspace{-0.5em}
    Ha lo scopo di registrare in modo formale e strutturato tutte le deliberazioni e i compiti emersi durante l'incontro. La tabella funge da riferimento ufficiale e chiaro per tutto il gruppo, 
    riassumendo cosa è stato stabilito e quali attività devono essere svolte.\\
    Queste informazioni non rimangono confinate al documento, ma vengono integrate e tracciate all'interno del nostro sistema di ticketing. Ciò garantisce che ogni elemento sia documentato per riferimento futuro e attivo per la sua esecuzione, 
    collegando direttamente la decisione$^G$ al task.

\vspace{0.7em}
\subsubsubsection{Verbali Esterni}
I verbali esterni sono documenti che registrano ufficialmente gli incontri avvenuti con soggetti esterni al gruppo di lavoro, in particolare con l'ente proponente del progetto. Tali documenti hanno lo scopo di tracciare le discussioni, 
le decisioni concordate e gli impegni assunti da entrambe le parti durante la riunione.
    \begin{enumerate}[label=\arabic*., itemsep=0pt, parsep=1pt, leftmargin=*]
        \item \textbf{Informazioni generali}
        \begin{itemize}[label=--, itemsep=0pt, parsep=1pt, leftmargin=1.5em]
            \item Tipo di riunione: Esterna;
            \item Luogo della riunione: concordato con il proponente;
            \item Data della riunione;
            \item Orario di inizio; 
            \item Orario di fine;
            \item Scriba, ovvero il nome di chi si occupa di redigere il Verbale;
            \item Partecipanti: ovvero i referenti dell'azienda proponente con i quali viene svolto l'incontro. 
        \end{itemize}
        \item \textbf{Ordine del giorno} 
    \end{enumerate}

    \vspace{-0.5em}
    L'ordine del giorno ha lo scopo di delineare ciò che verrà discusso durante la riunione, con particolare riferimento a chiarimenti dei dubbi emersi e 
    alle domande sollevate dal gruppo in preparazione dell'incontro con l'azienda.

    \begin{enumerate}[resume*, itemsep=0pt, parsep=1pt, leftmargin=*, label=\arabic*.]
        \item \textbf{Diario della riunione}
    \end{enumerate}

    \vspace{-0.5em}
    Documenta in modo sintetico ma completo lo svolgimento dell'incontro, registrando le discussioni principali, le decisioni prese e le attività concordate. Fornisce un resoconto strutturato degli argomenti trattati, mantenendo traccia di tutti gli aspetti rilevanti emersi durante il confronto.

    \begin{enumerate}[resume*, itemsep=0pt, parsep=1pt, leftmargin=*, label=\arabic*.]
        \item \textbf{Approvazione esterna}
    \end{enumerate}

    \vspace{-0.5em}
    È l'ultima sezione del documento che attesta che i relativi verbali esterni siano approvati dalla proponente tramite firma ed eventuale timbro del rappresentante.

\vspace{0.7em}
\subsubsubsection{Diari di Bordo}
    I diari di bordo sono presentazioni utilizzate durante gli incontri settimanali con l'obiettivo di verificare in modo condiviso lo stato di avanzamento di ciascun gruppo ammesso al I lotto.\\
    La struttura tipica di un diario di bordo comprende le seguenti sezioni:
    \begin{itemize}[itemsep=0pt, parsep=1pt, label=$\scriptstyle\bullet$]
        \item \textbf{Risultati}: descrive le attività completate nel periodo corrente e le confronta con quanto inizialmente pianificato;
        \item \textbf{Problematiche riscontrate}: consente di illustrare le problematiche affrontate e i dubbi ancora irrisolti, sono volti alla richiesta di supporto o chiarimenti;
        \item \textbf{Attività future}: elenca i compiti da svolgere nel prossimo intervallo di lavoro.
    \end{itemize}

\vspace{0.3em}
\subsubsubsection{Altri documenti}
Di seguito sono elencati tutti i documenti redatti e mantenuti durante l'intero ciclo di vita del progetto ciascuno dei quali risponde a uno scopo specifico, contribuendo alla tracciabilità, alla gestione e alla comunicazione delle attività di progetto 
La struttura iniziale di tali documenti corrisponde con quanto definito al punto 3.1.4:
    \begin{itemize}[itemsep=0pt, parsep=1pt, label=$\scriptstyle\bullet$]
        \item \textbf{Valutazione dei capitolati};
        \item \textbf{Dichiarazione degli impegni};
        \item \textbf{Lettera di presentazione};
        \item \textbf{Norme di progetto};
        \item \textbf{Analisi dei requisiti};
        \item \textbf{Piano di progetto};
        \item \textbf{Piano di qualifica};
        \item \textbf{Glossario};
        \item \textbf{Specifica Tecnica};
        \item \textbf{Manuale Utente}.
    \end{itemize}


\subsection{Gestione della Configurazione}

\subsection{Gestione della Qualità}

\subsection{Verifica}

\subsection{Validazione}