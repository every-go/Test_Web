\section{Processi di Supporto}
\subsection{Documentazione}
    Il processo$^G$ di documentazione è un elemento cardine di tutti i processi primari. Il suo output è fondamentale per tracciare le decisioni prese e per facilitare il lavoro asincrono, che nel nostro contesto si rivela notevolmente più produttivo di quello sincrono. \\
    Nello specifico, questo processo$^G$ si occupa di registrare le informazioni generate da ciascuna attività o processo$^G$ del ciclo di vita del prodotto; comprende quindi tutte le operazioni di pianificazione$^G$, progettazione$^G$, sviluppo$^G$, produzione$^G$, modifica$^G$, distribuzione$^G$ e manutenzione$^G$ dei documenti destinati a tutti i soggetti coinvolti.

\subsubsection{Linguaggio di Markup}
    Per la redazione dei documenti il gruppo ha deciso di utilizzare \textbf{LaTeX$^G$} ovvero un linguaggio di markup consolidato e ampiamente utilizzato per la stesura di documenti tecnici e scientifici. LaTeX$^G$ consente di mantenere un’elevata qualità tipografica e di gestire in modo efficiente la struttura del documento.\\
    NullPointers Group utilizza LaTeX$^G$ per la produzione$^G$ di tutta la documentazione, facendo uso di pacchetti e template appositamente sviluppati dai membri del gruppo.

\subsubsection{Versionamento}
    Il gruppo utilizza \textbf{GitHub$^G$} come piattaforma principale per la gestione del versionamento$^G$ e della collaborazione nella redazione dei documenti.
    Attraverso il sistema di controllo di versione Git$^G$, è possibile mantenere uno storico completo di tutte le modifiche, garantendo tracciabilità, ordine e coordinamento tra i membri del gruppo.

\subsubsection{Caricamento in Repository}
    Ogni volta che si inserisce un nuovo documento o si effettua una modifica$^G$ nel Repository$^G$ si utilizza un branch$^G$ feature/* personalizzato, la cui nomenclatura è definita nella sezione \ref{sec: nomenclatura-branch}.\\
    Una volta creato un commit nel branch$^G$ una GitHub$^G$ Action$^G$ si occupa di creare automaticamente una PR$^G$, la quale deve essere approvata dai Verificatori.

\subsubsection{Struttura base dei documenti}
\label{sec: struttura-base-doc}
\begin{center}
\textbf{Intestazione}
\end{center}
\vspace{-0.5em}
La prima pagina funge da intestazione del documento e contiene i seguenti elementi:
\begin{itemize}[itemsep=3pt, parsep=1pt, label=$\scriptstyle\bullet$]
        \item Logo dell'Università degli Studi di Padova;
        \item Logo NullPointers Group;
        \item \textbf{Nome del documento};
        \item \textbf{Stato}: se il documento è stato \vrs{Approvato} o se è ancora \vrs{In Approvazione};
        \item \textbf{Versione}: ultima versione verificata o approvata del documento;
        \item \textbf{Data ultima modifica$^G$}: ultima data in cui è stato modificato il documento (se ritenuta necessaria);
        \item \textbf{Redattori}: coloro che hanno partecipato alla redazione del documento;
        \item \textbf{Verificatori}: coloro che hanno partecipato alla verifica$^G$ di parti del documento (presente in documenti diversi da Verbali);
        \item \textbf{Partecipanti}: coloro che partecipano alla riunione, interna o esterna che sia (presente solo nei Verbali);
        \item \textbf{Destinatari} del documento, ovvero a chi è rivolto.
    \end{itemize}

\phantomsection
\label{sec:registro-modifiche}
\begin{center}
\textbf{Registro delle modifiche}
\end{center}
\vspace{-0.5em}
Segue il Registro delle modifiche in forma tabellare che consente la tracciabilità delle modifiche apportate al documento, nel quale viene specificato: 
    \begin{itemize}[itemsep=3pt, parsep=1pt, label=$\scriptstyle\bullet$]
        \item \textbf{Versione}: numero della versione del documento (identificativo unico);
        \item \textbf{Data}: data di approvazione della versione del documento;
        \item \textbf{Autore}: persona che ha apportato modifiche;
        \item \textbf{Verificatore$^G$}: persona che ha approvato le modifiche;
        \item \textbf{Descrizione}: breve descrizione delle modifiche apportate al documento.
    \end{itemize}

\begin{center}
\textbf{Contents}
\end{center}
\vspace{-0.5em}
Nella pagina successiva al registro delle modifiche è presente l'indice generale, nel quale vengono elencate tutte le sezioni che compongono il documento.

\begin{center}
\textbf{Indice delle tabelle e delle immagini}
\end{center}
\vspace{-0.5em}
Successivamente all'indice, qualora il documento contenga elementi grafici o tabellari, vengono riportati l'indice delle tabelle e l'indice delle immagini. 
Tali indici descrivono il contenuto di ciascun elemento e ne specificano la collocazione all'interno del documento.

\begin{center}
\textbf{Contenuto principale}
\end{center}
\vspace{-0.5em}
Il contenuto del documento è strutturato in modo gerarchico per organizzare al meglio i contenuti:
\begin{itemize}[itemsep=3pt, parsep=1pt, label=$\scriptstyle\bullet$]
    \item \textbf{Capitoli}: rappresentano le macro-aree tematiche;
    \item \textbf{Sezioni}: suddividono i capitoli in argomenti specifici;
    \item \textbf{Sottosezioni}: se necessarie, approfondiscono i dettagli di ogni argomento.
\end{itemize}
\vspace{0.1em}

\subsubsubsection{Verbali Interni}
\label{sec:verbali-interni}
I Verbali Interni seguono la struttura base dei documenti esposta alla sezione \ref{sec: struttura-base-doc} ma diversamente dal resto della documentazione non contengono la sezione del Registro delle modifiche.\\
Il contenuto principale dei Verbali Interni segue una struttura standardizzata composta dai seguenti elementi: \vspace{-0.5em}

    \begin{enumerate}[label=\arabic*., itemsep=0pt, parsep=1pt, leftmargin=*]
        \item \textbf{Informazioni generali}
        \begin{itemize}[label=--, itemsep=0pt, parsep=1pt, leftmargin=1.5em]
            \item Tipo di riunione: Interna;
            \item Luogo della riunione: in presenza o sulla piattaforma Discord$^G$;
            \item Data della riunione;
            \item Orario di inizio; 
            \item Orario di fine;
            \item Scriba, ovvero chi si occupa di redigere il Verbale che, come stabilito nella sezione \ref{sec: verbali-interni}, spetta alla figura del Responsabile$^G$.
        \end{itemize}
        \item \textbf{Ordine del giorno} 
    \end{enumerate}

    \vspace{-0.5em}
    Ha lo scopo di delineare in modo strutturato e puntuale gli argomenti che verranno discussi durante la riunione.

    \begin{enumerate}[resume*, itemsep=0pt, parsep=1pt, leftmargin=*, label=\arabic*.]
        \item \textbf{Diario della riunione}
    \end{enumerate}

    \vspace{-0.5em}
    Documenta in modo sintetico ma completo lo svolgimento della riunione, registrando le discussioni principali, le decisioni prese e le attività concordate.

    \begin{enumerate}[resume*, itemsep=0pt, parsep=1pt, leftmargin=*, label=\arabic*.]
        \item \textbf{Decisioni e Azioni}
    \end{enumerate}

    \vspace{-0.5em}
    Ha lo scopo di registrare in modo formale e strutturato tutte le deliberazioni e i compiti emersi durante l'incontro.\\
    La tabella funge da riferimento ufficiale e chiaro per tutto il gruppo, riassumendo cosa è stato stabilito e quali attività devono essere svolte.\\
    Alcune di queste informazioni non rimangono confinate al documento, ma vengono integrate e tracciate all'interno del nostro sistema di ticketing.\\
    Ciò garantisce che ogni elemento sia documentato per riferimento futuro e attivo per la sua esecuzione, collegando direttamente la decisione$^G$ al task$^G$.

\vspace{0.7em}
\subsubsubsection{Verbali Esterni}
\label{sec:verbali-esterni}
I verbali esterni sono documenti che registrano ufficialmente gli incontri avvenuti con soggetti esterni al gruppo di lavoro, in particolare con l'ente proponente$^G$ del progetto$^G$.\\
Tali documenti hanno lo scopo di tracciare le discussioni, le decisioni concordate e gli impegni assunti da entrambe le parti durante la riunione.\\
Seguono la struttura base dei documenti esposta alla sezione \ref{sec: struttura-base-doc} ma diversamente dal resto della documentazione non contengono la sezione del Registro delle Modifiche.
    \begin{enumerate}[label=\arabic*., itemsep=0pt, parsep=1pt, leftmargin=*]
        \item \textbf{Informazioni generali}
        \begin{itemize}[label=--, itemsep=0pt, parsep=1pt, leftmargin=1.5em]
            \item Tipo di riunione: Esterna;
            \item Luogo della riunione: concordato con il proponente$^G$;
            \item Data della riunione;
            \item Orario di inizio; 
            \item Orario di fine;
            \item Scriba, ovvero chi si occupa di redigere il Verbale che, come stabilito nella sezione \ref{sec: verbali-interni}, spetta alla figura del Responsabile$^G$.;
            \item Partecipanti: ovvero i referenti dell'azienda proponente$^G$ con i quali viene svolto l'incontro. 
        \end{itemize}
        \item \textbf{Ordine del giorno} 
    \end{enumerate}
    \vspace{-0.5em}
    L'ordine del giorno ha lo scopo di delineare ciò che verrà discusso durante la riunione, con particolare riferimento a chiarimenti dei dubbi emersi e alle domande sollevate dal gruppo in preparazione dell'incontro con l'azienda.

    \begin{enumerate}[resume*, itemsep=0pt, parsep=1pt, leftmargin=*, label=\arabic*.]
        \item \textbf{Diario della riunione}
    \end{enumerate}
    \vspace{-0.5em}
    Documenta in modo sintetico ma completo lo svolgimento dell'incontro, registrando le discussioni principali, le decisioni prese e le attività concordate. Fornisce un resoconto strutturato degli argomenti trattati, mantenendo traccia di tutti gli aspetti rilevanti emersi durante il confronto.

    \begin{enumerate}[resume*, itemsep=0pt, parsep=1pt, leftmargin=*, label=\arabic*.]
        \item \textbf{Decisioni e Azioni}
    \end{enumerate}
    \vspace{-0.5em}
    Questa sezione ha lo scopo di registrare le decisioni e i relativi compiti emersi dall'incontro, fungendo da riferimento ufficiale per il gruppo e permettendo la tracciabilità delle decisioni.\\
    Eventuali task$^G$, se tracciabili, verranno integrati nel sistema di ticketing.
    \begin{enumerate}[resume*, itemsep=0pt, parsep=1pt, leftmargin=*, label=\arabic*.]
        \item \textbf{Approvazione esterna}
    \end{enumerate}
    \vspace{-0.5em}
    È l'ultima sezione del documento che attesta che i relativi verbali esterni siano approvati dalla proponente$^G$ tramite firma ed eventuale timbro del rappresentante.

\vspace{0.7em}
\subsubsubsection{Diari di Bordo}
    I diari di bordo sono presentazioni utilizzate durante gli incontri settimanali con l'obiettivo di verificare in modo condiviso lo stato di avanzamento di ciascun gruppo ammesso al I lotto.\\
    La struttura tipica di un diario di bordo comprende le seguenti sezioni:
    \begin{itemize}[itemsep=3pt, parsep=1pt, label=$\scriptstyle\bullet$]
        \item \textbf{Risultati}: descrive le attività completate nel periodo corrente e le confronta con quanto inizialmente pianificato;
        \item \textbf{Problematiche riscontrate}: consente di illustrare le problematiche affrontate e i dubbi ancora irrisolti, sono volti alla richiesta di supporto o chiarimenti;
        \item \textbf{Attività future}: elenca i compiti da svolgere nel prossimo intervallo di lavoro.
    \end{itemize}

\vspace{0.3em}
\subsubsubsection{Altri documenti}
Di seguito sono elencati tutti i documenti redatti e mantenuti durante l'intero ciclo di vita del progetto$^G$ ciascuno dei quali risponde a uno scopo specifico, contribuendo alla tracciabilità, alla gestione e alla comunicazione delle attività di progetto$^G$.\\
La struttura iniziale di tali documenti corrisponde con quanto definito al punto 3.1.4:
    \begin{itemize}[itemsep=3pt, parsep=1pt, label=$\scriptstyle\bullet$]
        \item \textbf{Valutazione dei Capitolati};
        \item \textbf{Dichiarazione degli Impegni};
        \item \textbf{Lettera di Presentazione};
        \item \textbf{Norme di Progetto$^G$};
        \item \textbf{Analisi dei Requisiti$^G$};
        \item \textbf{Piano di Progetto$^G$};
        \item \textbf{Piano di Qualifica$^G$};
        \item \textbf{Glossario};
        \item \textbf{Specifica Tecnica};
        \item \textbf{Manuale Utente}.
    \end{itemize}

\subsubsection{Stesura dei documenti}
\label{sec:stesura-documenti}
La Redazione di un documento segue i seguenti passaggi: 
\begin{enumerate}\setlength{\itemsep}{0pt}
    \item \textbf{Creazione della Issue$^G$}: viene aperta una issue$^G$ su GitHub$^G$, assegnandola a uno o più membri in base al ruolo da loro ricoperto (vedi sezione 4.1.2), per tracciare le attività da svolgere.
    \item \textbf{Stesura del documento}: coloro a cui è stata assegnata la issue$^G$ avviano o continuano la stesura del documento e procede al caricamento su GitHub$^G$ tramite un branch$^G$ feature/* in base alla funzionalità$^G$ o alla sezione su cui stanno lavorando.
    \item \textbf{Creazione della Pull Request$^G$}: una volta completato il lavoro sul branch$^G$, viene aperta una Pull Request$^G$ con richiesta di verifica$^G$ della nuova sezione da un altro membro del gruppo, segnalato nel Registro delle Modifiche.
    \item \textbf{Revisione e Approvazione}: il membro del team designato procede ad esaminare il codice e il contenuto della PR$^G$, fornendo feedback e richiedendo modifiche se necessario.
    \item \textbf{Merge$^G$:} dopo essersi accertato che le modifiche siano conformi, avviene il merge$^G$ con il ramo principale, chiudendo automaticamente la issue$^G$ associata.
\end{enumerate}
Il procedimento che segue si applica solo a verbali o per l’approvazione di Documenti ritenuti definitivi e da approvare per termini di Baseline$^G$:
\begin{enumerate} [resume]
    \item \textbf{Approvazione e pubblicazione:} A seguito di esito positivo della revisione, il Responsabile$^G$ approva il documento completando la pull request$^G$ e procede al merge$^G$ del branch$^G$ secondario in quello principale.\\ Questa operazione attiva una GitHub$^G$ Action$^G$ che, oltre a procedere con la generazione del .pdf dai .tex, aggiorna automaticamente il sito web del progetto$^G$ con i nuovi documenti approvati.
\end{enumerate}

\subsubsection{Nomenclatura}
\subsubsubsection{File nel repository}
Per convenzione i documenti caricati nel repository$^G$ seguiranno la seguente denominazione:\\
Verbali interni:
\begin{center}
    AAAA-MM-GG\_verbale\_interno
\end{center}
Verbali esterni:
\begin{center}
    AAAA-MM-GG\_verbale\_esterno
\end{center}
Diari di bordo:
\begin{center}
    AAAA-MM-GG\_Diario\_di\_Bordo
\end{center}
Le date dovranno essere espresse nel formato ISO$^G$ 8601 (AAAA-MM-GG), che prevede quattro cifre per l'anno, due per il mese e due per il giorno. Questo standard garantisce un ordinamento cronologico automatico e inequivocabile sia per i verbali (interni ed esterni) che per i diario di bordo, semplificandone la consultazione e il raggruppamento.\\
Per tutti gli altri documenti, la convenzione di denominazione prevede la forma:
\begin{center}
    Titolo\_documento
\end{center}
\subsubsubsection{Documenti nel sito web}
Diversamente è stato deciso di fare per il sito della Documentazione \\
(https://nullpointersgroup.github$^G$.io/Documentazione/) dove, sempre per convenzione, è stato stabilito come segue:
\begin{center}
    AAAA-MM-GG\_TIPO v\#
\end{center}
In questo caso sarà presente il TIPO, ovvero:
\begin{itemize}[itemsep=5pt, parsep=1pt, label=$\scriptstyle\bullet$]
    \item VI$^G$: Verbale Interno;
    \item VE$^G$: Verbale Esterno;
    \item DB: Diario di Bordo.
\end{itemize}
La nomenclatura dei restanti documenti consisterà in:
\begin{center}
    Titolo\_documento v\#
\end{center}
Inoltre viene indicata la versione attuale di ogni documento con v\# dove \# funge da modificatore di versione, che viene incrementato progressivamente a ogni aggiornamento significativo del documento.

\subsubsection{Manutenzione}
L'attività di manutenzione$^G$ della documentazione viene attivata ogni qualvolta un documento richiede integrazioni o modifiche per rimanere accurato e allineato allo stato del progetto$^G$.\\
Il flusso di aggiornamento ripercorre le fasi principali della stesura iniziale: partendo dalla creazione di una issue$^G$ di tracciamento, si procede con le modifiche in un branch$^G$ dedicato, per concludere con una verifica$^G$ formale attraverso una Pull Request$^G$ prima dell'integrazione nel documento definitivo (il punto \ref{sec:stesura-documenti} descrive la procedura).

\subsubsection{Documentazione del Codice}
Questa sezione definisce gli standard da adottare per la scrittura e la documentazione del codice sorgente del progetto$^G$ SmartOrder.\\ 
Il suo scopo è garantire:
\begin{itemize}[itemsep=3pt, parsep=1pt, label=$\scriptstyle\bullet$]
    \item \textbf{Leggibilità e Manutenibilità}: un codice ben documentato è più facile da comprendere, correggere ed estendere.
    \item \textbf{Tracciabilità}: collegamento chiaro tra il codice, i requisiti e le decisioni progettuali.
    \item \textbf{Coerenza del Team}: uniformità nello stile, essenziale per il lavoro collaborativo e asincrono.
\end{itemize}

\subsubsubsection{Convenzioni per la Documentazione del Codice}
Per garantire tracciabilità e manutenibilità, il codice sorgente deve essere documentato utilizzando uno standard ispirato da \textbf{Doxygen$^G$}.\\
I blocchi di documentazione devono essere inseriti immediatamente prima della dichiarazione di:
\begin{itemize}[itemsep=3pt, parsep=1pt, label=$\scriptstyle\bullet$]
    \item Funzioni e metodi in Python$^G$.
    \item Componenti React$^G$, funzioni e metodi in JavaScript/TypeScript.
\end{itemize}

\subsubsubsection{Tag previsti}
Ogni blocco di documentazione deve includere i seguenti tag, ove applicabili:
\begin{table}[H]
\centering
\renewcommand{\arraystretch}{1.15}
\begin{tabular}{|p{0.22\textwidth}|p{0.7\textwidth}|}
    \hline
    \textbf{Tag} & \textbf{Scopo e Formato} \\
    \hline
    \texttt{@brief} & \textbf{Brief description}: descrizione concisa dello scopo e della funzionalità$^G$. \\
    \hline
    \texttt{@param} & \textbf{Type Description}: descrizione di un parametro, specificandone il tipo e una breve descrizione. \\
    \hline
    \texttt{@raise/@throws} & \textbf{ExceptionType Condition or description}: eccezioni/errori sollevati, con tipo e condizioni che li provocano. \\
    \hline
    \texttt{@bug$^G$} & \textbf{Actual problems}: descrizione di problemi noti non ancora risolti nel codice. \\
    \hline
    \texttt{@return} & \textbf{Type Description}: descrizione del valore di ritorno e del suo tipo. \\
    \hline
    \texttt{@req} & \textbf{Requisito associato}: identificativo del requisito associato (es. \texttt{RF-OB\_\_1}). Fondamentale per la tracciabilità. \\
    \hline
\end{tabular}
\caption{Tag previsti per la documentazione del codice}
\end{table}

\subsubsubsection{Verifica della Conformità}
	La conformità a questi standard è parte integrante del \textbf{processo$^G$ di verifica$^G$ del codice}:
	\begin{enumerate}[itemsep=3pt, parsep=1pt, label=\arabic*.]
		\item Il \textbf{Programmatore$^G$} è responsabile$^G$ della documentazione del proprio codice.
		\item Il \textbf{Verificatore$^G$}, durante la revisione di una Pull Request$^G$, deve controllare:
		\begin{itemize}[itemsep=2pt, parsep=1pt, label=$\circ$]
			\item La presenza dei blocchi di documentazione per le nuove funzionalità$^G$.
			\item La completezza dei tag corretti (\texttt{@brief}, \texttt{@param}, \texttt{@return}, \texttt{@req}).
			\item La correttezza del formato, in particolare per il tag \texttt{@req} che deve contenere un identificativo valido di requisito.
			\item La coerenza tra il codice e quanto descritto nella documentazione.
		\end{itemize}
		\item La mancata conformità è motivo di richiesta di modifica$^G$ prima dell'approvazione della PR$^G$.
	\end{enumerate}

\subsection{Gestione della Configurazione}
\label{sec:gestione-configurazione}
La gestione delle configurazioni è il processo$^G$ che garantisce il controllo e la tracciabilità di tutte le componenti di un progetto$^G$ software. Consente di organizzare le modifiche al codice e alla documentazione, permettendo un controllo su tutte le modifiche e le versioni rilasciate.\\
Per garantire questo controllo, ogni elemento del progetto$^G$ viene trattato come elemento di configurazione e sottoposto a versionamento$^G$ e tracciamento.\\
Gli elementi sottoposti a verifica$^G$ sono documenti, esclusi i verbali siccome questi non hanno necessità di registro di modifiche, e il codice.\\
Questo approccio assicura un avanzamento ordinato dello sviluppo$^G$, mantenendo una traccia storica dell’evoluzione di ciascun elemento.

\subsubsection{Strumenti di Controllo della Configurazione}
\label{sec:ticketing}
Il gruppo utilizza il sistema di Issue$^G$ Tracking e le Project Board integrate in GitHub$^G$ come strumenti principali per la gestione e l’organizzazione delle attività.\\
Questi strumenti consentono di pianificare il lavoro, monitorarne l’avanzamento e mantenere un controllo costante sulla qualità e sulla coerenza del processo$^G$ di sviluppo$^G$.
\subsubsubsection{Issue}
Ogni attività, modifica$^G$ o produzione$^G$ di documentazione viene gestita attraverso una issue$^G$, che deve contenere:
\begin{itemize}[itemsep=3pt, parsep=1pt, label=$\scriptstyle\bullet$]
    \item \textbf{Titolo}: una descrizione sintetica del compito da svolgere;
    \item \textbf{Descrizione}: informazioni dettagliate sul lavoro richiesto, con eventuali riferimenti ai documenti da aggiornare;
    \item \textbf{Assegnatario}: il membro incaricato;
    \item \textbf{Etichette} (Labels): utilizzate per classificare la tipologia del compito;
    \item \textbf{Milestone$^G$}: indica l’obiettivo di riferimento (RTB$^G$ o PB$^G$);
    \item \textbf{Board e Stato}: identifica la project board di appartenenza e lo stato corrente (Todo, In Progress, Done);
\end{itemize}
Le issue$^G$ vengono create inizialmente dal Responsabile$^G$, ma possono essere aggiornate da qualsiasi membro qualora emergano nuove informazioni, modifiche o necessità di riformulazione del compito. Ogni aggiornamento deve mantenere chiarezza, coerenza e tracciabilità.\\

Il mantenimento del sistema di Issue$^G$ Tracking avviene tramite una procedura stabilita:
\begin{enumerate}[itemsep=3pt, parsep=1pt, leftmargin=*]
    \item \textbf{Creazione della issue$^G$} da parte del Responsabile$^G$ o del membro che individua la necessità del compito.
    \item \textbf{Aggiornamento continuo della issue$^G$} da parte dell’assegnatario quando si presentano cambiamenti o integrazioni.
    \item \textbf{Registrazione dello stato di avanzamento} tramite lo spostamento della issue$^G$ nella colonna corrispondente della Project Board.
    \item \textbf{Revisione obbligatoria}: ogni attività che comporta una modifica$^G$ ai file del repository$^G$ richiede una verifica$^G$ formale.
    \item Quando un membro effettua un push nel branch$^G$ relativo alla issue$^G$, il sistema genera automaticamente una \textbf{pull request$^G$}.
    \item \textbf{Assegnazione della PR$^G$} al Verificatore$^G$ designato.
    \item \textbf{Verifica$^G$ della PR$^G$}: il Verificatore$^G$ controlla che il contenuto sia conforme agli standard, coerente con la documentazione e privo di errori.
    \item \textbf{Merge$^G$ nel main}: se la verifica$^G$ ha esito positivo, il Verificatore$^G$ effettua il merge$^G$ e chiude la issue$^G$. In caso contrario, richiede ulteriori modifiche.
\end{enumerate}
Questa procedura permette tracciabilità completa del lavoro e sincronizzazione costante tra issue$^G$, project board e repository$^G$ riducendo il rischio di incoerenze dei documenti.

\subsubsubsection{Project Board}
Il gruppo mantiene due Project Board principali:
\begin{itemize}[itemsep=3pt, parsep=1pt, label=$\scriptstyle\bullet$]
    \item \textbf{RTB$^G$} (Requirements and Technology Baseline$^G$)
    \item \textbf{PB$^G$} (Product Baseline$^G$)
\end{itemize}
Queste board permettono di organizzare e visualizzare in modo chiaro tutte le attività necessarie al raggiungimento delle rispettive milestone$^G$.\\
Ogni issue$^G$ viene collegata alla board pertinente e inserita nella colonna appropriata, così da garantire una visione aggiornata e condivisa dello stato del progetto$^G$.\\
Le board vengono aggiornate:
\begin{itemize}[itemsep=3pt, parsep=1pt, label=$\scriptstyle\bullet$]
    \item quando una issue$^G$ cambia stato;
    \item quando viene aggiunto un nuovo compito;
    \item quando un’attività richiede una modifica$^G$ o riformulazione.
\end{itemize}
Il Responsabile$^G$ verifica$^G$ periodicamente la coerenza delle board con le milestone$^G$ attive.

\subsubsubsection{Labels}
Per migliorare l'organizzazione, il gruppo utilizza un sistema di etichette che identifica chiaramente la tipologia di lavoro, introducendo una famiglia dedicata specificamente alla gestione documentale.\\
Il principio è semplice: viene sempre aggiunta un'etichetta specifica, come \vr{Piano di Progetto$^G$} o \vr{Analisi dei Requisiti$^G$}, che indica precisamente quale file è coinvolto.\\
Queste label vengono applicate sia alle issue$^G$ settimanali, sia alle attività associate alle milestone$^G$ RTB$^G$ e PB$^G$.\\
Ciò permette di individuare rapidamente quali attività richiedono attenzione e tracciare il carico di lavoro per tipologia di attività oltre che avere una visione completa dell'avanzamento complessivo del progetto$^G$.

\subsubsubsection{Nomenclatura dei branch}
\label{sec: nomenclatura-branch}
Per garantire uniformità e tracciabilità nella gestione dei branch$^G$ relativi alla documentazione, è stata adottata la seguente convenzione di denominazione:
\begin{table}[H]
    \centering
    \renewcommand{\arraystretch}{1.15}
    \begin{tabular}{|p{0.37\textwidth}|p{0.61\textwidth}|}
        \hline
        \textbf{Nome Branch$^G$} & \textbf{Descrizione} \\
        \hline
        feature/glossario & Per modifiche al documento Glossario \\
        \hline
        feature/analisi-requisiti & Per modifiche al documento Analisi dei Requisiti$^G$ \\
        \hline
        feature/specifica-tecnica & Per modifiche al documento Specifica Tecnica \\
        \hline
        feature/manuale-utente & Per modifiche al documento Manuale Utente \\
        \hline
        feature/norme-progetto$^G$ & Per modifiche al documento Norme di Progetto$^G$ \\
        \hline
        feature/piano-progetto$^G$ & Per modifiche al documento Piano di Progetto$^G$ \\
        \hline
        feature/piano-qualifica & Per modifiche al documento Piano di Qualifica\\
        \hline
        feature/YYYYMMDDverbint & Per creazione o modifica$^G$ di un verbale interno (YYYYMMDD = data) \\
        \hline
        feature/YYYYMMDDverbest & Per creazione o modifica$^G$ di un verbale esterno (YYYYMMDD = data) \\
        \hline
        feature/script & Per creazione o modifica$^G$ di script \\
        \hline
        feature/aggiornamenti & Per aggiornamenti generali a tutti i documenti \\
        \hline
        feature/YYYYMMDDdiario & Per creazione o modifica$^G$ di un diario di bordo (YYYYMMDD = data) \\
        \hline
    \end{tabular}
    \caption{Nomenclatura dei branch}
\end{table}

\subsubsection{Registrazione dello Stato di Configurazione}
L'attività di Registrazione dello Stato di Configurazione è fondamentale per garantire la tracciabilità e la conoscenza storica delle evoluzioni di tutti gli Elementi di Configurazione (codice e documentazione) prodotti dal gruppo.

\subsubsubsection{Documentazione}
In base a quanto definito nel \href{https://nullpointersgroup.github.io/Documentazione/output/RTB/Verbali\%20Interni/2025-11-06\_verbale\_interno.pdf}{verbale interno del 6 novembre} per il tracciamento durante il ciclo di vita dei documenti, NullPointers Group ha deciso di utilizzare una politica di versionamento$^G$ basata sulla specifica MAJOR$^G$.MINOR$^G$.PATCH$^G$. Questa specifica riflette lo stato di avanzamento e la validazione$^G$ formale del documento. La sua implementazione avviene come segue:
\begin{itemize}[itemsep=3pt, parsep=1pt, label=$\scriptstyle\bullet$]
    \item \textbf{MAJOR$^G$}: Incrementato solo in seguito all'Approvazione Formale (Approvazione) del file da parte del Responsabile$^G$.
    \item \textbf{MINOR$^G$}: Incrementato una volta completato l'intero ciclo di modifica$^G$, che include sia lo sviluppo$^G$ del contenuto sia la verifica$^G$ formale da parte del Verificatore$^G$ designato.
    \item \textbf{PATCH$^G$}: Incrementato per modifiche di entità minore, come correzioni ortografiche, formattazione, o aggiustamenti di sintassi che non alterano il contenuto sostanziale.
\end{itemize}
In aggiunta al versionamento$^G$, ogni documento conterrà un Registro delle Modifiche dettagliato (come descritto nella sezione \ref{sec:registro-modifiche}), che elenca cronologicamente tutte le variazioni apportate per ciascuna versione rilasciata.

\subsection{Gestione della Qualità}
\label{sec:gestione-qualita}
La Gestione della Qualità è il processo$^G$ che definisce, pianifica e coordina tutte le attività necessarie affinché i prodotti sviluppati dal gruppo rispettino gli standard stabiliti nel Piano di Qualifica$^G$ e risultino coerenti con gli obiettivi progettuali.\\
Tale processo$^G$ assicura che ogni artefatto prodotto (documenti, codice, modelli o deliverable intermedi) soddisfi i criteri qualitativi prefissati e fornisce un quadro operativo per il miglioramento continuo.\\
La qualità è garantita da un sistema integrato di strumenti, procedure e responsabilità, che accompagna e supporta l'intero ciclo di sviluppo$^G$. Questo approccio consente di prevenire l'introduzione di errori, rilevare tempestivamente eventuali incongruenze e garantire che ogni output sia verificato prima di essere considerato completo, oltre a mantenere elevata la tracciabilità delle attività.\\
La Gestione della Qualità opera in stretta sinergia con la Gestione della Configurazione (sezione \ref{sec:gestione-configurazione}) e con il processo$^G$ di Verifica$^G$ (sezione \ref{sec:verifica}), costituendo un sistema integrato di controllo e miglioramento del prodotto.

\subsubsection{Piano di Qualifica}
Il gruppo adotta una pianificazione$^G$ esplicita delle attività di qualità, formalizzata nel Piano di Qualifica$^G$, che definisce:
\begin{itemize}[itemsep=3pt, parsep=1pt, label=$\scriptstyle\bullet$]
    \item Gli obiettivi qualitativi specifici da raggiungere, distinti per il prodotto finale e per i processi di sviluppo$^G$;
    \item Metriche quantitative per misurare oggettivamente la qualità del prodotto ma anche del processo$^G$.
    \item Strumenti utilizzati sia per l'analisi statica che dinamica;
    \item Strategie di testing specifiche (unitario, di integrazione, di sistema e di accettazione) per verificare il prodotto e garantire che soddisfi i requisiti;
    \item Un cruscotto di valutazione che consente il monitoraggio, per la durata del ciclo di vita del progetto$^G$, delle metriche di qualità definite.
\end{itemize}
Il Piano di Qualifica$^G$ viene aggiornato nel corso del progetto$^G$ per l'aggiornamento dei dati e la continuazione del cruscotto di valutazione, oltre che per eventuali miglioramenti o variazioni negli strumenti adottati.

\subsubsection{Metriche di Qualità}
Per garantire un controllo oggettivo e ripetibile della qualità, il gruppo adotta un insieme di metriche che vengono applicate regolarmente ai prodotti sviluppati.\\
Le metriche (dettaglio nella sezione \ref{sec:metriche-qualita}) sono suddivise in due categorie:
\begin{itemize}[itemsep=3pt, parsep=1pt, label=$\scriptstyle\bullet$]
    \item \textbf{Metriche di Qualità del prodotto}: indicatori per misurare le caratteristiche del software finale, come affidabilità e usabilità.
    \item \textbf{Metriche di Qualità del processo$^G$}: indicatori per valutare l'efficacia$^G$ e l'efficienza$^G$ delle attività di sviluppo$^G$.
\end{itemize}

\subsubsection{Struttura e identificazione delle metriche}
\label{sec:struttura-metriche}
Per garantire chiarezza, coerenza interna e una tracciabilità efficace, tutte le metriche adottate nel progetto$^G$ e riportate nella sezione \ref{sec:metriche-qualita} sono strutturate secondo un formato standardizzato.\\
Ogni metrica è definita dai seguenti elementi essenziali:
\begin{itemize}[itemsep=3pt, parsep=1pt, label=$\scriptstyle\bullet$]
    \item \textbf{Codice Identificativo Univoco}, che utilizza il formato 
    \begin{center}
        \texttt{M[Abbreviazione]\_[Numero]}
    \end{center}
    dove:
    \begin{itemize}[label=\textendash, itemsep=0pt, parsep=1pt, leftmargin=1.5em]
        \item \texttt{MQD\_xx}: identifica le Metriche di Qualità del Prodotto.
        \item \texttt{MQC\_xx}: identifica le Metriche di Qualità del Processo$^G$.
    \end{itemize}
    Il numero progressivo (xx) garantisce l'unicità all'interno di ciascuna categoria.
    \item \textbf{Nome}: il nome completo e descrittivo della metrica.
    \item \textbf{Descrizione}: una spiegazione chiara della sua finalità e ambito di applicazione.
    \item \textbf{Formula}: la definizione matematica o procedurale per il suo calcolo, specificata per le metriche quantitative.
\end{itemize}
Questa struttura facilita il riconoscimento immediato della tipologia di metrica (di prodotto o di processo$^G$), standardizza la documentazione e ne supporta l'utilizzo sistematico in tutte le fasi di monitoraggio e valutazione.

\subsubsection{Miglioramento Continuo}
Il gruppo segue un approccio in cui la qualità migliora progressivamente, analizzando i risultati dopo ogni ciclo di lavoro.\\ 
Se emergono problemi, come errori frequenti o metriche insufficienti, si valutano azioni correttive. Questo può significare introdurre nuovi strumenti, aggiornare quelli esistenti o rivedere procedure e checklist.\\ Le modifiche importanti vengono discusse in gruppo, approvate e registrate formalmente nei verbali interni, per assicurare la tracciabilità e una corretta applicazione futura.

\subsection{Verifica}
\label{sec:verifica}
\subsubsection{Scopo e descrizione}
Il processo$^G$ di Verifica$^G$ è un'attività sistematica che deve essere applicata a tutti i prodotti di progetto$^G$ (documenti e codice) prima che possano essere considerati completi e rilasciati.\\ 
Il suo scopo principale è rispondere alla domanda \vr{\textit{Did I build$^G$ the system right?}}, ovvero assicurare che quanto prodotto sia stato realizzato correttamente, in piena conformità con i vincoli, le regole e gli standard qualitativi definiti nel Piano di Qualifica$^G$ e nelle Norme di Progetto$^G$.\\ 
Le attività di Verifica$^G$ sono svolte dai Verificatori, il cui compito è controllare, correggere e segnalare ogni discrepanza rispetto ai requisiti prestabiliti.

\subsubsection{Strumenti a supporto}
Per supportare le attività di Verifica$^G$ descritte, il gruppo utilizza GitHub$^G$, imponendo che ogni modifica$^G$ alla documentazione o al codice venga prima sottoposta a revisione attraverso una Pull Request$^G$.\\ 
Le regole di protezione del ramo principale obbligano che ogni PR$^G$ sia approvata da almeno un Verificatore$^G$ diverso dall'autore, garantendo un controllo qualità sistematico e tracciabile prima che le modifiche vengano integrate.

\subsubsection{Analisi statica}
L'analisi statica è una tecnica di verifica$^G$ che si applica senza eseguire il codice o il prodotto.\\
Il suo scopo è individuare problemi di sintassi, logica o conformità agli standard prima che si manifestino durante l'esecuzione.\\
Si può eseguire tramite metodi formali (dimostrazioni matematiche) oppure, più frequentemente, tramite metodi di lettura manuali o semi-automatici.

I metodi di lettura più diffusi sono due:
\begin{itemize}[itemsep=3pt, parsep=1pt, label=$\scriptstyle\bullet$]
    \item \textbf{Walkthrough$^G$}: è una tecnica di verifica$^G$ basata sull'ipotesi che esista un difetto, pur in assenza di informazioni sulla sua natura o localizzazione. Ne consegue l'obbligo di un'analisi approfondita ed esaustiva dell'oggetto sotto esame. Questa caratteristica lo rende una metodologia ad alto impegno di risorse, di limitata applicabilità pratica e scarsamente automatizzabile.
    \item \textbf{Ispezione}: utilizza una lista di controllo predeterminata, concentrandosi su possibili errori o criteri di qualità specifici. Pur essendo meno esaustiva di un walkthrough$^G$ (specie nelle prime fasi di sviluppo$^G$) offre un vantaggio decisivo: è facilmente automatizzabile. Per questo risulta la scelta preferibile quando occorre verificare grandi quantità di codice o documenti in modo sistematico e ripetibile.
\end{itemize}

\subsubsection{Analisi dinamica}
L'analisi dinamica viene condotta eseguendo il prodotto software attraverso una serie di test$^G$. Questa procedura misura la qualità funzionale del codice, assicurandosi che raggiunga il suo scopo e che il suo comportamento sia conforme alle specifiche.\\
Permette di individuare e risolvere gli errori nel codice che causano comportamenti indesiderati.\\
Ogni test$^G$ è definito da uno stato iniziale, input specifici e output attesi.\\
Le attività di test$^G$ devono essere ripetibili, per poter confermare le correzioni, e automatizzabili, per essere eseguiti in modo continuo durante tutto il ciclo di vita del prodotto.

\subsubsection{Classificazione dei Test}
Tutti i test$^G$ pianificati, eseguiti e i loro esiti sono tracciati nel Piano di Qualifica$^G$.\\
I test$^G$ sono classificati in categorie gerarchiche, ognuna con un codice identificativo univoco nel formato: