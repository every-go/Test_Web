\section{Standard di Progetto}

\subsection{Standard ISO/IEC 12207 - 1995}
\subsubsection{Scopo dello standard}
Lo standard ISO/IEC 12207 definisce un insieme strutturato di processi, attività e compiti che coprono l’intero ciclo di vita del software.\\
Fornisce un modello di riferimento per organizzare in modo sistematico le attività di sviluppo$^G$, verifica$^G$, gestione e manutenzione$^G$, garantendo coerenza e tracciabilità in tutte le fasi del progetto$^G$.

L’adozione di tale standard consente al gruppo di:
\begin{itemize}
    \setlength\itemsep{-0.1em}
    \item definire responsabilità e flussi operativi chiari;
    \item assicurare un controllo costante sulla qualità dei processi;
    \item mantenere allineamento tra pianificazione$^G$, sviluppo$^G$ e verifica$^G$;
    \item favorire la riproducibilità e la trasparenza del lavoro svolto.
\end{itemize}

\subsubsection{Applicazione nel progetto SmartOrder}
Per il progetto$^G$ SmartOrder, lo standard ISO/IEC 12207 definisce un modello di riferimento per l’organizzazione del ciclo di vita del software.\\
La sua adozione consente di strutturare le attività del progetto$^G$ in modo controllato e tracciabile, garantendo che ogni fase sia condotta secondo criteri uniformi di qualità.\\
Lo standard suddivide il ciclo di vita in tre categorie di processi:
\begin{itemize}
    \setlength\itemsep{-0.1em}
    \item \textbf{Processi primari}: rappresentano le attività direttamente legate alla realizzazione del prodotto. Comprendono l’analisi e la definizione dei requisiti, la progettazione$^G$ dell’architettura, lo sviluppo$^G$ del software, le attività di verifica$^G$ e validazione$^G$ necessarie per garantire che il sistema soddisfi quanto concordato con la proponente$^G$. Attraverso questi processi, il progetto$^G$ avanza in modo controllato verso la costruzione del prodotto finale.
    \item \textbf{Processi di supporto}: sono attività trasversali che affiancano e sostengono i processi primari. Includono la gestione della configurazione, il controllo della qualità, la produzione$^G$ della documentazione e le attività di verifica$^G$. Il loro ruolo è assicurare ordine, coerenza e affidabilità durante tutte le fasi del progetto$^G$.
    \item \textbf{Processi organizzativi}: regolano la gestione complessiva del progetto$^G$, la pianificazione$^G$, il coordinamento interno e le attività di miglioramento. Attraverso questi processi vengono definiti i ruoli, le responsabilità, gli obiettivi e le modalità operative del gruppo, garantendo un ambiente di lavoro strutturato e orientato al raggiungimento dei risultati.
\end{itemize}
L’utilizzo di ISO/IEC 12207 permette di mantenere un controllo metodico sul ciclo di vita, particolarmente importante in un progetto$^G$ che integra componenti di intelligenza artificiale$^G$, modelli multimodali e pipeline di elaborazione complesse.

\subsection{Standard ISO/IEC 9126}
\subsubsection{Scopo dello standard}
Lo standard ISO/IEC 9126 definisce un modello strutturato per valutare la qualità di un prodotto software. Il suo scopo è trasformare il concetto generico di "qualità" in un insieme di caratteristiche misurabili, permettendo valutazioni oggettive e confrontabili.

Il modello identifica sei dimensioni fondamentali della qualità, che sono:
\begin{itemize}
    \setlength\itemsep{-0.1em}
    \item \textbf{Funzionalità$^G$:} capacità del sistema di soddisfare requisiti e bisogni specificati.
    \item \textbf{Affidabilità:} misura la capacità del sistema di funzionare senza interruzioni e di resistere a condizioni anomale o a guasti.
    \item \textbf{Usabilità:} considera la facilità con cui un utente può imparare a usare il software e interagire con esso in modo efficace.
    \item \textbf{Efficienza$^G$:} si riferisce alle prestazioni del sistema e al suo utilizzo ottimale delle risorse hardware (come tempo di risposta e consumo di memoria).
    \item \textbf{Manutenibilità:} valuta la facilità con cui il software può essere modificato per correggere errori, migliorarlo o adattarlo a nuove esigenze.
    \item \textbf{Portabilità:} analizza la capacità del sistema di essere installato e di funzionare in ambienti diversi (ad esempio, su sistemi operativi diversi).
\end{itemize}
L'adozione di questo approccio garantisce che la qualità non sia un controllo solo finale, ma una caratteristica intrinseca, costruita e verificata progressivamente in ogni passaggio dello sviluppo$^G$.

\subsubsection{Applicazione nel progetto SmartOrder}
Nel progetto$^G$ SmartOrder, questo standard costituisce il quadro di riferimento principale per definire gli obiettivi di qualità del prodotto e le relative metriche (dettagliate nella Sezione \ref{sec:metriche-qualita}). Viene applicato per guidare lo sviluppo$^G$ verso traguardi concreti:
\begin{itemize}
    \setlength\itemsep{-0.1em}
    \item validare la correttezza dell’elaborazione multimodale (testi, immagini, audio);
    \item garantire robustezza e stabilità all’interno dei moduli di AI e NLP$^G$;
    \item assicurare tempi di risposta adeguati per un’interazione fluida con l’utente;
    \item preservare la separazione e la modularità tra i componenti, semplificando manutenzione$^G$ e futuri aggiornamenti;
    \item offrire un’interfaccia chiara e intuitiva nonostante la complessità architetturale del sistema.
\end{itemize}
L'adozione dello standard permette quindi di stabilire criteri di qualità precisi e misurabili, supportando sia la verifica$^G$ finale che il monitoraggio continuo durante tutto lo sviluppo$^G$.

\subsection{Motivazione dell’adozione degli standard}
La scelta di adottare gli standard ISO/IEC 12207 (per il processo$^G$) e ISO/IEC 9126 (per il prodotto) porta dei vantaggi strategici al progetto$^G$ SmartOrder:
\begin{itemize}
    \setlength\itemsep{-0.1em}
    \item consente di organizzare tutte le attività, dall’analisi iniziale fino al rilascio finale;
    \item permette di esercitare un controllo accurato e continuo sulla qualità sia del processo$^G$ di sviluppo$^G$ sia del prodotto realizzato;
    \item introduce metriche oggettive, misurabili e verificabili a supporto della valutazione delle prestazioni;
    \item supporta una gestione metodica di componenti tecnologicamente avanzati, tra cui modelli di intelligenza artificiale$^G$ e pipeline multimodali;
    \item garantisce affidabilità, manutenibilità e coerenza nel progetto$^G$.
\end{itemize}
L’adozione di tali standard costituisce quindi un fondamento essenziale per mantenere coerenza, trasparenza e qualità nello sviluppo$^G$ di SmartOrder.