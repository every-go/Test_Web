\section{Metriche di Qualità}
\label{sec:metriche-qualita}
Le metriche rappresentano uno strumento essenziale per garantire un monitoraggio continuo e oggettivo della qualità del prodotto software e dei processi impiegati per realizzarlo.\\
Permettono di individuare tempestivamente anomalie, inefficienze o deviazioni dagli standard prefissati, favorendo una gestione consapevole dell’evoluzione del progetto$^G$.\\
Per garantire chiarezza e coerenza interna, ogni metrica è classificata secondo la seguente notazione:
\begin{itemize}
    \item MQC\_xx - Metriche di qualità del processo$^G$
    \item MQD\_xx - Metriche di qualità del prodotto
\end{itemize}

% ============================================================
%              6.2 METRICHE DI PROCESSO$^G$
% ============================================================
\subsection{Metriche di qualità di Processo}
Le metriche qui definite consentono di monitorare l’andamento dei processi primari e di supporto, permettendo di verificare l’aderenza alla pianificazione$^G$, l’efficacia$^G$ delle attività svolte e la qualità complessiva del ciclo di sviluppo$^G$.
% --------------------- 6.2.1 Processi Primari ---------------------
\subsubsection{Processi Primari}
\begin{itemize}
% MQC_01
\item \textbf{MQC\_01 - Earned Value$^G$}:
    \begin{itemize}
        \item \textbf{Formula}:
        \[
            \text{Earned Value$^G$} =
            \text{Planned Value$^G$} \times \text{Completeness Issue$^G$}
        \]
        \item \textbf{Descrizione}: misura il valore del lavoro effettivamente completato rispetto alla pianificazione$^G$. Riflette l’avanzamento reale del progetto$^G$ in termini economici ed è utile per identificare scostamenti precoci rispetto al piano.
    \end{itemize}

% MQC_02
\item \textbf{MQC\_02 - Planned Value$^G$}:
    \begin{itemize}
        \item \textbf{Formula}:
        \[
            \text{Planned Value$^G$} = \text{Costo preventivato}
        \]
        \item \textbf{Descrizione}: indica il valore del lavoro che avrebbe dovuto essere completato in un dato momento secondo la pianificazione$^G$. È utilizzato come riferimento per confrontare l’avanzamento reale con quello previsto.
    \end{itemize}

% MQC_03
\item \textbf{MQC\_03 - Actual Cost$^G$}:
    \begin{itemize}
        \item \textbf{Formula}:
        \[
            \text{Actual Cost$^G$} = \text{Consuntivo$^G$}
        \]
        \item \textbf{Descrizione}: rappresenta il costo reale sostenuto fino a una certa data. Confrontato con l’Earned Value$^G$ consente di valutare l’efficienza$^G$ economica del progetto$^G$.
    \end{itemize}

% MQC_04
\item \textbf{MQC\_04 - Cost Performance Index}:
    \begin{itemize}
        \item \textbf{Formula}:
        \[
            \text{Cost Performance Index} =
            \frac{\text{Earned Value$^G$}}{\text{Actual Cost$^G$}}
        \]
        \item \textbf{Descrizione}: 
        Misura l'efficienza$^G$ economica del progetto$^G$ confrontando il valore realizzato con il costo effettivamente sostenuto.\\
        Un valore maggiore di 1 indica che il lavoro è stato svolto a un costo inferiore al previsto (sotto budget), un valore uguale a 1 indica che i costi sono in linea con le previsioni, mentre un valore inferiore a 1 segnala che il costo supera il previsto (sopra budget). \\
        Valori persistentemente inferiori a 1 richiedono interventi correttivi per riallineare il progetto$^G$ al budget.
    \end{itemize}

% MQC_05
\item \textbf{MQC\_05 - Schedule Performance Index}:
    \begin{itemize}
        \item \textbf{Formula}:
        \[
            \text{Schedule Performance Index} =
            \frac{\text{Earned Value$^G$}}{\text{Planned Value$^G$}}
        \]
        \item \textbf{Descrizione}: valuta l’efficacia$^G$ dell’avanzamento temporale. Un valore pari o superiore a 1 indica che il progetto$^G$ è in linea con la pianificazione$^G$ o in anticipo rispetto ad essa.
    \end{itemize}

% MQC_06
\item \textbf{MQC\_06 - Estimate at Completion$^G$}:
    \begin{itemize}
        \item \textbf{Formula}:
        \[
            \text{Estimate at Completion$^G$} = 
            \frac{\text{Budget at Completion$^G$}}{\text{Cost Performance Index}}
        \]
        \item \textbf{Descrizione}: stima il costo totale previsto a completamento. Viene utilizzato per prevedere scostamenti rispetto al budget pianificato e per aggiornare le proiezioni economiche.
    \end{itemize}

% MQC_07
\item \textbf{MQC\_07 - Estimate to Complete}:
    \begin{itemize}
        \item \textbf{Formula}:
        \[
            \text{Estimate to Complete} = \text{Estimate at Completion$^G$} - \text{Actual Cost$^G$}
        \]
        \item \textbf{Descrizione}: stima il costo necessario per completare le attività rimanenti. Consente di valutare l’impatto delle performance correnti sul budget finale.
    \end{itemize}

% MQC_08
\item \textbf{MQC\_08 - Time Estimate at Completion$^G$}:
    \begin{itemize}
        \item \textbf{Formula}:
        \[
            \text{Time Estimate at Completion$^G$} =
            \frac{\text{Tempo previsto}}{\text{Schedule Performance Index}}
        \]
        \item \textbf{Descrizione}: indica la data stimata di completamento del progetto$^G$ sulla base dell’andamento attuale. È utile per valutare possibili ritardi e adottare misure correttive.
    \end{itemize}

% MQC_09
\item \textbf{MQC\_09 - Completeness Issue$^G$}:
    \begin{itemize}
        \item \textbf{Formula}:
        \[
            \text{Completeness Issue$^G$} =
            \frac{\text{Issue$^G$ chiuse}}{\text{Issue$^G$ totali}}
        \]
        \item \textbf{Descrizione}: misura il grado di completamento delle attività pianificate, considerando il rapporto tra le issue$^G$ chiuse e quelle registrate. Un valore elevato indica un buon avanzamento delle attività di sviluppo$^G$.
    \end{itemize}
\end{itemize}

% --------------------- 6.2.2 Processi di Supporto ---------------------
\subsubsection{Processi di Supporto}
\begin{itemize}
% MQC_10
\item \textbf{MQC\_10 - Indice di Gulpease}:
    \begin{itemize}
        \item \textbf{Formula}:
        \[
            \text{Indice di Gulpease} =
            89 + \frac{300 \times \text{frasi} - 10 \times \text{lettere}}{\text{parole}}
        \]
        \item \textbf{Descrizione}: valuta la leggibilità dei documenti prodotti. Un indice elevato indica testi più chiari e facilmente comprensibili, migliorando la qualità complessiva della documentazione.
    \end{itemize}

% MQC_11
\item \textbf{MQC\_11 - Test$^G$ Success Rate}:
    \begin{itemize}
        \item \textbf{Formula}:
        \[
            \text{Test$^G$ Success Rate} =
            \frac{\text{Test$^G$ superati}}{\text{Test$^G$ totali}} \times 100
        \]
        \item \textbf{Descrizione}: indica la percentuale di test$^G$ automatizzati o manuali che si concludono con esito positivo. Riflette la stabilità delle funzionalità$^G$ verificate.
    \end{itemize}

% MQC_12
\item \textbf{MQC\_12 - Test$^G$ Density and Automation}:
    \begin{itemize}
        \item \textbf{Formula}:
        \[
            \text{Test$^G$ Density and Automation} =
            \frac{\text{Test$^G$ automatizzati}}{\text{Test$^G$ totali}} \times 100
        \]
        \item \textbf{Descrizione}: misura il grado di automazione$^G$ del processo$^G$ di verifica$^G$. Un’elevata densità di test$^G$ automatizzati migliora l’affidabilità della pipeline di testing e riduce i tempi di validazione$^G$.
    \end{itemize}
\end{itemize}

% --------------------- 6.2.3 Processi Organizzativi ---------------------
\subsubsection{Processi Organizzativi}
\begin{itemize}
% MQC_13
\item \textbf{MQC\_13 - Quality Metrics Satisfied}:
    \begin{itemize}
        \item \textbf{Formula}:
        \[
            \text{Quality Metrics Satisfied} =
            \frac{\text{Metriche Rispettate}}{\text{Metriche Totali}} \times 100
        \]
        \item \textbf{Descrizione}: valuta la percentuale di metriche qualitative che hanno raggiunto il valore accettabile o ottimale. È un indicatore complessivo dell’efficacia$^G$ del processo$^G$ di gestione della qualità.
    \end{itemize}
\end{itemize}

% ============================================================
%                 6.1 METRICHE DI PRODOTTO
% ============================================================
\subsection{Metriche di qualità di Prodotto}

% --------------------- 6.1.1 Funzionalità$^G$ ---------------------
\subsubsection{Funzionalità}
\begin{itemize}
% MQD_01
\item \textbf{MQD\_01 - Requisiti obbligatori soddisfatti}:
    \begin{itemize}
        \item \textbf{Formula}:
        \[
            \text{Requisiti obbligatori soddisfatti} = 
            \left(
                \frac{\text{Requisiti obbligatori soddisfatti}}
                     {\text{Requisiti obbligatori totali}}
            \right) \times 100
        \]
        \item \textbf{Descrizione}: misura la percentuale dei requisiti classificati come obbligatori che risultano implementati e verificati con successo. Una percentuale pari al 100\% è condizione necessaria per garantire la correttezza funzionale minima del prodotto.
    \end{itemize}

% MQD_02
\item \textbf{MQD\_02 - Requisiti desiderabili soddisfatti}:
    \begin{itemize}
        \item \textbf{Formula}:
        \[
			\text{Requisiti desiderabili soddisfatti} =
            \left(
                \frac{\text{Requisiti desiderabili soddisfatti}}
                     {\text{Requisiti desiderabili totali}}
            \right) \times 100
        \]
        \item \textbf{Descrizione}: indica il grado di completamento dei requisiti desiderabili, utili a migliorare l’esperienza d’uso o arricchire il sistema con funzionalità$^G$ non essenziali.
    \end{itemize}

% MQD_03
\item \textbf{MQD\_03 - Requisiti opzionali soddisfatti}:
    \begin{itemize}
        \item \textbf{Formula}:
        \[
			\text{Requisiti opzionali soddisfatti} =
            \left(
                \frac{\text{Requisiti opzionali soddisfatti}}
                     {\text{Requisiti opzionali totali}}
            \right) \times 100
        \]
        \item \textbf{Descrizione}: valuta l’implementazione di funzionalità$^G$ opzionali che aggiungono valore al sistema ma non influiscono sulla sua correttezza di base.
    \end{itemize}
\end{itemize}

% --------------------- 6.1.2 Affidabilità ---------------------
\subsubsection{Affidabilità}
\begin{itemize}
% MQD_04
\item \textbf{MQD\_04 - Code Coverage}:
    \begin{itemize}
        \item \textbf{Formula}:
        \[
            \text{Code Coverage} =
            \frac{\text{LOC$^G$ testate}}{\text{LOC$^G$ totali}} \times 100
        \]
		Dove LOC$^G$ = Lines of code
        \item \textbf{Descrizione}: misura la percentuale di linee di codice che risultano coperte da test$^G$ automatici. Una copertura elevata contribuisce a ridurre la probabilità di malfunzionamenti e garantisce maggiore stabilità nelle fasi di manutenzione$^G$.
    \end{itemize}

% MQD_05
\item \textbf{MQD\_05 - Branch$^G$ Coverage}:
    \begin{itemize}
        \item \textbf{Formula}:
        \[
            \text{Branch$^G$ Coverage} =
            \frac{\text{Rami coperti}}{\text{Rami totali}} \times 100
        \]
        \item \textbf{Descrizione}: misura la percentuale di rami decisionali testati. È una metrica fondamentale per verificare la robustezza del sistema in condizioni operative variabili.
    \end{itemize}

% MQD_06
\item \textbf{MQD\_06 - Statement Coverage}:
    \begin{itemize}
        \item \textbf{Formula}:
        \[
            \text{Statement Coverage} =
            \frac{\text{Istruzioni eseguite}}{\text{Istruzioni totali}} \times 100
        \]
        \item \textbf{Descrizione}: misura la percentuale di istruzioni eseguibili del codice che vengono effettivamente eseguite durante l’esecuzione della suite di test$^G$. Essa consente di valutare quanto i test$^G$ riescano a coprire il flusso operativo del software, identificando eventuali sezioni di codice non esercitate e quindi potenzialmente soggette a difetti non rilevati.
    \end{itemize}
\end{itemize}

% --------------------- 6.1.3 Usabilità ---------------------
\subsubsection{Usabilità}
\begin{itemize}
% MQD_07
\item \textbf{MQD\_07 - Time on Task$^G$}:
    \begin{itemize}
        \item \textbf{Formula}:
    \[
        \text{Time on Task$^G$} = \frac{\sum_{i=1}^{n} T_i}{n}
    \]
    dove:\\
    $T_i$ = tempo impiegato dall'utente $i$ per completare il task$^G$ specifico\\
    $n$ = numero di utenti che hanno completato il task$^G$
        \item \textbf{Descrizione}: misura quanto tempo, in media, un utente impiega a completare un'attività specifica nel sistema. Serve a capire se l'interfaccia è facile da usare e se ci sono passaggi che rallentano l'utente.
    \end{itemize}
\end{itemize}

% --------------------- 6.1.4 Efficienza$^G$ ---------------------
\subsubsection{Efficienza}
\begin{itemize}
% MQD_08
\item \textbf{MQD\_08 - Response time}:
    \begin{itemize}
        \item \textbf{Formula}:
        \[
            \text{Response time} = \text{Tempo risposta medio}
        \]
        \item \textbf{Descrizione}: misura il tempo di risposta del sistema alle interazioni dell’utente. Un valore ridotto è indicatore di un’applicazione performante e reattiva.
    \end{itemize}
\end{itemize}

% --------------------- 6.1.5 Manutenibilità ---------------------
\subsubsection{Manutenibilità}
\begin{itemize}
% MQD_09
\item \textbf{MQD\_09 - Code Smells per KLOC$^G$}:
    \begin{itemize}
        \item \textbf{Formula}:
        \[
            \text{Code Smells per KLOC$^G$} =
            \frac{\text{Numero di code smells}}{\text{KLOC$^G$}}
        \]
        Dove KLOC$^G$ = Kilo Lines of Code
        \item \textbf{Descrizione}: valuta la qualità del codice attraverso l’analisi dei “code smell”, ovvero pattern che segnalano complessità o debolezza nella struttura del software. Un numero contenuto favorisce la manutenibilità e la pulizia architetturale.
    \end{itemize}

% MQD_10
\item \textbf{MQD\_10 - Coefficient of Coupling}:
    \begin{itemize}
        \item \textbf{Formula}:
        \[
            \text{Coefficient of Coupling} =
            \frac{\text{Dipendenze esterne}}{\text{Moduli totali}}
        \]
        \item \textbf{Descrizione}: misura il livello di accoppiamento tra i moduli del sistema. Un basso accoppiamento è desiderabile perché rende il prodotto più stabile, testabile e semplice da estendere nel tempo.
    \end{itemize}

% MQD_11
\item \textbf{MQD\_11 - Cyclomatic Complexity$^G$}:
    \begin{itemize}
        \item \textbf{Formula}:
        \[
            \text{Cyclomatic Complexity$^G$} = E - N + 2P
        \]
        Dove:
        \begin{itemize}
            \item $E$: numero di archi
            \item $N$: numero di nodi
            \item $P$: componenti connessi
        \end{itemize}
        \item \textbf{Descrizione}: misura la complessità del flusso di controllo del codice contando i percorsi logicamente indipendenti. Valori elevati indicano funzioni difficili da comprendere e testare.
    \end{itemize}
\end{itemize}