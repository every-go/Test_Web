\section{Processi Primari}

    La creazione di un software di qualità non si basa solo sulla scrittura del codice e su test$^G$ isolati. Per realizzare un prodotto duraturo e ampiamente utilizzato, è essenziale adottare un modello di sviluppo$^G$ che definisca processi chiari da seguire. Lo standard ISO/IEC 12207, a tal proposito, identifica tra i processi primari quelli di \textbf{Fornitura} e \textbf{Sviluppo$^G$}.

	\subsection{Fornitura}
	Il processo$^G$ di Fornitura definisce le modalità con cui il gruppo si impegna a realizzare e consegnare il prodotto \textbf{SmartOrder} al proponente$^G$ Ergon Informatica Srl, nel rispetto di tempi, costi, qualità e requisiti concordati. \\
	Questo processo$^G$, che comprende tutte le attività di pianificazione$^G$, negoziazione, esecuzione, controllo e consegna, inizia con un'analisi preliminare dei requisiti per definire le funzionalità$^G$ del sistema, sulla base della quale è possibile negoziare i dettagli con la proponente$^G$, presentare una pianificazione$^G$ di progetto$^G$ e stimare una data di consegna plausibile.
	
	\subsubsection{Attività}
	\begin{itemize}[itemsep=3pt, parsep=1pt, label=$\scriptstyle\bullet$]
	    \item \textbf{Fase iniziale}: definizione degli obiettivi del progetto$^G$, studio di fattibilità tecnica ed economica, identificazione dei rischi e delle risorse necessarie, analisi dei capitolati d'appalto.
	    \item \textbf{Sviluppo$^G$ della proposta progettuale}: stesura della documentazione progettuale completa, inclusi Lettera di Presentazione, Preventivo dei Costi e Valutazione dei Capitolati.
	    \item \textbf{Consolidamento dei requisiti}: negoziazione con la proponente$^G$ su requisiti, tempi, costi, modalità di collaborazione e criteri di accettazione del prodotto finale.
	    \item \textbf{Pianificazione$^G$ operativa}: definizione del Piano di Progetto$^G$ dettagliato, assegnazione delle risorse, definizione delle milestones e delle scadenze critiche.
	    \item \textbf{Implementazione e monitoraggio}: monitoraggio dell’avanzamento, gestione delle modifiche e reporting periodico al proponente$^G$.
	    \item \textbf{Analisi dei risultati}: verifica$^G$ periodica dello stato del progetto$^G$ rispetto agli obiettivi prefissati, analisi degli scostamenti e valutazione delle performance del team.
	    \item \textbf{Rilascio del prodotto}: release del prodotto conforme ai requisiti, documentazione finale completa, chiusura formale del progetto$^G$.
	\end{itemize}
	
	\subsubsection{Strumenti a supporto}
	\phantomsection\label{sec:strumenti-supporto}
	Per la comunicazione interna al gruppo abbiamo deciso di utilizzare: \vspace{-0.5em}
	\begin{itemize}[itemsep=3pt, parsep=1pt, label=$\scriptstyle\bullet$]
		\item \textbf{GitHub$^G$}: per il versioning del codice e della documentazione, nonché per la gestione del backlog e del sistema di ticketing, strumenti essenziali per monitorare lo stato di avanzamento e le attività da completare;
	    \item \textbf{Discord$^G$}: per incontri a distanza;
	    \item \textbf{WhatsApp$^G$}: per la comunicazione asincrona del team.
	\end{itemize}
	Mentre per la comunicazione con l'azienda proponente$^G$ abbiamo concordato l'utilizzo di: \vspace{-0.5em}
	\begin{itemize}[itemsep=3pt, parsep=1pt, label=$\scriptstyle\bullet$]
	    \item \textbf{Google Mail}: per la comunicazione formale con l'Azienda;
	    \item \textbf{Google Meet$^G$}: per le riunioni periodiche e il coordinamento;
	    \item \textbf{WhatsApp$^G$}: per la comunicazione asincrona tra il team e la proponente$^G$.
	\end{itemize}

	\subsubsection{Documentazione fornita}
	NullPointers Group, in linea con gli obiettivi del progetto$^G$, si impegna a fornire a Ergon Informatica Srl e ai committenti tutta la documentazione relativa al progetto$^G$:

	\subsubsubsection{Lettera di Presentazione}
    \textbf{Obiettivo del documento}:\\
	Il documento si articola in tre tipologie:
    \begin{itemize}[itemsep=3pt, parsep=1pt, label=$\scriptstyle\bullet$]
		\item La \href{https://nullpointersgroup.github.io/Documentazione/output/Candidatura/Lettera\_di\_Presentazione.pdf}{Lettera di Presentazione} per la candidatura al progetto$^G$;
		\item La Lettera di Presentazione per la Requirements and Technology Baseline$^G$ (RTB$^G$);
	\end{itemize}
	Ciascuna lettera ha lo scopo di presentare formalmente il gruppo NullPointers Group, illustrando l’impegno nel rispettare gli obiettivi e i vincoli delle rispettive Baseline$^G$, 
	fornendo una panoramica di competenze, risorse e metodologie adottate durante lo sviluppo$^G$ del progetto$^G$.\\[0.5em]
	\medskip
	\textbf{Redattore$^G$}: Responsabile\\
	\medskip
	\textbf{Utilizzo}: Esterno\\
	\medskip 
	\begin{tabular}{@{}l l}
		\textbf{Destinatari:} & - Prof. Tullio Vardanega\\
							  & - Prof. Riccardo Cardin
	\end{tabular}

	\subsubsubsection{Preventivo dei Costi}
	\textbf{Obiettivo del documento}:\\
	Il \href{https://nullpointersgroup.github.io/Documentazione/output/Candidatura/Preventivo\_Costi.pdf}{Preventivo Costi} serve a mostrare come NullPointers Group ha pianificato il progetto$^G$, stimando ore e costi per ciascun ruolo e definendo le responsabilità dei membri del team.\\[0.5em]
	\medskip
	\textbf{Redattore$^G$}: Non Determinato\\
	\medskip
	\textbf{Utilizzo}: Esterno\\
	\medskip 
	\begin{tabular}{@{}l l}
		\textbf{Destinatari:} & - NullPointers Group\\
		                      & - Prof. Tullio Vardanega\\
							  & - Prof. Riccardo Cardin
	\end{tabular}

	\subsubsubsection{Valutazione dei Capitolati}
	\textbf{Obiettivo del documento}:\\
	La \href{https://nullpointersgroup.github.io/Documentazione/output/Candidatura/Valutazione\_Capitolati.pdf}{Valutazione dei Capitolati} è un documento redatto da NullPointer Group che analizza ogni capitolato$^G$ proposto, individuando punti di forza e criticità.\\
	Ogni analisi comprende la descrizione del capitolato$^G$ (informazioni sull’azienda proponente$^G$ e sul prodotto da sviluppare), una descrizione dei punti di forza e delle criticità riscontrate, derivanti dalla sua realizzazione. Con riguardo al primo capitolato$^G$ e ai due successivi di interesse, è stato inoltre descritto lo stack tecnologico previsto per la loro realizzazione.\\[0.5em]
	\medskip
	\textbf{Redattore$^G$}: Non Determinato\\
	\medskip
	\textbf{Utilizzo}: Esterno\\
	\medskip 
	\begin{tabular}{@{}l l}
		\textbf{Destinatari:} & - NullPointers Group\\
		                      & - Prof. Tullio Vardanega\\
							  & - Prof. Riccardo Cardin
	\end{tabular}

	\subsubsubsection{Verbali Esterni}
	\textbf{Obiettivo del documento}:\\
	Il Verbale Esterno è un documento redatto dal gruppo in occasione delle riunioni che coinvolgono soggetti esterni, come i referenti dell’azienda proponente$^G$.\\
	Riporta in modo strutturato tutti i contenuti discussi durante l’incontro, incluse richieste, chiarimenti, decisioni, vincoli, feedback e indicazioni fornite dai referenti esterni. Il suo scopo è garantire trasparenza, tracciabilità e un allineamento costante tra il team e la proponente$^G$, riducendo il rischio di fraintendimenti o interpretazioni errate.\\[0.5em]
	\medskip
	\textbf{Redattore$^G$}: Responsabile\\
	\medskip
	\textbf{Utilizzo}: Esterno\\
	\medskip 
	\begin{tabular}{@{}l l}
		\textbf{Destinatari:} & - NullPointers Group\\
		                      & - Ergon Informatica Srl
	\end{tabular}

	\subsubsubsection{Verbali Interni}
	\label{sec: verbali-interni}
	\textbf{Obiettivo del documento}:\\
	Il Verbale Interno è un documento redatto dal gruppo in occasione delle riunioni svolte esclusivamente tra i membri del team.\\
	Ha lo scopo di registrare tutte le decisioni prese, le attività pianificate e eventuali problematiche affrontate durante gli incontri interni.\\
	Il documento permette di mantenere una traccia chiara e condivisa dell’avanzamento del lavoro, favorendo il coordinamento tra i membri del gruppo e garantendo continuità anche quando le attività vengono svolte in modo asincrono.\\[0.5em]
	\medskip
	\textbf{Redattore$^G$}: Responsabile\\
	\medskip
	\textbf{Utilizzo}: Interno\\
	\medskip 
	\begin{tabular}{@{}l l}
		\textbf{Destinatari:} & - NullPointers Group
	\end{tabular}

	\subsubsubsection{Analisi dei Requisiti}
	\textbf{Obiettivo del documento}:\\
	L’\href{https://nullpointersgroup.github.io/Documentazione/output/RTB/Documenti\%20Esterni/Analisi\_dei\_Requisiti.pdf}{Analisi dei Requisiti$^G$} ha lo scopo di definire in modo chiaro e completo le funzionalità$^G$, i vincoli e le caratteristiche attese dal sistema SmartOrder. Questo documento costituisce un riferimento univoco per tutto il team, riducendo ambiguità e incomprensioni durante le fasi di progettazione$^G$ e sviluppo$^G$.\\
	Nel documento sono individuati gli attori principali e secondari, insieme ai principali casi d’uso.\\ 
	I requisiti sono organizzati in funzionali, non funzionali e di vincolo, e distinti per priorità in obbligatori, desiderabili e opzionali, fornendo una guida chiara per sviluppo$^G$, progettazione$^G$ e verifica$^G$.\\[0.5em]
	\medskip
	\textbf{Redattore$^G$}: Analista\\
	\medskip
	\textbf{Utilizzo}: Esterno\\
	\medskip 
	\begin{tabular}{@{}l l}
		\textbf{Destinatari:} & - NullPointers Group\\
		                      & - Ergon Informatica Srl\\
		                      & - Prof. Tullio Vardanega\\
							  & - Prof. Riccardo Cardin
	\end{tabular}

	\subsubsubsection{Glossario}
	\textbf{Obiettivo del documento}:\\
	Il \href{https://nullpointersgroup.github.io/Documentazione/output/RTB/Documenti\%20Interni/Glossario.pdf}{Glossario} 
	definisce in modo univoco termini tecnici, acronimi e concetti rilevanti del progetto$^G$ SmartOrder, è ritenuto dal team uno strumento fondamentale in quanto permette di standardizzare la terminologia e facilitare la comunicazione tra tutti i soggetti coinvolti.\\
	Le voci sono ordinate alfabeticamente e, quando citate nei documenti, riportano l'apice.\\[0.5em]
	\medskip
	\textbf{Redattore$^G$}: Analista\\
	\medskip
	\textbf{Utilizzo}: Interno\\
	\medskip 
	\begin{tabular}{@{}l l}
		\textbf{Destinatari:} & - NullPointers Group
	\end{tabular}

	\subsubsubsection{Norme di Progetto}
	Le \href{https://nullpointersgroup.github.io/Documentazione/output/RTB/Documenti\%20Interni/Norme\_di\_Progetto.pdf}{Norme di Progetto$^G$}, ovvero il presente documento, definiscono il Way of Working$^G$ adottato dal team NullPointers Group per lo sviluppo$^G$ del progetto$^G$ SmartOrder.\\
	Mirano a stabilire regole, metodologie e standard per garantire coerenza, qualità e uniformità nella produzione$^G$ di documenti, codice e artefatti di progetto$^G$.\\
	Le norme disciplinano la gestione e il versionamento$^G$ dei file, la comunicazione interna ed esterna, il tracciamento dei requisiti e delle decisioni, nonché le procedure di revisione e approvazione.\\ 
	Il rispetto di queste regole è obbligatorio per tutti i membri del gruppo e rappresenta un riferimento costante per l’esecuzione delle attività di progetto$^G$.\\[0.5em]
	\medskip
	\textbf{Redattore$^G$}: Amministratore\\
	\medskip
	\textbf{Utilizzo}: Interno\\
	\medskip 
	\begin{tabular}{@{}l l}
		\textbf{Destinatari:} & - NullPointers Group
	\end{tabular}

	\subsubsubsection{Piano di Progetto}
	Il \href{https://nullpointersgroup.github.io/Documentazione/output/RTB/Documenti\%20Esterni/Piano\_di\_Progetto.pdf}{Piano di Progetto$^G$} 
	definisce l’organizzazione, le attività, le risorse e i tempi necessari per lo sviluppo$^G$ del progetto$^G$ da parte del team. \\
	Per ogni sprint$^G$ è inclusa una tabella con i preventivi e i ruoli assunti da ciascun componente. \\
	Vengono inoltre forniti: l'analisi dei rischi, le scadenze e la stima di tempi e costi, con l'aggiornamento del consumo orario e dei costi previsti ed effettivi.\\ 
	Rappresenta uno strumento operativo fondamentale per coordinare il team, gestire le risorse, tracciare le decisioni e verificare la conformità agli standard di qualità definiti nelle Norme di Progetto$^G$, garantendo il rispetto di obiettivi, requisiti e vincoli del progetto$^G$.\\[0.5em]
	\medskip
	\textbf{Redattore$^G$}: Responsabile\\
	\medskip
	\textbf{Utilizzo}: Esterno\\
	\medskip 
	\begin{tabular}{@{}l l}
		\textbf{Destinatari:} & - NullPointers Group\\
		                      & - Ergon Informatica Srl\\
		                      & - Prof. Tullio Vardanega\\
							  & - Prof. Riccardo Cardin
	\end{tabular}

	\subsubsubsection{Piano di Qualifica}
	Il \href{https://nullpointersgroup.github.io/Documentazione/output/RTB/Documenti\%20Esterni/Piano\_di\_Qualifica.pdf}{Piano di Qualifica$^G$} 
	definisce strategie, criteri e procedure per verificare e validare il sistema SmartOrder, garantendo il rispetto dei requisiti funzionali, 
	non funzionali e di vincolo.\\
	Include la pianificazione$^G$ delle attività di test$^G$, la descrizione dei casi di verifica$^G$, i criteri di accettazione e le modalità di tracciamento delle anomalie.\\
	Rappresenta dunque un riferimento operativo per assicurare la qualità del prodotto, monitorare l’avanzamento dei test$^G$ e supportare decisioni sul rilascio 
	delle componenti del sistema.\\[0.5em]
	\medskip
	\textbf{Redattore$^G$}: Amministratore\\
	\medskip
	\textbf{Utilizzo}: Esterno\\
	\medskip 
	\begin{tabular}{@{}l l}
		\textbf{Destinatari:} & - NullPointers Group\\
		                      & - Ergon Informatica Srl\\
		                      & - Prof. Tullio Vardanega\\
							  & - Prof. Riccardo Cardin
	\end{tabular}
	
	\subsection{Sviluppo}
	Il processo$^G$ di sviluppo$^G$ è un insieme strutturato di attività che guidano la realizzazione del software, dalla definizione dei requisiti fino al rilascio del prodotto finale.\\ 
	Comprende fasi di analisi, progettazione$^G$, implementazione, integrazione, test$^G$ e validazione$^G$, seguendo un approccio iterativo e modulare, con ciascuna parte del sistema sviluppata, testata e integrata passo dopo passo.\\
	Lo scopo del processo$^G$ è garantire che il prodotto soddisfi i requisiti, ridurre i rischi, mantenere la qualità e la tracciabilità delle attività, e assicurare che ogni rilascio sia coerente con gli obiettivi concordati con il committente$^G$.
	
	\subsubsection{Attività}
	Il progetto$^G$ SmartOrder è stato sviluppato seguendo un approccio incrementale e agile$^G$, garantendo al contempo la conformità agli standard internazionali 
	\href{https://www.math.unipd.it/~tullio/IS-1/2009/Approfondimenti/ISO_12207-1995.pdf}{ISO/IEC 12207:1995}.\\ 
	Questo approccio permette di mantenere tracciabilità, qualità e gestione efficace delle risorse durante tutte le fasi del progetto$^G$.\\
	Le attività di sviluppo$^G$ comprendono:
	\begin{enumerate}[itemsep=3pt, parsep=1pt, label=$\scriptstyle\bullet$]
		\item \textbf{Raccolta e Analisi dei Requisiti$^G$}: individuazione delle esigenze della proponente$^G$ e definizione dei casi d’uso, con distinzione tra requisiti funzionali, non funzionali e di vincolo.
		\item \textbf{Progettazione$^G$ dell’Architettura}: definizione della struttura del sistema, dei componenti principali e delle loro interazioni.
		\item \textbf{Sviluppo$^G$ Incrementale}: implementazione delle funzionalità$^G$ in moduli indipendenti, ciascuno verificato prima dell’integrazione.
		\item \textbf{Monitoraggio e Revisione}: attività continua durante tutto il progetto$^G$, dedicata al controllo dell’avanzamento, alla gestione delle modifiche e alla verifica$^G$ della qualità.
		\item \textbf{Integrazione e Test$^G$}: unione dei moduli sviluppati e verifica$^G$ del corretto funzionamento complessivo tramite test$^G$ di sistema.
		\item \textbf{Rilascio e Distribuzione$^G$}: consegna delle funzionalità$^G$ approvate e della documentazione associata.
	\end{enumerate}
	Queste attività sono cicliche e iterativamente aggiornate ad ogni sprint$^G$, permettendo un progresso costante e la tracciabilità di tutte le decisioni.

	\subsubsection{Analisi dei Requisiti}
	L’analisi dei requisiti$^G$ è un processo$^G$ che ha lo scopo di identificare in maniera univoca tutte le esigenze funzionali ed i requisiti che il sistema software dovrà soddisfare.\\
	Il risultato di questo processo$^G$ viene formalizzato nell’apposito documento di \href{https://nullpointersgroup.github.io/Documentazione/output/RTB/Documenti\%20Esterni/Analisi\_dei\_Requisiti.pdf}{Analisi dei Requisiti$^G$} il quale costituisce la base per le successive fasi di progettazione$^G$, codifica e test$^G$. 
	Espone nel dettaglio i casi d’uso e i requisiti che compongono gli elementi fondamentali per lo sviluppo$^G$ del progetto$^G$.
		\subsubsubsection{Casi d'uso}
		Per la definizione e codifica dei casi d’uso è stata adottata la seguente nomenclatura:
		\begin{center} UC\_Principale.Secondario \end{center}
		dove:
		\begin{itemize}[itemsep=3pt, parsep=1pt, label=$\scriptstyle\bullet$]
			\item \textbf{UC} è l’acronimo di Use Case (Caso d’Uso);
			\item  \textbf{Principale} è un numero intero, incrementale che identifica univocamente il caso d’uso principale;
			\item \textbf{Secondario} è un numero intero, incrementale che identifica un caso d’uso derivato dal caso d’uso principale.
		\end{itemize}
		Nel caso in cui un caso d’uso non presenti sotto-casi verrà utilizzato il formato:
		\begin{center} UC\_Principale \end{center}
		I casi d’uso principali sono univoci all’interno del sistema: non possono esistere due casi d’uso con lo stesso valore Principale.\\
		È invece possibile che casi d’uso secondari condividano il medesimo indice Secondario, purché appartengano a casi d’uso principali differenti.\\
		Questa convenzione ha lo scopo di garantire l’univocità e la tracciabilità di ciascun Caso d’Uso. Ad ogni identificatore inoltre è associato un nome ed una breve descrizione che riassume in maniera concisa lo scopo del caso d’uso.
		\subsubsubsection{Requisiti}
		Per la definizione e la codifica dei Requisiti è stata adottata la seguente nomenclatura:
		\begin{center} RTipologia-Priorità\_\# \end{center}
		dove:
		\begin{itemize}
			\item \textbf{R} è l'abbreviazione di Requisito;
			\item \textbf{Tipologia} identifica la tipologia del requisito, le possibili tipologie sono:
				\begin{itemize}[itemsep=3pt, parsep=1pt, label=$\scriptstyle\bullet$]
					\item \textbf{F}: Requisito Funzionale$^G$;
					\item \textbf{Q}: Requisito di Qualità;
					\item \textbf{V}: Requisito di Vincolo$^G$.
				\end{itemize}
			\item \textbf{Priorità} identifica la priorità di sviluppo$^G$ del requisito, le possibili tipologie sono:
				\begin{itemize}[itemsep=3pt, parsep=1pt, label=$\scriptstyle\bullet$]
					\item \textbf{OB}: che indica Obbligatorio;
					\item \textbf{DE}: che indica Desiderabile;
					\item \textbf{OP}: che indica Opzionale.
				\end{itemize}
			\item \textbf{\#} è un numero intero incrementale che identifica univocamente il requisito all’interno della sua tipologia e priorità.
		\end{itemize}
		Per maggiore chiarezza sui casi d’uso ed i requisiti si consiglia la lettura del documento di \href{https://nullpointersgroup.github.io/Documentazione/output/RTB/Documenti\%20Esterni/Analisi\_dei\_Requisiti.pdf}{Analisi dei Requisiti$^G$}.
	
	\subsubsection{Codifica}
	Il compito del Programmatore$^G$ è tradurre in codice eseguibile quanto definito durante l’analisi e la progettazione$^G$. Per garantire coerenza, manutenibilità e qualità del prodotto software, è imperativo che ogni sviluppatore aderisca alle seguenti regole e convenzioni.

	\subsubsubsection{Convenzioni sintattiche}
	La verifica$^G$ della conformità alle convenzioni di codifica viene effettuata tramite strumenti automatici e revisione manuale durante le Pull Request$^G$.\\
	Per garantire coerenza, leggibilità e manutenibilità del codice in tutto il progetto$^G$, il team adotta le seguenti convenzioni:
	\begin{itemize}[itemsep=3pt, parsep=1pt, label=\textbullet]
		\item \textbf{Nomenclatura}: i nomi di variabili, funzioni, metodi e classi devono essere scritti in inglese.
		\item Per il \textbf{codice di produzione$^G$}: usare \texttt{camelCase} per metodi, funzioni, variabili e \texttt{PascalCase} per i nomi delle classi.
        \item Per i \textbf{test$^G$}: si utilizza il prefisso \texttt{test\_}.
		\item \textbf{Lingua dei commenti}: i commenti devono essere scritti in italiano. Questo favorisce la comprensione immediata da parte di tutti i membri del team durante le revisioni.
		\item \textbf{Scope delle Variabili}: limitare al massimo l'uso di variabili globali.
		\item \textbf{Funzioni}: favorire funzioni concise e monolitiche (che svolgono un unico compito ben definito).
	\end{itemize}

