\section{Processi Primari}

    La creazione di un software di qualità non si basa solo sulla scrittura del codice e su test$^G$ isolati. Per realizzare un prodotto duraturo e ampiamente utilizzato, è essenziale adottare un modello di sviluppo$^G$ che definisca processi chiari da seguire. 
	Lo standard ISO/IEC 12207, a tal proposito, identifica tra i processi primari quelli di \textbf{Fornitura} e \textbf{Sviluppo}.

	\subsection{Fornitura}
	Il processo di Fornitura definisce le modalità con cui il gruppo si impegna a realizzare e consegnare il prodotto \textbf{SmartOrder} al proponente Ergon Informatica Srl, nel rispetto di tempi, costi, qualità e requisiti concordati. 
	Questo processo, che comprende tutte le attività di pianificazione, negoziazione, esecuzione, controllo e consegna, inizia con un'analisi preliminare dei requisiti per definire le funzionalità$^G$ del sistema, sulla base della quale è possibile 
	negoziare i dettagli con la proponente, presentare una pianificazione$^G$ di progetto e stimare una data di consegna plausibile.
	
	\subsubsection{Attività}
	\begin{itemize}[itemsep=5pt, parsep=1pt, label=$\scriptstyle\bullet$]
	    \item \textbf{Fase iniziale}: Definizione degli obiettivi del progetto, studio di fattibilità tecnica ed economica, identificazione dei rischi e delle risorse necessarie, analisi dei capitolati d'appalto.
	    \item \textbf{Sviluppo$^G$ della proposta progettuale}: Stesura della documentazione progettuale completa, inclusi lettera di presentazione, preventivo dei costi e valutazione dei capitolati.
	    \item \textbf{Consolidamento dei requisiti}: Negoziazione con la proponente su requisiti, tempi, costi, modalità di collaborazione e criteri di accettazione del prodotto finale.
	    \item \textbf{Pianificazione$^G$ operativa}: Definizione del piano di progetto$^G$ dettagliato, assegnazione delle risorse, definizione delle milestones e delle scadenze critiche.
	    \item \textbf{Implementazione e monitoraggio}: Monitoraggio dell’avanzamento, gestione delle modifiche e reporting periodico al proponente.
	    \item \textbf{Analisi dei risultati}: Verifica$^G$ periodica dello stato del progetto rispetto agli obiettivi prefissati, analisi degli scostamenti e valutazione delle performance del team.
	    \item \textbf{Rilascio del prodotto}: Release del prodotto conforme ai requisiti, documentazione finale completa, chiusura formale del progetto.
	\end{itemize}
	
	\subsubsection{Strumenti a supporto}
	Per la comunicazione interna al gruppo abbiamo deciso di utilizzare: \vspace{-0.5em}
	\begin{itemize}[itemsep=0pt, parsep=1pt, label=$\scriptstyle\bullet$]
		\item \textbf{GitHub}: per il versioning del codice e della documentazione, nonché per la gestione del backlog e del sistema di ticketing, strumenti essenziali per monitorare lo stato di avanzamento e le attività da completare;
	    \item \textbf{Discord}: per incontri a distanza;
	    \item \textbf{WhatsApp}: per la comunicazione sincrona del team.
	\end{itemize}
	Mentre per la comunicazione con l'azienda proponente abbiamo concordato l'utilizzo di: \vspace{-0.5em}
	\begin{itemize}[itemsep=0pt, parsep=1pt, label=$\scriptstyle\bullet$]
	    \item \textbf{Google Mail}: per la comunicazione formale con l'Azienda;
	    \item \textbf{Google Meet}: per le riunioni periodiche e il coordinamento;
	    \item \textbf{WhatsApp}: per la comunicazione asincrona tra il team e la proponente.
	\end{itemize}

	\subsubsection{Documentazione fornita}
	NullPointers Group, in linea con gli obiettivi del progetto, si impegna a fornire a Ergon Informatica Srl e ai committenti tutta la documentazione relativa al progetto:

	\subsubsubsection{Lettera di Presentazione}
    \textbf{Obiettivo del documento}:\\
	Il documento si articola in tre tipologie:
    \begin{itemize}[itemsep=0pt, parsep=1pt, label=$\scriptstyle\bullet$]
		\item La \ulhref{https://nullpointersgroup.github.io/Documentazione/output/Candidatura/Lettera\_di\_Presentazione.pdf}{Lettera di presentazione} per la candidatura al progetto;
		\item La \ulhref{https://nullpointersgroup.github.io/Documentazione/output/RTB/Documenti\%20Esterni/Lettera\_di\_Presentazione.pdf}{Lettera di presentazione} per la Requirements and Technology Baseline$^G$ (RTB)$^{G}$;
		\item La \ulhref{https://nullpointersgroup.github.io/Documentazione/output/PB/Documenti\%20Esterni/Lettera\_di\_Presentazione.pdf}{Lettera di presentazione} per la Product Baseline$^G$ (PB)$^{G}$.
	\end{itemize}
	Ciascuna lettera ha lo scopo di presentare formalmente il gruppo NullPointers Group, illustrando l’impegno nel rispettare gli obiettivi e i vincoli delle rispettive Baseline$^{G}$, 
	fornendo una panoramica di competenze, risorse e metodologie adottate durante lo sviluppo$^G$ del progetto.\\[0.5em]
	\medskip
	\textbf{Redattore}: Responsabile\\
	\medskip
	\textbf{Utilizzo}: Esterno\\
	\medskip 
	\begin{tabular}{@{}l l}
		\textbf{Destinatari:} & - Prof. Tullio Vardanega\\
							& - Prof. Riccardo Cardin\\
							& - Ergon Informatica Srl\\
							& - NullPointers Group
	\end{tabular}

	\subsubsubsection{Preventivo dei Costi}
	\textbf{Obiettivo del documento}:\\
	La \ulhref{https://nullpointersgroup.github.io/Documentazione/output/Candidatura/Preventivo\_Costi.pdf}{Preventivo Costi} serve a mostrare come NullPointers Group ha pianificato il progetto, stimando ore e costi per ciascun ruolo e definendo le responsabilità dei membri del team.\\[0.5em]
	\medskip
	\textbf{Redattore}: Non Determinato\\
	\medskip
	\textbf{Utilizzo}: Esterno\\
	\medskip 
	\begin{tabular}{@{}l l}
		\textbf{Destinatari:} & - Prof. Tullio Vardanega\\
							& - Prof. Riccardo Cardin\\
							& - Ergon Informatica Srl\\
							& - NullPointers Group
	\end{tabular}

	\subsubsubsection{Valutazione dei Capitolati}
	\textbf{Obiettivo del documento}:\\
	La \ulhref{https://nullpointersgroup.github.io/Documentazione/output/Candidatura/Valutazione\_Capitolati.pdf}{Valutazione dei Capitolati} è un documento redatto da NullPointer Group che analizza ogni capitolato$^G$ proposto, individuando punti di forza e criticità.\\
	Ogni analisi comprende la descrizione del capitolato$^G$ (informazioni sull’azienda proponente e sul prodotto da sviluppare), una descrizione dei punti di forza e delle criticità riscontrate, derivanti dalla sua realizzazione.
	Con riguardo al primo capitolato$^G$ e ai due successivi di interesse, è stato inoltre descritto lo stack tecnologico previsto per la loro realizzazione.\\[0.5em]
	\medskip
	\textbf{Redattore}: Non Determinato\\
	\medskip
	\textbf{Utilizzo}: Esterno\\
	\medskip 
	\begin{tabular}{@{}l l}
		\textbf{Destinatari:} & - Prof. Tullio Vardanega\\
							& - Prof. Riccardo Cardin\\
							& - Ergon Informatica Srl\\
							& - NullPointers Group
	\end{tabular}

	\subsubsubsection{Verbali Esterni}
	\textbf{Obiettivo del documento}:\\
	Il Verbale Esterno è un documento redatto dal gruppo in occasione delle riunioni che coinvolgono soggetti esterni, come i referenti dell’azienda proponente.\\
	Riporta in modo strutturato tutti i contenuti discussi durante l’incontro, incluse richieste, chiarimenti, decisioni, vincoli, feedback e indicazioni fornite dai referenti esterni.
	Il suo scopo è garantire trasparenza, tracciabilità e un allineamento costante tra il team e la proponente, riducendo il rischio di fraintendimenti o interpretazioni errate.\\[0.5em]
	\medskip
	\textbf{Redattore}: Responsabile\\
	\medskip
	\textbf{Utilizzo}: Esterno\\
	\medskip 
	\begin{tabular}{@{}l l}
		\textbf{Destinatari:} & - Prof. Tullio Vardanega\\
							& - Prof. Riccardo Cardin\\
							& - Ergon Informatica Srl\\
							& - NullPointers Group
	\end{tabular}

	\subsubsubsection{Verbali Interni}
	\textbf{Obiettivo del documento}:\\
	Il Verbale Interno è un documento redatto dal gruppo in occasione delle riunioni svolte esclusivamente tra i membri del team.\\
	Ha lo scopo di registrare tutte le decisioni prese, le attività pianificate e eventuali problematiche affrontate durante gli incontri interni.\\
	Il documento permette di mantenere una traccia chiara e condivisa dell’avanzamento del lavoro, favorendo il coordinamento tra i membri del gruppo 
	e garantendo continuità anche quando le attività vengono svolte in modo asincrono.\\[0.5em]
	\medskip
	\textbf{Redattore}: Responsabile\\
	\medskip
	\textbf{Utilizzo}: Interno\\
	\medskip 
	\begin{tabular}{@{}l l}
		\textbf{Destinatari:} & - NullPointers Group
	\end{tabular}

	\subsubsubsection{Analisi dei Requisiti}
	\textbf{Obiettivo del documento}:\\
	L’\ulhref{https://nullpointersgroup.github.io/Documentazione/output/RTB/Documenti\%20Esterni/Analisi\_dei\_Requisiti.pdf}{analisi dei requisiti} ha lo scopo di definire in modo chiaro e completo le funzionalità, i vincoli e le caratteristiche attese dal sistema SmartOrder. Questo documento costituisce 
	un riferimento univoco per tutto il team, riducendo ambiguità e incomprensioni durante le fasi di progettazione$^G$ e sviluppo.\\
	Nel documento sono individuati gli attori principali e secondari, insieme ai principali casi d’uso.\\ I requisiti sono organizzati in funzionali, non funzionali e di vincolo, 
	e distinti per priorità in obbligatori, desiderabili e opzionali, fornendo una guida chiara per sviluppo, progettazione$^G$ e verifica.\\[0.5em]
	\medskip
	\textbf{Redattore}: Analista\\
	\medskip
	\textbf{Utilizzo}: Esterno\\
	\medskip 
	\begin{tabular}{@{}l l}
		\textbf{Destinatari:} & - Prof. Tullio Vardanega\\
							& - Prof. Riccardo Cardin\\
							& - Ergon Informatica Srl\\
							& - NullPointers Group
	\end{tabular}

	\subsubsubsection{Glossario}
	\textbf{Obiettivo del documento}:\\
	Il \ulhref{https://nullpointersgroup.github.io/Documentazione/output/RTB/Documenti\%20Interni/Glossario.pdf}{glossario} 
	definisce in modo univoco termini tecnici, acronimi e concetti rilevanti del progetto SmartOrder, è ritenuto dal team uno strumento fondamentale 
	in quanto permette di standardizzare la terminologia e facilitare la comunicazione tra tutti i soggetti coinvolti.\\
	Le voci sono ordinate alfabeticamente e, quando citate nei documenti, riportano l'apice$^{G}$.\\
	La figura dell'Amministratore$^G$ si occupa di aggiornare periodicamente il glossario.\\[0.5em]
	\medskip
	\textbf{Redattore}: Analista\\
	\medskip
	\textbf{Utilizzo}: Interno\\
	\medskip 
	\begin{tabular}{@{}l l}
		\textbf{Destinatari:} & - NullPointers Group
	\end{tabular}

	\subsubsubsection{Norme di Progetto}
	Le \ulhref{https://nullpointersgroup.github.io/Documentazione/output/RTB/Documenti\%20Interni/Norme\_di\_Progetto.pdf}{Norme di Progetto},
	ovvero il seguente documento, definiscono il Way of Working$^G$ adottato dal team NullPointers Group per lo sviluppo$^G$ del progetto SmartOrder. 
	Mirano a stabilire regole, metodologie e standard per garantire coerenza, qualità e uniformità nella produzione$^G$ di documenti, codice e artefatti di progetto.\\
	Le norme disciplinano la gestione e il versionamento dei file, la comunicazione interna ed esterna, il tracciamento dei requisiti e delle decisioni, nonché 
	le procedure di revisione e approvazione.\\ 
	Il rispetto di queste regole è obbligatorio per tutti i membri del gruppo e rappresenta un riferimento costante per l’esecuzione delle attività di progetto.\\[0.5em]
	\medskip
	\textbf{Redattore}: Amministratore\\
	\medskip
	\textbf{Utilizzo}: Interno\\
	\medskip 
	\begin{tabular}{@{}l l}
		\textbf{Destinatari:} & - NullPointers Group
	\end{tabular}

	\subsubsubsection{Piano di Progetto}
	Il \ulhref{https://nullpointersgroup.github.io/Documentazione/output/RTB/Documenti\%20Esterni/Piano\_di\_Progetto.pdf}{Piano di Progetto} 
	definisce l’organizzazione, le attività, le risorse e i tempi necessari per lo sviluppo$^G$ del progetto da parte del team. \\
	Per ogni sprint$^G$ è inclusa una tabella con i preventivi e i ruoli assunti da ciascun componente. \\Vengono inoltre forniti: l'analisi dei rischi, le scadenze e la stima di tempi e costi, 
	con l'aggiornamento del consumo orario e dei costi previsti ed effettivi.\\ Rappresenta uno strumento operativo fondamentale per coordinare il team, gestire le risorse, tracciare le decisioni e 
	verificare la conformità agli standard di qualità definiti nelle Norme di Progetto, garantendo il rispetto di obiettivi, requisiti e vincoli del progetto.\\[0.5em]
	\medskip
	\textbf{Redattore}: Responsabile\\
	\medskip
	\textbf{Utilizzo}: Esterno\\
	\medskip 
	\begin{tabular}{@{}l l}
		\textbf{Destinatari:} & - Prof. Tullio Vardanega\\
							& - Prof. Riccardo Cardin\\
							& - Ergon Informatica Srl\\
							& - NullPointers Group
	\end{tabular}

	\subsubsubsection{Piano di Qualifica}
	Il \ulhref{https://nullpointersgroup.github.io/Documentazione/output/RTB/Documenti\%20Esterni/Piano\_di\_Qualifica.pdf}{Piano di Qualifica} 
	definisce strategie, criteri e procedure per verificare e validare il sistema SmartOrder, garantendo il rispetto dei requisiti funzionali, 
	non funzionali e di vincolo.\\
	Include la pianificazione$^G$ delle attività di test, la descrizione dei casi di verifica, i criteri di accettazione e le modalità di tracciamento delle anomalie.\\
	Rappresenta dunque un riferimento operativo per assicurare la qualità del prodotto, monitorare l’avanzamento dei test$^G$ e supportare decisioni sul rilascio 
	delle componenti del sistema.\\[0.5em]
	\medskip
	\textbf{Redattore}: Amministratore\\
	\medskip
	\textbf{Utilizzo}: Esterno\\
	\medskip 
	\begin{tabular}{@{}l l}
		\textbf{Destinatari:} & - Prof. Tullio Vardanega\\
							& - Prof. Riccardo Cardin\\
							& - Ergon Informatica Srl\\
							& - NullPointers Group
	\end{tabular}
	
	\subsection{Sviluppo}
	Il processo di sviluppo$^G$ è un insieme strutturato di attività che guidano la realizzazione del software, dalla definizione dei requisiti 
	fino al rilascio del prodotto finale.\\ Comprende fasi di analisi, progettazione, implementazione, integrazione, test$^G$ e validazione, 
	seguendo un approccio iterativo e modulare, con ciascuna parte del sistema sviluppata, testata e integrata passo dopo passo.\\
	Lo scopo del processo è garantire che il prodotto soddisfi i requisiti, ridurre i rischi, mantenere la qualità e la tracciabilità delle attività, 
	e assicurare che ogni rilascio sia coerente con gli obiettivi concordati con il committente.
	
	\subsubsection{Attività}
	Il progetto SmartOrder è stato sviluppato seguendo un approccio incrementale e agile, garantendo al contempo la conformità agli standard internazionali 
	\ulhref{https://www.math.unipd.it/~tullio/IS-1/2009/Approfondimenti/ISO_12207-1995.pdf}{ISO/IEC 12207:1995}.\\ 
	Questo approccio permette di mantenere tracciabilità, qualità e gestione efficace delle risorse durante tutte le fasi del progetto.
	Le attività di sviluppo$^G$ comprendono:
	\begin{enumerate}[itemsep=0pt, parsep=1pt, label=$\scriptstyle\bullet$]
		\item \textbf{Raccolta e Analisi dei Requisiti}: individuazione delle esigenze della proponente e definizione dei casi d’uso, con distinzione tra requisiti funzionali, non funzionali e di vincolo.
		\item \textbf{Progettazione$^G$ dell’Architettura}: definizione della struttura del sistema, dei componenti principali e delle loro interazioni.
		\item \textbf{Sviluppo$^G$ Incrementale}: implementazione delle funzionalità$^G$ in moduli indipendenti, ciascuno verificato prima dell’integrazione.
		\item \textbf{Monitoraggio e Revisione}: attività continua durante tutto il progetto, dedicata al controllo dell’avanzamento, alla gestione delle modifiche e alla verifica$^G$ della qualità.
		\item \textbf{Integrazione e Test}: unione dei moduli sviluppati e verifica$^G$ del corretto funzionamento complessivo tramite test$^G$ di sistema.
		\item \textbf{Rilascio e Distribuzione}: consegna delle funzionalità$^G$ approvate e della documentazione associata.
	\end{enumerate}
	Queste attività sono cicliche e iterativamente aggiornate ad ogni sprint, permettendo un progresso costante e la tracciabilità di tutte le decisioni.
