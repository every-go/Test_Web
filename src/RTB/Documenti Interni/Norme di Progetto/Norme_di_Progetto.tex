% Configurazione
\documentclass{article}

\usepackage{titling} % Required for inserting the subtitle
\usepackage{graphicx} % Required for inserting images
\usepackage{tabularx} % Per l'ambiente tabularx (tabelle)
\usepackage{calc} % Sempre per le tabelle
\usepackage{tocloft}
\renewcommand{\cftsecleader}{\cftdotfill{\cftdotsep}}
\usepackage[hidelinks]{hyperref} % Per i collegamenti ipertestuali, ad esempio sulla table of contents
\usepackage{xcolor} % Per colorare il testo
\usepackage{colortbl} % Per colorare le celle delle tabelle
\usepackage{lipsum} % Per generare lorem ipsum
\usepackage[normalem]{ulem} % Per sottolineare il testo
\usepackage{array} % Per la visualizzazione fluttuante di array di domande e risposte
\usepackage{ragged2e} % Pacchetto necessario per \justifying che giustifica il testo di tabelle
\usepackage{placeins} % Impedisce a figure e tabelle di andare oltre le barriere definite
\usepackage{fancyhdr} % Intestazioni e piè di pagina
\usepackage{lastpage} % Per contare il numero totale di pagine
\usepackage{enumitem} 
\usepackage{amssymb} %per personalizzare gli elenchi

\newcommand{\ulhref}[2]{\href{#1}{\underline{#2}}} % Sottolineatura
\newcommand{\ulref}[1]{\uline{\ref{#1}}} % Sottolinea riferimenti a figure
\setlength{\parindent}{0pt} % Nessun rientro dei paragrafi

\graphicspath{ {immagini/} {../../../shared/images/} }

% --- Impostazioni intestazione e piè di pagina ---
\pagestyle{fancy}
\fancyhf{}
% Intestazione standard
\fancyhead[L]{NullPointers Group}
\fancyhead[R]{Norme di Progetto}
\renewcommand{\headrulewidth}{0.4pt}
% Piè di pagina
\fancyfoot[C]{Pagina \thepage{} di \pageref{LastPage}}
\renewcommand{\footrulewidth}{0.4pt}

% --- Ridefinizione dello stile plain per gli indici ---
\fancypagestyle{plain}{
  \fancyhf{}
  \fancyhead[L]{NullPointers Group}
  \fancyhead[R]{Norme di Progetto}
  \fancyfoot[C]{Pagina \thepage{} di \pageref{LastPage}}
  \renewcommand{\headrulewidth}{0.4pt}
  \renewcommand{\footrulewidth}{0.4pt}
}

% --- Solo piè di pagina per la prima pagina ---
\fancypagestyle{firstpage}{
  \fancyhf{}
  \fancyfoot[C]{Pagina \thepage{} di \pageref{LastPage}}
  \renewcommand{\headrulewidth}{0pt}
  \renewcommand{\footrulewidth}{0.4pt}
}

% --- Creazione livello 4 (subsubsubsection) ---
\usepackage{titlesec}

\titleclass{\subsubsubsection}{straight}[\subsubsection]

\newcounter{subsubsubsection}[subsubsection]
\renewcommand{\thesubsubsubsection}{\thesubsubsection.\arabic{subsubsubsection}}

\titleformat{\subsubsubsection}
  {\normalfont\normalsize\bfseries}
  {\thesubsubsubsection}{0.5em}{}

\titlespacing*{\subsubsubsection}{0pt}{1ex}{1ex}

% --- Registrazione livello 4 per tocloft ---
\makeatletter
\newcommand{\l@subsubsubsection}{\@dottedtocline{4}{7.0em}{3.7em}}
\makeatother

% --- Toc depth ---
\setcounter{secnumdepth}{4}
\setcounter{tocdepth}{4}

%struttura
\begin{document}
\thispagestyle{firstpage} % Copertina con solo piè di pagina

%parte costante
\begin{minipage}{0.4\textwidth}
    \includegraphics[width=0.6\textwidth]{logo_unipd.png}
\end{minipage}
\begin{minipage}{0.55\textwidth}
    \textcolor{red}{\textbf{Università degli Studi di Padova}} \\
    \textcolor{red}{Laurea: Informatica} \\
    \textcolor{red}{Corso: Ingegneria del Software} \\
    \textcolor{red}{Anno Accademico: 2025/2026}
\end{minipage}

\begin{minipage}{0.4\textwidth}
    \includegraphics[width=0.6\textwidth]{logo_gruppo.jpeg}
\end{minipage}
\begin{minipage}{0.55\textwidth}
    \textbf{Gruppo: NullPointers Group} \\
    Email: \textsf{groupnullpointers@gmail.com}
\end{minipage}

\vspace{2cm}
{
		\centering
		\Huge\bfseries Norme di Progetto\par
		\vspace{1.0cm}
		\Large\bfseries \par
	}
\begin{center}
\begin{tabular}{r|l}
    Stato & In Approvazione \\[0.5em]
    Versione & 0.2.0 \\[0.5em]
    Data ultima modifica & 15/11/2025 \\[0.5em]
    Redattori & Lisa Casagrande \\
              & Matteo Mazzaretto \\[0.5em]
    Verificatori & Matteo Mazzaretto \\
                 & Tommaso Ceron \\[0.5em]
    Destinatari & Prof. Tullio Vardanega \\
    & Prof. Riccardo Cardin \\
    & Ergon Informatica Srl \\
    & NullPointers Group \\
\end{tabular}
\end{center}
\newpage
	
%variabile prima dell'indice
\section*{Registro delle modifiche}

\begin{table}[h]
	\centering
	\resizebox{\textwidth}{!}{
		\begin{tabular}{|c|c|c|c|c|}
			\hline
			\rowcolor[gray]{0.9}
			Vers & Data & Autore & Verificatore & Descrizione \\
			1.0.0 & 13-11 & L. Casagrande & L. Pieripolli & Approvazione documento \\
			\hline
			0.1.0 & 12-11 & L. Casagrande & M. Mazzaretto & Creazione e stesura documento \\
			\hline
	\end{tabular}}
\end{table}
\newpage
\hypersetup{linkcolor=black}

% Indice generale
\tableofcontents

\newpage
% Indice delle tabelle
\listoftables

\newpage
% Indice delle immagini
\listoffigures

%parte variabile
\newpage
\section{Introduzione}
	\subsection{Scopo del documento}
	Il presente documento nasce per descrivere il Way of Working$^G$ adottato da \textbf{\vr{NullPointers Group}} durante lo svolgimento del progetto$^G$ SmartOrder.\\
	Lo standard di riferimento è l'ISO/IEC 12207:1995, il quale prevede tre tipologie di processi.
	\begin{itemize}
	    \setlength\itemsep{-0.1em}
	    \item \textbf{Processi primari:} processi fondamentali senza i quali un progetto$^G$ non può definirsi tale;
	    \item \textbf{Processi di Supporto:} processi che coadiuvano i processi primari nello svolgimento delle rispettive azioni;
	    \item \textbf{Processi organizzativi:} processi di carattere più generale che aiutano la realizzazione del progetto$^G$.
	\end{itemize}

	La stesura di questo documento è incrementale, cioè una stesura passo passo con modifiche, aggiunte e cancellazioni a seguito di miglioramenti del metodo di lavoro. I membri del gruppo si impegnano a visionare costantemente questo documento e a rispettare rigorosamente le regole definite in esso, per svolgere il progetto$^G$ in modo professionale, coerente ed uniforme.

\subsection{Scopo del prodotto}
	La gestione automatizzata degli ordini di acquisto in contesti multicanale rappresenta una sfida complessa per le aziende moderne, che devono affrontare la necessità di interpretare richieste provenienti da fonti eterogenee come email, chat, messaggi vocali e immagini.\\
	Il capitolato$^G$ numero C8 di Ergon Informatica propone di sviluppare una piattaforma intelligente in grado di analizzare input multimodali e convertirli automaticamente in ordini strutturati, pronti per l'inserimento nei sistemi gestionali aziendali.

	L'obiettivo che si è posto questo gruppo è realizzare un sistema basato su architettura a microservizi che integri tecniche avanzate di intelligenza artificiale$^G$, machine learning$^G$ e natural language processing$^G$, in grado di riconoscere le intenzioni del cliente, estrarre le informazioni rilevanti e validarle in maniera coerente con il catalogo prodotti aziendale. Questo approccio consentirà di ridurre drasticamente l'intervento umano nelle fasi ripetitive, migliorando al contempo l'efficienza$^G$ complessiva e la soddisfazione del cliente finale.

	Il progetto$^G$ SmartOrder si propone quindi di dimostrare come le tecnologie di intelligenza artificiale$^G$ possano essere applicate con successo a processi reali di business, trasformando un compito complesso e frammentato in un flusso lineare, automatizzato e scalabile. L'obiettivo è realizzare questo progetto$^G$ entro il 30 Aprile 2026 con un budget a disposizione di: Euro 11.440.
	
\subsection{Glossario}
    La realizzazione di un sistema software complesso come SmartOrder richiede, ancora prima della scrittura del codice, un'importante operazione di confronto, analisi e progettazione$^G$. Per supportare e facilitare il lavoro asincrono tra i membri del gruppo e garantire una comunicazione efficace con il committente$^G$, tutte le informazioni derivanti da questa attività saranno appositamente documentate in un glossario condiviso, utile per evitare ambiguità o incomprensioni riguardanti la nomenclatura adottata in tutti i documenti visionabili.

    In accordo con quanto stabilito nel \href{https://nullpointersgroup.github.io/Documentazione/output/RTB/Verbali\%20Interni/2025-11-06\_verbale\_interno.pdf}{verbale interno del 6 novembre 2025}, si è deciso 
	di adottare il glossario come strumento ufficiale per la standardizzazione della terminologia di progetto$^G$ e di assegnare la responsabilità della sua manutenzione$^G$ alla figura dell'Analista$^G$.

    La nomenclatura utilizzata per segnalare che la definizione di una parola è contenuta nel glossario è la seguente: 
    \begin{center}
        termine$^{G}$
    \end{center}

    I termini sono ordinati alfabeticamente per facilitarne la consultazione e vengono inclusi sia termini tecnici che acronimi significativi.

    Il gruppo si impegna a visionare il Glossario periodicamente, per permettere la più completa comprensione di ogni tipo di documento pubblicato e per mantenere un allineamento semantico costante tra tutti i partecipanti al progetto$^G$. 

    \subsection{Riferimenti}
	\subsubsection{Riferimenti normativi}

	\begin{itemize}[itemsep=5pt, parsep=5pt, label=$\scriptstyle\bullet$]

    \item \textbf{Capitolato$^G$ C8 - Ergon Informatica Srl - SmartOrder}\\
    \url{https://www.math.unipd.it/~tullio/IS-1/2025/Progetto/C8.pdf}\\[3pt]
    \textbf{Ultima consultazione: 30 Novembre 2025}

	\end{itemize}

	\subsubsection{Riferimenti informativi}
	\begin{itemize}[itemsep=5pt, parsep=5pt, label=$\scriptstyle\bullet$]

    \item \textbf{Standard ISO/IEC 9126}\\
     \url{https://en.wikipedia.org/wiki/ISO/IEC\_9126}\\[3pt]
    \textbf{Ultima consultazione: 13 Dicembre 2025}

    \item \textbf{Standard ISO/IEC/IEEE$^G$ 12207:1995}\\
     \url{https://www.math.unipd.it/~tullio/IS-1/2009/Approfondimenti/ISO\_12207-1995.pdf}\\[3pt]
    \textbf{Ultima consultazione: 12 Dicembre 2025}

	\item \textbf{Glossario, versione 1.0.0}\\
     \url{https://nullpointersgroup.github.io/Documentazione/output/RTB/Documenti\%20Interni/Glossario.pdf}\\[3pt]
    \textbf{Ultima consultazione: 13 Dicembre 2025}

	\end{itemize}


\newpage
\section{Processi Primari}
	\subsection{Fornitura}
	
	
	\subsection{Sviluppo}

\newpage
\section{Processi di Supporto}
\subsection{Documentazione}
    Il processo di documentazione è un elemento cardine di tutti i processi primari. Il suo output è fondamentale per tracciare le decisioni$^{G}$ prese e per facilitare il lavoro asincrono, 
    che nel nostro contesto si rivela notevolmente più produttivo di quello sincrono. \\
    Nello specifico, questo processo si occupa di registrare le informazioni generate da ciascuna attività o processo del ciclo di vita del prodotto; comprende quindi tutte le operazioni di pianificazione$^{G}$, 
    progettazione$^{G}$, sviluppo$^{G}$, produzione$^{G}$, modifica$^{G}$, distribuzione$^{G}$ e manutenzione$^{G}$ dei documenti destinati a tutti i soggetti coinvolti.

\subsubsection{Linguaggio di Markup}
    Per la redazione dei documenti il gruppo ha deciso di utilizzare \textbf{LaTeX}$^{G}$ ovvero un linguaggio di markup consolidato e ampiamente utilizzato per la stesura di documenti tecnici e scientifici. LaTeX$^G$ consente 
    di mantenere un’elevata qualità tipografica e di gestire in modo efficiente la struttura del documento.\\
    NullPointers Group utilizza LaTeX$^G$ per la produzione$^G$ di tutta la documentazione, facendo uso di pacchetti e template appositamente sviluppati dai membri del gruppo.

\subsubsection{Versionamento}
    Il gruppo utilizza \textbf{GitHub}$^{G}$ come piattaforma principale per la gestione del versionamento$^{G}$ e della collaborazione nella redazione dei documenti.
    Attraverso il sistema di controllo di versione Git, è possibile mantenere uno storico completo di tutte le modifiche, garantendo tracciabilità, ordine e coordinamento tra i membri del gruppo.

\subsubsection{Caricamento in Repository}
    Ogni volta che si inserisce un nuovo documento o si effettua una modifica$^G$ nel Repository$^{G}$ si utilizza un branch$^G$ feature/* personalizzato.\\
    Una volta creato un commit nel branch$^{G}$ una GitHub$^G$ Action$^{G}$ si occupa di creare automaticamente una pull request$^{G}$, la quale deve essere approvata dai Verificatori$^{G}$.

\subsubsection{Struttura base dei documenti}
\begin{center}
\textbf{Intestazione}
\end{center}
\vspace{-0.5em}
La prima pagina funge da intestazione del documento e contiene i seguenti elementi:
\begin{itemize}[itemsep=0pt, parsep=1pt, label=$\scriptstyle\bullet$]
        \item Logo dell'Università degli Studi di Padova;
        \item Logo NullPointers Group;
        \item \textbf{Nome del documento};
        \item \textbf{Stato}: se il documento è stato Approvato o se è ancora In Approvazione;
        \item \textbf{Versione}: ultima versione verificata o approvata del documento;
        \item \textbf{Data ultima modifica}: ultima data in cui è stato modificato il documento (se ritenuta necessaria);
        \item \textbf{Redattori}: coloro che hanno partecipato alla redazione del documento;
        \item \textbf{Verificatori}: coloro che hanno partecipato alla verifica$^{G}$ di parti del documento (presente in documenti diversi da Verbali);
        \item \textbf{Partecipanti}: coloro che partecipano alla riunione, interna o esterna che sia (presente solo nei Verbali);
        \item \textbf{Destinatari} del documento, ovvero a chi è rivolto.
    \end{itemize}

\begin{center}
\textbf{Registro delle modifiche}
\end{center}
\vspace{-0.5em}
Segue il Registro delle modifiche in forma tabellare che consente la tracciabilità delle modifiche apportate al documento, nel quale viene specificato: 
    \begin{itemize}[itemsep=0pt, parsep=1pt, label=$\scriptstyle\bullet$]
        \item \textbf{Versione}: numero della versione del documento (identificativo unico);
        \item \textbf{Data}: data di approvazione della versione del documento;
        \item \textbf{Autore}: persona che ha apportato modifiche;
        \item \textbf{Verificatore}: persona che ha approvato le modifiche;
        \item \textbf{Descrizione}: breve descrizione delle modifiche apportate al documento.
    \end{itemize}

\begin{center}
\textbf{Contents}
\end{center}
\vspace{-0.5em}
Nella pagina successiva al registro delle modifiche è presente l'indice generale, nel quale vengono elencate tutte le sezioni che compongono il documento.

\begin{center}
\textbf{Indice delle tabelle e delle immagini}
\end{center}
\vspace{-0.5em}
Successivamente all'indice, qualora il documento contenga elementi grafici o tabellari, vengono riportati l'indice delle tabelle e l'indice delle immagini. 
Tali indici descrivono il contenuto di ciascun elemento e ne specificano la collocazione all'interno del documento.

\begin{center}
\textbf{Contenuto principale}
\end{center}
\vspace{-0.5em}
Il contenuto del documento è strutturato in modo gerarchico per organizzare al meglio i contenuti:
\begin{itemize}[itemsep=0pt, parsep=1pt, label=$\scriptstyle\bullet$]
    \item \textbf{Capitoli}: rappresentano le macro-aree tematiche;
    \item \textbf{Sezioni}: suddividono i capitoli in argomenti specifici;
    \item \textbf{Sottosezioni}: se necessarie, approfondiscono i dettagli di ogni argomento.
\end{itemize}
\vspace{0.1em}

\subsubsubsection{Verbali Interni}
Il contenuto principale dei verbali interni segue una struttura standardizzata composta dai seguenti elementi: \vspace{-0.5em}

    \begin{enumerate}[label=\arabic*., itemsep=0pt, parsep=1pt, leftmargin=*]
        \item \textbf{Informazioni generali}
        \begin{itemize}[label=--, itemsep=0pt, parsep=1pt, leftmargin=1.5em]
            \item Tipo di riunione: Interna;
            \item Luogo della riunione: in presenza o sulla piattaforma Discord;
            \item Data della riunione;
            \item Orario di inizio; 
            \item Orario di fine;
            \item Scriba, ovvero il nome di chi si occupa di redigere il Verbale.
        \end{itemize}
        \item \textbf{Ordine del giorno} 
    \end{enumerate}

    \vspace{-0.5em}
    Ha lo scopo di delineare in modo strutturato e puntuale gli argomenti che verranno discussi durante la riunione.

    \begin{enumerate}[resume*, itemsep=0pt, parsep=1pt, leftmargin=*, label=\arabic*.]
        \item \textbf{Diario della riunione}
    \end{enumerate}

    \vspace{-0.5em}
    Documenta in modo sintetico ma completo lo svolgimento della riunione, registrando le discussioni principali, le decisioni prese e le attività concordate.

    \begin{enumerate}[resume*, itemsep=0pt, parsep=1pt, leftmargin=*, label=\arabic*.]
        \item \textbf{Decisioni e Azioni}
    \end{enumerate}

    \vspace{-0.5em}
    Ha lo scopo di registrare in modo formale e strutturato tutte le deliberazioni e i compiti emersi durante l'incontro. La tabella funge da riferimento ufficiale e chiaro per tutto il gruppo, 
    riassumendo cosa è stato stabilito e quali attività devono essere svolte.\\
    Queste informazioni non rimangono confinate al documento, ma vengono integrate e tracciate all'interno del nostro sistema di ticketing. Ciò garantisce che ogni elemento sia documentato per riferimento futuro e attivo per la sua esecuzione, 
    collegando direttamente la decisione$^G$ al task.

\vspace{0.7em}
\subsubsubsection{Verbali Esterni}
I verbali esterni sono documenti che registrano ufficialmente gli incontri avvenuti con soggetti esterni al gruppo di lavoro, in particolare con l'ente proponente del progetto. Tali documenti hanno lo scopo di tracciare le discussioni, 
le decisioni concordate e gli impegni assunti da entrambe le parti durante la riunione.
    \begin{enumerate}[label=\arabic*., itemsep=0pt, parsep=1pt, leftmargin=*]
        \item \textbf{Informazioni generali}
        \begin{itemize}[label=--, itemsep=0pt, parsep=1pt, leftmargin=1.5em]
            \item Tipo di riunione: Esterna;
            \item Luogo della riunione: concordato con il proponente;
            \item Data della riunione;
            \item Orario di inizio; 
            \item Orario di fine;
            \item Scriba, ovvero il nome di chi si occupa di redigere il Verbale;
            \item Partecipanti: ovvero i referenti dell'azienda proponente con i quali viene svolto l'incontro. 
        \end{itemize}
        \item \textbf{Ordine del giorno} 
    \end{enumerate}

    \vspace{-0.5em}
    L'ordine del giorno ha lo scopo di delineare ciò che verrà discusso durante la riunione, con particolare riferimento a chiarimenti dei dubbi emersi e 
    alle domande sollevate dal gruppo in preparazione dell'incontro con l'azienda.

    \begin{enumerate}[resume*, itemsep=0pt, parsep=1pt, leftmargin=*, label=\arabic*.]
        \item \textbf{Diario della riunione}
    \end{enumerate}

    \vspace{-0.5em}
    Documenta in modo sintetico ma completo lo svolgimento dell'incontro, registrando le discussioni principali, le decisioni prese e le attività concordate. Fornisce un resoconto strutturato degli argomenti trattati, mantenendo traccia di tutti gli aspetti rilevanti emersi durante il confronto.

    \begin{enumerate}[resume*, itemsep=0pt, parsep=1pt, leftmargin=*, label=\arabic*.]
        \item \textbf{Approvazione esterna}
    \end{enumerate}

    \vspace{-0.5em}
    È l'ultima sezione del documento che attesta che i relativi verbali esterni siano approvati dalla proponente tramite firma ed eventuale timbro del rappresentante.

\vspace{0.7em}
\subsubsubsection{Diari di Bordo}
    I diari di bordo sono presentazioni utilizzate durante gli incontri settimanali con l'obiettivo di verificare in modo condiviso lo stato di avanzamento di ciascun gruppo ammesso al I lotto.\\
    La struttura tipica di un diario di bordo comprende le seguenti sezioni:
    \begin{itemize}[itemsep=0pt, parsep=1pt, label=$\scriptstyle\bullet$]
        \item \textbf{Risultati}: descrive le attività completate nel periodo corrente e le confronta con quanto inizialmente pianificato;
        \item \textbf{Problematiche riscontrate}: consente di illustrare le problematiche affrontate e i dubbi ancora irrisolti, sono volti alla richiesta di supporto o chiarimenti;
        \item \textbf{Attività future}: elenca i compiti da svolgere nel prossimo intervallo di lavoro.
    \end{itemize}

\vspace{0.3em}
\subsubsubsection{Altri documenti}
Di seguito sono elencati tutti i documenti redatti e mantenuti durante l'intero ciclo di vita del progetto ciascuno dei quali risponde a uno scopo specifico, contribuendo alla tracciabilità, alla gestione e alla comunicazione delle attività di progetto 
La struttura iniziale di tali documenti corrisponde con quanto definito al punto 3.1.4:
    \begin{itemize}[itemsep=0pt, parsep=1pt, label=$\scriptstyle\bullet$]
        \item \textbf{Valutazione dei capitolati};
        \item \textbf{Dichiarazione degli impegni};
        \item \textbf{Lettera di presentazione};
        \item \textbf{Norme di progetto};
        \item \textbf{Analisi dei requisiti};
        \item \textbf{Piano di progetto};
        \item \textbf{Piano di qualifica};
        \item \textbf{Glossario};
        \item \textbf{Specifica Tecnica};
        \item \textbf{Manuale Utente}.
    \end{itemize}


\subsection{Gestione della Configurazione}

\subsection{Gestione della Qualità}

\subsection{Verifica}

\subsection{Validazione}

\newpage
\section{Processi Organizzativi}
I processi organizzativi definiscono un insieme di operazioni di supporto per lo sviluppo$^G$ software che operano trasversalmente rispetto al ciclo di vita del software garantendo che il gruppo possieda l’organizzazione, le infrastrutture e le competenze necessarie per sostenere i processi primari.\\
Assicurano la buona esecuzione di tutti i processi adottati e eventuali miglioramenti.\\
Si individuano i seguenti processi:
\begin{itemize}[itemsep=3pt, parsep=1pt, label=$\scriptstyle\bullet$]
    \item Gestione dei Processi;
    \item Gestione dell'Infrastruttura;
    \item Processo$^G$ di Miglioramento;
    \item Processo$^G$ di Formazione.
\end{itemize}

\subsection{Gestione dei Processi}
Secondo lo standard ISO$^G$ 12207:1995, \vr{La gestione dei processi comprende le attività e i compiti che possono essere svolti da qualsiasi soggetto che debba gestire i propri processi}.\\
Sulla base di questo principio, il suo scopo principale è stabilire come un processo$^G$ deve essere pianificato e monitorato secondo le relative responsabilità dei membri del gruppo.\\
Un altro obiettivo fondamentale è garantire un flusso comunicativo efficace, sia interno che esterno assicurando: coerenza, controllo e miglioramento continuo.
 \subsubsection{Attività previste}
 Per assicurare il raggiungimento degli obiettivi nel rispetto di tempi e qualità, il processo$^G$ è strutturato nelle seguenti attività, che definiscono un flusso di lavoro chiaro e responsabilità precise:
    \subsubsubsection{Inizializzazione}
    L’avvio del processo$^G$ avviene tramite la selezione dei Requisiti, presenti nel documento \vr{Analisi dei Requisiti$^G$}, da portare a termine tramite le attività di tale processo$^G$.\\
    Il Responsabile$^G$ valuta preliminarmente la fattibilità del processo$^G$: se alcuni requisiti risultano irrealizzabili per vincoli di tempo, risorse o competenze, e previo accordo di tutto il gruppo, i requisiti del processo$^G$ possono essere modificati in questa fase per garantire il raggiungimento dei criteri di completamento.
    \subsubsubsection{Pianificazione}
    L’attività di pianificazione$^G$, portata a termine dal Responsabile$^G$, ha lo scopo di preparare il piano di esecuzione delle attività del processo$^G$. 
    In particolare deve verificare la disponibilità delle risorse necessarie (budget residuo, della disponibilità dei componenti del gruppo, competenza, etc.) per completare il processo$^G$ entro i tempi stabiliti.\\
    Infine assegna le attività del processo$^G$ ai membri del team in base ai loro ruoli.
    \subsubsubsection{Esecuzione e controllo}
    Durante l’esecuzione: i membri del team portano a termine le attività assegnategli, mentre il Responsabile$^G$ ha il compito di monitorare l’andamento delle attività.\\
    Qualora si presentassero problemi, il Responsabile$^G$ deve essere immediatamente notificato e in caso di stallo, contattare la proponente$^G$ o il committente$^G$ per un chiarimento.
    \subsubsubsection{Verifica}
    La Verifica$^G$ del prodotto realizzato dai membri del team è di competenza del Verificatore$^G$, che assicura la corretta realizzazione del singolo requisito tramite la procedura di Verifica$^G$. 
    \subsubsubsection{Chiusura}
    La chiusura del processo$^G$ consegue la terminazione di tutte le attività che ne hanno preso parte.\\
    È compito del Responsabile$^G$ o del Verificatore$^G$ approvare il merge$^G$ della Pull Request$^G$ nel branch$^G$ main; una volta fatto il merge$^G$, il processo$^G$ è da definirsi chiuso. %da confermare se è il responsabile$^G$ 

\subsubsection{Tracciamento delle ore}
Per monitorare il tempo dedicato ai diversi ruoli durante il progetto$^G$, viene utilizzato uno spreadsheet dedicato, accessibile al gruppo su Google Drive.\\
Al termine di ogni sprint$^G$, viene generato il consuntivo$^G$ dello sprint$^G$ appena concluso e il preventivo di quello successivo. \\
Il membro che ricopre il ruolo di Responsabile$^G$ ha il compito di inserire tali informazioni nel documento \vr{Piano di Progetto$^G$}.
\subsubsubsection{Preventivo}
Il preventivo delle ore è presentato in forma tabellare e riporta le ore stimate per ciascun membro del gruppo, suddivise per ruolo e riferite al singolo sprint$^G$.\\
Viene inoltre creato un diagramma circolare che illustra la distribuzione$^G$ delle ore previste per ogni ruolo, fornendo una rappresentazione immediata e intuitiva delle risorse che si prevede di allocare per quello sprint$^G$.
\subsubsubsection{Consuntivo}
Il consuntivo$^G$ delle ore è anch’esso presentato tramite una tabella che riporta le ore effettivamente registrate da ciascun membro, suddivise per ruolo e relative allo sprint$^G$.\\ 
Anche in questo caso viene incluso un diagramma circolare che visualizza la ripartizione delle ore effettive per ruolo, fornendo una visione immediata delle risorse realmente impiegate durante lo sprint$^G$.

\subsubsection{Ruoli}
\begin{table}[H]
\centering
\renewcommand{\arraystretch}{1.15}
\begin{tabular}{ |p{0.25\textwidth}|p{0.65\textwidth}|}
    \hline
    \textbf{Ruolo} & \textbf{Compiti} \\
    \hline
    Responsabile$^G$ & Il Responsabile$^G$ coordina le attività del gruppo garantendo una pianificazione$^G$ efficace. \\
    \hline
    Amministratore$^G$ & L’Amministratore$^G$ si occupa della configurazione e gestione dell’infrastruttura IT di supporto al progetto$^G$. \\
    \hline
    Analista$^G$ & L’Analista$^G$ si occupa di identificare e chiarire i requisiti, interpretando le esigenze degli utilizzatori finali per garantire una corretta definizione delle funzionalità$^G$. \\
    \hline
    Verificatore$^G$ & Il Verificatore$^G$ si occupa di assicurare la qualità dei prodotti e dei processi adottati, effettuando revisioni e test$^G$. \\
    \hline
    Programmatore$^G$ & Il Programmatore$^G$ è responsabile$^G$ dello sviluppo$^G$ del codice sorgente del progetto$^G$, traducendo il design in codice funzionante e testabile dal proponente$^G$. \\
    \hline
    Progettista$^G$ & Il Progettista$^G$ traduce i requisiti del sistema in un’architettura software dettagliata, definendo moduli, interfacce, flussi dati e vincoli tecnici. \\
    \hline
\end{tabular}
\caption{Ruoli}
\end{table}

\subsubsubsection{Rotazione dei ruoli}
La rotazione dei ruoli avviene ogni due settimane, in modo da garantire una distribuzione$^G$ equilibrata delle competenze e favorire l’apprendimento trasversale all’interno del gruppo.\\
Ci riserviamo tuttavia la possibilità di effettuare modifiche ai ruoli anche a metà sprint$^G$, qualora necessario.\\
Tale decisione$^G$ può derivare dall’esito dell’incontro intermedio, durante il quale viene valutato l’avanzamento delle attività e individuati eventuali ritardi, blocchi o membri sottoutilizzati.\\
In questi casi, l’assegnazione di nuovi incarichi o la riorganizzazione dei ruoli permette una migliore suddivisione dei task$^G$ in unità più piccole e gestibili, favorendo il completamento efficace delle attività pianificate.

\subsubsection{Coordinamento}
Il coordinamento è un fattore fondamentale per il buon andamento del progetto$^G$, perché permette di gestire in modo efficace le attività del team e le relazioni con la proponente$^G$ e i committenti.\\
Per assicurare un flusso informativo continuo e aggiornato, sono pianificate riunioni regolari e l’impiego di canali di comunicazione adeguati.

\subsubsubsection{Riunioni}
Al fine di garantire un flusso di comunicazione efficace, un costante allineamento del gruppo e un confronto attivo con la proponente$^G$, sono previsti due tipi distinti di incontri:
\subsubsubsubsection{Riunioni interne}
Le riunioni interne vengono organizzate regolarmente, in genere ogni lunedì della settimana.\\
Durante questi incontri il gruppo effettua un punto della situazione: vengono analizzate le attività svolte, quelle ancora in corso e le eventuali criticità emerse.\\
Il Responsabile$^G$, grazie al confronto tra i membri, può così ottenere una visione aggiornata dell’avanzamento del progetto$^G$ e pianificare al meglio le attività successive.
\subsubsubsubsection{Riunioni esterne}
Le riunioni esterne coinvolgono i membri del gruppo e la proponente$^G$.\\
Tali incontri non seguono una periodicità fissa: vengono programmati secondo necessità tramite richiesta via canale concordato con la proponente$^G$ (come specificato al punto \ref{sec:strumenti-supporto}).\\
Durante le riunioni esterne il gruppo presenta lo stato di avanzamento del lavoro, chiarisce eventuali dubbi e riceve indicazioni utili per proseguire nelle attività di sviluppo$^G$.\\
Al termine di ogni riunione, interna o esterna, viene redatto un verbale (specifiche a sezioni \ref{sec:verbali-interni} e \ref{sec:verbali-esterni}) che documenta gli argomenti trattati e le decisioni prese, garantendo tracciabilità e condivisione delle informazioni.

\subsubsubsection{Comunicazioni}
Le comunicazioni costituiscono un elemento fondamentale per garantire coordinamento, continuità informativa e tempestività nelle attività del gruppo. Gli strumenti utilizzati per la comunicazione sono descritti nella sezione \ref{sec:strumenti-supporto}; in questa sezione viene invece specificato come tali strumenti vengono impiegati all’interno dei processi organizzativi.
\subsubsubsubsection{Comunicazioni interne}
Le comunicazioni interne hanno lo scopo di supportare il lavoro quotidiano del team, favorire la condivisione di informazioni e consentire un rapido confronto operativo.
\begin{itemize}[itemsep=3pt, parsep=1pt, label=$\scriptstyle\bullet$]
    \item Le comunicazioni rapide o di servizio avvengono mediante il canale di messaggistica adottato dal gruppo (WhatsApp$^G$);
    \item Le discussioni che richiedono un confronto più approfondito o decisioni condivise vengono svolte tramite incontri a distanza (Discord$^G$);
    \item La documentazione dei task$^G$, degli avanzamenti e delle attività pianificate è gestita attraverso lo strumento di versionamento$^G$ e ticketing adottato dal team (GitHub$^G$).
\end{itemize}
Questa organizzazione consente di distinguere in modo chiaro tra comunicazioni informali, operative e momenti di coordinamento strutturato.
\subsubsubsubsection{Comunicazioni esterne}
Le comunicazioni esterne garantiscono il mantenimento di un flusso regolare e trasparente con la proponente$^G$ e con i committenti.
\begin{itemize}[itemsep=3pt, parsep=1pt, label=$\scriptstyle\bullet$]
    \item Le comunicazioni formali, come approvazioni di verbali esterni, avvengono tramite l’indirizzo e-mail istituzionale del gruppo (Google Mail);
    \item Le riunioni esterne sono organizzate attraverso la piattaforma concordata per gli incontri a distanza (Google Meet$^G$);
    \item Per eventuali comunicazioni asincrone o rapide è previsto anche un canale condiviso con la proponente$^G$ (WhatsApp$^G$), come stabilito in fase iniziale.
\end{itemize}
Tutte le comunicazioni esterne rilevanti ai fini del progetto$^G$ vengono tracciate mediante la produzione$^G$ dei verbali esterni.

\subsection{Gestione dell'Infrastruttura}
Il processo$^G$ di infrastruttura ha lo scopo di fornire, configurare e mantenere l’ambiente di lavoro necessario all'esecuzione di tutti i processi di sviluppo$^G$ e documentazione.\\
Esso comprende la gestione delle risorse, siano esse hardware o software, garantendone la disponibilità e l'efficienza$^G$ per l'intera durata del progetto$^G$.
\subsubsection{Attività previste}
\subsubsubsection{Implementazione}
Per supportare il lavoro asincrono, la tracciabilità e la qualità dei prodotti NullPointers Group adotta i seguenti strumenti che costituiscono l’infrastruttura del progetto$^G$:
\begin{itemize}[itemsep=3pt, parsep=1pt, label=$\scriptstyle\bullet$]
    \item Gestione del versionamento$^G$: Git$^G$
    \item Piattaforma: GitHub$^G$
    \item Automazione$^G$: GitHub$^G$ Action$^G$, script Python$^G$ e Lua$^G$
    \item Comunicazione: Discord$^G$ e Whatsapp$^G$
\end{itemize}

\subsubsubsection{Predisposizione}
L’attività di predisposizione stabilisce le regole di interazione tra i membri del gruppo e l’ambiente di lavoro, inoltre definisce la natura dell’infrastruttura utilizzata.\\
L’infrastruttura adottata è finalizzata a minimizzare gli errori e a garantire la coerenza del prodotto.\\
Vengono riportati gli strumenti principali:

\begin{table}[!h]
\centering
\renewcommand{\arraystretch}{1.15}
\begin{tabular}{|p{0.23\textwidth}|p{0.7\textwidth}|}
    \hline
    \textbf{Strumento} & \textbf{Predisposizione} \\
    \hline
    Git$^G$ & Definizione di un file .gitignore condiviso per escludere i file temporanei e di build$^G$ garantendo che la repository$^G$ contenga solamente i file sorgente. \\
    \hline
    GitHub$^G$ & È stata creata una repository$^G$ dedicata alla documentazione del progetto$^G$. È stata applicata una  branch$^G$ protection rule sul ramo main: ogni modifica$^G$ deve provenire da una pull request$^G$ e richiede l’approvazione di un Verificatore$^G$ per il suo merge$^G$.\\
    \hline
    Discord$^G$ e \newline Whatsapp$^G$ & Per consentire al gruppo di riunirsi settimanalmente, e venire incontro al fatto che ci sono significative distanze tra i membri, è stato creato un server sulla piattaforma Discord$^G$, un’applicazione che consente videochiamate e scambio di messaggi; ideale per il nostro scopo. È stato inoltre creato un gruppo Whatsapp$^G$ per questioni minori che non richiedono una videochiamata. \\
    \hline
    Labels GitHub$^G$ & Sono state implementate delle Labels per categorizzare le attività e Milestones per tracciare l’avanzamento del progetto$^G$.\\
    \hline
    GitHub$^G$ Action$^G$, script Python$^G$ e script Lua$^G$ & Sono state configurate le GitHub$^G$ Action$^G$ per l’esecuzione automatica degli script che compilano i file sorgente LaTeX$^G$ ad ogni push garantendo che la versione PDF visibile sul sito sia sempre sincronizzata con l’ultima versione dei documenti. \\
    \hline
\end{tabular}
\caption{Strumenti di predisposizione}
\end{table}

\newpage
\subsubsubsection{Manutenzione}
Data la complessità del progetto$^G$ è probabile che l’infrastruttura subisca dei cambiamenti nel corso del tempo per l’aggiornamento o il miglioramento delle sue funzionalità$^G$.\\
È compito dell’Amministratore$^G$ la manutenzione$^G$ dell’infrastruttura ovvero le attività di controllo delle funzionalità$^G$ ed aggiornamento/creazione degli script di automazione$^G$.\\
Successivamente verranno illustrate le norme da seguire per mantenere e aggiornare l’infrastruttura affinché il flusso di lavoro non venga spezzato.

\subsubsubsubsection{Git}
Git$^G$ non ha bisogno di particolari configurazioni, è sufficiente accedere localmente con le credenziali che il membro usa per accedere a GitHub$^G$.

\subsubsubsubsection{GitHub}
Su GitHub$^G$, l’account di NullPointers Group è gestito come organizzazione, ovvero un account che serve da contenitore per il lavoro condiviso tra membri di un team.\\
Sono state create 3 repository$^G$ dentro l’organizzazione:
\begin{itemize}[itemsep=3pt, parsep=1pt, label=$\scriptstyle\bullet$]
    \item \textbf{Documentazione}: repository$^G$ dove viene salvata e versionata tutta la documentazione in merito al capitolato$^G$ SmartOrder e non solo.
    \item \textbf{PoC$^G$}: repository$^G$ dove viene salvato e versionato il codice sorgente per il Proof of Concept$^G$.
    \item \textbf{SmartOrder}: repository$^G$ dove viene salvato e versionato il codice sorgente dell’applicativo SmartOrder.
    \item \textbf{.github$^G$}: repository$^G$ che serve a creare il README.md del gruppo, gestire i workflow condivisi fra le varie repository$^G$. In generale, serve a centralizzare configurazioni e contenuti condivisi.
\end{itemize}
Affinché le nuove impostazioni vengano effettivamente applicate nelle repository$^G$ di documentazione o di codice sorgente, il verificatore$^G$ dovrà assicurarsi che la nuova infrastruttura proposta superi le metriche di qualità di processo$^G$ definite in seguito.

\subsubsubsubsection{Action e script ausiliari}
Nelle repository$^G$ di Documentazione e SmartOrder, si impiegano strumenti di CI messi a disposizione da GitHub$^G$: le GitHub$^G$ Action$^G$, le quali vengono definite in un file “.yml” dentro la cartella “.github/workflows”.\\
Affinché le GitHub$^G$ Action$^G$ potessero assolvere allo scopo per cui sono state configurate, sono stati sviluppati script ausiliari, in Python$^G$ e Lua$^G$, da far eseguire a quest'ultime.\\
La configurazione di nuove GitHub$^G$ o la modifica$^G$ di GitHub$^G$ Action$^G$ esistenti spetta agli Amministratori.\\
La creazione di nuove GitHub$^G$ Action$^G$ viene richiesta agli Amministratori dal Responsabile$^G$, sotto comune accordo dai membri del gruppo.\\

\subsubsubsubsection{Infrastruttura di Tracking}
La configurazione e la manutenzione$^G$ ordinaria degli strumenti di GitHub$^G$ (la cui organizzazione è descritta nella sezione \ref{sec:ticketing}) seguono le procedure definite successivamente.\\[1em]
\textbf{Sistema di Label}\\
Viene definita una palette standard di label e le convenzioni di nomenclatura. Periodicamente si esegue una pulizia per rimuovere quelle obsolete o unificarne di duplicate.\\
Qualsiasi modifica$^G$ allo standard viene discussa e approvata dal gruppo.\\[1em]
\textbf{Project Board}\\
Le board per le milestone$^G$ (RTB$^G$, PB$^G$) vengono create da un template predefinito. Al termine di una milestone$^G$, le board vengono archiviate dopo aver chiuso le issue$^G$ residue.\\
Mensilmente si verifica$^G$ che la posizione delle issue$^G$ nelle colonne corrisponda al loro stato effettivo.\\[1em]
\textbf{Issue$^G$ e Automazioni}\\
Si mantengono aggiornati i template predefiniti per la creazione di nuove issue$^G$. Vengono identificate e chiuse periodicamente le issue$^G$ inattive o completate. \\

\subsubsubsubsection{Discord}
Discord$^G$ è la piattaforma principale per la comunicazione interna del gruppo.\\
Il server è organizzato in canali testuali dedicati a specifiche aree, come la documentazione e il codice sorgente, per mantenere le discussioni focalizzate.\\
La creazione di nuovi canali avviene su richiesta di qualsiasi membro, ma è riservata all'Amministratore$^G$ del server, che provvede a configurarli, garantire l'accesso a tutti e mantenere la moderazione.

\subsection{Processo di Miglioramento}
Il miglioramento è un processo$^G$ finalizzato a valutare, misurare, controllare e ottimizzare il ciclo di vita del software.\\
Il suo obiettivo è garantire che il prodotto non solo risponda alle aspettative, ma raggiunga e mantenga standard elevati di qualità ed efficienza$^G$, attraverso revisioni periodiche e perfezionamenti incrementali.\\
L'approccio è ciclico e, secondo il modello Plan-Do-Check-Act (PDCA), si compone di quattro fasi principali, applicate periodicamente per garantire un progresso continuo.

\subsubsection{Ciclo PDCA}
Il miglioramento continuo si articola nelle seguenti quattro fasi, iterabili su base periodica o al verificarsi di eventi significativi:
\begin{itemize}[itemsep=3pt, parsep=1pt, label=$\scriptstyle\bullet$]
    \item \textbf{Plan} (Pianificazione$^G$): identificazione e documentazione dei processi organizzativi da applicare al ciclo di vita del software.
    \item \textbf{Do} (Implementazione): integrazione dei miglioramenti nei processi esistenti, con aggiornamento della documentazione e raccolta di dati storici e tecnici.
    \item \textbf{Check} (Valutazione): analisi dei processi rispetto agli obiettivi prefissati e alle metriche adottate, mediante i dati raccolti e revisioni periodiche.
    \item \textbf{Act} (Standardizzazione): consolidamento dei miglioramenti efficaci nei processi, per evitarne la regressione e assicurarne l'evoluzione costante.
\end{itemize}

\subsection{Processo di Formazione}
Il processo$^G$ di formazione è un'attività di supporto volta a garantire che tutti i membri del gruppo possiedano le competenze necessarie per svolgere i compiti assegnati e gestire le tecnologie richieste dal progetto$^G$.

\subsubsection{Attività}
La formazione del team si articola in un percorso strutturato che comprende:
\begin{itemize}[itemsep=3pt, parsep=1pt, label=$\scriptstyle\bullet$]
    \item \textbf{Analisi dei bisogni}: identificare le competenze richieste dal progetto$^G$ e le eventuali lacune del gruppo, partendo dai requisiti della proponente$^G$ e dalle tecnologie scelte.
    \item \textbf{Formazione con la Proponente$^G$}: partecipazione a incontri tecnici dedicati, organizzati dalla proponente$^G$, sulle tecnologie cardine del progetto$^G$ (Docker$^G$, FastAPI, LangChain, React$^G$). Queste sessioni accelerano la curva di apprendimento e allineano il gruppo con gli standard tecnici e le metodologie di sviluppo$^G$ adottate dall'azienda.
    \item \textbf{Apprendimento individuale}: studio autonomo, da parte di ciascun membro, degli strumenti, linguaggi e framework$^G$ necessari per il proprio ruolo e per le attività comuni.
    \item \textbf{Condivisione delle conoscenze} (Knowledge Sharing): diffusione interna delle competenze, facilitata dai membri più esperti in un dato ambito attraverso discussioni tecniche, sessioni di pair programming o documentazione informale.
    \item \textbf{Consolidamento e uniformità}: l'obiettivo è garantire un livello di preparazione omogeneo e adeguato agli obiettivi di progetto$^G$, promuovendo la crescita professionale di tutti i componenti.
\end{itemize}

\subsubsection{Implementazione del processo}
In base a quanto previsto dallo standard ISO/IEC 12207:1995, è necessario realizzare una revisione dei requisiti del progetto$^G$.\\
Questo passaggio serve a comprendere le competenze che i membri del NullPointers Group dovranno sviluppare per completare il progetto$^G$ didattico.\\
Come gruppo abbiamo quindi definito quali sono le tecnologie necessarie da approfondire ed utilizzare per la corretta realizzazione del progetto$^G$.\\
Queste tecnologie sono:
\begin{enumerate}
    \item \textbf{React$^G$}: framework$^G$ JavaScript/TypeScript per la realizzazione dell'interfaccia web.
    \item \textbf{Python$^G$}: linguaggio di programmazione versatile, utilizzato per implementare API$^G$ REST$^G$, gestire le comunicazioni con modelli di machine learning$^G$ e con il database$^G$, e realizzare il back-end.
    \item \textbf{PostgreSQL$^G$}: sistema di gestione di database$^G$ relazionale open source, utilizzato per memorizzare e gestire i dati dell’applicazione.
    \item \textbf{Docker$^G$}: piattaforma per la containerizzazione, che consente di distribuire e eseguire l’applicazione in ambienti isolati e replicabili.
    \item \textbf{LaTeX$^G$}: linguaggio di markup utilizzato per produrre la documentazione
\end{enumerate}

%da continuare...

\end{document}
