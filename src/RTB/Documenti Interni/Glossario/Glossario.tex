% Configurazione
\documentclass{article}

\usepackage{titling} % Required for inserting the subtitle
\usepackage{graphicx} % Required for inserting images
\usepackage{tabularx} % Per l'ambiente tabularx (tabelle)
\usepackage{calc} % Sempre per le tabelle
\usepackage{tocloft}
\renewcommand{\cftsecleader}{\cftdotfill{\cftdotsep}}
\usepackage[colorlinks=true,
linkcolor=black,
urlcolor=blue,
citecolor=blue]{hyperref}
% Per i collegamenti ipertestuali, ad esempio sulla table of contents
\usepackage{xcolor} % Per colorare il testo
\usepackage{colortbl} % Per colorare le celle delle tabelle
\usepackage{lipsum} % Per generare lorem ipsum
\usepackage[normalem]{ulem} % Per sottolineare il testo
\usepackage{array} % Per la visualizzazione fluttuante di array di domande e risposte
\usepackage{ragged2e} % Pacchetto necessario per \justifying che giustifica il testo di tabelle

\newcommand{\ulhref}[2]{\href{#1}{\uline{#2}}} % Nuovo comando per sottolineare i link
\newcommand{\ulref}[1]{\uline{\ref{#1}}} % Nuovo comando per sottolineare i collegamenti a immagini
\setlength{\parindent}{0pt} % Rimuove il rientro automatico dei paragrafi

\graphicspath{ {immagini/} {../../../shared/images/} }

\usepackage{fancyhdr}
\usepackage{lastpage}

% --- Impostazioni intestazione e piè di pagina ---
\pagestyle{fancy}
\fancyhf{}
% Intestazione standard
\fancyhead[L]{NullPointers Group}
\fancyhead[R]{Norme di Progetto}
\renewcommand{\headrulewidth}{0.4pt}
% Piè di pagina
\fancyfoot[C]{Pagina \thepage{} di \pageref{LastPage}}
\renewcommand{\footrulewidth}{0.4pt}

% --- Ridefinizione dello stile plain per gli indici ---
\fancypagestyle{plain}{
  \fancyhf{}
  \fancyhead[L]{NullPointers Group}
  \fancyhead[R]{Norme di Progetto}
  \fancyfoot[C]{Pagina \thepage{} di \pageref{LastPage}}
  \renewcommand{\headrulewidth}{0.4pt}
  \renewcommand{\footrulewidth}{0.4pt}
}

% --- Solo piè di pagina per la prima pagina ---
\fancypagestyle{firstpage}{
  \fancyhf{}
  \fancyfoot[C]{Pagina \thepage{} di \pageref{LastPage}}
  \renewcommand{\headrulewidth}{0pt}
  \renewcommand{\footrulewidth}{0.4pt}
}

%struttura
\begin{document}
\thispagestyle{firstpage} % Copertina con solo piè di pagina
% Parte costante
\begin{minipage}{0.4\textwidth}
    \includegraphics[width=0.6\textwidth]{logo_unipd.png}
\end{minipage}
\begin{minipage}{0.55\textwidth}
    \textcolor{red}{\textbf{Università degli Studi di Padova}} \\
    \textcolor{red}{Laurea: Informatica} \\
    \textcolor{red}{Corso: Ingegneria del Software} \\
    \textcolor{red}{Anno Accademico: 2025/2026}
\end{minipage}

\begin{minipage}{0.4\textwidth}
    \includegraphics[width=0.6\textwidth]{logo_gruppo.jpeg}
\end{minipage}
\begin{minipage}{0.55\textwidth}
    \textbf{Gruppo: NullPointers Group} \\
    Email: \textsf{groupnullpointers@gmail.com}
\end{minipage}

\vspace{2cm}
{
		\centering
		\Huge\bfseries Norme di Progetto\par
		\vspace{1.0cm}
		\Large\bfseries \par
	}
\begin{center}
\begin{tabular}{r|l}
    Stato & In Approvazione \\[0.5em]
    Versione & 0.2.0 \\[0.5em]
    Data ultima modifica & 15/11/2025 \\[0.5em]
    Redattori & Lisa Casagrande \\
              & Matteo Mazzaretto \\[0.5em]
    Verificatori & Matteo Mazzaretto \\
                 & Tommaso Ceron \\[0.5em]
    Destinatari & Prof. Tullio Vardanega \\
    & Prof. Riccardo Cardin \\
    & Ergon Informatica Srl \\
    & NullPointers Group \\
\end{tabular}
\end{center}
\newpage
% variabile, prima dell'indice
\section*{Registro delle modifiche}

\begin{table}[h]
	\centering
	\resizebox{\textwidth}{!}{
		\begin{tabular}{|c|c|c|c|c|}
			\hline
			\rowcolor[gray]{0.9}
			Vers & Data & Autore & Verificatore & Descrizione \\
			1.0.0 & 13-11 & L. Casagrande & L. Pieripolli & Approvazione documento \\
			\hline
			0.1.0 & 12-11 & L. Casagrande & M. Mazzaretto & Creazione e stesura documento \\
			\hline
	\end{tabular}}
\end{table}
\newpage
\hypersetup{linkcolor=black}
\begin{center}
\renewcommand{\contentsname}{Indice}
\tableofcontents
\end{center}

\newpage

\hypersetup{linkcolor=blue}
% Parte variabile
\section{Introduzione}
	\subsection{Scopo del documento}
	Il presente documento nasce per descrivere il Way of Working$^G$ adottato da \textbf{\vr{NullPointers Group}} durante lo svolgimento del progetto$^G$ SmartOrder.\\
	Lo standard di riferimento è l'ISO/IEC 12207:1995, il quale prevede tre tipologie di processi.
	\begin{itemize}
	    \setlength\itemsep{-0.1em}
	    \item \textbf{Processi primari:} processi fondamentali senza i quali un progetto$^G$ non può definirsi tale;
	    \item \textbf{Processi di Supporto:} processi che coadiuvano i processi primari nello svolgimento delle rispettive azioni;
	    \item \textbf{Processi organizzativi:} processi di carattere più generale che aiutano la realizzazione del progetto$^G$.
	\end{itemize}

	La stesura di questo documento è incrementale, cioè una stesura passo passo con modifiche, aggiunte e cancellazioni a seguito di miglioramenti del metodo di lavoro. I membri del gruppo si impegnano a visionare costantemente questo documento e a rispettare rigorosamente le regole definite in esso, per svolgere il progetto$^G$ in modo professionale, coerente ed uniforme.

\subsection{Scopo del prodotto}
	La gestione automatizzata degli ordini di acquisto in contesti multicanale rappresenta una sfida complessa per le aziende moderne, che devono affrontare la necessità di interpretare richieste provenienti da fonti eterogenee come email, chat, messaggi vocali e immagini.\\
	Il capitolato$^G$ numero C8 di Ergon Informatica propone di sviluppare una piattaforma intelligente in grado di analizzare input multimodali e convertirli automaticamente in ordini strutturati, pronti per l'inserimento nei sistemi gestionali aziendali.

	L'obiettivo che si è posto questo gruppo è realizzare un sistema basato su architettura a microservizi che integri tecniche avanzate di intelligenza artificiale$^G$, machine learning$^G$ e natural language processing$^G$, in grado di riconoscere le intenzioni del cliente, estrarre le informazioni rilevanti e validarle in maniera coerente con il catalogo prodotti aziendale. Questo approccio consentirà di ridurre drasticamente l'intervento umano nelle fasi ripetitive, migliorando al contempo l'efficienza$^G$ complessiva e la soddisfazione del cliente finale.

	Il progetto$^G$ SmartOrder si propone quindi di dimostrare come le tecnologie di intelligenza artificiale$^G$ possano essere applicate con successo a processi reali di business, trasformando un compito complesso e frammentato in un flusso lineare, automatizzato e scalabile. L'obiettivo è realizzare questo progetto$^G$ entro il 30 Aprile 2026 con un budget a disposizione di: Euro 11.440.
	
\subsection{Glossario}
    La realizzazione di un sistema software complesso come SmartOrder richiede, ancora prima della scrittura del codice, un'importante operazione di confronto, analisi e progettazione$^G$. Per supportare e facilitare il lavoro asincrono tra i membri del gruppo e garantire una comunicazione efficace con il committente$^G$, tutte le informazioni derivanti da questa attività saranno appositamente documentate in un glossario condiviso, utile per evitare ambiguità o incomprensioni riguardanti la nomenclatura adottata in tutti i documenti visionabili.

    In accordo con quanto stabilito nel \href{https://nullpointersgroup.github.io/Documentazione/output/RTB/Verbali\%20Interni/2025-11-06\_verbale\_interno.pdf}{verbale interno del 6 novembre 2025}, si è deciso 
	di adottare il glossario come strumento ufficiale per la standardizzazione della terminologia di progetto$^G$ e di assegnare la responsabilità della sua manutenzione$^G$ alla figura dell'Analista$^G$.

    La nomenclatura utilizzata per segnalare che la definizione di una parola è contenuta nel glossario è la seguente: 
    \begin{center}
        termine$^{G}$
    \end{center}

    I termini sono ordinati alfabeticamente per facilitarne la consultazione e vengono inclusi sia termini tecnici che acronimi significativi.

    Il gruppo si impegna a visionare il Glossario periodicamente, per permettere la più completa comprensione di ogni tipo di documento pubblicato e per mantenere un allineamento semantico costante tra tutti i partecipanti al progetto$^G$. 

    \subsection{Riferimenti}
	\subsubsection{Riferimenti normativi}

	\begin{itemize}[itemsep=5pt, parsep=5pt, label=$\scriptstyle\bullet$]

    \item \textbf{Capitolato$^G$ C8 - Ergon Informatica Srl - SmartOrder}\\
    \url{https://www.math.unipd.it/~tullio/IS-1/2025/Progetto/C8.pdf}\\[3pt]
    \textbf{Ultima consultazione: 30 Novembre 2025}

	\end{itemize}

	\subsubsection{Riferimenti informativi}
	\begin{itemize}[itemsep=5pt, parsep=5pt, label=$\scriptstyle\bullet$]

    \item \textbf{Standard ISO/IEC 9126}\\
     \url{https://en.wikipedia.org/wiki/ISO/IEC\_9126}\\[3pt]
    \textbf{Ultima consultazione: 13 Dicembre 2025}

    \item \textbf{Standard ISO/IEC/IEEE$^G$ 12207:1995}\\
     \url{https://www.math.unipd.it/~tullio/IS-1/2009/Approfondimenti/ISO\_12207-1995.pdf}\\[3pt]
    \textbf{Ultima consultazione: 12 Dicembre 2025}

	\item \textbf{Glossario, versione 1.0.0}\\
     \url{https://nullpointersgroup.github.io/Documentazione/output/RTB/Documenti\%20Interni/Glossario.pdf}\\[3pt]
    \textbf{Ultima consultazione: 13 Dicembre 2025}

	\end{itemize}

\newpage
\begin{center}\section*{A}\end{center}
\addcontentsline{toc}{section}{A}

% Stile per i termini del glossario
\newcommand{\term}[1]{%
  \par % chiude il paragrafo precedente
  \vspace{1.2em} % spazio prima del termine
  \begin{center}
    \textbf{\large #1}
  \end{center}
  \vspace{0.4em} % piccolo spazio dopo il termine
  \noindent % inizio della definizione senza rientro
}


\term{Amministratore}
	L'Amministratore si occupa della configurazione e gestione dell'infrastruttura IT di supporto ad un progetto.
	Il suo ruolo è particolarmente importante nelle fasi iniziali e durante il deployment, dove raggiunge il picco di impegno per garantire un deploy corretto.

\term{Analisi dei Requisiti}
    Documento contenente i Casi d'Uso identificati da un Team per i requisiti di un determinato progetto. Contiene anche i Requisiti funzionali, Requisiti non funzionali e Requisiti di Vincolo.

\term{Analista}
	Si occupa di identificare e chiarire i requisiti, interpretando le esigenze degli utilizzatori finali per garantire una corretta definizione delle funzionalità.

\term{API}
  	Acronimo di \textit{Application Programming Interface}, è un'interfaccia utile a permettere o facilitare la comunicazione tra diversi software o parti di un singolo software.

\term{Attore}
  	Un'entità esterna e non controllabile dal progetto, ma che interagisce con esso, con attività o obiettivi specifici da soddisfare.

\term{Automazione}
	Esecuzione deterministica e ripetibile di processi (build, test, deploy, verifiche, provisioning) attivati da eventi del repository o da pianificazioni, 
	definita tramite workflow dichiarativi in YAML, eseguita in ambienti isolati, senza intervento manuale.

\term{Azione}
  	L'insieme di processi intrapresi per mettere in pratica una determinata Decisione.





\newpage
\begin{center}\section*{B}\end{center}
\addcontentsline{toc}{section}{B}

\term{Backend}
La parte di un progetto che si occupa del funzionamento a livello logico del sistema.

\term{Baseline}
Stato fissato di artefatti (requisiti, design, codice, documentazione) considerato stabile e sufficientemente maturo, dal quale retrocedere comporterebbe costo o rischio elevato.

\term{Branch}
Uno specifico ramo di lavoro all'interno di una repository, atta a mantenere una distinzione tra ramo "principale" e rami di lavoro in cui ciascun membro del gruppo può lavorare in maniera asincrona senza rischio di conflitto.

\newpage
\begin{center}\section*{C}\end{center}
\addcontentsline{toc}{section}{C}

\term{Capitolato}
Documento fornito dalla proponente, che espone le principali specifiche e condizioni di un progetto. Stabilisce delle regole e delle aspettative che il fornitore 
deve rispettare.

\term{Caso d'uso}
Insieme di azioni all'interno di uno scenario, con in comune uno scopo finale per un Attore.

\term{Consuntivo}
Rendiconto redatto dal gruppo al termine di uno sprint che riporta le ore effettivamente impiegate dai membri del gruppo, permettendo il calcolo del costo.


\newpage
\begin{center}
\section*{D}
\end{center}
\addcontentsline{toc}{section}{D}

\term{Database}
Un database (in italiano: base di dati) è un sistema per la conservazione e gestione dei dati. Comprende una componente fisica, che è il supporto di memorizzazione dei dati, e una componente software, detta DBMS (Database Management System), che gestisce l’accesso ai dati, ne garantisce la persistenza e fornisce un metodo sicuro ed efficiente per leggerli e modificarli.

\term{Debug}
Processo di individuazione, analisi e correzione degli errori presenti nel codice di un programma.

\term{Decisione}
Un accordo all'interno del Team, riguardante metodologie o convenzioni da adottare per soddisfare un determinato bisogno.

\term{Discord}
Piattaforma di videochiamate e messaggistica asincrona. Caratterizzata dalla possibilità di avere spazi indipendenti: server; dove si possono creare vari canali testuali indipendenti, solitamente suddivisi per argomenti, così come canali vocali indipendenti, utilizzati per chiamate contemporanee tra i vari componenti del server.

\term{Distribuzione}
Processo finale dello sviluppo software in cui il prodotto viene reso disponibile in una versione stabile per l’utilizzo da parte degli utenti.

\term{Docker}
Software che permette di eseguire applicazioni in ambienti isolati chiamati container.

\term{Doxygen}
Software che permette la generazione automatica di documentazione a partire dal codice sorgente di un software.

\term{Draw.io}
Software utilizzato per la creazione di diagrammi UML.


\newpage
\begin{center}\section*{E}\end{center}
\addcontentsline{toc}{section}{E}

\term{EasyOCR}
Libreria python di tipo OCR per il riconoscimento di testo da immagini. 

\term{Efficacia}
Indica il modo con cui un insieme di attività raggiunge gli obiettivi attesi.

\term{Efficienza}
Rapporto tra risultati ottenuti e risorse impiegate per ottenerli.
\newpage
\begin{center}\section*{F}\end{center}
\addcontentsline{toc}{section}{F}

\term{Frontend}
La parte di un progetto che si occupa dell'interazione diretta con l'utilizzatore finale. 

\term{Funzionalità}
Una caratteristica offerta da un Sistema, spesso con lo scopo di soddisfare un bisogno.
\newpage
\begin{center}\section*{G}\end{center}
\addcontentsline{toc}{section}{G}

\term{Git}
Sistema di controllo di versione che traccia le modifiche al codice sorgente nel tempo. Permette a più sviluppatori di collaborare simultaneamente sullo stesso progetto, gestendo la cronologia delle versioni e supportando lo sviluppo parallelo.

\term{GitFlow}
Modello che sfrutta i branch di Git definendo una struttura organizzata per la gestione dei rami di sviluppo; facilita il lavoro parallelo e consente la netta separazione della versione in produzione.

\term{GitHub}
Servizio cloud che permette di gestire una o più repository Git, offre strumenti che semplificano la collaborazione e l'automazione di processi.

\term{GitHub Action}
Strumento messo a disposizione da GitHub per effettuare automazioni all'interno di una repository.

\term{Google Meet}
Piattaforma di videocomunicazione sviluppata da Google, utilizzata dal gruppo NullPointers Group per le riunioni online con l'azienda.

\term{Google Vision}
Servizio di intelligenza artificiale sviluppato da Google che permette di analizzare e comprendere il contenuto delle immagini attraverso modelli di machine learning.

\term{GPT}
Modello linguistico sviluppato da OpenAI e basato su reti neurali, specializzato nella comprensione e generazione del linguaggio naturale. È progettato per fungere da struttura base su cui costruire applicazioni software e sistemi di intelligenza artificiale.
\newpage
\begin{center}\section*{H}\end{center}
\addcontentsline{toc}{section}{H}
\newpage
\begin{center}\section*{I}\end{center}
\addcontentsline{toc}{section}{I}

\term{Intelligenza Artificiale}
Disciplina dell'informatica che studia e sviluppa sistemi in grado di eseguire compiti che richiedono tipicamente l'intelligenza umana, come il ragionamento, l'apprendimento e la comprensione del linguaggio naturale.

\term{ISO}
Ente di standardizzazione internazionale che definisce norme tecniche condivise per garantire interoperabilità, qualità e uniformità tra prodotti, processi e sistemi.

\term{Issue}
All'interno dell'ecosistema GitHub, rappresenta un'attività che richiede modifiche al codice sorgente per essere svolta, può trattarsi di un bug da risolvere, una feature da aggiungere, di una correzione minore o anche di qualcosa definito dagli sviluppatori.

\newpage
\begin{center}
\section*{J}
\end{center}
\addcontentsline{toc}{section}{J}

\term{JSON}
JSON (JavaScript Object Notation) è un formato leggero per la rappresentazione di dati strutturati, leggibile da esseri umani e facilmente parsabile, ovvero facilmente analizzato o interpretato da un parser.


\newpage
\begin{center}
\section*{K}
\end{center}
\addcontentsline{toc}{section}{K}

\term{KLOC}
Acronimo di \textit{Kilo Lines of Code}, le linee di codice totali che compongono un progetto, nell'ordine delle migliaia.


\newpage
\begin{center}\section*{L}\end{center}
\addcontentsline{toc}{section}{L}

\term{LaTeX}
Linguaggio di markup utilizzato dal gruppo per produrre tutti i documenti.

\term{Licenza MIT}
Una licenza Open Source che permette di utilizzare, modificare e distribuire liberamente il software a patto che la licenza venga distribuita insieme al codice originale.

\term{LlaMa}
Famiglia di modelli linguistici sviluppata da Meta, progettata per essere efficiente e disponibile anche offline. Consente l'esecuzione locale, senza la necessità di appoggiarsi a servizi cloud.

\term{LLM}
Large Language Models. Indica i modelli addestrati su enormi quantità di testo come ChatGPT e DeepSeek.
\newpage
\begin{center}\section*{M}
	\addcontentsline{toc}{section}{M}
\end{center}

\term{Major}
Una modifica atta ad approvare un documento in una sua versione "pronta al rilascio". In un sistema di versionamento x.y.z, la Major è rappresentata dalla lettera x, e ogni
modifica di tipo Major aggiorna un file dalla versione x.y.z alla versione (x+1).0.0.

\term{Manutenzione}
Insieme di attività atte a risolvere i problemi e migliorare le funzionalità di un prodotto.

\term{Merge}
Processo con il quale, in un sistema di versionamento come git, un ramo di lavoro viene fatto confluire in un altro, generalmente in un grado gerarchico più elevato.   

\term{Metodo Agile}
Metodo di sviluppo software per la gestione di progetti complessi basato sui cicli iterativi brevi chiamati \textit{Sprint}, che favorisce la flessibilità del progetto e la collaborazione continua.

\term{Milestone}
Periodo di tempo entro il quale il team si pone il raggiungimento di determinati obiettivi.

\term{Minimum Viable Product}
La prima versione di un prodotto che assolve le funzionalità minime per rispettare i requisiti obbligatori stabiliti dalla proponente.

\term{Minor}
Una modifica atta ad aggiungere migliorie o funzionalità significative. In un sistema di versionamento x.y.z, la Minor è rappresentata dalla lettera y, e ogni modifica di tipo Minor aggiorna un file dalla versione x.y.z alla versione x.(y+1).0.

\term{Modifica}
Qualsiasi intervento sul codice esistente, sulla documentazione o sul metodo di lavoro volto a correggere degli errori, aggiornare i contenuti o adattarli ai nuovi requisiti.

\term{MVP}
Acronimo di \textit{Minimum Viable Product}, per la definizione vedere 'Minimum Viable Product'
\newpage
\begin{center}
\section*{N}
\end{center}
\addcontentsline{toc}{section}{N}

\term{Natural Language Processing}
É il ramo dell'informatica e dell'intelligenza artificiale che sviluppa metodi e algoritmi per far comprendere, generare e interagire i computer con il linguaggio umano.

\term{NLP}
Acronimo di \textit{Natural Language Processing}, per la definizione vedere \vrs{Natural Language Processing}

\term{Norme di Progetto}
Documento che descrive il Way of Working del gruppo. Contiene le regole che NullPointers Group rispetta per garantire un lavoro efficace ed efficiente.


\newpage
\begin{center}\section*{O}\end{center}
\addcontentsline{toc}{section}{O}

\term{OCR}
Processo di riconoscimento ottico di caratteri che converte un'immagine o un documento scannerizzato in un formato leggibile dalla macchina.

\term{OpenAI}
Azienda di ricerca e sviluppo specializzata nell'intelligenza artificiale, nota per la creazione dei modelli GPT. Fornisce inoltre API e servizi per l'integrazione di queste tecnologie all'interno delle applicazioni software.
\newpage
\begin{center}\section*{P}\end{center}
\addcontentsline{toc}{section}{P}

\term{Patch}
Una modifica atta a risolvere errori ortografici o di struttura. In un sistema di versionamento x.y.z, la patch è rappresentata dalla lettera z, e ogni modifica di tipo patch
aggiorna la versione di un file da x.y.z a x.y.(z+1).
\term{PB}
Acronimo di \textit{Product Baseline}, per la definizione vedere 'Product Baseline'

\term{Pianificazione}
Processo di definizione degli obiettivi del progetto identificandone le risorse ed i tempi richiesti per la sua realizzazione.

\term{Piano di Progetto}
Documento che si occupa di definire l'organizzazione del gruppo, specificando l'assegnazione dei ruoli, il monte ore previsto per ogni risorsa, le tempistiche e il conto economico preventivo in funzione dell'avanzamento del progetto.

\term{Piano di Qualifica}
Documento che si occupa di garantire il rispetto dei requisiti del progetto definendo le strategie di verifica e validazione affinché vengano rispettate le attese della proponente.

\term{PoC}
Acronimo di \textit{Proof of Concept}, per la definizione vedere 'Proof of Concept'

\term{PostgreSQL}
Sistema di gestione di database relazionionali open source, noto per la sua scalabilità e la conformità agli standard SQL.

\term{PR}
Acronimo di \textit{Pull Request}, per la definizione vedere 'Pull Request'

\term{Product Baseline}
Fase finale in un progetto, che prevede la realizzazione di un Minimum Viable Product, a seguito del superamento dei test concordati nella 
Requirements and Technology Baseline.

\term{Produzione}
Processo di traduzione della progettazione in codice sorgente mediante la programmazione ed i test.

\term{Progettazione}
Processo che definisce la struttura di un sistema, identificandone componenti, relazioni e livello di dettaglio necessari per la realizzazione. Serve a semplificarne lo sviluppo e la manutenzione futura.

\term{Proof of Concept}
Prodotto dimostrativo atto a presentare la fattibilità dell'idea.

\term{Pull Request}
All'interno dell'ecosistema GitHub, uno strumento che permette di sospendere momentaneamente la pubblicazione di una modifica nel ramo principale della repository, in attesa 
di approvazione e/o workflow automatici (e.g. GitHub Action).

\term{Python}
Linguaggio di programmazione di alto livello, open source, noto per la sua sintassi chiara; è ampiamente utilizzato nel campo dell'intelligenza artificiale e del machine learning.
\newpage
\begin{center}\section*{Q}\end{center}
\addcontentsline{toc}{section}{Q}
\newpage
\begin{center}\section*{R}\end{center}
\addcontentsline{toc}{section}{R}

\term{React}
Framework open source sviluppato da Meta, basato su JavaScript, utilizzato per costruire interfacce utente dinamiche nelle applicazioni web front-end.

\term{Redattore}
Colui che scrive il documento. Viene identificato come "Autore" nel registro delle modifiche.

\term{Repository}
Uno spazio condiviso tra tutti i membri del Team, in cui viene memorizzato e versionato tutto il codice sorgente prodotto.

\term{Requisito di Vincolo}
Tipo di requisito che descrive le restrizioni imposte dall'azienda proponente che il progetto deve rispettare.

\term{Requisito funzionale}
Definiscono cosa il sistema deve fare, descrivendo le sue azioni e funzioni specifiche. Sono le capacità che l'utente si aspetta, come "aggiungere un prodotto al carrello" in un negozio online.

\term{Requisito non funzionale}
Specificano come il sistema deve comportarsi mentre svolge quelle funzioni. Si concentrano su qualità come le prestazioni, la sicurezza o l'affidabilità, ad esempio "il sistema deve supportare 1000 utenti contemporanei senza rallentamenti".

\term{Responsabile}
Il Responsabile coordina le attività del gruppo e garantisce una pianificazione efficace.

\term{REST}
REST (Representational State Transfer) è uno stile architetturale per lo sviluppo di servizi web che utilizza protocolli standard (tipicamente HTTP) per operazioni su risorse identificate tramite URL, basandosi su metodi come GET, POST, PUT, DELETE.

\term{RTB}
Baseline iniziale e vincolante di requisiti e tecnologie approvate per il progetto. Comprende requisiti funzionali, non funzionali, di sicurezza e conformità, insieme a strumenti, linguaggi, piattaforme e infrastrutture scelte. Costituisce il riferimento stabile per verificare coerenza e avanzamento. Qualsiasi modifica richiede revisione e approvazione formale.
\newpage
\begin{center}\section*{S}\end{center}
\addcontentsline{toc}{section}{S}

\term{SM}
Codice stabilito dal Team, rappresenta un'attività, con corrispettiva issue in GitHub all'interno della Repository 'SmartOrder', indica dunque un'attività da svolgere 
relativa al progetto e al \textit{PoC}.
Il codice è seguito da un numero incrementale che permette di distinguere univocamente una singola attività.

\term{SMD}
Codice stabilito dal Team, rappresenta un'attività, con corrispettiva issue in GitHub all'interno della Repository 'Documentazione', indica dunque un'attività da svolgere
relativa alla documentazione di progetto.
Il codice è seguito da un numero incrementale che permette di distinguere univocamente una singola attività.

\term{Sprint}
Ciclo di lavoro di durata fissa, tipico dello sviluppo Agile, durante il quale il team si impegna a completare un insieme prefissato di attività per produrre un incremento del software.

\term{Sviluppo}
Attività pratica di scrittura, test e integrazione del codice necessaria a implementare le funzionalità definite durante la fase di progettazione.
\newpage
\begin{center} \section*{T} 
    \addcontentsline{toc}{section}{T} 
\end{center}

\term{Task}
Un'attività ben definita, assegnata ad uno o più membri del gruppo, che contribuisce allo sviluppo del progetto. Una task può essere parte di un'attività più grande o un obiettivo specifico da raggiungere.

\term{TD}
Acronimo di \textit{To Do}, indica un'attività futura decisa in un incontro interno/esterno, presenta uno o più assegnatari. Il codice è seguito da un numero incrementale che permette di distinguere univocamente una singola attività.

\term{Test}
Un processo dello sviluppo software che si occupa di verificare la funzionalità di una o più parti del prodotto.

\term{Token}
Unità fondamentale di testo (come una parola, un carattere o un frammento di parola) in cui viene suddiviso l'input per essere elaborato dai modelli di linguaggio.

\term{Tokenizzazione}
Processo di segmentazione del testo in unità elementari chiamate \textit{Token}, passaggio necessario agli LLM per l'interpretazione del linguaggio naturale.
\newpage
\begin{center}\section*{U}\end{center}
\addcontentsline{toc}{section}{U}
\newpage
\begin{center}
\section*{V}
\end{center}
\addcontentsline{toc}{section}{V}

\term{Validazione}
Insieme di controlli per assicurarsi che un prodotto (o una parte di esso) soddisfi in modo congruo ciò per cui è stato creato.

\term{VE}
Acronimo di \textit{Verbale Esterno}, viene utilizzato per tracciare nel tempo decisioni e azioni intraprese in un incontro esterno. Il codice è seguito da un numero incrementale che permette di distinguere univocamente una singola attività.

\term{Verifica}
Insieme di processi atti a garantire che una funzione all'interno di un prodotto soddisfi determinati requisiti.

\term{Verificatore}
Il Verificatore si occupa di assicurare la qualità dei prodotti e dei processi adottati, effettuando revisioni e test.

\term{Versionamento}
Metodo per gestire e tenere traccia di documenti, codice o altri artefatti digitali, organizzandoli in versioni identificate da un codice univoco, in modo da poter monitorare modifiche, confrontare stati diversi e ripristinare versioni precedenti.

\term{VI}
Acronimo di \textit{Verbale Interno}, viene utilizzato per tracciare nel tempo decisioni e azioni intraprese in un incontro interno. Il codice è seguito da un numero incrementale che permette di distinguere univocamente una singola attività.


\newpage
\begin{center}\section*{W}\end{center}
    \addcontentsline{toc}{section}{W}
    
\term{Way of Working}
Insieme di metodologie e strutture organizzative approvate dall'intero Team per svolgere in modo coerente i propri compiti.

\term{WhatsApp}
Applicazione di messaggistica istantanea adottata dal gruppo NullPointers Group per le comunicazioni interne e con l'azienda.

\term{Whisper}
Modello di riconoscimento vocale sviluppato da OpenAI specializzato nella trascrizione di audio in testo.

\newpage
\begin{center}\section*{X}\end{center}
\addcontentsline{toc}{section}{X}

\term{XML}
XML (eXtensible Markup Language) è un formato di testo per rappresentare dati strutturati con tag personalizzabili, progettato per trasportare e archiviare informazioni in modo leggibile e interoperabile.
\newpage
\begin{center}\section*{Y}\end{center}
\addcontentsline{toc}{section}{Y}
\newpage
\begin{center}\section*{Z}\end{center}
\addcontentsline{toc}{section}{Z}

\end{document}