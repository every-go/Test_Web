\begin{center}\section*{G}\end{center}
\addcontentsline{toc}{section}{G}

\term{Git}
Sistema di controllo di versione che traccia le modifiche al codice sorgente nel tempo. Permette a più sviluppatori di collaborare simultaneamente sullo stesso progetto, gestendo la cronologia delle versioni e supportando lo sviluppo parallelo.

\term{GitFlow}
Modello che sfrutta i branch di Git definendo una struttura organizzata per la gestione dei rami di sviluppo; facilita il lavoro parallelo e consente la netta separazione della versione in produzione.

\term{GitHub}
Servizio cloud che permette di gestire una o più repository Git, offre strumenti che semplificano la collaborazione e l'automazione di processi.

\term{GitHub Action}
Strumento messo a disposizione da GitHub per effettuare automazioni all'interno di una repository.

\term{Google Meet}
Piattaforma di videocomunicazione sviluppata da Google, utilizzata dal gruppo NullPointers Group per le riunioni online con l'azienda.

\term{Google Vision}
Servizio di intelligenza artificiale sviluppato da Google che permette di analizzare e comprendere il contenuto delle immagini attraverso modelli di machine learning.

\term{GPT}
Modello linguistico sviluppato da OpenAI e basato su reti neurali, specializzato nella comprensione e generazione del linguaggio naturale. È progettato per fungere da struttura base su cui costruire applicazioni software e sistemi di intelligenza artificiale.