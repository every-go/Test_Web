\begin{center}
\section*{C}
\end{center}
\addcontentsline{toc}{section}{C}

\term{Capitolato}
Documento fornito dalla proponente, che espone le principali specifiche e condizioni di un progetto. Stabilisce delle regole e delle aspettative che il fornitore deve rispettare.

\term{Caso d'uso}
Insieme di azioni all'interno di uno scenario, con in comune uno scopo finale per un Attore.

\term{Code Smell}
Caratteristica negativa di un codice sorgente, sono di diverso tipo e possono indicare difetti di design o di implementazione, diminuendo la qualità del software e aumentandone il debito tecnico.

\term{Committente}
Persona, ente o azienda che richiede e finanzia la realizzazione di un progetto software, di un sistema informatico o di un servizio IT. Il committente definisce gli obiettivi, approva i requisiti e verifica i risultati, ma non esegue direttamente lo sviluppo.

\term{Consuntivo}
Rendiconto redatto dal gruppo al termine di uno sprint che riporta le ore effettivamente impiegate dai membri del gruppo, permettendo il calcolo del costo.

\term{Continuous Integration}
Pratica di sviluppo del software che prevede che gli sviluppatori effettuino frequentemente commit verso una repository centrale, alla quale possono seguire test automatizzati.

\term{Cyclomatic Complexity}
Una metrica utilizzata per misurare la complessità di un programma; nello specifico misura il numero di cammini linearmente indipendenti di un metodo, una classe o un progetto intero.

