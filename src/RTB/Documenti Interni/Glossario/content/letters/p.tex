\begin{center}
\section*{P}
\end{center}
\addcontentsline{toc}{section}{P}

\term{Patch}
Una modifica atta a risolvere errori ortografici o di struttura. In un sistema di versionamento x.y.z, la patch è rappresentata dalla lettera z, e ogni modifica di tipo patch aggiorna un file dalla versione x.y.z alla versione x.y.(z+1).

\term{PB}
Acronimo di \textit{Product Baseline}, per la definizione vedere \vrs{Product Baseline}.

\term{Pianificazione}
Processo di definizione degli obiettivi del progetto identificandone le risorse ed i tempi richiesti per la sua realizzazione.

\term{Piano di Progetto}
Documento che si occupa di definire l'organizzazione del gruppo, specificando l'assegnazione dei ruoli, il monte ore previsto per ogni risorsa, le tempistiche e il conto economico preventivo in funzione dell'avanzamento del progetto.

\term{Piano di Qualifica}
Documento che si occupa di garantire il rispetto dei requisiti del progetto definendo le strategie di verifica e validazione affinché vengano rispettate le attese della proponente.

\term{Planned Value}
È la metrica che rappresenta il valore del lavoro pianificato rispetto al budget preventivato previsto.

\term{PoC}
Acronimo di \textit{Proof of Concept}, per la definizione vedere \vrs{Proof of Concept}.

\term{PostgreSQL}
Sistema di gestione di database relazionionali open source, noto per la sua scalabilità e la conformità agli standard SQL.

\term{PR}
Acronimo di \textit{Pull Request}, per la definizione vedere \vrs{Pull Request}.

\term{Processo}
Insieme di attività correlate e coese che trasformano ingressi (bisogni) in uscite (prodotti) secondo regole date, consumando risorse nel farlo.

\term{Product Baseline}
Fase finale in un progetto, che prevede la realizzazione di un Minimum Viable Product, a seguito del superamento dei test concordati nella Requirements and Technology Baseline.

\term{Produzione}
Processo di traduzione della progettazione in codice sorgente mediante la programmazione ed i test.

\term{Progettazione}
Processo che definisce la struttura di un sistema, identificandone componenti, relazioni e livello di dettaglio necessari per la realizzazione. Serve a semplificarne lo sviluppo e la manutenzione futura.

\term{Progettista}
Ruolo dedito alla progettazione dell'architettura di un progetto, al fine di assicurarsi codice più corretto e manutenibile.

\term{Progetto}
Insieme ordinato di attività, istanziate dai processi di ciclo di vita che soddisfano gli obiettivi dati.

\term{Programmatore}
La persona incaricata di svolgere le attività di codifica, data l'architettura.

\term{Proof of Concept}
Prodotto dimostrativo atto a presentare la fattibilità dell'idea.

\term{Proponente}
La persona, il gruppo o l’organizzazione che propone un progetto, un’idea o un’iniziativa, e ne definisce obiettivi, requisiti e motivazioni. La proponente di NullPointers Group è Ergon s.r.l.

\term{Pull Request}
All'interno dell'ecosistema GitHub, uno strumento che permette di sospendere momentaneamente la pubblicazione di una modifica nel ramo principale della repository, in attesa di approvazione e/o workflow automatici (e.g. GitHub Action).

\term{PV}
Acronimo di \textit{Planned Value}, per la definizione vedere \vrs{Planned Value}.

\term{Python}
Linguaggio di programmazione di alto livello, open source, noto per la sua sintassi chiara; è ampiamente utilizzato nel campo dell'intelligenza artificiale e del machine learning.

