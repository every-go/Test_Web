\begin{center}\section*{P}\end{center}
\addcontentsline{toc}{section}{P}

\term{Patch}
Una modifica atta a risolvere errori ortografici o di struttura. In un sistema di versionamento x.y.z, la patch è rappresentata dalla lettera z, e ogni modifica di tipo patch
aggiorna la versione di un file da x.y.z a x.y.(z+1).
\term{PB}
Acronimo di \textit{Product Baseline}, per la definizione vedere 'Product Baseline'

\term{Pianificazione}
Processo di definizione degli obiettivi del progetto identificandone le risorse ed i tempi richiesti per la sua realizzazione.

\term{Piano di Progetto}
Documento che si occupa di definire l'organizzazione del gruppo, specificando l'assegnazione dei ruoli, il monte ore previsto per ogni risorsa, le tempistiche e il conto economico preventivo in funzione dell'avanzamento del progetto.

\term{Piano di Qualifica}
Documento che si occupa di garantire il rispetto dei requisiti del progetto definendo le strategie di verifica e validazione affinché vengano rispettate le attese della proponente.

\term{PoC}
Acronimo di \textit{Proof of Concept}, per la definizione vedere 'Proof of Concept'

\term{PostgreSQL}
Sistema di gestione di database relazionionali open source, noto per la sua scalabilità e la conformità agli standard SQL.

\term{PR}
Acronimo di \textit{Pull Request}, per la definizione vedere 'Pull Request'

\term{Product Baseline}
Fase finale in un progetto, che prevede la realizzazione di un Minimum Viable Product, a seguito del superamento dei test concordati nella 
Requirements and Technology Baseline.

\term{Produzione}
Processo di traduzione della progettazione in codice sorgente mediante la programmazione ed i test.

\term{Progettazione}
Processo che definisce la struttura di un sistema, identificandone componenti, relazioni e livello di dettaglio necessari per la realizzazione. Serve a semplificarne lo sviluppo e la manutenzione futura.

\term{Proof of Concept}
Prodotto dimostrativo atto a presentare la fattibilità dell'idea.

\term{Pull Request}
All'interno dell'ecosistema GitHub, uno strumento che permette di sospendere momentaneamente la pubblicazione di una modifica nel ramo principale della repository, in attesa 
di approvazione e/o workflow automatici (e.g. GitHub Action).

\term{Python}
Linguaggio di programmazione di alto livello, open source, noto per la sua sintassi chiara; è ampiamente utilizzato nel campo dell'intelligenza artificiale e del machine learning.