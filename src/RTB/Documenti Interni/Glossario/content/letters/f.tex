\begin{center}
\section*{F}
\end{center}
\addcontentsline{toc}{section}{F}

\term{Framework}
Insieme strutturato di componenti software, librerie e regole che forniscono una base riutilizzabile per sviluppare applicazioni. Un framework offre funzionalità già pronte e un’architettura predefinita, guidando lo sviluppatore nelle scelte progettuali e riducendo il lavoro necessario per creare nuovi programmi.

\term{Frontend}
La parte di un progetto che si occupa dell'interazione diretta con l'utilizzatore finale.

\term{Funzionalità}
Una caratteristica offerta da un Sistema, spesso con lo scopo di soddisfare un bisogno.

