\begin{center}
\section*{B}
\end{center}
\addcontentsline{toc}{section}{B}

\term{BAC}
Acronimo di \textit{Budget at Completion}, per la definizione vedere \vrs{Budget at Completion}.

\term{Backend}
La parte di un progetto che si occupa del funzionamento a livello logico del sistema.

\term{Baseline}
Stato fissato di artefatti (requisiti, design, codice, documentazione) considerato stabile e sufficientemente maturo, dal quale retrocedere comporterebbe costo o rischio elevato.

\term{Branch}
Uno specifico ramo di lavoro all'interno di una repository, atta a mantenere una distinzione tra ramo \vr{principale} e rami di lavoro in cui ciascun membro del gruppo può lavorare in maniera asincrona senza rischio di conflitto.

\term{Budget at Completion}
È il budget complessivo preventivato all'inizio di un progetto, che rappresenta il costo totale stimato per completare tutte le attività e consegnare il prodotto finale.

\term{Bug}
Difetto o errore nel codice di un programma che provoca comportamenti imprevisti o malfunzionamenti. Può derivare da errori di logica, sintassi, progettazione o interazioni inattese tra componenti software.

\term{Build}
Processo con cui il codice sorgente, insieme alle sue dipendenze, viene trasformato in un artefatto eseguibile o distribuibile (ad esempio un binario, una libreria o un pacchetto).

