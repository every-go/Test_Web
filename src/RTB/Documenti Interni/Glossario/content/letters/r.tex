\begin{center}\section*{R}\end{center}
\addcontentsline{toc}{section}{R}

\term{React}
Framework open source sviluppato da Meta, basato su JavaScript, utilizzato per costruire interfacce utente dinamiche nelle applicazioni web front-end.

\term{Redattore}
Colui che scrive il documento. Viene identificato come "Autore" nel registro delle modifiche.

\term{Repository}
Uno spazio condiviso tra tutti i membri del Team, in cui viene memorizzato e versionato tutto il codice sorgente prodotto.

\term{Requisito di Vincolo}
Tipo di requisito che descrive le restrizioni imposte dall'azienda proponente che il progetto deve rispettare.

\term{Requisito funzionale}
Definiscono cosa il sistema deve fare, descrivendo le sue azioni e funzioni specifiche. Sono le capacità che l'utente si aspetta, come "aggiungere un prodotto al carrello" in un negozio online.

\term{Requisito non funzionale}
Specificano come il sistema deve comportarsi mentre svolge quelle funzioni. Si concentrano su qualità come le prestazioni, la sicurezza o l'affidabilità, ad esempio "il sistema deve supportare 1000 utenti contemporanei senza rallentamenti".

\term{Responsabile}
Il Responsabile coordina le attività del gruppo e garantisce una pianificazione efficace.

\term{REST}
REST (Representational State Transfer) è uno stile architetturale per lo sviluppo di servizi web che utilizza protocolli standard (tipicamente HTTP) per operazioni su risorse identificate tramite URL, basandosi su metodi come GET, POST, PUT, DELETE.

\term{RTB}
Baseline iniziale e vincolante di requisiti e tecnologie approvate per il progetto. Comprende requisiti funzionali, non funzionali, di sicurezza e conformità, insieme a strumenti, linguaggi, piattaforme e infrastrutture scelte. Costituisce il riferimento stabile per verificare coerenza e avanzamento. Qualsiasi modifica richiede revisione e approvazione formale.