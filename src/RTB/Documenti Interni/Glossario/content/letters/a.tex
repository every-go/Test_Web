\begin{center}\section*{A}\end{center}
\addcontentsline{toc}{section}{A}

% Stile per i termini del glossario
\newcommand{\term}[1]{%
  \par % chiude il paragrafo precedente
  \vspace{1.2em} % spazio prima del termine
  \begin{center}
    \textbf{\large #1}
  \end{center}
  \vspace{0.4em} % piccolo spazio dopo il termine
  \noindent % inizio della definizione senza rientro
}


\term{Amministratore}
	L'Amministratore si occupa della configurazione e gestione dell'infrastruttura IT di supporto ad un progetto.
	Il suo ruolo è particolarmente importante nelle fasi iniziali e durante il deployment, dove raggiunge il picco di impegno per garantire un deploy corretto.

\term{Analisi dei Requisiti}
    Documento contenente i Casi d'Uso identificati da un Team per i requisiti di un determinato progetto. Contiene anche i Requisiti funzionali, Requisiti non funzionali e Requisiti di Vincolo.

\term{Analista}
	Si occupa di identificare e chiarire i requisiti, interpretando le esigenze degli utilizzatori finali per garantire una corretta definizione delle funzionalità.

\term{API}
  	Acronimo di \textit{Application Programming Interface}, è un'interfaccia utile a permettere o facilitare la comunicazione tra diversi software o parti di un singolo software.

\term{Attore}
  	Un'entità esterna e non controllabile dal progetto, ma che interagisce con esso, con attività o obiettivi specifici da soddisfare.

\term{Automazione}
	Esecuzione deterministica e ripetibile di processi (build, test, deploy, verifiche, provisioning) attivati da eventi del repository o da pianificazioni, 
	definita tramite workflow dichiarativi in YAML, eseguita in ambienti isolati, senza intervento manuale.

\term{Azione}
  	L'insieme di processi intrapresi per mettere in pratica una determinata Decisione.




