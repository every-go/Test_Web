\begin{center}
\section*{A}
\end{center}
\addcontentsline{toc}{section}{A}

\term{AC}
Acronimo di \textit{Actual Cost}, per la definizione vedere \vrs{Actual Cost}.

\term{Actual Cost}
È una metrica che rappresenta il costo in euro sostenuto fino al momento corrente per svolgere le ore produttive rendicontate.

\term{Agile}
In informatica, Agile è un approccio di lavoro basato su iterazioni brevi e frequenti feedback. Nel metodo Scrum, uno dei più diffusi, il lavoro viene suddiviso in Sprint di una o due settimane. Ogni Sprint inizia con una Sprint Planning, in cui si definiscono gli obiettivi, e si conclude con una Sprint Retrospective, in cui il team valuta cosa è andato bene e cosa migliorare. Durante lo Sprint si svolgono riunioni giornaliere (Daily Scrum) per monitorare l’avanzamento.\\

\term{Amministratore}
L'Amministratore si occupa della configurazione e gestione dell'infrastruttura IT di supporto ad un progetto.

\term{Analisi dei Requisiti}
Documento contenente i Casi d'Uso identificati da un Team per i requisiti di un determinato progetto. Contiene anche i Requisiti funzionali, Requisiti di Qualità e Requisiti di Vincolo.

\term{Analista}
Si occupa di identificare e chiarire i requisiti, interpretando le esigenze degli utilizzatori finali per garantire una corretta definizione delle funzionalità.

\term{API}
Acronimo di \textit{Application Programming Interface}, è un'interfaccia utile a permettere o facilitare la comunicazione tra diversi software o parti di un singolo software.

\term{Attore}
Un'entità esterna e non controllabile dal progetto, ma che interagisce con esso, con attività o obiettivi specifici da soddisfare.

\term{Automazione}
Esecuzione deterministica e ripetibile di processi (build, test, deploy, verifiche, provisioning) attivati da eventi del repository o da pianificazioni, definita tramite workflow dichiarativi in YAML, eseguita in ambienti isolati, senza intervento manuale.

\term{Azione}
L'insieme di processi intrapresi per mettere in pratica una determinata Decisione.

