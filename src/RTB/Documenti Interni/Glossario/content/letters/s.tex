\begin{center}
\section*{S}
\end{center}
\addcontentsline{toc}{section}{S}

\term{SM}
Codice stabilito dal Team, rappresenta un'attività, con corrispettiva issue in GitHub all'interno della Repository \vrs{SmartOrder}, indica dunque un'attività da svolgere relativa al progetto e al \textit{PoC}. Il codice è seguito da un numero incrementale che permette di distinguere univocamente una singola attività.

\term{SMD}
Codice stabilito dal Team, rappresenta un'attività, con corrispettiva issue in GitHub all'interno della Repository \vrs{Documentazione}, indica dunque un'attività da svolgere relativa alla documentazione di progetto. Il codice è seguito da un numero incrementale che permette di distinguere univocamente una singola attività.

\term{Sprint}
Ciclo di lavoro di durata fissa, tipico dello sviluppo Agile, durante il quale il team si impegna a completare un insieme prefissato di attività per produrre un incremento del software.

\term{SQL}
Acronimo di \textit{Structured Query Language}, è un linguaggio di programmazione con lo scopo di creare, filtrare ed eliminare dati memorizzati all'intero di tabelle nei database relazionali.

\term{Stakeholder}
Qualunque persona, gruppo o organizzazione che ha interesse, influenza o è coinvolta nel progetto, includendo clienti, utenti, sviluppatori, dirigenti, fornitori e persino organizzazioni esterne, ed è fondamentale per definire requisiti.

\term{Sviluppo}
Attività pratica di scrittura, test e integrazione del codice necessaria a implementare le funzionalità definite durante la fase di progettazione.

