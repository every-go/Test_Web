\begin{center}
\section*{M}
\end{center}
\addcontentsline{toc}{section}{M}

\term{Machine Learning}
Una branca dell'intelligenza artificiale che permette ad un computer di svolgere un compito apprendendo dai dati forniti e migliorando le sue prestazioni nel tempo tramite più dati ed esperienze.

\term{Major}
Una modifica atta ad approvare un documento in una sua versione \vr{pronta al rilascio}. In un sistema di versionamento x.y.z, la Major è rappresentata dalla lettera x, e ogni modifica di tipo Major aggiorna un file dalla versione x.y.z alla versione (x+1).y.z.

\term{Manutenzione}
Insieme di attività atte a risolvere i problemi e migliorare le funzionalità di un prodotto.

\term{Merge}
Processo con il quale, in un sistema di versionamento come git, un ramo di lavoro viene fatto confluire in un altro, generalmente in un grado gerarchico più elevato.

\term{Metodo Agile}
Metodo di sviluppo software per la gestione di progetti complessi basato sui cicli iterativi brevi chiamati \textit{Sprint}, che favorisce la flessibilità del progetto e la collaborazione continua.

\term{Milestone}
Periodo di tempo entro il quale il team si pone il raggiungimento di determinati obiettivi.

\term{Minimum Viable Product}
La prima versione di un prodotto che assolve le funzionalità minime per rispettare i requisiti obbligatori stabiliti dalla proponente.

\term{Minor}
Una modifica atta ad aggiungere migliorie o funzionalità significative. In un sistema di versionamento x.y.z, la Minor è rappresentata dalla lettera y, e ogni modifica di tipo Minor aggiorna un file dalla versione x.y.z alla versione x.(y+1).0.

\term{ML}
Acronimo di \textit{Machine Learning}, per la definizione vedere \vrs{Machine Learning}.

\term{Modifica}
Qualsiasi intervento sul codice esistente, sulla documentazione o sul metodo di lavoro volto a correggere degli errori, aggiornare i contenuti o adattarli ai nuovi requisiti.

\term{MVP}
Acronimo di \textit{Minimum Viable Product}, per la definizione vedere \vrs{Minimum Viable Product}.

