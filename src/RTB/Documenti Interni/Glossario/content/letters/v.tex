\begin{center}\section*{V}\end{center}
\addcontentsline{toc}{section}{V}

\term{Validazione}
Insieme di controlli per assicurarsi che un prodotto (o una parte di esso) soddisfi in modo congruo ciò per cui è stato creato.

\term{Verifica}
	Insieme di processi atti a garantire che una funzione all'interno di un prodotto soddisfi determinati requisiti.

\term{Verificatore}
	Il Verificatore si occupa di assicurare la qualità dei prodotti e dei processi adottati, effettuando revisioni e test.

\term{VI}
	Acronimo di \textit{Verbale Interno}, viene utilizzato per tracciare nel tempo decisioni e azioni intraprese in un incontro interno. Il codice è seguito da un numero incrementale che permette di distinguere univocamente una singola attività.

\term{VE}
	Acronimo di \textit{Verbale Esterno}, viene utilizzato per tracciare nel tempo decisioni e azioni intraprese in un incontro esterno. Il codice è seguito da un numero incrementale che permette di distinguere univocamente una singola attività.
	



