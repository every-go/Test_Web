\begin{center} \section*{T} 
    \addcontentsline{toc}{section}{T} 
\end{center}

\term{Task}
Un'attività ben definita, assegnata ad uno o più membri del gruppo, che contribuisce allo sviluppo del progetto. Una task può essere parte di un'attività più grande o un obiettivo specifico da raggiungere.

\term{TD}
Acronimo di \textit{To Do}, indica un'attività futura decisa in un incontro interno/esterno, presenta uno o più assegnatari. Il codice è seguito da un numero incrementale che permette di distinguere univocamente una singola attività.

\term{Test}
Un processo dello sviluppo software che si occupa di verificare la funzionalità di una o più parti del prodotto.

\term{Token}
Unità fondamentale di testo (come una parola, un carattere o un frammento di parola) in cui viene suddiviso l'input per essere elaborato dai modelli di linguaggio.

\term{Tokenizzazione}
Processo di segmentazione del testo in unità elementari chiamate \textit{Token}, passaggio necessario agli LLM per l'interpretazione del linguaggio naturale.