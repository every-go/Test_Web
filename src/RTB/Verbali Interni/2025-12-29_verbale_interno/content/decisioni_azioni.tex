\section{Decisioni e Azioni}

\begin{table}[h]
	\centering
	\renewcommand{\arraystretch}{1.15}
	\resizebox{\textwidth}{!}{
		\begin{tabular}{|p{0.12\textwidth}|p{0.65\textwidth}|}
			\hline
			\rowcolor[gray]{0.9}
			Codice & Descrizione \\
			\hline
			VI$^G$ 15.1 & Creazione repository$^G$ dedicata in vista del PoC$^G$.\\
			\hline
			VI$^G$ 15.2 & Dichiarare eventuali casi d'uso derivanti da altri casi d'uso.\\
			\hline
			VI$^G$ 15.3 & Utilizzare sempre elenco puntato per sezioni "Pre-Condizioni" e "Post-Condizioni".\\
			\hline
			VI$^G$ 15.4 & Utilizzare elenco numerato per la sezione "Sezione Principale".\\
			\hline
			VI$^G$ 15.5 & Utilizzare il comando latex$^G$ "hyperref" per i collegamenti ipertestuali".\\
			\hline
			VI$^G$ 15.6 & Utilizzare sempre caption per l'immagine raffigurante lo schema UML$^G$ del caso d'uso$^G$.\\
			\hline
			\href{https://github.com/NullPointersGroup/Documentazione/issues/120}{SMD$^G$ 27}& Continuazione Analisi di Requisiti: sezione Casi d'Uso (30-35). \\
			\hline
			\href{https://github.com/NullPointersGroup/Documentazione/issues/121}{SMD$^G$ 28}& Continuazione Analisi di Requisiti: stesura requisiti. \\
			\hline
			\href{https://github.com/NullPointersGroup/Documentazione/issues/122}{SMD$^G$ 29}& Continuazione Analisi di Requisiti: stesura tracciamento. \\
			\hline
		\end{tabular}
	}
	\vspace{0.3cm}
\end{table}


Qualsiasi modifica$^G$ alle issue$^G$ relativa all'Analisi dei Requisiti$^G$, dovute a ricontrolli durante la stesura, saranno tracciate nel sistema di progetto$^G$.
