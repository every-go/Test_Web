% Configurazione
\documentclass{article}

\usepackage{titling} % Required for inserting the subtitle
\usepackage{graphicx} % Required for inserting images
\usepackage{tabularx} % Per l'ambiente tabularx (tabelle)
\usepackage{calc} % Sempre per le tabelle
\usepackage[colorlinks=true,
linkcolor=blue,
urlcolor=blue,
citecolor=blue]{hyperref}
 % Per i collegamenti ipertestuali, ad esempio sulla table of contents
\usepackage{xcolor} % Per colorare il testo
\usepackage{colortbl} % Per colorare le celle delle tabelle
\usepackage{lipsum} % Per generare lorem ipsum
\usepackage[normalem]{ulem} % Per sottolineare il testo
\usepackage{array} % Per la visualizzazione fluttuante di array di domande e risposte
\usepackage{ragged2e} % Pacchetto necessario per \justifying che giustifica il testo di tabelle

\newcommand{\ulhref}[2]{\href{#1}{\uline{#2}}} % Nuovo comando per sottolineare i link
\newcommand{\ulref}[1]{\uline{\ref{#1}}} % Nuovo comando per sottolineare i collegamenti a immagini
\setlength{\parindent}{0pt} % Rimuove il rientro automatico dei paragrafi

\graphicspath{ {immagini/} {../../../shared/images/} }


% Variabili
\newcommand{\data}{6 Novembre 2025}


% Struttura
\begin{document}

% Parte costante
\begin{minipage}{0.4\textwidth}
    \includegraphics[width=0.6\textwidth]{logo_unipd.png}
\end{minipage}
\begin{minipage}{0.55\textwidth}
    \textcolor{red}{\textbf{Università degli Studi di Padova}} \\
    \textcolor{red}{Laurea: Informatica} \\
    \textcolor{red}{Corso: Ingegneria del Software} \\
    \textcolor{red}{Anno Accademico: 2025/2026}
\end{minipage}

\begin{minipage}{0.4\textwidth}
    \includegraphics[width=0.6\textwidth]{logo_gruppo.jpeg}
\end{minipage}
\begin{minipage}{0.55\textwidth}
    \textbf{Gruppo: NullPointers Group} \\
    Email: \textsf{groupnullpointers@gmail.com}
\end{minipage}

\vspace{2cm}
{
		\centering
		\Huge\bfseries Norme di Progetto\par
		\vspace{1.0cm}
		\Large\bfseries \par
	}
\begin{center}
\begin{tabular}{r|l}
    Stato & In Approvazione \\[0.5em]
    Versione & 0.2.0 \\[0.5em]
    Data ultima modifica & 15/11/2025 \\[0.5em]
    Redattori & Lisa Casagrande \\
              & Matteo Mazzaretto \\[0.5em]
    Verificatori & Matteo Mazzaretto \\
                 & Tommaso Ceron \\[0.5em]
    Destinatari & Prof. Tullio Vardanega \\
    & Prof. Riccardo Cardin \\
    & Ergon Informatica Srl \\
    & NullPointers Group \\
\end{tabular}
\end{center}
\newpage
% variabile, prima dell'indice
\section*{Registro delle modifiche}

\begin{table}[h]
	\centering
	\resizebox{\textwidth}{!}{
		\begin{tabular}{|c|c|c|c|c|}
			\hline
			\rowcolor[gray]{0.9}
			Vers & Data & Autore & Verificatore & Descrizione \\
			1.0.0 & 13-11 & L. Casagrande & L. Pieripolli & Approvazione documento \\
			\hline
			0.1.0 & 12-11 & L. Casagrande & M. Mazzaretto & Creazione e stesura documento \\
			\hline
	\end{tabular}}
\end{table}
\newpage
\hypersetup{linkcolor=black}
\tableofcontents
\clearpage
\newpage

\hypersetup{linkcolor=blue}
% Parte variabile
\section{Informazioni generali}

\begin{itemize}
    \item \textbf{Tipo di riunione:} Interna
    \item \textbf{Luogo:} Meeting Discord$^G$
    \item \textbf{Data:} 29/12/2025
    \item \textbf{Ora inizio:} 14:00
    \item \textbf{Ora fine:} 17:00
    \item \textbf{Scriba:} Marco Brunello
\end{itemize}
\newpage
\section{Ordine del giorno}

Secondo colloquio con la proponente$^G$ Ergon Informatica.\\
In questo incontro, svolta a distanza tramite la piattaforma Google Meet$^G$, discutiamo principalmente gli scenari dei casi d'uso e dei requisiti.\\
Oltretutto, poniamo alcune domande e dubbi che ci sono sorti:
\begin{itemize}
	\item È preferibile utilizzare il formato TOON oppure il più classico formato JSON?
	\item Con quale tecnologia si implementa l'API$^G$ REST? Suggerimenti a riguardo
	\item Si può considerare il sistema di machine learning$^G$ come un qualcosa definito e modificato da noi, e quindi non considerabile né attore$^G$ principale né attore$^G$ secondario?
\end{itemize}
\newpage
\section{Diario della riunione}

	Abbiamo inviato mail alle aziende per fissare i colloqui per i nostri capitolati d'appalto preferiti.\\
	Di conseguenza abbiamo iniziato a scrivere le domande per le proponenti.\\
	Abbiamo discusso per il sito web e abbiamo deciso di creare l'automazione, purtroppo non riuscendo ad automatizzare perfettamente come vorremmo.\\
	
\newpage
\section{Decisioni}

	
\begin{table}[h]
	\centering
	\resizebox{\textwidth}{!}{
		\begin{tabular}{|c|c|p{0.5\textwidth}|}
			\hline
			\rowcolor[gray]{0.9}
			Codice & TD di riferimento & Decisione \\
			\hline
			VI 7.1 & - & Assegnazione ruoli e compiti \\
			\hline
			VI 7.2 & TD 6.2 & Scartata l'idea di creare un branch dev \\
			\hline
			VI 7.3 & - & Creati codici SM e SMD \\
			\hline
			VI 7.4 & - & Scelto standard ISO/IEC/IEEE 29148:2018 per Analisi Requisiti \\
			\hline
			VI 7.5 & - & Nuovo sistema di versionamento \\
			\hline
			\href{https://github.com/NullPointersGroup/Documentazione/issues/1}{SMD 1} & - & Miglioramento script \\
			\hline
			\href{https://github.com/NullPointersGroup/Documentazione/issues/2}{SMD 2} & TD 6.1 & Creata GitHub Action per le pull request personalizzate approvate dai verificatori \\
			\hline
	\end{tabular}}
\end{table}
\newpage
\section{Todo}

Durante la riunione abbiamo deciso le seguenti task con degli assegnatari:

\vspace{0.5cm}

\begin{table}[htbp]
	\begin{tabular}{|c|c|p{0.5\textwidth}|}
		\hline
		\rowcolor[gray]{0.9}
		Codice & Assegnatario & Task Todo \\
		\hline
		TD 6.1 & Matteo Mazzaretto & Creazione YAML della github-action. \\
		\hline
		TD 6.2 & Marco Brunello & Modifica impostazioni organizzazione così da evitare push diretti sul main e creazione branch dev.\\
		\hline
	\end{tabular}
\end{table}

\end{document}