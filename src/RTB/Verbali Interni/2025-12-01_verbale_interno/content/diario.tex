\section{Diario della riunione}
\subsection{Rotazione ruoli}
In questo sprint$^G$, fino al 15/12/2025, i ruoli saranno i seguenti: \\
\begin{center}
	Matteo Mazzaretto $\rightarrow$ Responsabile$^G$ \\
	Lisa Casagrande, Luca Marcuzzo $\rightarrow$ Amministratore$^G$ \\
	Marco Brunello, Tommaso Ceron, Laura Pieripolli $\rightarrow$ Analista$^G$
\end{center}
È comunque prevista una sessione di analisi dei requisiti$^G$ del gruppo, da effettuare prima del 10/12 per l'incontro con la proponente$^G$.

\subsection{Definizione issue}
In questo sprint$^G$ vengono definite le issue$^G$ per il diario di bordo del 15/12, per l'inizio della stesura del Piano di Qualifica$^G$ e per la prosecuzione della stesura delle Norme di Progetto$^G$. Tutto il resto del lavoro sarà in documenti non condivisi col pubblico perché bozze di scenari da discutere con la proponente$^G$.

\subsection{Definizione milestone}
È stato deciso di definire due milestone$^G$: una per il completamento della documentazione e una per il completamento del \vrs{Proof of Concept$^G$}. \\
Entrambe avranno la stessa data di termine, ma servono per suddividere le attività. \\
Ogni documento avrà la issue$^G$ finale di approvazione da parte del responsabile$^G$.

\subsection{Definizione nomenclatura branch e nomenclatura issue}
È stata definita la nomenclatura corretta per i branch$^G$ nella repo Documentazione. \\
Tutti i branch$^G$ saranno sempre di tipo feature/* come prima, ma abbiamo una più chiara definizione dei nomi:
\begin{center}
	\begin{tabular}{|c|c|}
		\hline
		Nome documento & Branch$^G$ \\
		\hline
		Analisi dei Requisiti$^G$ & feature/analisi-requisiti \\
		\hline
		Norme di Progetto$^G$ & feature/norme-progetto$^G$ \\
		\hline
		Piano di Progetto$^G$ & feature/piano-progetto$^G$ \\
		\hline
		Glossario & feature/glossario \\
		\hline
		Piano di Qualifica$^G$ & feature/piano-qualifica \\
		\hline
		Specifica Tecnica & feature/specifica-tecnica \\
		\hline
		Manuale Utente & feature/manuale-utente \\
		\hline
	\end{tabular}
\end{center}
È stato deciso che le issue$^G$ non avranno nomenclatura particolare, l'importante è che il nome e la descrizione siano sintetiche e concise.

\subsection{Definizione metriche di qualifica}
Sono stati definiti i codici per le metriche di qualifica. \\
MQC\_\# per \vrs{Metriche Qualità Processo$^G$}. \\
MQD\_\# per \vrs{Metriche Qualità Prodotto}. \\
\# indica un numero incrementale.\\
Le metriche di qualifica individuate sono:
\begin{itemize}
	\item MQC\_01: Completeness Issue$^G$.
	\item MQC\_02: Budget Value.
	\item MQC\_03: Earned Value$^G$.
	\item MQC\_04: Planned Value$^G$.
	\item MQC\_05: Actual Cost$^G$.
	\item MQC\_06: Cost Performance Index.
	\item MQC\_07: Schedule Performance Index.
	\item MQC\_08: Estimate At Completion$^G$.
	\item MQC\_09: Estimate To Complete.
	\item MQC\_10: Time Estimate At Completion$^G$.
	\item MQC\_11: Indice di Gulpease.
	\item MQC\_12: Code Coverage.
	\item MQC\_13: Test$^G$ Success Rate.
	\item MQC\_14: Quality Metrics Satisfied.
	\item MQD\_01: Requisiti Obbligatori Soddisfatti.
	\item MQD\_02: Requisiti Desiderabili Soddisfatti.
	\item MQD\_03: Requisiti Opzionali Soddisfatti.
	\item MQD\_04: Branch$^G$ Coverage.
	\item MQD\_05: Statement Coverage.
	\item MQD\_06: Time on Task$^G$.
	\item MQD\_07: Response Time.
	\item MQD\_08: Code Smells.
	\item MQD\_09: Coupling Coefficient.
	\item MQD\_10: Cyclomatic Complexity$^G$.
\end{itemize}
Ogni metrica è definita in maniera rigorosa nelle Norme di Progetto$^G$.\\
Ogni eventuale modifica$^G$ sarà tracciata nel documento appena menzionato.