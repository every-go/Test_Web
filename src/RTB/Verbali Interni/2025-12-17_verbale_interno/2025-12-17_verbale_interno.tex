% Configurazione
\documentclass{article}

\usepackage{titling} % Required for inserting the subtitle
\usepackage{graphicx} % Required for inserting images
\usepackage{tabularx} % Per l'ambiente tabularx (tabelle)
\usepackage{calc} % Sempre per le tabelle
\usepackage{xcolor} % Per colorare il testo
\usepackage[colorlinks=true,
linkcolor=blue,
urlcolor=blue,
citecolor=blue]{hyperref}
\usepackage{colortbl} % Per colorare le celle delle tabelle
\usepackage{lipsum} % Per generare lorem ipsum
\usepackage[normalem]{ulem} % Per sottolineare il testo
\usepackage{array} % Per la visualizzazione fluttuante di array di domande e risposte
\usepackage{ragged2e} % Pacchetto necessario per \justifying che giustifica il testo di tabelle
\usepackage{makecell}%Per l'altezza delle tabelle
\usepackage[T1]{fontenc} %per {} come ""

\newcommand{\ulhref}[2]{\href{#1}{\underline{#2}}} % Sottolineatura
\newcommand{\ulref}[1]{\uline{\ref{#1}}} % Sottolinea riferimenti a figure
\setlength{\parindent}{0pt} % Nessun rientro dei paragrafi
\newcommand{\vr}[1]{\textquotedblleft#1\textquotedblright}% per " "
\newcommand{\vrs}[1]{\textquoteleft#1\textquoteright}% per ' '

\graphicspath{ {immagini/} {../../../shared/images/} }


% Variabili
\newcommand{\data}{17 Dicembre 2025}


% Struttura
\begin{document}

% Parte costante
\begin{minipage}{0.4\textwidth}
    \includegraphics[width=0.6\textwidth]{logo_unipd.png}
\end{minipage}
\begin{minipage}{0.55\textwidth}
    \textcolor{red}{\textbf{Università degli Studi di Padova}} \\
    \textcolor{red}{Laurea: Informatica} \\
    \textcolor{red}{Corso: Ingegneria del Software} \\
    \textcolor{red}{Anno Accademico: 2025/2026}
\end{minipage}

\begin{minipage}{0.4\textwidth}
    \includegraphics[width=0.6\textwidth]{logo_gruppo.jpeg}
\end{minipage}
\begin{minipage}{0.55\textwidth}
    \textbf{Gruppo: NullPointers Group} \\
    Email: \textsf{groupnullpointers@gmail.com}
\end{minipage}

\vspace{2cm}
{
		\centering
		\Huge\bfseries Norme di Progetto\par
		\vspace{1.0cm}
		\Large\bfseries \par
	}
\begin{center}
\begin{tabular}{r|l}
    Stato & In Approvazione \\[0.5em]
    Versione & 0.2.0 \\[0.5em]
    Data ultima modifica & 15/11/2025 \\[0.5em]
    Redattori & Lisa Casagrande \\
              & Matteo Mazzaretto \\[0.5em]
    Verificatori & Matteo Mazzaretto \\
                 & Tommaso Ceron \\[0.5em]
    Destinatari & Prof. Tullio Vardanega \\
    & Prof. Riccardo Cardin \\
    & Ergon Informatica Srl \\
    & NullPointers Group \\
\end{tabular}
\end{center}
\newpage
% variabile, prima dell'indice
\hypersetup{linkcolor=black}
\renewcommand{\contentsname}{Indice}
\tableofcontents
\clearpage
\newpage

\hypersetup{linkcolor=blue}
% Parte variabile
\section{Informazioni generali}

\begin{itemize}
    \item \textbf{Tipo di riunione:} Interna
    \item \textbf{Luogo:} Meeting Discord$^G$
    \item \textbf{Data:} 29/12/2025
    \item \textbf{Ora inizio:} 14:00
    \item \textbf{Ora fine:} 17:00
    \item \textbf{Scriba:} Marco Brunello
\end{itemize}
\newpage
\section{Ordine del giorno}

Secondo colloquio con la proponente$^G$ Ergon Informatica.\\
In questo incontro, svolta a distanza tramite la piattaforma Google Meet$^G$, discutiamo principalmente gli scenari dei casi d'uso e dei requisiti.\\
Oltretutto, poniamo alcune domande e dubbi che ci sono sorti:
\begin{itemize}
	\item È preferibile utilizzare il formato TOON oppure il più classico formato JSON?
	\item Con quale tecnologia si implementa l'API$^G$ REST? Suggerimenti a riguardo
	\item Si può considerare il sistema di machine learning$^G$ come un qualcosa definito e modificato da noi, e quindi non considerabile né attore$^G$ principale né attore$^G$ secondario?
\end{itemize}
\newpage
\section{Diario della riunione}

	Abbiamo inviato mail alle aziende per fissare i colloqui per i nostri capitolati d'appalto preferiti.\\
	Di conseguenza abbiamo iniziato a scrivere le domande per le proponenti.\\
	Abbiamo discusso per il sito web e abbiamo deciso di creare l'automazione, purtroppo non riuscendo ad automatizzare perfettamente come vorremmo.\\
	
\newpage
\section{Decisioni e Azioni}

\begin{table}[h]
	\centering
	\renewcommand{\arraystretch}{1.15}
	\resizebox{\textwidth}{!}{
		\begin{tabular}{|p{0.12\textwidth}|p{0.65\textwidth}|}
			\hline
			\rowcolor[gray]{0.9}
			Codice & Descrizione \\
			\hline
			VI$^G$ 16.1 & Aggiunta metrica di \vr{Cognitive Complexity}.\\
			\hline
			VI$^G$ 16.2 & Test$^G$ di unità, integrazione e regressione verranno concepiti durante la PB$^G$.\\
			\hline
			VI$^G$ 16.3 & Creazione nomenclatura per il tracciamento dei test$^G$.\\
			\hline
			VI$^G$ 16.4 & Continuazione stesura analisi dei requisiti$^G$.\\
			\hline
			\href{https://github.com/NullPointersGroup/Documentazione/issues/128}{SMD$^G$ 30} & Stesura diario di bordo.\\
			\hline
			\href{https://github.com/NullPointersGroup/Documentazione/issues/134}{SMD$^G$31} & Consuntivo$^G$ quarto sprint$^G$ e Preventivo quinto sprint$^G$.\\
			\hline
		\end{tabular}
	}
	\vspace{0.3cm}
\end{table}

\end{document}
