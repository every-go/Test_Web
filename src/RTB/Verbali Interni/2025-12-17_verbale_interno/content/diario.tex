\section{Diario della riunione}
A seguito della discussione dei casi d'uso avvenuta con la proponente$^G$, di cui il relativo \href{https://nullpointersgroup.github.io/Documentazione/output/RTB/Verbali\%20Esterni/2025-12-10\_verbale\_esterno.pdf}{verbale esterno in data 10 Dicembre}, si è deciso di rivedere alcuni casi d'uso.

\subsection{Gestione delle ambiguità}
Nel corso della riunione è stata esaminata e ridefinita la procedura per la gestione delle ambiguità riscontrate durante il processo$^G$ di ordinazione dell'utente.\\
Il processo$^G$ prende avvio quando l'utente crea un ordine, per il quale è stato stabilito un limite massimo di 15 articoli.\\
La gestione delle ambiguità avverrà interamente attraverso la chatbox. Nel momento in cui il sistema rileva un'ambiguità nella descrizione di un articolo fornita dall'utente, la chatbox interverrà elencando i dubbi.\\
A seguito di questo messaggio, sarà compito dell'utente riformulare con precisione l'oggetto che intende ordinare.\\
Qualora, dopo questa prima riformulazione, la chatbox non riesca ancora a comprendere la richiesta in modo univoco, chiederà all'utente di specificare nuovamente. Se invece è stato compreso correttamente, il sistema procederà all'aggiunta o alla modifica$^G$ dell'articolo.\\
Per quanto riguarda la gestione dell'ordine, sono stati definiti comandi specifici. L'utente potrà utilizzare il comando /carrello per visualizzare il resoconto completo del proprio ordine, e il comando /invia per confermare e inviare l'ordine definitivo.\\
È stato tuttavia stabilito un vincolo fondamentale per garantire la correttezza degli ordini: l'invio dell'ordine non sarà possibile senza una preventiva verifica$^G$ obbligatoria.\\
Di conseguenza, l'utente dovrà necessariamente aver visualizzato il riepilogo del carrello tramite il comando /carrello prima che il comando /invia diventi effettivo e permetta la trasmissione dell'ordine.\\
Questo passaggio è stato ritenuto necessario dalla proponente$^G$ e introdotto come misura di sicurezza per ridurre al minimo errori e fraintendimenti.

\subsection{In generale}
La funzionalità$^G$ di autenticazione di utenti e admin è stata ridefinita come caso d'uso$^G$ secondario e desiderabile.\\
Si è proceduto alla scomposizione di alcuni casi d'uso facendo in modo che fossero il più atomici possibile.\\
Suddivisione del lavoro tra gli Analisti del gruppo per poter individuare i requisiti relativi ad ogni caso d'uso$^G$.