\section{Diario della riunione}
\subsection{Aggiunta metrica di \vr{Cognitive Complexity}}
Il gruppo, svolgendo ricerche col fine di automatizzare il calcolo delle metriche, si è imbattuto nella metrica \vr{Cognitive Complexity}. \\
La quale è calcolata assegnando valori incrementali a varie strutture che aumentano lo sforzo mentale per capire una funzione, come for e if annidati.\\

\subsection{Codici per tracciamento dei test}
Nel corso della riunione il gruppo si è accordato che i test$^G$ di unità, di integrazione e regressione verranno concepiti durante la PB$^G$, in quanto richiedono valori attesi che al momento il gruppo non sa prevedere. \\
In compenso, il gruppo si è accordato sulla convenzione della nomenclatura dei test$^G$ per il tracciamento:
\begin{itemize}
	\item TU\_XX: \textbf{T}est di \textbf{U}nità numero xx
	\item TI\_XX: \textbf{T}est di \textbf{I}ntegrazione numero xx
	\item TS\_XX: \textbf{T}est di \textbf{S}istema numero xx
	\item TA\_XX: \textbf{T}est di \textbf{A}ccettazione numero xx
\end{itemize}
Inoltre, associato ad ogni singolo test$^G$ ci sarà il suo stato:
\begin{itemize}
	\item S: Superato
	\item I: Implementato
	\item NI: Non Implementato
\end{itemize}

\subsection{Accertamento stato di avanzamento}
Nei precedenti sprint$^G$ il gruppo aveva pianificato di finire l'analisi dei requisiti$^G$ nel giorno odierno.\\
Tuttavia il gruppo ha sovvrastimato il carico di lavoro che avrebbe comportato la terminazione entro il giorno 5 gennaio, perciò si è prefissato di continuare la stesura dell'analisi dei requisiti$^G$.