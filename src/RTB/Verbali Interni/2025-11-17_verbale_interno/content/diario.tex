\section{Diario della riunione}
\subsection{Assegnazione ruoli e compiti}
Nella riunione di oggi sono stati stabiliti i ruoli da adottare per questo sprint$^G$ che ha durata da oggi fino al 01-12-2025 compreso.\\
\begin{center}
	Lisa Casagrande $\to$ Responsabile$^G$ \\
	Marco Brunello, Laura Pieripolli, Matteo Mazzaretto $\to$ Amministratore\\
	Tommaso Ceron, Luca Marcuzzo $\to$ Analista$^G$ \\
\end{center}
\subsection{Definizione codici Analisi dei Requisiti}
Si è deciso di adottare i seguenti codici all'interno del documento di Analisi dei Requisiti$^G$ al fine di definire in maniera univoca e standardizzata ogni requisito che verrà riportato.
\begin{itemize}
	\item RF - \textit{Requisiti Funzionali}
	\item RN - \textit{Requisiti Non Funzionali}
	\item RV - \textit{Requisiti di Vincolo}
\end{itemize}
A ciascuno di essi viene affiancato un codice che ne identifica l'importanza: OB, DE, OP che indicano rispettivamente \textit{Vincolo Obbligatorio}, \textit{Vincolo Desiderabile} e \textit{Vincolo Opzionale}.\\
I codici verranno seguiti da un \#, ovvero un numero incrementale che permette di distinguere in modo univoco il requisito in questione.\\
L’identificatore completo è composto dalla tipologia del requisito, dal suo livello di priorità e dal numero identificativo; ad esempio un requisito funzionale, obbligatorio verrà segnalato come RF-OB\_\#
