% Configurazione
\documentclass{article}

\usepackage{titling} % Required for inserting the subtitle
\usepackage{graphicx} % Required for inserting images
\usepackage{tabularx} % Per l'ambiente tabularx (tabelle)
\usepackage{calc} % Sempre per le tabelle
\usepackage{tocloft}
\renewcommand{\cftsecleader}{\cftdotfill{\cftdotsep}}
\usepackage[hidelinks]{hyperref} % Per i collegamenti ipertestuali, ad esempio sulla table of contents
\usepackage{xcolor} % Per colorare il testo
\usepackage{colortbl} % Per colorare le celle delle tabelle
\usepackage{lipsum} % Per generare lorem ipsum
\usepackage[normalem]{ulem} % Per sottolineare il testo
\usepackage{array} % Per la visualizzazione fluttuante di array di domande e risposte
\usepackage{ragged2e} % Pacchetto necessario per \justifying che giustifica il testo di tabelle
\usepackage{placeins} % Impedisce a figure e tabelle di andare oltre le barriere definite
\usepackage{fancyhdr} % Intestazioni e piè di pagina
\usepackage{lastpage} % Per contare il numero totale di pagine
\usepackage{float}
\usepackage{enumitem}

\newcommand{\ulhref}[2]{\href{#1}{\underline{#2}}} % Sottolineatura
\newcommand{\ulref}[1]{\uline{\ref{#1}}} % Sottolinea riferimenti a figure
\setlength{\parindent}{0pt} % Nessun rientro dei paragrafi

\setcounter{secnumdepth}{4}

% Definizione del nuovo livello
\newcounter{subsubsubsection}[subsubsection]
\renewcommand{\thesubsubsubsection}{%
	\thesubsubsection.\arabic{subsubsubsection}}

\newcommand{\subsubsubsection}[1]{%
	\refstepcounter{subsubsubsection}%
	\par\medskip
	\textbf{\thesubsubsubsection\quad #1}\par
	\medskip
}

\graphicspath{ {immagini/} {../../../shared/images} }

% --- Impostazioni intestazione e piè di pagina ---
\pagestyle{fancy}
\fancyhf{}
% Intestazione standard
\fancyhead[L]{NullPointers Group}
\fancyhead[R]{Piano di Progetto}
\renewcommand{\headrulewidth}{0.4pt}
% Piè di pagina
\fancyfoot[C]{Pagina \thepage{} di \pageref{LastPage}}
\renewcommand{\footrulewidth}{0.4pt}

% --- Ridefinizione dello stile plain per gli indici ---
\fancypagestyle{plain}{
  \fancyhf{}
  \fancyhead[L]{NullPointers Group}
  \fancyhead[R]{Piano di Progetto}
  \fancyfoot[C]{Pagina \thepage{} di \pageref{LastPage}}
  \renewcommand{\headrulewidth}{0.4pt}
  \renewcommand{\footrulewidth}{0.4pt}
}

% --- Solo piè di pagina per la prima pagina ---
\fancypagestyle{firstpage}{
  \fancyhf{}
  \fancyfoot[C]{Pagina \thepage{} di \pageref{LastPage}}
  \renewcommand{\headrulewidth}{0pt}
  \renewcommand{\footrulewidth}{0.4pt}
}

%struttura
\begin{document}
\thispagestyle{firstpage} % Copertina con solo piè di pagina

%parte costante
\begin{minipage}{0.4\textwidth}
    \includegraphics[width=0.6\textwidth]{logo_unipd.png}
\end{minipage}
\begin{minipage}{0.55\textwidth}
    \textcolor{red}{\textbf{Università degli Studi di Padova}} \\
    \textcolor{red}{Laurea: Informatica} \\
    \textcolor{red}{Corso: Ingegneria del Software} \\
    \textcolor{red}{Anno Accademico: 2025/2026}
\end{minipage}

\begin{minipage}{0.4\textwidth}
    \includegraphics[width=0.6\textwidth]{logo_gruppo.jpeg}
\end{minipage}
\begin{minipage}{0.55\textwidth}
    \textbf{Gruppo: NullPointers Group} \\
    Email: \textsf{groupnullpointers@gmail.com}
\end{minipage}

\vspace{2cm}
{
		\centering
		\Huge\bfseries Norme di Progetto\par
		\vspace{1.0cm}
		\Large\bfseries \par
	}
\begin{center}
\begin{tabular}{r|l}
    Stato & In Approvazione \\[0.5em]
    Versione & 0.2.0 \\[0.5em]
    Data ultima modifica & 15/11/2025 \\[0.5em]
    Redattori & Lisa Casagrande \\
              & Matteo Mazzaretto \\[0.5em]
    Verificatori & Matteo Mazzaretto \\
                 & Tommaso Ceron \\[0.5em]
    Destinatari & Prof. Tullio Vardanega \\
    & Prof. Riccardo Cardin \\
    & Ergon Informatica Srl \\
    & NullPointers Group \\
\end{tabular}
\end{center}
\newpage
	
%variabile prima dell'indice
\section*{Registro delle modifiche}

\begin{table}[h]
	\centering
	\resizebox{\textwidth}{!}{
		\begin{tabular}{|c|c|c|c|c|}
			\hline
			\rowcolor[gray]{0.9}
			Vers & Data & Autore & Verificatore & Descrizione \\
			1.0.0 & 13-11 & L. Casagrande & L. Pieripolli & Approvazione documento \\
			\hline
			0.1.0 & 12-11 & L. Casagrande & M. Mazzaretto & Creazione e stesura documento \\
			\hline
	\end{tabular}}
\end{table}
\newpage
\hypersetup{linkcolor=black}
\tableofcontents

\newpage
\renewcommand{\listtablename}{Indice delle Tabelle}
\listoftables

\newpage
\renewcommand{\listfigurename}{Indice delle Immagini}
\listoffigures

%parte variabile
\newpage
\section{Introduzione}
	\subsection{Scopo del documento}
	Il presente documento nasce per descrivere il Way of Working$^G$ adottato da \textbf{\vr{NullPointers Group}} durante lo svolgimento del progetto$^G$ SmartOrder.\\
	Lo standard di riferimento è l'ISO/IEC 12207:1995, il quale prevede tre tipologie di processi.
	\begin{itemize}
	    \setlength\itemsep{-0.1em}
	    \item \textbf{Processi primari:} processi fondamentali senza i quali un progetto$^G$ non può definirsi tale;
	    \item \textbf{Processi di Supporto:} processi che coadiuvano i processi primari nello svolgimento delle rispettive azioni;
	    \item \textbf{Processi organizzativi:} processi di carattere più generale che aiutano la realizzazione del progetto$^G$.
	\end{itemize}

	La stesura di questo documento è incrementale, cioè una stesura passo passo con modifiche, aggiunte e cancellazioni a seguito di miglioramenti del metodo di lavoro. I membri del gruppo si impegnano a visionare costantemente questo documento e a rispettare rigorosamente le regole definite in esso, per svolgere il progetto$^G$ in modo professionale, coerente ed uniforme.

\subsection{Scopo del prodotto}
	La gestione automatizzata degli ordini di acquisto in contesti multicanale rappresenta una sfida complessa per le aziende moderne, che devono affrontare la necessità di interpretare richieste provenienti da fonti eterogenee come email, chat, messaggi vocali e immagini.\\
	Il capitolato$^G$ numero C8 di Ergon Informatica propone di sviluppare una piattaforma intelligente in grado di analizzare input multimodali e convertirli automaticamente in ordini strutturati, pronti per l'inserimento nei sistemi gestionali aziendali.

	L'obiettivo che si è posto questo gruppo è realizzare un sistema basato su architettura a microservizi che integri tecniche avanzate di intelligenza artificiale$^G$, machine learning$^G$ e natural language processing$^G$, in grado di riconoscere le intenzioni del cliente, estrarre le informazioni rilevanti e validarle in maniera coerente con il catalogo prodotti aziendale. Questo approccio consentirà di ridurre drasticamente l'intervento umano nelle fasi ripetitive, migliorando al contempo l'efficienza$^G$ complessiva e la soddisfazione del cliente finale.

	Il progetto$^G$ SmartOrder si propone quindi di dimostrare come le tecnologie di intelligenza artificiale$^G$ possano essere applicate con successo a processi reali di business, trasformando un compito complesso e frammentato in un flusso lineare, automatizzato e scalabile. L'obiettivo è realizzare questo progetto$^G$ entro il 30 Aprile 2026 con un budget a disposizione di: Euro 11.440.
	
\subsection{Glossario}
    La realizzazione di un sistema software complesso come SmartOrder richiede, ancora prima della scrittura del codice, un'importante operazione di confronto, analisi e progettazione$^G$. Per supportare e facilitare il lavoro asincrono tra i membri del gruppo e garantire una comunicazione efficace con il committente$^G$, tutte le informazioni derivanti da questa attività saranno appositamente documentate in un glossario condiviso, utile per evitare ambiguità o incomprensioni riguardanti la nomenclatura adottata in tutti i documenti visionabili.

    In accordo con quanto stabilito nel \href{https://nullpointersgroup.github.io/Documentazione/output/RTB/Verbali\%20Interni/2025-11-06\_verbale\_interno.pdf}{verbale interno del 6 novembre 2025}, si è deciso 
	di adottare il glossario come strumento ufficiale per la standardizzazione della terminologia di progetto$^G$ e di assegnare la responsabilità della sua manutenzione$^G$ alla figura dell'Analista$^G$.

    La nomenclatura utilizzata per segnalare che la definizione di una parola è contenuta nel glossario è la seguente: 
    \begin{center}
        termine$^{G}$
    \end{center}

    I termini sono ordinati alfabeticamente per facilitarne la consultazione e vengono inclusi sia termini tecnici che acronimi significativi.

    Il gruppo si impegna a visionare il Glossario periodicamente, per permettere la più completa comprensione di ogni tipo di documento pubblicato e per mantenere un allineamento semantico costante tra tutti i partecipanti al progetto$^G$. 

    \subsection{Riferimenti}
	\subsubsection{Riferimenti normativi}

	\begin{itemize}[itemsep=5pt, parsep=5pt, label=$\scriptstyle\bullet$]

    \item \textbf{Capitolato$^G$ C8 - Ergon Informatica Srl - SmartOrder}\\
    \url{https://www.math.unipd.it/~tullio/IS-1/2025/Progetto/C8.pdf}\\[3pt]
    \textbf{Ultima consultazione: 30 Novembre 2025}

	\end{itemize}

	\subsubsection{Riferimenti informativi}
	\begin{itemize}[itemsep=5pt, parsep=5pt, label=$\scriptstyle\bullet$]

    \item \textbf{Standard ISO/IEC 9126}\\
     \url{https://en.wikipedia.org/wiki/ISO/IEC\_9126}\\[3pt]
    \textbf{Ultima consultazione: 13 Dicembre 2025}

    \item \textbf{Standard ISO/IEC/IEEE$^G$ 12207:1995}\\
     \url{https://www.math.unipd.it/~tullio/IS-1/2009/Approfondimenti/ISO\_12207-1995.pdf}\\[3pt]
    \textbf{Ultima consultazione: 12 Dicembre 2025}

	\item \textbf{Glossario, versione 1.0.0}\\
     \url{https://nullpointersgroup.github.io/Documentazione/output/RTB/Documenti\%20Interni/Glossario.pdf}\\[3pt]
    \textbf{Ultima consultazione: 13 Dicembre 2025}

	\end{itemize}


\newpage
\section{Analisi e gestione dei rischi}
In questa sezione il gruppo descrive l'analisi dei potenziali rischi associati al progetto.\\
Per mitigare tali rischi, abbiamo scelto di adottare un consuntivo$^G$ più contenuto all'inizio di ogni sprint, in particolare nelle fasi iniziali del progetto quando la padronanza delle metodologie di lavoro è minore.\\
Ogni rischio è stato identificato ed analizzato, e sono state definite apposite strategie di mitigazione e controllo.\\
I rischi possono appartenere a tre macro–categorie: 
\begin{itemize}[itemsep=5pt, parsep=1pt, label=$\scriptstyle\bullet$]
    \item \textbf{RE}(Rischi Economici): rischi che influenzano tempi, costi, produttività o impegno orario del gruppo.
    \item \textbf{RP}(Rischi Personali): rischi legati ai singoli membri, alle loro attività, disponibilità, competenze e condizioni individuali.
    \item \textbf{RA}(Rischi Ambientali): rischi dovuti a fattori esterni, imprevisti o condizioni non controllabili direttamente dal gruppo.
\end{itemize} 
Ciascun rischio è identificato tramite un codice che ne identifica la categoria e un numero incrementale  \#.\\ (es. \textbf{RE\#}, \textbf{RP\#}, \textbf{RA\#}).\\
Ogni rischio possiede una frequenza probabile di avvenimento e un impatto, espressi su una scala: \{Molto basso, Basso, Medio, Elevato, Molto elevato\}.

\subsection{RE1: Rischi tecnologici}
\vspace{2mm}
\begin{table}[H]
	\centering
	\renewcommand{\arraystretch}{1.15}
	\resizebox{\textwidth}{!}{
		\begin{tabular}{|>{\bfseries}m{0.25\textwidth}|m{0.65\textwidth}|}
			\hline
			\multicolumn{1}{|c|}{\textbf{Tipologia Dato}} & \multicolumn{1}{c|}{\textbf{Descrizione}} \\ \hline
			Codice & RE1 \\ \hline
			Nome & Tecnologie \\ \hline
			Descrizione & Il gruppo non ha esperienza in progetti di queste dimensioni con front-end$^G$ e back-end$^G$ complessi e tecnologie eterogenee. Sono inclusi possibili errori e lunghe sessioni di debugging$^G$. \\ \hline
			Mitigazione & All'inizio di ogni sprint$^G$ si valuta il livello di conoscenza delle tecnologie, riducendo le ore produttive previste per includere il tempo di apprendimento. \\ \hline
			Frequenza \newline probabile di \newline avvenimento & Elevata \\ \hline
			Ripercussioni & Elevate \\ \hline
	\end{tabular}}
	\caption{Informazioni sul rischio RE1}
\end{table}

\subsection{RE2: Sovraccarico o stima errata del lavoro}
\vspace{2mm}
\begin{table}[H]
	\centering
	\renewcommand{\arraystretch}{1.15}
	\resizebox{\textwidth}{!}{
		\begin{tabular}{|>{\bfseries}m{0.25\textwidth}|m{0.65\textwidth}|}
			\hline
			\multicolumn{1}{|c|}{\textbf{Tipologia Dato}} & \multicolumn{1}{c|}{\textbf{Descrizione}} \\ \hline
			Codice & RE2 \\ \hline
			Nome & Stima errata delle attività \\ \hline
			Descrizione & Le task$^G$ possono richiedere più tempo del previsto, causando ritardi negli sprint$^G$ e sovraccarico dei membri del gruppo. \\ \hline
			Mitigazione & Pianificare task$^G$ realistiche e suddividere task$^G$ grandi in sotto-task. \\ \hline
			Frequenza \newline probabile di \newline avvenimento & Media \\ \hline
			Ripercussioni & Elevate \\ \hline
	\end{tabular}}
	\caption{Informazioni sul rischio RE2}
\end{table}



\subsection{RP1: Attività universitarie}
\vspace{2mm}
\begin{table}[H]
	\centering
	\renewcommand{\arraystretch}{1.15}
	\begin{tabular}{|>{\bfseries}m{0.25\textwidth}|m{0.65\textwidth}|}
		\hline
		\multicolumn{1}{|c|}{\textbf{Tipologia Dato}} & \multicolumn{1}{c|}{\textbf{Descrizione}} \\ \hline
		Codice & RP1 \\ \hline
		Nome & Attività universitarie \\ \hline
		Descrizione & Tutti i componenti del gruppo hanno almeno un altro esame oltre a questo corso di Ingegneria del Software. \\ \hline
		Mitigazione & All'inizio di ogni sprint$^G$ si tiene conto dell'impegno richiesto da altri corsi, riducendo il consuntivo$^G$ orario per renderlo più realistico. \\ \hline
		Frequenza \newline probabile di \newline avvenimento & Media \\ \hline
		Ripercussioni & Elevate \\ \hline
	\end{tabular}
	\caption{Informazioni sul rischio RP1}
\end{table}

\subsection{RP2: Attività extra-universitarie}
\vspace{2mm}
\begin{table}[H]
	\centering
	\renewcommand{\arraystretch}{1.15}
	\resizebox{\textwidth}{!}{
	\begin{tabular}{|>{\bfseries}m{0.25\textwidth}|m{0.65\textwidth}|}
		\hline
		\multicolumn{1}{|c|}{\textbf{Tipologia Dato}} & \multicolumn{1}{c|}{\textbf{Descrizione}} \\ \hline
		Codice & RP2 \\ \hline
		Nome & Attività extra-universitarie \\ \hline
		Descrizione & Essendo un gruppo a intensità media, le nostre attività extra-universitarie (lavoro, sport, impegni personali) sono frequenti. \\ \hline
		Mitigazione & All'inizio di ogni sprint$^G$ si tiene conto degli impegni extra, riducendo il consuntivo$^G$ orario per renderlo coerente alla disponibilità reale. \\ \hline
		Frequenza \newline probabile di \newline avvenimento & Elevata \\ \hline
		Ripercussioni & Medie \\ \hline
	\end{tabular}}
	\caption{Informazioni sul rischio RP2}
\end{table}

\subsection{RP3: Incomprensioni}
\vspace{2mm}
\begin{table}[H]
	\centering
	\renewcommand{\arraystretch}{1.15}
	\resizebox{\textwidth}{!}{
	\begin{tabular}{|>{\bfseries}m{0.25\textwidth}|m{0.65\textwidth}|}
		\hline
		\multicolumn{1}{|c|}{\textbf{Tipologia Dato}} & \multicolumn{1}{c|}{\textbf{Descrizione}} \\ \hline
		Codice & RP3 \\ \hline
		Nome & Incomprensioni \\ \hline
		Descrizione & Un membro può svolgere male una task$^G$ per incomprensioni o una definizione poco chiara del compito. \\ \hline
		Mitigazione & Il gruppo deve concordare task$^G$ sempre chiare e ogni membro deve verificare di aver compreso correttamente cosa deve fare. \\ \hline
		Frequenza \newline probabile di \newline avvenimento & Molto bassa \\ \hline
		Ripercussioni & Elevate \\ \hline
	\end{tabular}}
	\caption{Informazioni sul rischio RP3}
\end{table}



\subsection{RA1: Imprevisti}
\vspace{2mm}
\begin{table}[H]
	\centering
	\renewcommand{\arraystretch}{1.15}
	\resizebox{\textwidth}{!}{
	\begin{tabular}{|>{\bfseries}m{0.25\textwidth}|m{0.65\textwidth}|}
		\hline
		\multicolumn{1}{|c|}{\textbf{Tipologia Dato}} & \multicolumn{1}{c|}{\textbf{Descrizione}} \\ \hline
		Codice & RA1 \\ \hline
		Nome & Imprevisti \\ \hline
		Descrizione & È naturale che possano verificarsi imprevisti di tipo universitario, familiare o personale, che influenzano la produttività. \\ \hline
		Mitigazione & La persona coinvolta deve avvisare tempestivamente il gruppo per permettere una ridistribuzione del lavoro. \\ \hline
		Frequenza \newline probabile di \newline avvenimento & Bassa \\ \hline
		Ripercussioni & Molto elevate \\ \hline
	\end{tabular}}
	\caption{Informazioni sul rischio RA1}
\end{table}

\subsection{RA2: Rischi legati ai requisiti}
\vspace{2mm}
\begin{table}[H]
	\centering
	\renewcommand{\arraystretch}{1.15}
	\resizebox{\textwidth}{!}{
	\begin{tabular}{|>{\bfseries}m{0.25\textwidth}|m{0.65\textwidth}|}
		\hline
		\multicolumn{1}{|c|}{\textbf{Tipologia Dato}} & \multicolumn{1}{c|}{\textbf{Descrizione}} \\ \hline
		Codice & RA2 \\ \hline
		Nome & Requisiti instabili o incompleti \\ \hline
		Descrizione & I requisiti possono essere interpretati in modo errato o modificati durante il progetto, causando ritardi e lavoro non coerente con le aspettative. \\ \hline
		Mitigazione & Effettuare incontri chiarificatori con il proponente, validare i requisiti dopo ogni sprint$^G$ e produrre documentazione condivisa. \\ \hline
		Frequenza \newline probabile di \newline avvenimento & Media \\ \hline
		Ripercussioni & Elevate \\ \hline
	\end{tabular}}
	\caption{Informazioni sul rischio RA2}
\end{table}




\newpage
\section{Pianificazione nel lungo termine}

\subsection{Introduzione}
Come definito nel \href{https://nullpointersgroup.github.io/Documentazione/output/Candidatura/Preventivo_Costi.pdf}{Preventivo Costi}, il gruppo si impegna a concludere il progetto$^G$ entro il \textbf{30/04/2026}, con un costo insormontabile di \textbf{11.440€}.\\
Il preventivo iniziale, già definito nel documento menzionato precedentemente, è il seguente:\\
\begin{table}[h!]
	\centering
	\begin{tabular}{|c|c|c|c|}
		\hline
		\rowcolor{gray!25}
		Ruolo & Ore totali & Costo (€/h) & Costo totale (€) \\ \hline
		Responsabile$^G$ & 50 & 30 & 1500 \\ \hline
		Amministratore$^G$ & 42 & 20 & 840 \\ \hline
		Analista$^G$ & 95 & 25 & 2375 \\ \hline
		Progettista$^G$ & 125 & 25 & 3125 \\ \hline
		Verificatore$^G$ & 115 & 15 & 1725 \\ \hline
		Programmatore$^G$ & 125 & 15 & 1875 \\ \hline
	\end{tabular}
	\caption{Ore per ciascun ruolo e calcolo dei costi totali}
\end{table}

La data di consegna prevista per la RTB$^G$ è il \textbf{15/02/2026}.

\newpage

\subsection{Attività previste per la RTB}
Le attività previste per la RTB$^G$ (Requirements and Technology Baseline$^G$) sono riassunte nella seguente tabella:\\
\begin{table}[H]
	\centering
	\resizebox{\textwidth}{!}{
	\renewcommand{\arraystretch}{1.15}
	\begin{tabular}{|p{0.3\textwidth}|p{0.7\textwidth}|}
		\hline
		\rowcolor[gray]{0.9}
		\textbf{Attività} & \textbf{Descrizione} \\
		\hline
		\textbf{Analisi dei Requisiti$^G$} & Documento esterno che ha lo scopo di definire i requisiti funzionali, requisiti non funzionali e requisiti di vincolo . Inoltre illustra tutti i casi d'uso relativi a requisiti obbligatori, non obbligatori e di sistema. \\
		\hline
		\textbf{Piano di Progetto$^G$} & Documento esterno che esprime le attività svolte e da svolgere durante il progetto$^G$. \\
		\hline
		\textbf{Piano di Qualifica$^G$} &  Documento esterno che rappresenta un documento fondamentale per la gestione e il monitoraggio continuo della qualità del progetto$^G$ e dei processi coinvolti nel suo
		ciclo di vita. \\
		\hline
		\textbf{Norme di Progetto$^G$} & Documento interno che descrive il Way of Working$^G$ del gruppo nel dettaglio. \\
		\hline
		\textbf{Glossario} & Documento interno che permette di facilitare la comprensione di parole tecniche nella comunicazione fra il gruppo e la proponente$^G$. \\
		\hline
		\textbf{Proof of Concept$^G$} & Eseguibile che dimostra la coesione fra le tecnologie scelte. \\
		\hline
		\textbf{Lettera di presentazione RTB$^G$} & Simile alla Lettera di Presentazione per la Candidatura, esprime la volontà del gruppo di partecipare alla revisione per la RTB$^G$. \\
		\hline
	\end{tabular}}
\end{table}
Ogni attività che inizia, continua o finalizza la stesura delle documentazioni menzionate nella tabella è definita all'interno di ogni sprint$^G$ nella sezione \vr{Attività da svolgere}.

\newpage

\subsection{Attività previste per la PB}
Questa sezione verrà redatta dopo il superamento della RTB$^G$, in modo da consentire la scrittura più accurata durante la PB$^G$ (Product Baseline$^G$)

\newpage
\section{Pianificazione a breve termine}

\subsection{Introduzione}
NullPointers Group ha stabilito di procedere con un approccio Agile$^G$ allo svolgimento del progetto, definendo la durata dello Sprint$^G$ di 2 settimane per avere un'organizzazione migliore.\\
Seguendo questo principio, il gruppo si impegna a stabilire all’inizio di ogni sprint i ruoli all'interno del gruppo, che ruotano bisettimanalmente, e le attività per le settimane successive, mantenendo però la possibilità di effettuare il cambiamento qualora le esigenze organizzative lo imponessero.\\
Per verificare la continua e corretta prosecuzione delle attività, rimarremo in contatto con l'azienda \textbf{Ergon Informatica s.r.l.} sia tramite incontri telematici su \textbf{Google Meet}$^G$ sia tramite i canali di comunicazione diretti concordati.
Seguiranno ora le descrizioni dei vari periodi di lavoro, nelle quali verranno esposte le sezioni:
\begin{enumerate}
	\item Informazioni generali
	\item Attività da svolgere
	\item Prospetto consumo tempo e costi
	\item Rischi attesi
	\item Consumo tempo e costi effettivo
	\item Aggiornamento delle risorse rimanenti
	\item Retrospettiva, comprendente anche i rischi effettivamente riscontrati
\end{enumerate}

\newpage
\subsection{Requirements and Technology Baseline}

\subsubsection{Sprint 1}

\subsubsubsection{Informazioni generali e attività da svolgere}

\begin{tabular}{p{0.25\textwidth} p{0.2\textwidth}}
	\textbf{Inizio} & 6/11/2025 \\
	\textbf{Fine prevista} & 17/11/2025 \\
	\textbf{Fine reale} & 17/11/2025 \\
	\textbf{Giorni di ritardo} & 0
\end{tabular}
\newline \newline
Questo primo periodo dura 11 giorni invece di 2 settimane in modo da permetterci di allinearci al lunedì come data di fine sprint, siccome il 6/11 era di venerdì e il secondo lunedì disponibile cascava il 17/11.\\
Le nostre attività da svolgere, definite nel \underline{\href{https://nullpointersgroup.github.io/Documentazione/output/RTB/Verbali\%20Interni/2025-11-06_verbale_interno.pdf}{Verbale interno del 6/11/2025}} e nel \underline{\href{https://nullpointersgroup.github.io/Documentazione/output/RTB/Verbali\%20Interni/2025-11-12_verbale_interno.pdf}{Verbale interno del 12/11/2025}} sono:
\begin{itemize}
	\item Inizio stesura Analisi dei Requisiti
	\item Creazione Glossario
	\item Miglioramento sistema versionamento documenti
	\item Stesura diario di bordo 10/11 e 17/11
	\item Inizio stesura Norme di Progetto
	\item Creazione HTML e CSS per visualizzazione Glossario
	\item Creazione script per la trascrizione del glossario da .tex a .html
	\item Implementazione Branch Protection
\end{itemize}


\subsubsubsection{Rischi attesi}
I possibili rischi attesi sono:
\begin{itemize}
	\item R1: alcuni componenti del gruppo sono occupati con lo studio di altri corsi
	\item R3: un imprevisto a qualche componente
	\item R4: per l'implementazione della Branch Protection
	\item R5: rischio d'incomprensione del modo corretto con il quale stendere i documenti
\end{itemize}

\subsubsubsection{Preventivo}
Si prospetta l'utilizzo delle seguenti risorse:\newpage
\begin{table}[H]
	\centering
	% Prima tabella senza bordi per le intestazioni ruotate
	\begin{tabular}{p{0.2\textwidth}p{0.12\textwidth}p{0.12\textwidth}p{0.12\textwidth}p{0.12\textwidth}p{0.12\textwidth}p{0.12\textwidth}}
		& \rotatebox{45}{\textbf{Responsabile}} &
		\rotatebox{45}{\textbf{Amministratore}} & \rotatebox{45}{\textbf{Analista}} & \rotatebox{45}{\textbf{Programmatore}} & \rotatebox{45}{\textbf{Verificatore}} &
		\rotatebox{45}{\textbf{Progettista}} \\
	\end{tabular}
	
	\vspace{0.2cm} % Piccolo spazio tra le due tabelle
	
	% Seconda tabella con i dati e i bordi
	\begin{tabular}{|p{0.2\textwidth}|p{0.12\textwidth}|p{0.12\textwidth}|p{0.12\textwidth}|p{0.12\textwidth}|p{0.12\textwidth}|p{0.12\textwidth}|}
		\hline
		M. Mazzaretto & - & 6 & - & - & 1 & - \\ \hline
		T. Ceron      & - & 6 & - & - & 1 & - \\ \hline
		L. Pieripolli & 4 & - & - & - & 1 & - \\ \hline
		L. Marcuzzo   & - & - & 4 & - & - & - \\ \hline
		M. Brunello   & - & 6 & - & - & 1 & - \\ \hline
		L. Casagrande & - & - & 6 & - & - & - \\ \hline
	\end{tabular}
	\caption{Sprint 1: Preventivo}
\end{table}

È importante notare che sono considerate poche ore produttive poiché la maggior parte delle ore effettivamente previste sarebbero state dedicate allo studio della documentazione e al corretto modo di lavoro piuttosto che l'effettiva stesura.\\
Oltretutto L. Pieripolli e L. Marcuzzo hanno avuto un preventivo di ore minore poiché impegnati con lo studio per un esame parziale di un altro corso.

\begin{figure}[H]
	\centering
	\includegraphics[width=0.8\textwidth]{PianoProgetto/sprint01_preventivo}
	\caption{Sprint 1: Preventivo}
\end{figure}

\subsubsubsection{Consuntivo}
Le risorse effettivamente utilizzate sono nella seguente tabella:\\
\begin{table}[H]
	\centering
	% Prima tabella senza bordi per le intestazioni ruotate
	\begin{tabular}{p{0.2\textwidth}p{0.12\textwidth}p{0.12\textwidth}p{0.12\textwidth}p{0.12\textwidth}p{0.12\textwidth}p{0.12\textwidth}}
		& \rotatebox{45}{\textbf{Responsabile}} &
		\rotatebox{45}{\textbf{Amministratore}} & \rotatebox{45}{\textbf{Analista}} &
		\rotatebox{45}{\textbf{Progettista}} & \rotatebox{45}{\textbf{Verificatore}} & \rotatebox{45}{\textbf{Programmatore}} \\
	\end{tabular}
	
	\vspace{0.2cm} % Piccolo spazio tra le due tabelle
	
	% Seconda tabella con i dati e i bordi
	\begin{tabular}{|p{0.2\textwidth}|p{0.12\textwidth}|p{0.12\textwidth}|p{0.12\textwidth}|p{0.12\textwidth}|p{0.12\textwidth}|p{0.12\textwidth}|}
		\hline
		M. Mazzaretto & - & 4 & - & - & 1 & - \\ \hline
		T. Ceron      & - & 3.5 & - & - & 1 & - \\ \hline
		L. Pieripolli & 3 & - & - & - & 1 & - \\ \hline
		L. Marcuzzo   & - & - & 2.5 & - & - & - \\ \hline
		M. Brunello   & - & 5.5 & - & - & 1 & - \\ \hline
		L. Casagrande & - & - & 5 & - & - & - \\ \hline
	\end{tabular}
	\caption{Sprint 1: Consuntivo}
\end{table}

C'è stata una sovrastima delle ore produttive causate dal continuo studio della documentazione ma la poca messa in opera dell'effettiva stesura.\\ L'implementazione della Branch Protection ha portato ad un aumento di ore a M. Brunello maggiore dei suoi compagni a causa della sottostima del rischio tecnologico.

\begin{figure}[H]
	\centering
	\includegraphics[width=0.8\textwidth]{PianoProgetto/sprint01_consuntivo}
	\caption{Sprint 1: Consuntivo}
\end{figure}

\subsubsubsection{Aggiornamento delle risorse rimanenti}

\begin{table}[H]
	\centering
	\begin{tabular}{|p{0.185\textwidth}|p{0.1\textwidth}|p{0.05\textwidth}|p{0.1\textwidth}|p{0.13\textwidth}|p{0.12\textwidth}|}
		\hline
		\rowcolor{gray!25}
		Ruolo & Costo & Ore & Costo \newline effettivo & Ore \newline rimanenti & Budget \newline rimanente \\ \hline
		Responsabile   & 30(€/h) & 3 & 90 & 47 (-3) & 1410 (-90) \\ \hline
		Amministratore & 20(€/h) & 13 & 260 & 29 (-13) & 580 \newline (-260) \\ \hline
		Analista       & 25(€/h) & 7.5 & 187.5 & 87.5 (-7.5) & 2187.5 \newline (-187.5) \\ \hline
		Progettista    & 25(€/h) & - & - & 125 & 3125 \\ \hline
		Verificatore   & 15(€/h) & 4 & 60 & 111 (-4) & 1665 (-60) \\ \hline
		Programmatore  & 15(€/h) & - & - & 125 & 1875 \\ \hline
		Totale         & - & 27.5 & 597.5 & 524.5 (-27.5) & 10842.5 \newline (-597.5) \\ \hline
	\end{tabular}
	\caption{Sprint 1: Aggiornamento risorse}
\end{table}

\subsubsubsection{Rischi incontrati}
Sono stati incontrati i rischi R1 ed R4, già definiti nei rischi attesi.

\subsubsubsection{Retrospettiva}
Al termine dello Sprint, è stata confermata la chiusura di tutte le attività.

\newpage
\subsubsection{Sprint$^G$ 2}

\subsubsubsection{Informazioni generali e attività da svolgere}

\begin{tabular}{p{0.25\textwidth} p{0.2\textwidth}}
	\textbf{Inizio} & 17/11/2025 \\
	\textbf{Fine prevista} & 01/12/2025 \\
	\textbf{Fine reale} & \\
	\textbf{Giorni di ritardo} &
\end{tabular}
\newline \newline 
Le nostre attività da svolgere, definite nel \underline{\href{https://nullpointersgroup.github.io/Documentazione/output/RTB/Verbali\%20Interni/2025-11-17_verbale_interno.pdf}{Verbale interno del 17/11/2025}} e nel \underline{\href{https://nullpointersgroup.github.io/Documentazione/output/RTB/Verbali\%20Interni/2025-11-24_verbale_interno.pdf}{Verbale interno del 24/11/2025}} sono:
\begin{itemize}[itemsep=5pt, parsep=1pt, label=$\scriptstyle\bullet$]
	\item Inizio stesura Piano di Progetto
	\item Continuazione stesura Norme di Progetto
	\item Stesura diario di bordo 24/11 e 1/12
	\item Aggiornamento Glossario
	\item Continuazione individuazione Casi d'Uso e Requisiti
\end{itemize}


\subsubsubsection{Rischi attesi}
I possibili rischi attesi sono:
\begin{itemize}[itemsep=5pt, parsep=1pt, label=$\scriptstyle\bullet$]
	\item RA1: un imprevisto a qualche componente
	\item RP3: rischio d'incomprensione di qualche task
\end{itemize}


\subsubsubsection{Preventivo}
Si prospetta l'utilizzo delle seguenti risorse:\newpage
\begin{table}[H]
	\centering
	% Prima tabella senza bordi per le intestazioni ruotate
	\begin{tabular}{p{0.2\textwidth}p{0.12\textwidth}p{0.12\textwidth}p{0.12\textwidth}p{0.12\textwidth}p{0.12\textwidth}p{0.12\textwidth}}
		& \rotatebox{45}{\textbf{Responsabile}} &
		\rotatebox{45}{\textbf{Amministratore}} & \rotatebox{45}{\textbf{Analista}} & \rotatebox{45}{\textbf{Programmatore}} & \rotatebox{45}{\textbf{Verificatore}} &
		\rotatebox{45}{\textbf{Progettista}} \\
	\end{tabular}
	
	\vspace{0.2cm} % Piccolo spazio tra le due tabelle
	
	% Seconda tabella con i dati e i bordi
	\begin{tabular}{|p{0.2\textwidth}|p{0.12\textwidth}|p{0.12\textwidth}|p{0.12\textwidth}|p{0.12\textwidth}|p{0.12\textwidth}|p{0.12\textwidth}|}
		\hline
		M. Mazzaretto & - & 2 & 3 & - & 1 & - \\ \hline
		T. Ceron      & - & - & 6 & - & 1 & - \\ \hline
		L. Pieripolli & - & 5 & - & - & 1 & - \\ \hline
		L. Marcuzzo   & - & - & 6 & - & 1 & - \\ \hline
		M. Brunello   & - & 5 & - & - & - & - \\ \hline
		L. Casagrande & 4 & - & - & - & - & - \\ \hline
	\end{tabular}
	\caption{Sprint$^G$ 2: Preventivo}
\end{table}

\begin{figure}[H]
	\centering
	\includegraphics[width=0.8\textwidth]{PianoProgetto/sprint01_preventivo}
	\caption{Sprint$^G$ 2: Preventivo}
\end{figure}

\subsubsubsection{Consuntivo}

\subsubsubsection{Aggiornamento delle risorse rimanenti}

\subsubsubsection{Rischi incontrati}

\subsubsubsection{Retrospettiva}


\end{document}
