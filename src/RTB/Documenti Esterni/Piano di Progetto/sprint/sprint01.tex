\subsubsection{Sprint 1}

\subsubsubsection{Informazioni generali e attività da svolgere}

\begin{tabular}{p{0.25\textwidth} p{0.2\textwidth}}
	\textbf{Inizio} & 6/11/2025 \\
	\textbf{Fine prevista} & 17/11/2025 \\
	\textbf{Fine reale} & 17/11/2025 \\
	\textbf{Giorni di ritardo} & 0
\end{tabular}
\newline \newline
Questo primo periodo ha una durata di 11 giorni invece di 2 settimane in modo da permettere al gruppo di allinearsi al lunedì come data di chiusura dello sprint$^G$.\\
Le nostre attività da svolgere, definite nel \href{https://nullpointersgroup.github.io/Documentazione/output/RTB/Verbali\%20Interni/2025-11-06_verbale_interno.pdf}{Verbale interno del 6/11/2025} e nel \href{https://nullpointersgroup.github.io/Documentazione/output/RTB/Verbali\%20Interni/2025-11-12_verbale_interno.pdf}{Verbale interno del 12/11/2025} sono:
\begin{itemize}[itemsep=5pt, parsep=1pt, label=$\scriptstyle\bullet$]
	\item Inizio stesura Analisi dei Requisiti$^G$
	\item Creazione Glossario
	\item Miglioramento sistema versionamento$^G$ documenti
	\item Stesura diario di bordo 10/11 e 17/11
	\item Inizio stesura Norme di Progetto$^G$
	\item Creazione HTML e CSS per visualizzazione Glossario
	\item Creazione script per la trascrizione del glossario da .tex a .html
	\item Implementazione Branch$^G$ Protection
\end{itemize}


\subsubsubsection{Rischi attesi}
I possibili rischi attesi sono:
\begin{itemize}[itemsep=5pt, parsep=1pt, label=$\scriptstyle\bullet$]
	\item RP1: alcuni componenti del gruppo sono occupati con lo studio di altri corsi
	\item RA1: un imprevisto a qualche componente
	\item RE1: per l'implementazione della Branch$^G$ Protection
	\item RE2: rischio stima errata del lavoro
	\item RP3: rischio d'incomprensione del modo corretto con il quale stendere i documenti
\end{itemize}

\subsubsubsection{Preventivo}
Si prospetta l'utilizzo delle seguenti risorse:\newpage
\begin{table}[H]
	\centering
	% Prima tabella senza bordi per le intestazioni ruotate
	\begin{tabular}{p{0.2\textwidth}p{0.12\textwidth}p{0.12\textwidth}p{0.12\textwidth}p{0.12\textwidth}p{0.12\textwidth}p{0.12\textwidth}}
		& \rotatebox{45}{\textbf{Responsabile$^G$}} &
		\rotatebox{45}{\textbf{Amministratore$^G$}} & \rotatebox{45}{\textbf{Analista$^G$}} & \rotatebox{45}{\textbf{Programmatore$^G$}} & \rotatebox{45}{\textbf{Verificatore$^G$}} &
		\rotatebox{45}{\textbf{Progettista$^G$}} \\
	\end{tabular}
	
	\vspace{0.2cm} % Piccolo spazio tra le due tabelle
	
	% Seconda tabella con i dati e i bordi
	\begin{tabular}{|p{0.2\textwidth}|p{0.12\textwidth}|p{0.12\textwidth}|p{0.12\textwidth}|p{0.12\textwidth}|p{0.12\textwidth}|p{0.12\textwidth}|}
		\hline
		M. Mazzaretto & - & 6 & - & - & 1 & - \\ \hline
		T. Ceron      & - & 6 & - & - & 1 & - \\ \hline
		L. Pieripolli & 4 & - & - & - & 1 & - \\ \hline
		L. Marcuzzo   & - & - & 4 & - & - & - \\ \hline
		M. Brunello   & - & 6 & - & - & 1 & - \\ \hline
		L. Casagrande & - & - & 6 & - & - & - \\ \hline
	\end{tabular}
	\caption{Sprint 1: Preventivo}
\end{table}

Il totale preventivato per lo sprint$^G$ è \textbf{790€}.\\
È importante notare che sono considerate poche ore produttive poiché la maggior parte delle ore effettivamente previste sarebbero state dedicate allo studio della documentazione e al corretto modo di lavoro piuttosto che l'effettiva stesura.\\
Oltretutto L. Pieripolli e L. Marcuzzo hanno avuto un preventivo di ore minore poiché impegnati con lo studio per un esame parziale di un altro corso.

\begin{figure}[H]
	\centering
	\includegraphics[width=0.8\textwidth]{PianoProgetto/sprint01_preventivo}
	\caption{Sprint 1: Preventivo}
\end{figure}

\subsubsubsection{Consuntivo}
Le risorse effettivamente utilizzate sono nella seguente tabella:\\
\begin{table}[H]
	\centering
	% Prima tabella senza bordi per le intestazioni ruotate
	\begin{tabular}{p{0.2\textwidth}p{0.12\textwidth}p{0.12\textwidth}p{0.12\textwidth}p{0.12\textwidth}p{0.12\textwidth}p{0.12\textwidth}}
		& \rotatebox{45}{\textbf{Responsabile$^G$}} &
		\rotatebox{45}{\textbf{Amministratore$^G$}} & \rotatebox{45}{\textbf{Analista$^G$}} &
		\rotatebox{45}{\textbf{Programmatore$^G$}} & \rotatebox{45}{\textbf{Verificatore$^G$}} & \rotatebox{45}{\textbf{Progettista$^G$}} \\
	\end{tabular}
	
	\vspace{0.2cm} % Piccolo spazio tra le due tabelle
	
	% Seconda tabella con i dati e i bordi
	\begin{tabular}{|p{0.2\textwidth}|p{0.12\textwidth}|p{0.12\textwidth}|p{0.12\textwidth}|p{0.12\textwidth}|p{0.12\textwidth}|p{0.12\textwidth}|}
		\hline
		M. Mazzaretto & - & 4 \textcolor{green}{(-2)} & - & - & 1 & - \\ \hline
		T. Ceron      & - & 3.5 \textcolor{green}{(-2.5)} & - & - & 1 & - \\ \hline
		L. Pieripolli & 3 \textcolor{green}{(-1)} & - & - & - & 1 & - \\ \hline
		L. Marcuzzo   & - & - & 2.5 \textcolor{green}{(-1.5)} & - & - & - \\ \hline
		M. Brunello   & - & 5.5 \textcolor{green}{(-0.5)} & - & - & 1 & - \\ \hline
		L. Casagrande & - & - & 5 \textcolor{green}{(-1)} & - & - & - \\ \hline
	\end{tabular}
	\caption{Sprint 1: Consuntivo}
\end{table}

C'è stata una sovrastima delle ore produttive causate dal continuo studio della documentazione ma la poca messa in opera dell'effettiva stesura.\\ 
L'implementazione della Branch$^G$ Protection ha portato ad un aumento di ore a M. Brunello maggiore dei suoi compagni a causa della sottostima del rischio tecnologico.

\begin{figure}[H]
	\centering
	\includegraphics[width=0.8\textwidth]{PianoProgetto/sprint01_consuntivo}
	\caption{Sprint 1: Consuntivo}
\end{figure}

\subsubsubsection{Aggiornamento delle risorse rimanenti}

\begin{table}[H]
	\centering
	\begin{tabular}{|p{0.2\textwidth}|p{0.1\textwidth}|p{0.05\textwidth}|p{0.1\textwidth}|p{0.13\textwidth}|p{0.12\textwidth}|}
		\hline
		\rowcolor{gray!25}
		Ruolo & Costo & Ore & Costo \newline effettivo & Ore \newline rimanenti & Budget \newline rimanente \\ \hline
		Responsabile$^G$   & 30(€/h) & 3 & 90 & 47 \textcolor{red}{(-3)} & 1410 \textcolor{red}{(-90)} \\ \hline
		Amministratore$^G$ & 20(€/h) & 13 & 260 & 29 \textcolor{red}{(-13)} & 580 \newline \textcolor{red}{(-260)} \\ \hline
		Analista$^G$       & 25(€/h) & 7.5 & 187.5 & 87.5 \textcolor{red}{(-7.5)} & 2187.5 \newline \textcolor{red}{(-187.5)} \\ \hline
		Progettista$^G$    & 25(€/h) & - & - & 125 & 3125 \\ \hline
		Verificatore$^G$   & 15(€/h) & 4 & 60 & 111 \textcolor{red}{(-4)} & 1665 \textcolor{red}{(-60)} \\ \hline
		Programmatore$^G$  & 15(€/h) & - & - & 125 & 1875 \\ \hline
		Totale         & - & 27.5 & 597.5 & 524.5 \textcolor{red}{(-27.5)} & 10842.5 \newline \textcolor{red}{(-597.5)} \\ \hline
	\end{tabular}
	\caption{Sprint 1: Aggiornamento risorse}
\end{table}

\subsubsubsection{Rischi incontrati}
Sono stati incontrati i rischi RP1, RE1 ed RE2, già definiti nei rischi attesi. \\
Infatti, il primo rischio era di sicuro avvenimento in quanto lo studio per il parziale non era rimandabile.\\
Il secondo rischio invece è stato incontrato in quanto l'implementazione della Branch$^G$ Protection è stata molto più complessa del previsto.

\subsubsubsection{Efficacia delle strategie di gestione di mitigazione dei rischi}
Per il rischio RP1, è stata adottata la strategia di ridurre il carico di lavoro, diminuendo il livello di difficoltà delle attività assegnate a tutti i membri del gruppo. Questo per evitare di non completare lo Sprint$^G$ a causa dell'assegnazione di attività troppo ampie o impossibili da portare a termine con un numero ridotto di componenti.\\
Per il rischio RE1, trattandosi della nostra prima esperienza con un livello di complessità simile, abbiamo deciso di utilizzare un metodo simile anche nella repository$^G$ del codice sorgente del prodotto, in modo da trarre doppio vantaggio da quest'implementazione.\\
Per il rischio RE2, come gruppo cerchiamo di migliorare Sprint$^G$ per Sprint$^G$ nella scrittura del preventivo per renderlo più accurato, poiché le ore produttive sono state decisamente minori rispetto a quelle aspettate.

\subsubsubsection{Migliorie da attuare per le attività future}
In questo sprint$^G$ ci si è molto focalizzati sul miglioramento degli script utilizzati dalle GitHub$^G$ Action$^G$ e sull'implementazione della Branch$^G$ Protection.\\
Queste attività sono mirate a facilitare il lavoro futuro, compreso lo script che inserisce la G all'apice dei termini presenti nel glossario, non inizialmente definito nelle attività ma svolto nell'interesse del gruppo.\\
Nei prossimi sprint$^G$ c'è l'obiettivo di iniziare a scrivere i documenti rimanenti e di migliorare quelli già presenti.

\subsubsubsection{Retrospettiva}
Al termine dello Sprint$^G$, è stata confermata la chiusura di tutte le attività inserite nelle attività da svolgere con l'aggiunta dello script menzionato precedentemente.