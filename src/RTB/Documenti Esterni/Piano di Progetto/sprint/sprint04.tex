\subsubsection{Sprint 4}

\subsubsubsection{Informazioni generali e attività da svolgere}

\begin{tabular}{p{0.25\textwidth} p{0.2\textwidth}}
	\textbf{Inizio} & 15/12/2025 \\
	\textbf{Fine prevista} & 29/12/2025 \\
	\textbf{Fine reale} & 29/12/2025\\
	\textbf{Giorni di ritardo} & 0\\
\end{tabular}
\newline \newline 
Le nostre attività da svolgere, definite nel \href{https://nullpointersgroup.github.io/Documentazione/output/RTB/Verbali\%20Interni/2025-12-15\_verbale\_interno.pdf}{Verbale interno del 15/12/2025}, \href{https://nullpointersgroup.github.io/Documentazione/output/RTB/Verbali\%20Interni/2025-12-17\_verbale\_interno.pdf}{Verbale interno del 17/12/2025} e nel \href{https://nullpointersgroup.github.io/Documentazione/output/RTB/Verbali\%20Interni/2025-12-23\_verbale\_interno.pdf}{Verbale interno del 23/12/2025} sono:
\begin{itemize}[itemsep=5pt, parsep=1pt, label=$\scriptstyle\bullet$]
	\item Continuazione Analisi di Requisiti: sezione Attori, Requisiti di qualità e di vincolo
	\item Continuazione Analisi di Requisiti: sezione Casi d’Uso (1-29/36-51)
\end{itemize}


\subsubsubsection{Rischi attesi}
I possibili rischi attesi sono:
\begin{itemize}[itemsep=5pt, parsep=1pt, label=$\scriptstyle\bullet$]
	\item RP1: esame universitario a qualche componente
	\item RP2: attività extra-universitarie quali festività natalizie
	\item RP3: rischio d'incomprensione di qualche task$^G$
\end{itemize}


\subsubsubsection{Preventivo}
Si prospetta l'utilizzo delle seguenti risorse:\newpage
\begin{table}[H]
	\centering
	% Prima tabella senza bordi per le intestazioni ruotate
	\begin{tabular}{p{0.2\textwidth}p{0.12\textwidth}p{0.12\textwidth}p{0.12\textwidth}p{0.12\textwidth}p{0.12\textwidth}p{0.12\textwidth}}
		& \rotatebox{45}{\textbf{Responsabile$^G$}} &
		\rotatebox{45}{\textbf{Amministratore$^G$}} & \rotatebox{45}{\textbf{Analista$^G$}} & \rotatebox{45}{\textbf{Programmatore$^G$}} & \rotatebox{45}{\textbf{Verificatore$^G$}} &
		\rotatebox{45}{\textbf{Progettista$^G$}} \\
	\end{tabular}
	
	\vspace{0.2cm} % Piccolo spazio tra le due tabelle
	
	% Seconda tabella con i dati e i bordi
	\begin{tabular}{|p{0.2\textwidth}|p{0.12\textwidth}|p{0.12\textwidth}|p{0.12\textwidth}|p{0.12\textwidth}|p{0.12\textwidth}|p{0.12\textwidth}|}
		\hline
		M. Mazzaretto & - & - & 6 & - & - & - \\ \hline
		T. Ceron      & - & - & 6 & - & 1 & - \\ \hline
		L. Pieripolli & - & - & 6 & - & 1 & - \\ \hline
		L. Marcuzzo   & 4 & - & - & - & - & - \\ \hline
		M. Brunello   & - & - & 6 & - & 1 & - \\ \hline
		L. Casagrande & - & - & 6 & - & - & - \\ \hline
	\end{tabular}
	\caption{Sprint 4: Preventivo}
\end{table}

Il totale preventivato per lo sprint$^G$ è \textbf{915€}.\\

\begin{figure}[H]
	\centering
	\includegraphics[width=0.8\textwidth]{PianoProgetto/sprint04_preventivo}
	\caption{Sprint 4: Preventivo}
\end{figure}

\subsubsubsection{Consuntivo}
In questo sprint$^G$ sono state utilizzate le seguenti risorse:
\begin{table}[H]
	\centering
	% Prima tabella senza bordi per le intestazioni ruotate
	\begin{tabular}{p{0.2\textwidth}p{0.12\textwidth}p{0.12\textwidth}p{0.12\textwidth}p{0.12\textwidth}p{0.12\textwidth}p{0.12\textwidth}}
		& \rotatebox{45}{\textbf{Responsabile$^G$}} &
		\rotatebox{45}{\textbf{Amministratore$^G$}} & \rotatebox{45}{\textbf{Analista$^G$}} & \rotatebox{45}{\textbf{Programmatore$^G$}} & \rotatebox{45}{\textbf{Verificatore$^G$}} &
		\rotatebox{45}{\textbf{Progettista$^G$}} \\
	\end{tabular}
	
	\vspace{0.2cm} % Piccolo spazio tra le due tabelle
	
	% Seconda tabella con i dati e i bordi
	\begin{tabular}{|p{0.2\textwidth}|p{0.12\textwidth}|p{0.12\textwidth}|p{0.12\textwidth}|p{0.12\textwidth}|p{0.12\textwidth}|p{0.12\textwidth}|}
		\hline
		M. Mazzaretto & - & - & 5 \textcolor{green}{(-1)} & - & - & - \\ \hline
		T. Ceron      & - & - & 5 \textcolor{green}{(-1)} & - & 1 & - \\ \hline
		L. Pieripolli & - & - & 7 \textcolor{red}{(+1)} & - & 1 & - \\ \hline
		L. Marcuzzo   & 4 & - & - & - & - & - \\ \hline
		M. Brunello   & - & - & 6 & - & 1 & - \\ \hline
		L. Casagrande & - & - & 7 \textcolor{red}{(+1)} & - & - & - \\ \hline
	\end{tabular}
	\caption{Sprint 4: Consuntivo}
\end{table}

Da notare che, sebbene la distribuzione$^G$ di ore degli analisti per singoli membri del gruppo non sia stata perfettamente accurata durante il preventivo, le ore totali da analista$^G$ sono state come preventivate, sintomo del fatto che il gruppo sta iniziando a comprendere il carico di lavoro necessario per portare a termine le task$^G$ dello sprint$^G$. 

\begin{figure}[H]
	\centering
	\includegraphics[width=0.8\textwidth]{PianoProgetto/sprint04_consuntivo}
	\caption{Sprint 4: Consuntivo}
\end{figure}

\subsubsubsection{Aggiornamento delle risorse rimanenti}
\begin{table}[H]
	\centering
	\begin{tabular}{|p{0.2\textwidth}|p{0.1\textwidth}|p{0.05\textwidth}|p{0.1\textwidth}|p{0.13\textwidth}|p{0.12\textwidth}|}
		\hline
		\rowcolor{gray!25}
		Ruolo & Costo & Ore & Costo \newline effettivo & Ore \newline rimanenti & Budget \newline rimanente \\ \hline
		Responsabile$^G$   & 30(€/h) & 4 & 120 & 37 \textcolor{red}{(-4)} & 1110 \textcolor{red}{(-120)} \\ \hline
		Amministratore$^G$ & 20(€/h) & - & - & 11 & 220 \\ \hline
		Analista$^G$       & 25(€/h) & 30 & 750 & 26.5 \textcolor{red}{(-30)} & 662.5 \newline \textcolor{red}{(-750)} \\ \hline
		Progettista$^G$    & 25(€/h) & - & - & 125 & 3125 \\ \hline
		Verificatore$^G$   & 15(€/h) & 3 & 45 & 101 \textcolor{red}{(-3)} & 1515 \textcolor{red}{(-45)} \\ \hline
		Programmatore$^G$  & 15(€/h) & - & - & 125 & 1875 \\ \hline
		Totale         & - & 37 & 915 & 425.5 \textcolor{red}{(-37)} & 8507.5 \newline \textcolor{red}{(-915)} \\ \hline
	\end{tabular}
	\caption{Sprint 4: Aggiornamento risorse}
\end{table}

\subsubsubsection{Rischi incontrati}
Nessun rischio incontrato.
\subsubsubsection{Efficacia delle strategie di gestione di mitigazione dei rischi}
La strategia per l'assegnazione e la stima delle ore per ogni ruolo si è rivelata assai efficace, come detto nel consuntivo$^G$, abbiamo accuratamente previsto le ore necessarie per questo sprint$^G$.
\subsubsubsection{Migliorie da attuare per le attività future}
Con l'arrivo della sessione invernale dobbiamo migliorare nell'assegnazione delle attività cercando di essere il più meticolosi possibile, con l'obiettivo di mitigare il rischio RP1.
\subsubsubsection{Retrospettiva}
Tutte le attività pianificate sono state completate.