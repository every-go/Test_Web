\subsubsection{Sprint 3}

\subsubsubsection{Informazioni generali e attività da svolgere}

\begin{tabular}{p{0.25\textwidth} p{0.2\textwidth}}
	\textbf{Inizio} & 01/12/2025 \\
	\textbf{Fine prevista} & 15/12/2025 \\
	\textbf{Fine reale} & 15/12/2025 \\
	\textbf{Giorni di ritardo} & 0
\end{tabular}
\newline \newline 
Le nostre attività da svolgere, definite nel \href{https://nullpointersgroup.github.io/Documentazione/output/RTB/Verbali\%20Interni/2025-12-01_verbale_interno.pdf}{Verbale interno del 01/12/2025} sono:
\begin{itemize}[itemsep=5pt, parsep=1pt, label=$\scriptstyle\bullet$]
	\item Inizio stesura Piano di Qualifica$^G$
	\item Definizione metriche di qualità
	\item Continuazione stesura Norme di Progetto$^G$
	\item Stesura diario di bordo 15/12
	\item Terminazione individuazione Casi d'Uso e Requisiti
	\item Discussione con la proponente$^G$ del punto sopra
\end{itemize}


\subsubsubsection{Rischi attesi}
I possibili rischi attesi sono:
\begin{itemize}[itemsep=5pt, parsep=1pt, label=$\scriptstyle\bullet$]
	\item RA1: un imprevisto a qualche componente
	\item RE2: sovraccarico o stima errata del lavoro
	\item RP3: rischio d'incomprensione di qualche task$^G$
\end{itemize}


\subsubsubsection{Preventivo}
Si prospetta l'utilizzo delle seguenti risorse:\newpage
\begin{table}[H]
	\centering
	% Prima tabella senza bordi per le intestazioni ruotate
	\begin{tabular}{p{0.2\textwidth}p{0.12\textwidth}p{0.12\textwidth}p{0.12\textwidth}p{0.12\textwidth}p{0.12\textwidth}p{0.12\textwidth}}
		& \rotatebox{45}{\textbf{Responsabile$^G$}} &
		\rotatebox{45}{\textbf{Amministratore$^G$}} & \rotatebox{45}{\textbf{Analista$^G$}} & \rotatebox{45}{\textbf{Programmatore$^G$}} & \rotatebox{45}{\textbf{Verificatore$^G$}} &
		\rotatebox{45}{\textbf{Progettista$^G$}} \\
	\end{tabular}
	
	\vspace{0.2cm} % Piccolo spazio tra le due tabelle
	
	% Seconda tabella con i dati e i bordi
	\begin{tabular}{|p{0.2\textwidth}|p{0.12\textwidth}|p{0.12\textwidth}|p{0.12\textwidth}|p{0.12\textwidth}|p{0.12\textwidth}|p{0.12\textwidth}|}
		\hline
		M. Mazzaretto & 4 & - & - & - & 1 & - \\ \hline
		T. Ceron      & - & - & 8 & - & - & - \\ \hline
		L. Pieripolli & - & - & 8 & - & - & - \\ \hline
		L. Marcuzzo   & - & 5 & - & - & 1 & - \\ \hline
		M. Brunello   & - & - & 8 & - & - & - \\ \hline
		L. Casagrande & - & 5 & - & - & 1 & - \\ \hline
	\end{tabular}
	\caption{Sprint 3: Preventivo}
\end{table}

Il totale preventivato per lo sprint$^G$ è \textbf{965€}.\\

\begin{figure}[H]
	\centering
	\includegraphics[width=0.8\textwidth]{PianoProgetto/sprint03_preventivo}
	\caption{Sprint 3: Preventivo}
\end{figure}

\subsubsubsection{Consuntivo}
In questo sprint$^G$ sono state utilizzate le seguenti risorse:

\begin{table}[H]
	\centering
	% Prima tabella senza bordi per le intestazioni ruotate
	\begin{tabular}{p{0.2\textwidth}p{0.12\textwidth}p{0.12\textwidth}p{0.12\textwidth}p{0.12\textwidth}p{0.12\textwidth}p{0.12\textwidth}}
		& \rotatebox{45}{\textbf{Responsabile$^G$}} &
		\rotatebox{45}{\textbf{Amministratore$^G$}} & \rotatebox{45}{\textbf{Analista$^G$}} & \rotatebox{45}{\textbf{Programmatore$^G$}} & \rotatebox{45}{\textbf{Verificatore$^G$}} &
		\rotatebox{45}{\textbf{Progettista$^G$}} \\
	\end{tabular}
	
	\vspace{0.2cm} % Piccolo spazio tra le due tabelle
	
	% Seconda tabella con i dati e i bordi
	\begin{tabular}{|p{0.2\textwidth}|p{0.12\textwidth}|p{0.12\textwidth}|p{0.12\textwidth}|p{0.12\textwidth}|p{0.12\textwidth}|p{0.12\textwidth}|}
		\hline
		M. Mazzaretto & 3 \textcolor{green}{(-1)} & - & - & - & 1 & - \\ \hline
		T. Ceron      & - & - & 6 \textcolor{green}{(-2)} & - & - & - \\ \hline
		L. Pieripolli & - & - & 5 \textcolor{green}{(-3)} & - & - & - \\ \hline
		L. Marcuzzo   & - & 4 \textcolor{green}{(-1)} & - & - & 1 & - \\ \hline
		M. Brunello   & - & - & 5 \textcolor{green}{(-3)} & - & - & - \\ \hline
		L. Casagrande & - & 5 & - & - & 1 & - \\ \hline
	\end{tabular}
	\caption{Sprint 3: Consuntivo}
\end{table}

Si evidenzia come il processo$^G$ di analisi sia migliorato, nonostante ciò è stato necessario l'apporto di più persone.

\begin{figure}[H]
	\centering
	\includegraphics[width=0.8\textwidth]{PianoProgetto/sprint03_consuntivo}
	\caption{Sprint 3: Consuntivo}
\end{figure}

\subsubsubsection{Aggiornamento delle risorse rimanenti}

\begin{table}[H]
	\centering
	\begin{tabular}{|p{0.2\textwidth}|p{0.1\textwidth}|p{0.05\textwidth}|p{0.1\textwidth}|p{0.13\textwidth}|p{0.12\textwidth}|}
		\hline
		\rowcolor{gray!25}
		Ruolo & Costo & Ore & Costo \newline effettivo & Ore \newline rimanenti & Budget \newline rimanente \\ \hline
		Responsabile$^G$   & 30(€/h) & 3 & 90 & 41 \textcolor{red}{(-3)} & 1230 \textcolor{red}{(-90)} \\ \hline
		Amministratore$^G$ & 20(€/h) & 9 & 180 & 11 \textcolor{red}{(-9)} & 220 \newline \textcolor{red}{(-180)} \\ \hline
		Analista$^G$       & 25(€/h) & 16 & 400 & 56.5 \textcolor{red}{(-16)} & 1412.5 \newline \textcolor{red}{(-400)} \\ \hline
		Progettista$^G$    & 25(€/h) & - & - & 125 & 3125 \\ \hline
		Verificatore$^G$   & 15(€/h) & 3 & 45 & 104 \textcolor{red}{(-3)} & 1560 \textcolor{red}{(-45)} \\ \hline
		Programmatore$^G$  & 15(€/h) & - & - & 125 & 1875 \\ \hline
		Totale         & - & 31 & 715 & 462.5 \textcolor{red}{(-31)} & 9422.5 \newline \textcolor{red}{(-715)} \\ \hline
	\end{tabular}
	\caption{Sprint 3: Aggiornamento risorse}
\end{table}

\subsubsubsection{Rischi incontrati}
In questo sprint$^G$ non si sono verificati rischi.

\subsubsubsection{Efficacia delle strategie di gestione di mitigazione dei rischi}
La strategia di incrementare le ore assegnate agli analisti si è rivelata efficace e produttiva, consentendoci di definire in modo più chiaro e dettagliato gli use case e i relativi requisiti.\\
Inoltre, la redazione della documentazione è risultata rapida e fluida, confermando la correttezza della scelta di assegnare un numero inferiore di ore agli amministratori.

\subsubsubsection{Migliorie da attuare per le attività future}
Migliorare la fase di correzione dei documenti, assicurando un rispetto più rigoroso e uniforme degli standard definiti dal gruppo.\\
Inoltre, è necessario migliorare la stima delle ore per ogni ruolo, in particolare per gli analisti e i verificatori, al fine di ridurre le discrepanze tra preventivo e consuntivo$^G$ nei prossimi sprint$^G$.

\subsubsubsection{Retrospettiva}
Tutte le attività pianificate sono state completate; tuttavia, una di esse è stata portata avanti esclusivamente tramite un documento condiviso non pubblico e, per questo motivo, il relativo risultato non è ancora stato pubblicato nel repository$^G$.