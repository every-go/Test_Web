\subsubsection{Sprint 2}

\subsubsubsection{Informazioni generali e attività da svolgere}

\begin{tabular}{p{0.25\textwidth} p{0.2\textwidth}}
	\textbf{Inizio} & 17/11/2025 \\
	\textbf{Fine prevista} & 01/12/2025 \\
	\textbf{Fine reale} & 01/12/2025 \\
	\textbf{Giorni di ritardo} & 0
\end{tabular}
\newline \newline 
Le nostre attività da svolgere, definite nel \href{https://nullpointersgroup.github.io/Documentazione/output/RTB/Verbali\%20Interni/2025-11-17_verbale_interno.pdf}{Verbale interno del 17/11/2025} e nel \href{https://nullpointersgroup.github.io/Documentazione/output/RTB/Verbali\%20Interni/2025-11-24_verbale_interno.pdf}{Verbale interno del 24/11/2025} sono:
\begin{itemize}[itemsep=5pt, parsep=1pt, label=$\scriptstyle\bullet$]
	\item Inizio stesura Piano di Progetto$^G$
	\item Continuazione stesura Norme di Progetto$^G$
	\item Stesura diario di bordo 24/11 e 1/12
	\item Aggiornamento Glossario
	\item Continuazione individuazione Casi d'Uso e Requisiti
\end{itemize}


\subsubsubsection{Rischi attesi}
I possibili rischi attesi sono:
\begin{itemize}[itemsep=5pt, parsep=1pt, label=$\scriptstyle\bullet$]
	\item RA1: un imprevisto a qualche componente
	\item RE2: sovraccarico o stima errata del lavoro
	\item RP3: rischio d'incomprensione di qualche task$^G$
\end{itemize}


\subsubsubsection{Preventivo}
Si prospetta l'utilizzo delle seguenti risorse:\newpage
\begin{table}[H]
	\centering
	% Prima tabella senza bordi per le intestazioni ruotate
	\begin{tabular}{p{0.2\textwidth}p{0.12\textwidth}p{0.12\textwidth}p{0.12\textwidth}p{0.12\textwidth}p{0.12\textwidth}p{0.12\textwidth}}
		& \rotatebox{45}{\textbf{Responsabile$^G$}} &
		\rotatebox{45}{\textbf{Amministratore$^G$}} & \rotatebox{45}{\textbf{Analista$^G$}} & \rotatebox{45}{\textbf{Programmatore$^G$}} & \rotatebox{45}{\textbf{Verificatore$^G$}} &
		\rotatebox{45}{\textbf{Progettista$^G$}} \\
	\end{tabular}
	
	\vspace{0.2cm} % Piccolo spazio tra le due tabelle
	
	% Seconda tabella con i dati e i bordi
	\begin{tabular}{|p{0.2\textwidth}|p{0.12\textwidth}|p{0.12\textwidth}|p{0.12\textwidth}|p{0.12\textwidth}|p{0.12\textwidth}|p{0.12\textwidth}|}
		\hline
		M. Mazzaretto & - & 2 & 3 & - & 1 & - \\ \hline
		T. Ceron      & - & - & 6 & - & 1 & - \\ \hline
		L. Pieripolli & - & 5 & - & - & 1 & - \\ \hline
		L. Marcuzzo   & - & - & 6 & - & 1 & - \\ \hline
		M. Brunello   & - & 5 & - & - & - & - \\ \hline
		L. Casagrande & 4 & - & - & - & - & - \\ \hline
	\end{tabular}
	\caption{Sprint 2: Preventivo}
\end{table}

Il totale preventivato per lo sprint$^G$ è \textbf{795€}.\\

\begin{figure}[H]
	\centering
	\includegraphics[width=0.8\textwidth]{PianoProgetto/sprint02_preventivo}
	\caption{Sprint 2: Preventivo}
\end{figure}

\subsubsubsection{Consuntivo}
In questo sprint$^G$ sono state utilizzate le seguenti risorse:

\begin{table}[H]
	\centering
	% Prima tabella senza bordi per le intestazioni ruotate
	\begin{tabular}{p{0.2\textwidth}p{0.12\textwidth}p{0.12\textwidth}p{0.12\textwidth}p{0.12\textwidth}p{0.12\textwidth}p{0.12\textwidth}}
		& \rotatebox{45}{\textbf{Responsabile$^G$}} &
		\rotatebox{45}{\textbf{Amministratore$^G$}} & \rotatebox{45}{\textbf{Analista$^G$}} & \rotatebox{45}{\textbf{Programmatore$^G$}} & \rotatebox{45}{\textbf{Verificatore$^G$}} &
		\rotatebox{45}{\textbf{Progettista$^G$}} \\
	\end{tabular}
	
	\vspace{0.2cm} % Piccolo spazio tra le due tabelle
	
	% Seconda tabella con i dati e i bordi
	\begin{tabular}{|p{0.2\textwidth}|p{0.12\textwidth}|p{0.12\textwidth}|p{0.12\textwidth}|p{0.12\textwidth}|p{0.12\textwidth}|p{0.12\textwidth}|}
		\hline
		M. Mazzaretto & - & 2.5 \textcolor{green}{(-0.5)} & 5 \textcolor{red}{(+2)} & - & 1 & - \\ \hline
		T. Ceron      & - & - & 5 \textcolor{green}{(-1)} & - & 0.5 \textcolor{green}{(-0.5)} & - \\ \hline
		L. Pieripolli & - & 3 \textcolor{green}{(-2)} & - & - & 0.5 \textcolor{green}{(-0.5)} & - \\ \hline
		L. Marcuzzo   & - & - & 5 \textcolor{green}{(-1)} & - & 0.5 \textcolor{green}{(-0.5)} & - \\ \hline
		M. Brunello   & - & 3.5 \textcolor{green}{(-1.5)} & - & - & 0.5 \textcolor{red}{(+0.5)} & - \\ \hline
		L. Casagrande & 3 \textcolor{green}{(-1)} & - & - & - & 1 \textcolor{red}{(+1)} & - \\ \hline
	\end{tabular}
	\caption{Sprint 2: Consuntivo}
\end{table}

Si evidenzia come il processo$^G$ di verifica$^G$ sia migliorato, nonostante ciò è stato necessario l'apporto di più persone. Per questo il gruppo ha l'obiettivo di migliorare i processi di verifica$^G$.

\begin{figure}[H]
	\centering
	\includegraphics[width=0.8\textwidth]{PianoProgetto/sprint02_consuntivo}
	\caption{Sprint 2: Consuntivo}
\end{figure}

\subsubsubsection{Aggiornamento delle risorse rimanenti}

\begin{table}[H]
	\centering
	\begin{tabular}{|p{0.2\textwidth}|p{0.1\textwidth}|p{0.05\textwidth}|p{0.1\textwidth}|p{0.13\textwidth}|p{0.12\textwidth}|}
		\hline
		\rowcolor{gray!25}
		Ruolo & Costo & Ore & Costo \newline effettivo & Ore \newline rimanenti & Budget \newline rimanente \\ \hline
		Responsabile$^G$   & 30(€/h) & 3 & 90 & 44 \textcolor{red}{(-3)} & 1320 \textcolor{red}{(-90)} \\ \hline
		Amministratore$^G$ & 20(€/h) & 9 & 180 & 20 \textcolor{red}{(-9)} & 400 \newline \textcolor{red}{(-180)} \\ \hline
		Analista$^G$       & 25(€/h) & 15 & 375 & 72.5 \textcolor{red}{(-15)} & 1812.5 \newline \textcolor{red}{(-375)} \\ \hline
		Progettista$^G$    & 25(€/h) & - & - & 125 & 3125 \\ \hline
		Verificatore$^G$   & 15(€/h) & 4 & 60 & 107 \textcolor{red}{(-4)} & 1605 \textcolor{red}{(-60)} \\ \hline
		Programmatore$^G$  & 15(€/h) & - & - & 125 & 1875 \\ \hline
		Totale         & - & 31 & 705 & 493.5 \textcolor{red}{(-31)} & 10137.5 \newline \textcolor{red}{(-705)} \\ \hline
	\end{tabular}
	\caption{Sprint 2: Aggiornamento risorse}
\end{table}

\subsubsubsection{Rischi incontrati}
Anche in questo sprint$^G$ è stato incontrato il rischio RE2, che sottolinea la difficoltà del gruppo di trovare una quadra per la stesura corretta di un buon preventivo, seppur consapevoli che sia praticamente impossibile avere il 100\% di accuratezza in ogni Sprint$^G$.

\subsubsubsection{Efficacia delle strategie di gestione di mitigazione dei rischi}
La strategia di migliorare gli script e le GitHub$^G$ Action$^G$ si è rivelata efficace, non ci sono stati problemi nella verifica$^G$ dei documenti e ogni componente ha svolto il suo ruolo al meglio delle sue possibilità.\\
È stato implementato un altro significativo miglioramento negli script, che ora ritornano un'eccezione se falliscono, permettendo un controllo immediato dell'errore e impedendo l'inserimento in remoto di file compromessi oppure non compilabili.\\
Questa strategia sarà utile anche per il futuro, perché sarà una base di partenza per la scrittura del PoC$^G$ e in generale del codice.

\subsubsubsection{Migliorie da attuare per le attività future}
Nella prossima fase, il gruppo dovrà definire in modo chiaro e conciso cosa inserire nel documento \vrs{Analisi dei Requisiti$^G$}.\\
Per questo motivo, è stato deciso di inserire nel preventivo molte più ore alla figura dell'\vrs{Analista$^G$}.\\
Inoltre, dato che ormai tutti noi abbiamo studiato al meglio la scrittura dei documenti, sono state dedicate meno ore alla figura dell'\vrs{Amministratore$^G$} poiché si prevede che la scrittura della documentazione sarà più veloce e fluida.

\subsubsubsection{Retrospettiva}
Sono state completate tutte le attività definite, seppur una di queste è stata continuata solo in un documento condiviso non pubblico, per questo non è ancora stato pubblicato in repo il risultato.