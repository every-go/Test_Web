\section{Introduzione}
	\subsection{Informazioni generali}
	Il Piano di Progetto$^G$ è un documento che esprime le attività svolte e da svolgere durante il progetto$^G$ di Ingegneria del Software. \\
	Viene utilizzato dal gruppo per la mitigazione dei rischi e una corretta pianificazione$^G$ dei ruoli e delle attività, con lo scopo di automiglioramento, utilizzando un processo$^G$ incrementale.\\
	Le modalità operative e i dettagli relativi al way of working$^G$ sono riportati nelle Norme di Progetto$^G$, consultabili nei Riferimenti normativi.
	
	\subsection{Definizioni, acronimi e abbreviazioni}
	Per tutti i termini tecnici, gli acronimi e le definizioni utilizzate nel documento si rimanda al \underline{\href{https://nullpointersgroup.github.io/Documentazione/output/RTB/Documentazione_interna/Glossario.pdf}{Glossario}}, disponibile come documento separato.\\
	Ogni parola presente nel glossario viene segnata come segue:
	\begin{center}
		termine$^{G}$
	\end{center}
	
	\subsection{Riferimenti}
	\subsubsection{Riferimenti normativi}
	\begin{itemize}[itemsep=5pt, parsep=5pt, label=$\scriptstyle\bullet$]
	
	\item \textbf{Capitolato$^G$ C8 - Ergon Informatica Srl - SmartOrder}\\
	\url{https://www.math.unipd.it/~tullio/IS-1/2025/Progetto/C8.pdf}\\[3pt]
	\textbf{Ultima consultazione: 22 Novembre 2025}
	
	\item \textbf{Norme di Progetto$^G$}\\
	\url{https://nullpointersgroup.github.io/Documentazione/output/RTB/Documenti\%20Interni/Norme\_di\_Progetto.pdf}\\[3pt]
	\textbf{Ultima consultazione: 2 Dicembre 2025}
	\end{itemize}

	\subsubsection{Riferimenti informativi}
	\begin{itemize}[itemsep=5pt, parsep=5pt, label=$\scriptstyle\bullet$]

    \item \textbf{Specifica dei requisiti software secondo IEEE$^G$ 830-1998}\\
     \url{https://ieeexplore.ieee.org/document/720574}\\[3pt]
    \textbf{Ultima consultazione: 22 Novembre 2025}

    \item \textbf{Standard ISO/IEC/IEEE$^G$ 12207:1995}\\
     \url{https://www.math.unipd.it/~tullio/IS-1/2009/Approfondimenti/ISO\_12207-1995.pdf}\\[3pt]
    \textbf{Ultima consultazione: 22 Novembre 2025}

	\item \textbf{Glossario, versione 1.0.0}\\
     \url{https://nullpointersgroup.github.io/Documentazione/output/RTB/Documenti\%20Interni/Glossario.pdf}\\[3pt]
    \textbf{Ultima consultazione: 27 Dicembre 2025}

	\end{itemize}