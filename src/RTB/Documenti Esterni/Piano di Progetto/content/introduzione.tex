\section{Introduzione}
	\subsection{Informazioni generali}
	Il Piano di Progetto$^G$ è un documento che esprime le attività svolte e da svolgere durante il progetto di Ingegneria del Software. \\
	Viene utilizzato dal gruppo per la mitigazione dei rischi e una corretta pianificazione$^G$ dei ruoli e delle attività, con lo scopo di automiglioramento, utilizzando un processo incrementale.\\
	Le modalità operative e i dettagli relativi al way of working$^G$ sono riportati nelle Norme di Progetto, consultabili nei Riferimenti normativi.
	
	\subsection{Definizioni, acronimi e abbreviazioni}
	Per tutti i termini tecnici, gli acronimi e le definizioni utilizzate nel documento si rimanda al \underline{\href{https://nullpointersgroup.github.io/Documentazione/output/RTB/Documentazione_interna/Glossario.pdf}{Glossario}}, disponibile come documento separato.\\
	Ogni parola presente nel glossario viene segnata come segue:
	\begin{center}
		termine$^G$
	\end{center}

	\subsection{Riferimenti}
	\subsubsection{Riferimenti normativi}
	\href{https://www.math.unipd.it/~tullio/IS-1/2025/Progetto/C8.pdf}{%
	\uline{Capitolato$^G$ d’appalto C8 – SmartOrder: Analisi multimodale per la creazione automatica di ordini}%
	} \\[0.5em]
	\underline{\href{https://nullpointersgroup.github.io/Documentazione/output/RTB/Documenti\%20Interni/Norme\_di\_Progetto.pdf}{Norme di Progetto}} 
	
	\subsubsection{Riferimenti informativi}
	\underline{\href{https://ieeexplore.ieee.org/document/720574}{Specifica dei requisiti software secondo IEEE 830-1998}} \\[0.5em]
	\underline{\href{https://www.math.unipd.it/~tullio/IS-1/2009/Approfondimenti/ISO_12207-1995.pdf}{Standard ISO/IEC 12207:1995}} \\[0.5em]
    \underline{\href{https://nullpointersgroup.github.io/Documentazione/output/RTB/Documentazione_interna/Glossario.pdf}{Glossario}} \\[0.5em]