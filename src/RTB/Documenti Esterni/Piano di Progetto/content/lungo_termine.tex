\section{Pianificazione nel lungo termine}

\subsection{Introduzione}
Come definito nel \href{https://nullpointersgroup.github.io/Documentazione/output/Candidatura/Preventivo_Costi.pdf}{Preventivo Costi}, il gruppo si impegna a concludere il progetto$^G$ entro il \textbf{30/04/2026}, con un costo insormontabile di \textbf{11.440€}.\\
Il preventivo iniziale, già definito nel documento menzionato precedentemente, è il seguente:\\
\begin{table}[h!]
	\centering
	\begin{tabular}{|c|c|c|c|}
		\hline
		\rowcolor{gray!25}
		Ruolo & Ore totali & Costo (€/h) & Costo totale (€) \\ \hline
		Responsabile$^G$ & 50 & 30 & 1500 \\ \hline
		Amministratore$^G$ & 42 & 20 & 840 \\ \hline
		Analista$^G$ & 95 & 25 & 2375 \\ \hline
		Progettista$^G$ & 125 & 25 & 3125 \\ \hline
		Verificatore$^G$ & 115 & 15 & 1725 \\ \hline
		Programmatore$^G$ & 125 & 15 & 1875 \\ \hline
	\end{tabular}
	\caption{Ore per ciascun ruolo e calcolo dei costi totali}
\end{table}

La data di consegna prevista per la RTB$^G$ è il \textbf{15/02/2026}.

\newpage

\subsection{Attività previste per la RTB}
Le attività previste per la RTB$^G$ (Requirements and Technology Baseline$^G$) sono riassunte nella seguente tabella:\\
\begin{table}[H]
	\centering
	\resizebox{\textwidth}{!}{
	\renewcommand{\arraystretch}{1.15}
	\begin{tabular}{|p{0.3\textwidth}|p{0.7\textwidth}|}
		\hline
		\rowcolor[gray]{0.9}
		\textbf{Attività} & \textbf{Descrizione} \\
		\hline
		\textbf{Analisi dei Requisiti$^G$} & Documento esterno che ha lo scopo di definire i requisiti funzionali, requisiti non funzionali e requisiti di vincolo . Inoltre illustra tutti i casi d'uso relativi a requisiti obbligatori, non obbligatori e di sistema. \\
		\hline
		\textbf{Piano di Progetto$^G$} & Documento esterno che esprime le attività svolte e da svolgere durante il progetto$^G$. \\
		\hline
		\textbf{Piano di Qualifica$^G$} &  Documento esterno che rappresenta un documento fondamentale per la gestione e il monitoraggio continuo della qualità del progetto$^G$ e dei processi coinvolti nel suo
		ciclo di vita. \\
		\hline
		\textbf{Norme di Progetto$^G$} & Documento interno che descrive il Way of Working$^G$ del gruppo nel dettaglio. \\
		\hline
		\textbf{Glossario} & Documento interno che permette di facilitare la comprensione di parole tecniche nella comunicazione fra il gruppo e la proponente$^G$. \\
		\hline
		\textbf{Proof of Concept$^G$} & Eseguibile che dimostra la coesione fra le tecnologie scelte. \\
		\hline
		\textbf{Lettera di presentazione RTB$^G$} & Simile alla Lettera di Presentazione per la Candidatura, esprime la volontà del gruppo di partecipare alla revisione per la RTB$^G$. \\
		\hline
	\end{tabular}}
\end{table}
Ogni attività che inizia, continua o finalizza la stesura delle documentazioni menzionate nella tabella è definita all'interno di ogni sprint$^G$ nella sezione \vr{Attività da svolgere}.

\newpage

\subsection{Attività previste per la PB}
Questa sezione verrà redatta dopo il superamento della RTB$^G$, in modo da consentire la scrittura più accurata durante la PB$^G$ (Product Baseline$^G$)