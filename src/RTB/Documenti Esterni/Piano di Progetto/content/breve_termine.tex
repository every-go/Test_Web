\section{Pianificazione a breve termine}

\subsection{Introduzione}
NullPointers Group ha stabilito di procedere con un approccio Agile$^G$ allo svolgimento del progetto, definendo la durata dello Sprint$^G$ di 2 settimane per avere un'organizzazione migliore.\\
Seguendo questo principio, il gruppo si impegna a stabilire all’inizio di ogni sprint i ruoli all'interno del gruppo, che ruotano bisettimanalmente, e le attività per le settimane successive, mantenendo però la possibilità di effettuare il cambiamento qualora le esigenze organizzative lo imponessero.\\
Per verificare la continua e corretta prosecuzione delle attività, rimarremo in contatto con l'azienda \textbf{Ergon Informatica s.r.l.} sia tramite incontri telematici su \textbf{Google Meet}$^G$ sia tramite i canali di comunicazione diretti concordati.
Seguiranno ora le descrizioni dei vari periodi di lavoro, nelle quali verranno esposte le sezioni:
\begin{enumerate}
	\item Informazioni generali
	\item Attività da svolgere
	\item Prospetto consumo tempo e costi
	\item Rischi attesi
	\item Consumo tempo e costi effettivo
	\item Aggiornamento delle risorse rimanenti
	\item Retrospettiva, comprendente anche i rischi effettivamente riscontrati
\end{enumerate}

\newpage
\subsection{Requirements and Technology Baseline}

\subsubsection{Sprint 1}

\subsubsubsection{Informazioni generali e attività da svolgere}

\begin{tabular}{p{0.25\textwidth} p{0.2\textwidth}}
	\textbf{Inizio} & 6/11/2025 \\
	\textbf{Fine prevista} & 17/11/2025 \\
	\textbf{Fine reale} & 17/11/2025 \\
	\textbf{Giorni di ritardo} & 0
\end{tabular}
\newline \newline
Questo primo periodo dura 11 giorni invece di 2 settimane in modo da permetterci di allinearci al lunedì come data di fine sprint, siccome il 6/11 era di venerdì e il secondo lunedì disponibile cascava il 17/11.\\
Le nostre attività da svolgere, definite nel \underline{\href{https://nullpointersgroup.github.io/Documentazione/output/RTB/Verbali\%20Interni/2025-11-06_verbale_interno.pdf}{Verbale interno del 6/11/2025}} e nel \underline{\href{https://nullpointersgroup.github.io/Documentazione/output/RTB/Verbali\%20Interni/2025-11-12_verbale_interno.pdf}{Verbale interno del 12/11/2025}} sono:
\begin{itemize}
	\item Inizio stesura Analisi dei Requisiti
	\item Creazione Glossario
	\item Miglioramento sistema versionamento documenti
	\item Stesura diario di bordo 10/11 e 17/11
	\item Inizio stesura Norme di Progetto
	\item Creazione HTML e CSS per visualizzazione Glossario
	\item Creazione script per la trascrizione del glossario da .tex a .html
	\item Implementazione Branch Protection
\end{itemize}


\subsubsubsection{Rischi attesi}
I possibili rischi attesi sono:
\begin{itemize}
	\item R1: alcuni componenti del gruppo sono occupati con lo studio di altri corsi
	\item R3: un imprevisto a qualche componente
	\item R4: per l'implementazione della Branch Protection
	\item R5: rischio d'incomprensione del modo corretto con il quale stendere i documenti
\end{itemize}

\subsubsubsection{Preventivo}
Si prospetta l'utilizzo delle seguenti risorse:\newpage
\begin{table}[H]
	\centering
	% Prima tabella senza bordi per le intestazioni ruotate
	\begin{tabular}{p{0.2\textwidth}p{0.12\textwidth}p{0.12\textwidth}p{0.12\textwidth}p{0.12\textwidth}p{0.12\textwidth}p{0.12\textwidth}}
		& \rotatebox{45}{\textbf{Responsabile}} &
		\rotatebox{45}{\textbf{Amministratore}} & \rotatebox{45}{\textbf{Analista}} & \rotatebox{45}{\textbf{Programmatore}} & \rotatebox{45}{\textbf{Verificatore}} &
		\rotatebox{45}{\textbf{Progettista}} \\
	\end{tabular}
	
	\vspace{0.2cm} % Piccolo spazio tra le due tabelle
	
	% Seconda tabella con i dati e i bordi
	\begin{tabular}{|p{0.2\textwidth}|p{0.12\textwidth}|p{0.12\textwidth}|p{0.12\textwidth}|p{0.12\textwidth}|p{0.12\textwidth}|p{0.12\textwidth}|}
		\hline
		M. Mazzaretto & - & 6 & - & - & 1 & - \\ \hline
		T. Ceron      & - & 6 & - & - & 1 & - \\ \hline
		L. Pieripolli & 4 & - & - & - & 1 & - \\ \hline
		L. Marcuzzo   & - & - & 4 & - & - & - \\ \hline
		M. Brunello   & - & 6 & - & - & 1 & - \\ \hline
		L. Casagrande & - & - & 6 & - & - & - \\ \hline
	\end{tabular}
	\caption{Sprint 1: Preventivo}
\end{table}

È importante notare che sono considerate poche ore produttive poiché la maggior parte delle ore effettivamente previste sarebbero state dedicate allo studio della documentazione e al corretto modo di lavoro piuttosto che l'effettiva stesura.\\
Oltretutto L. Pieripolli e L. Marcuzzo hanno avuto un preventivo di ore minore poiché impegnati con lo studio per un esame parziale di un altro corso.

\begin{figure}[H]
	\centering
	\includegraphics[width=0.8\textwidth]{PianoProgetto/sprint01_preventivo}
	\caption{Sprint 1: Preventivo}
\end{figure}

\subsubsubsection{Consuntivo}
Le risorse effettivamente utilizzate sono nella seguente tabella:\\
\begin{table}[H]
	\centering
	% Prima tabella senza bordi per le intestazioni ruotate
	\begin{tabular}{p{0.2\textwidth}p{0.12\textwidth}p{0.12\textwidth}p{0.12\textwidth}p{0.12\textwidth}p{0.12\textwidth}p{0.12\textwidth}}
		& \rotatebox{45}{\textbf{Responsabile}} &
		\rotatebox{45}{\textbf{Amministratore}} & \rotatebox{45}{\textbf{Analista}} &
		\rotatebox{45}{\textbf{Progettista}} & \rotatebox{45}{\textbf{Verificatore}} & \rotatebox{45}{\textbf{Programmatore}} \\
	\end{tabular}
	
	\vspace{0.2cm} % Piccolo spazio tra le due tabelle
	
	% Seconda tabella con i dati e i bordi
	\begin{tabular}{|p{0.2\textwidth}|p{0.12\textwidth}|p{0.12\textwidth}|p{0.12\textwidth}|p{0.12\textwidth}|p{0.12\textwidth}|p{0.12\textwidth}|}
		\hline
		M. Mazzaretto & - & 4 & - & - & 1 & - \\ \hline
		T. Ceron      & - & 3.5 & - & - & 1 & - \\ \hline
		L. Pieripolli & 3 & - & - & - & 1 & - \\ \hline
		L. Marcuzzo   & - & - & 2.5 & - & - & - \\ \hline
		M. Brunello   & - & 5.5 & - & - & 1 & - \\ \hline
		L. Casagrande & - & - & 5 & - & - & - \\ \hline
	\end{tabular}
	\caption{Sprint 1: Consuntivo}
\end{table}

C'è stata una sovrastima delle ore produttive causate dal continuo studio della documentazione ma la poca messa in opera dell'effettiva stesura.\\ L'implementazione della Branch Protection ha portato ad un aumento di ore a M. Brunello maggiore dei suoi compagni a causa della sottostima del rischio tecnologico.

\begin{figure}[H]
	\centering
	\includegraphics[width=0.8\textwidth]{PianoProgetto/sprint01_consuntivo}
	\caption{Sprint 1: Consuntivo}
\end{figure}

\subsubsubsection{Aggiornamento delle risorse rimanenti}

\begin{table}[H]
	\centering
	\begin{tabular}{|p{0.185\textwidth}|p{0.1\textwidth}|p{0.05\textwidth}|p{0.1\textwidth}|p{0.13\textwidth}|p{0.12\textwidth}|}
		\hline
		\rowcolor{gray!25}
		Ruolo & Costo & Ore & Costo \newline effettivo & Ore \newline rimanenti & Budget \newline rimanente \\ \hline
		Responsabile   & 30(€/h) & 3 & 90 & 47 (-3) & 1410 (-90) \\ \hline
		Amministratore & 20(€/h) & 13 & 260 & 29 (-13) & 580 \newline (-260) \\ \hline
		Analista       & 25(€/h) & 7.5 & 187.5 & 87.5 (-7.5) & 2187.5 \newline (-187.5) \\ \hline
		Progettista    & 25(€/h) & - & - & 125 & 3125 \\ \hline
		Verificatore   & 15(€/h) & 4 & 60 & 111 (-4) & 1665 (-60) \\ \hline
		Programmatore  & 15(€/h) & - & - & 125 & 1875 \\ \hline
		Totale         & - & 27.5 & 597.5 & 524.5 (-27.5) & 10842.5 \newline (-597.5) \\ \hline
	\end{tabular}
	\caption{Sprint 1: Aggiornamento risorse}
\end{table}

\subsubsubsection{Rischi incontrati}
Sono stati incontrati i rischi R1 ed R4, già definiti nei rischi attesi.

\subsubsubsection{Retrospettiva}
Al termine dello Sprint, è stata confermata la chiusura di tutte le attività.

\newpage
\subsubsection{Sprint$^G$ 2}

\subsubsubsection{Informazioni generali e attività da svolgere}

\begin{tabular}{p{0.25\textwidth} p{0.2\textwidth}}
	\textbf{Inizio} & 17/11/2025 \\
	\textbf{Fine prevista} & 01/12/2025 \\
	\textbf{Fine reale} & \\
	\textbf{Giorni di ritardo} &
\end{tabular}
\newline \newline 
Le nostre attività da svolgere, definite nel \underline{\href{https://nullpointersgroup.github.io/Documentazione/output/RTB/Verbali\%20Interni/2025-11-17_verbale_interno.pdf}{Verbale interno del 17/11/2025}} e nel \underline{\href{https://nullpointersgroup.github.io/Documentazione/output/RTB/Verbali\%20Interni/2025-11-24_verbale_interno.pdf}{Verbale interno del 24/11/2025}} sono:
\begin{itemize}[itemsep=5pt, parsep=1pt, label=$\scriptstyle\bullet$]
	\item Inizio stesura Piano di Progetto
	\item Continuazione stesura Norme di Progetto
	\item Stesura diario di bordo 24/11 e 1/12
	\item Aggiornamento Glossario
	\item Continuazione individuazione Casi d'Uso e Requisiti
\end{itemize}


\subsubsubsection{Rischi attesi}
I possibili rischi attesi sono:
\begin{itemize}[itemsep=5pt, parsep=1pt, label=$\scriptstyle\bullet$]
	\item RA1: un imprevisto a qualche componente
	\item RP3: rischio d'incomprensione di qualche task
\end{itemize}


\subsubsubsection{Preventivo}
Si prospetta l'utilizzo delle seguenti risorse:\newpage
\begin{table}[H]
	\centering
	% Prima tabella senza bordi per le intestazioni ruotate
	\begin{tabular}{p{0.2\textwidth}p{0.12\textwidth}p{0.12\textwidth}p{0.12\textwidth}p{0.12\textwidth}p{0.12\textwidth}p{0.12\textwidth}}
		& \rotatebox{45}{\textbf{Responsabile}} &
		\rotatebox{45}{\textbf{Amministratore}} & \rotatebox{45}{\textbf{Analista}} & \rotatebox{45}{\textbf{Programmatore}} & \rotatebox{45}{\textbf{Verificatore}} &
		\rotatebox{45}{\textbf{Progettista}} \\
	\end{tabular}
	
	\vspace{0.2cm} % Piccolo spazio tra le due tabelle
	
	% Seconda tabella con i dati e i bordi
	\begin{tabular}{|p{0.2\textwidth}|p{0.12\textwidth}|p{0.12\textwidth}|p{0.12\textwidth}|p{0.12\textwidth}|p{0.12\textwidth}|p{0.12\textwidth}|}
		\hline
		M. Mazzaretto & - & 2 & 3 & - & 1 & - \\ \hline
		T. Ceron      & - & - & 6 & - & 1 & - \\ \hline
		L. Pieripolli & - & 5 & - & - & 1 & - \\ \hline
		L. Marcuzzo   & - & - & 6 & - & 1 & - \\ \hline
		M. Brunello   & - & 5 & - & - & - & - \\ \hline
		L. Casagrande & 4 & - & - & - & - & - \\ \hline
	\end{tabular}
	\caption{Sprint$^G$ 2: Preventivo}
\end{table}

\begin{figure}[H]
	\centering
	\includegraphics[width=0.8\textwidth]{PianoProgetto/sprint01_preventivo}
	\caption{Sprint$^G$ 2: Preventivo}
\end{figure}

\subsubsubsection{Consuntivo}

\subsubsubsection{Aggiornamento delle risorse rimanenti}

\subsubsubsection{Rischi incontrati}

\subsubsubsection{Retrospettiva}
