\section{Analisi e gestione dei rischi}
In questa sezione il gruppo descrive l'analisi dei potenziali rischi associati al progetto.\\
Per mitigare tali rischi, abbiamo scelto di adottare un consuntivo più contenuto all'inizio di ogni sprint, in particolare nelle fasi iniziali del progetto quando la padronanza delle metodologie di lavoro è minore.\\
Ogni rischio è stato identificato ed analizzato, in seguito sono state inserite delle possibili tecniche di mitigazione e di controllo.\\
Siccome la maggior parte dei rischi può appartenere a più categorie, ovvero individuale, di gruppo oppure legata a fattori esterni, è stato deciso di descrivere ogni rischio con un unico codice: \textbf{R\#}, dove \# è un identificatore incrementale che indica il numero del rischio.\\
Ogni rischio possiede una frequenza probabile di avvenimento e una ripercussione, le quali appartengono all'insieme \{Molto basso, Basso, Medio, Elevato, Molto elevato\}.

\subsection{R1: Rischi da altre attività universitarie}

\vspace{2mm}

\begin{table}[h]
	\centering
	\renewcommand{\arraystretch}{1.15}
	\begin{tabular}{|>{\bfseries}m{0.25\textwidth}|m{0.65\textwidth}|}
		\hline
		\multicolumn{1}{|c|}{\textbf{Tipologia Dato}} & \multicolumn{1}{c|}{\textbf{Descrizione}} \\
		\hline
		Codice & R1 \\
		\hline
		Nome & Attività universitarie. \\
		\hline
		Descrizione & Tutti i componenti del gruppo hanno almeno un altro esame oltre a questo corso di Ingegneria del Software. \\
		\hline
		Mitigazione & All'inizio di ogni sprint si tiene conto di quanto si è impegnati con lo studio di altri corsi, riducendo il consuntivo orario per renderlo più realistico. \\
		\hline
		Frequenza \newline probabile di \newline avvenimento & Media \\
		\hline
		Ripercussioni & Elevate \\
		\hline
	\end{tabular}
	\caption{Analisi del rischio - Attività universitarie}
	\label{tab:rischio-universitario}
\end{table}

\subsection{R2: Rischi da altre attività extra-universitarie}

\vspace{2mm}

\begin{table}[H]
	\centering
	\resizebox{\textwidth}{!}{
		\begin{tabular}{|>{\bfseries}m{0.25\textwidth}|m{0.65\textwidth}|}
			\hline
			\multicolumn{1}{|c|}{\textbf{Tipologia Dato}} & \multicolumn{1}{c|}{\textbf{Descrizione}} \\
			\hline
			\textbf{Codice} & R2 \\
			\hline
			\textbf{Nome} & Attività extra-universitarie. \\
			\hline
			\textbf{Descrizione} & Essendo un gruppo ad intensità media, le nostre attività extra-universitarie sono elevate, come lavoro, sport praticati o altri impegni personali.\\
			\hline
			\textbf{Mitigazione} & All'inizio di ogni sprint si tiene conto di quanto si è impegnati con altre attività, riducendo il consuntivo orario per renderlo più realistico. \\
			\hline
			\textbf{Frequenza probabile di avvenimento} & Elevata \\
			\hline
			\textbf{Ripercussioni} & Medie, grazie alla nostra intensità dichiarata \\
			\hline
	\end{tabular}}
	\caption{Informazioni sul rischio R2}
\end{table}

\subsection{R3: Rischi da imprevisti}

\vspace{2mm}

\begin{table}[H]
	\centering
	\resizebox{\textwidth}{!}{
		\begin{tabular}{|>{\bfseries}m{0.25\textwidth}|m{0.65\textwidth}|}
			\hline
			\multicolumn{1}{|c|}{\textbf{Tipologia Dato}} & \multicolumn{1}{c|}{\textbf{Descrizione}} \\
			\hline
			\textbf{Codice} & R3 \\
			\hline
			\textbf{Nome} & Da imprevisti. \\
			\hline
			\textbf{Descrizione} & Abbiamo deciso di inserire questo rischio poiché è naturale che qualcuno possa avere un imprevisto di qualunque tipo, che sia universitario, familiare o di qualsiasi altra natura.\\
			\hline
			\textbf{Mitigazione} & La persona che ha un imprevisto è tenuta ad avvisare subito gli altri componenti del gruppo per permettere la suddivisione del suo lavoro fra gli altri membri. \\
			\hline
			\textbf{Frequenza probabile di avvenimento} & Bassa \\
			\hline
			\textbf{Ripercussioni} & Molto elevate \\
			\hline
	\end{tabular}}
	\caption{Informazioni sul rischio R3}
\end{table}

\subsection{R4: Rischi tecnologici}

\vspace{2mm}

\begin{table}[H]
	\centering
	\resizebox{\textwidth}{!}{
		\begin{tabular}{|>{\bfseries}m{0.25\textwidth}|m{0.65\textwidth}|}
			\hline
			\multicolumn{1}{|c|}{\textbf{Tipologia Dato}} & \multicolumn{1}{c|}{\textbf{Descrizione}} \\
			\hline
			\textbf{Codice} & R4 \\
			\hline
			\textbf{Nome} & Da tecnologie. \\
			\hline
			\textbf{Descrizione} & Nessun componente del gruppo ha mai lavorato a progetti di queste dimensioni con front-end$^G$ e back-end$^G$ così complessi, con tecnologie eterogenee fra di loro.\newline
			Sono considerati anche i potenziali errori nel codice, i quali possono portare ad una sessione di debugging$^G$ molto lunga e complessa portando via molto tempo produttivo.\\
			\hline
			\textbf{Mitigazione} & All'inizio di ogni Sprint si considera il livello di conoscenza delle tecnologie coinvolte, riducendo di conseguenza le ore produttive previste, poiché gran parte del tempo sarà dedicata ad acquisire le competenze necessarie per svolgere le task$^G$ dello Sprint. \\
			\hline
			\textbf{Frequenza probabile di avvenimento} & Elevata \\
			\hline
			\textbf{Ripercussioni} & Elevate \\
			\hline
	\end{tabular}}
	\caption{Informazioni sul rischio R4}
\end{table}

\subsection{R5: Rischi da incomprensione}

\vspace{2mm}

\begin{table}[H]
	\centering
	\resizebox{\textwidth}{!}{
		\begin{tabular}{|>{\bfseries}m{0.25\textwidth}|m{0.65\textwidth}|}
			\hline
			\multicolumn{1}{|c|}{\textbf{Tipologia Dato}} & \multicolumn{1}{c|}{\textbf{Descrizione}} \\
			\hline
			\textbf{Codice} & R5 \\
			\hline
			\textbf{Nome} & Da incomprensioni. \\
			\hline
			\textbf{Descrizione} & Ogni individuo incaricato può svolgere male il proprio compito a causa di un'incomprensione della task o di una sua definizione poco chiara.\\
			\hline
			\textbf{Mitigazione} & Ogni componente del gruppo deve essere sicuro di ciò che sta per fare e il gruppo deve definire in maniera chiara e precisa le task. \\
			\hline
			\textbf{Frequenza probabile di avvenimento} & Molto bassa \\
			\hline
			\textbf{Ripercussioni} & Elevate \\
			\hline
	\end{tabular}}
	\caption{Informazioni sul rischio R5}
\end{table}

