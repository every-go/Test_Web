\section{Analisi e gestione dei rischi}
In questa sezione il gruppo descrive l'analisi dei potenziali rischi associati al progetto.\\
Per mitigare tali rischi, abbiamo scelto di adottare un consuntivo$^G$ più contenuto all'inizio di ogni sprint, in particolare nelle fasi iniziali del progetto quando la padronanza delle metodologie di lavoro è minore.\\
Ogni rischio è stato identificato ed analizzato, e sono state definite apposite strategie di mitigazione e controllo.\\
I rischi possono appartenere a tre macro–categorie: 
\begin{itemize}[itemsep=5pt, parsep=1pt, label=$\scriptstyle\bullet$]
    \item \textbf{RE}(Rischi Economici): rischi che influenzano tempi, costi, produttività o impegno orario del gruppo.
    \item \textbf{RP}(Rischi Personali): rischi legati ai singoli membri, alle loro attività, disponibilità, competenze e condizioni individuali.
    \item \textbf{RA}(Rischi Ambientali): rischi dovuti a fattori esterni, imprevisti o condizioni non controllabili direttamente dal gruppo.
\end{itemize} 
Ciascun rischio è identificato tramite un codice che ne identifica la categoria e un numero incrementale  \#.\\ (es. \textbf{RE\#}, \textbf{RP\#}, \textbf{RA\#}).\\
Ogni rischio possiede una frequenza probabile di avvenimento e un impatto, espressi su una scala: \{Molto basso, Basso, Medio, Elevato, Molto elevato\}.

\subsection{RE1: Rischi tecnologici}
\vspace{2mm}
\begin{table}[H]
	\centering
	\renewcommand{\arraystretch}{1.15}
	\resizebox{\textwidth}{!}{
		\begin{tabular}{|>{\bfseries}m{0.25\textwidth}|m{0.65\textwidth}|}
			\hline
			\multicolumn{1}{|c|}{\textbf{Tipologia Dato}} & \multicolumn{1}{c|}{\textbf{Descrizione}} \\ \hline
			Codice & RE1 \\ \hline
			Nome & Tecnologie \\ \hline
			Descrizione & Il gruppo non ha esperienza in progetti di queste dimensioni con front-end$^G$ e back-end$^G$ complessi e tecnologie eterogenee. Sono inclusi possibili errori e lunghe sessioni di debugging$^G$. \\ \hline
			Mitigazione & All'inizio di ogni sprint$^G$ si valuta il livello di conoscenza delle tecnologie, riducendo le ore produttive previste per includere il tempo di apprendimento. \\ \hline
			Frequenza \newline probabile di \newline avvenimento & Elevata \\ \hline
			Ripercussioni & Elevate \\ \hline
	\end{tabular}}
	\caption{Informazioni sul rischio RE1}
\end{table}

\subsection{RE2: Sovraccarico o stima errata del lavoro}
\vspace{2mm}
\begin{table}[H]
	\centering
	\renewcommand{\arraystretch}{1.15}
	\resizebox{\textwidth}{!}{
		\begin{tabular}{|>{\bfseries}m{0.25\textwidth}|m{0.65\textwidth}|}
			\hline
			\multicolumn{1}{|c|}{\textbf{Tipologia Dato}} & \multicolumn{1}{c|}{\textbf{Descrizione}} \\ \hline
			Codice & RE2 \\ \hline
			Nome & Stima errata delle attività \\ \hline
			Descrizione & Le task$^G$ possono richiedere più tempo del previsto, causando ritardi negli sprint$^G$ e sovraccarico dei membri del gruppo. \\ \hline
			Mitigazione & Pianificare task$^G$ realistiche e suddividere task$^G$ grandi in sotto-task. \\ \hline
			Frequenza \newline probabile di \newline avvenimento & Media \\ \hline
			Ripercussioni & Elevate \\ \hline
	\end{tabular}}
	\caption{Informazioni sul rischio RE2}
\end{table}



\subsection{RP1: Attività universitarie}
\vspace{2mm}
\begin{table}[H]
	\centering
	\renewcommand{\arraystretch}{1.15}
	\begin{tabular}{|>{\bfseries}m{0.25\textwidth}|m{0.65\textwidth}|}
		\hline
		\multicolumn{1}{|c|}{\textbf{Tipologia Dato}} & \multicolumn{1}{c|}{\textbf{Descrizione}} \\ \hline
		Codice & RP1 \\ \hline
		Nome & Attività universitarie \\ \hline
		Descrizione & Tutti i componenti del gruppo hanno almeno un altro esame oltre a questo corso di Ingegneria del Software. \\ \hline
		Mitigazione & All'inizio di ogni sprint$^G$ si tiene conto dell'impegno richiesto da altri corsi, riducendo il consuntivo$^G$ orario per renderlo più realistico. \\ \hline
		Frequenza \newline probabile di \newline avvenimento & Media \\ \hline
		Ripercussioni & Elevate \\ \hline
	\end{tabular}
	\caption{Informazioni sul rischio RP1}
\end{table}

\subsection{RP2: Attività extra-universitarie}
\vspace{2mm}
\begin{table}[H]
	\centering
	\renewcommand{\arraystretch}{1.15}
	\resizebox{\textwidth}{!}{
	\begin{tabular}{|>{\bfseries}m{0.25\textwidth}|m{0.65\textwidth}|}
		\hline
		\multicolumn{1}{|c|}{\textbf{Tipologia Dato}} & \multicolumn{1}{c|}{\textbf{Descrizione}} \\ \hline
		Codice & RP2 \\ \hline
		Nome & Attività extra-universitarie \\ \hline
		Descrizione & Essendo un gruppo a intensità media, le nostre attività extra-universitarie (lavoro, sport, impegni personali) sono frequenti. \\ \hline
		Mitigazione & All'inizio di ogni sprint$^G$ si tiene conto degli impegni extra, riducendo il consuntivo$^G$ orario per renderlo coerente alla disponibilità reale. \\ \hline
		Frequenza \newline probabile di \newline avvenimento & Elevata \\ \hline
		Ripercussioni & Medie \\ \hline
	\end{tabular}}
	\caption{Informazioni sul rischio RP2}
\end{table}

\subsection{RP3: Incomprensioni}
\vspace{2mm}
\begin{table}[H]
	\centering
	\renewcommand{\arraystretch}{1.15}
	\resizebox{\textwidth}{!}{
	\begin{tabular}{|>{\bfseries}m{0.25\textwidth}|m{0.65\textwidth}|}
		\hline
		\multicolumn{1}{|c|}{\textbf{Tipologia Dato}} & \multicolumn{1}{c|}{\textbf{Descrizione}} \\ \hline
		Codice & RP3 \\ \hline
		Nome & Incomprensioni \\ \hline
		Descrizione & Un membro può svolgere male una task$^G$ per incomprensioni o una definizione poco chiara del compito. \\ \hline
		Mitigazione & Il gruppo deve concordare task$^G$ sempre chiare e ogni membro deve verificare di aver compreso correttamente cosa deve fare. \\ \hline
		Frequenza \newline probabile di \newline avvenimento & Molto bassa \\ \hline
		Ripercussioni & Elevate \\ \hline
	\end{tabular}}
	\caption{Informazioni sul rischio RP3}
\end{table}



\subsection{RA1: Imprevisti}
\vspace{2mm}
\begin{table}[H]
	\centering
	\renewcommand{\arraystretch}{1.15}
	\resizebox{\textwidth}{!}{
	\begin{tabular}{|>{\bfseries}m{0.25\textwidth}|m{0.65\textwidth}|}
		\hline
		\multicolumn{1}{|c|}{\textbf{Tipologia Dato}} & \multicolumn{1}{c|}{\textbf{Descrizione}} \\ \hline
		Codice & RA1 \\ \hline
		Nome & Imprevisti \\ \hline
		Descrizione & È naturale che possano verificarsi imprevisti di tipo universitario, familiare o personale, che influenzano la produttività. \\ \hline
		Mitigazione & La persona coinvolta deve avvisare tempestivamente il gruppo per permettere una ridistribuzione del lavoro. \\ \hline
		Frequenza \newline probabile di \newline avvenimento & Bassa \\ \hline
		Ripercussioni & Molto elevate \\ \hline
	\end{tabular}}
	\caption{Informazioni sul rischio RA1}
\end{table}

\subsection{RA2: Rischi legati ai requisiti}
\vspace{2mm}
\begin{table}[H]
	\centering
	\renewcommand{\arraystretch}{1.15}
	\resizebox{\textwidth}{!}{
	\begin{tabular}{|>{\bfseries}m{0.25\textwidth}|m{0.65\textwidth}|}
		\hline
		\multicolumn{1}{|c|}{\textbf{Tipologia Dato}} & \multicolumn{1}{c|}{\textbf{Descrizione}} \\ \hline
		Codice & RA2 \\ \hline
		Nome & Requisiti instabili o incompleti \\ \hline
		Descrizione & I requisiti possono essere interpretati in modo errato o modificati durante il progetto, causando ritardi e lavoro non coerente con le aspettative. \\ \hline
		Mitigazione & Effettuare incontri chiarificatori con il proponente, validare i requisiti dopo ogni sprint$^G$ e produrre documentazione condivisa. \\ \hline
		Frequenza \newline probabile di \newline avvenimento & Media \\ \hline
		Ripercussioni & Elevate \\ \hline
	\end{tabular}}
	\caption{Informazioni sul rischio RA2}
\end{table}


