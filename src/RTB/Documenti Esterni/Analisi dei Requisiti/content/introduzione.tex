\section{Introduzione}
	\subsection{Scopo del documento}
	Il presente documento di \textbf{Analisi dei Requisiti} definisce in modo formale, completo e strutturato i requisiti funzionali e non funzionali del sistema software \textbf{SmartOrder}, da sviluppare nell'ambito del progetto didattico del corso di Ingegneria del Software.
	Il documento descrive i casi d'uso principali del sistema, corredati da diagrammi UML per visualizzare le interazioni tra attori e funzionalità.
	Rappresenta il fondamento per le successive fasi di progettazione, implementazione, testing e validazione, garantendo l'allineamento del prodotto finale con le specifiche del proponente \textbf{Ergon Informatica Srl} e con gli obiettivi delineati nel capitolato C8.

	I requisiti identificati sono classificati secondo le seguenti priorità:
	\vspace{-0.3em}
	\begin{itemize}
	    \setlength\itemsep{-0.1em}
	    \item \textbf{Obbligatori:} essenziali per il funzionamento minimo del sistema e irrinunciabili per il proponente;
	    \item \textbf{Desiderabili:} non critici, ma in grado di apportare un valore aggiunto significativo all'utente finale;
	    \item \textbf{Opzionali:} implementabili in fasi successive o in estensioni future.
	\end{itemize}

	Il documento è redatto dal gruppo \textbf{``NullPointers Group''} ed è destinato a:

	\vspace{-0.3em}
	\begin{itemize}
	    \setlength\itemsep{-0.1em}
	    \item il \textbf{Committente} (Ergon Informatica Srl), per la verifica della corretta interpretazione delle richieste;
	    \item il \textbf{Team di Sviluppo} come linea guida per la progettazione architetturale e la codifica;
	    \item il \textbf{Team di Verifica} per la definizione delle strategie di test e validazione.
	\end{itemize}

    Il documento è inoltre destinato ad altre figure professionali coinvolte nello sviluppo, quali Amministratori e Responsabili di Progetto, per consentire loro di acquisire una piena comprensione delle specifiche di Sistema.

	\subsection{Prospettiva del prodotto}
	SmartOrder si propone come una piattaforma intelligente e multimodale per l'interpretazione automatica di ordini di acquisto provenienti da canali eterogenei – quali testo (email, chat), audio (chiamate, messaggi vocali) e immagini (foto, documenti) – e la loro trasformazione in ordini strutturati pronti per l'inserimento in sistemi gestionali ERP.
    Il sistema è progettato secondo un'\textbf{architettura modulare} e \textbf{scalabile}, in grado di integrare modelli avanzati di intelligenza artificiale (LLM, visione artificiale, speech-to-text) e di adattarsi a volumi elevati di dati mantenendo elevate prestazioni.
	L'obiettivo è ridurre drasticamente l'intervento umano in attività ripetitive e a basso valore aggiunto, minimizzando gli errori di interpretazione e migliorando l'efficienza operativa e la soddisfazione del cliente finale.
	
	\subsection{Funzioni del prodotto}
	Il sistema dovrà offrire le seguenti funzionalità principali:
	\begin{itemize}
	    \setlength\itemsep{-0.1em}
	    \item Acquisizione di input multimodali (testo, audio, immagini) da molteplici canali;
	    \item Estrazione automatica di informazioni tramite pipeline di NLP, visione artificiale e trascrizione audio;
	    \item Validazione, arricchimento semantico e normalizzazione dei dati estratti;
	    \item Fusione multimodale per un'interpretazione contestuale coerente;
	    \item Generazione di ordini strutturati in formati standard (JSON, XML) compatibili con ERP;
	    \item Integrazione con database aziendali tramite API REST;
	    \item Interfaccia web per il monitoraggio, la gestione e il feedback dei processi;
	    \item Meccanismi di logging e apprendimento continuo per il miglioramento del sistema.
	\end{itemize}
	I requisiti obbligatori sono stati definiti in accordo con le indicazioni del proponente e con quanto emerso dai colloqui preliminari.

	\subsection{Caratteristiche dell'utente}
	Il sistema si rivolge principalmente a:
	\begin{itemize}
	    \setlength\itemsep{-0.1em}
	    \item \textbf{Operatori aziendali} addetti all'inserimento e alla validazione degli ordini;
	    \item \textbf{Amministratori di sistema} per il monitoraggio, la configurazione e la manutenzione della piattaforma;
	    \item \textbf{Clienti finali} che inviano ordini tramite canali non strutturati (es. email, WhatsApp).
	\end{itemize}
	Non sono richieste competenze tecniche avanzate per l'utilizzo delle funzionalità base, mentre la configurazione avanzata e il monitoraggio sono riservati a utenti con ruolo amministrativo.

	\subsection{Definizioni, acronimi e abbreviazioni}
	Per tutti i termini tecnici, gli acronimi e le definizioni utilizzate nel documento si rimanda al \underline{\href{https://nullpointersgroup.github.io/Documentazione/output/RTB/Documentazione_interna/Glossario.pdf}{Glossario}}, disponibile come documento separato. \\
	Ogni parola presente nel Glossario viene segnata, solo alla prima occorrenza, come segue:
	\begin{center}
        termine$^{G}$
    \end{center}

	\subsection{Riferimenti}
	\subsubsection{Riferimenti normativi}
	\underline{\href{https://www.math.unipd.it/~tullio/IS-1/2025/Progetto/C8.pdf}{Capitolato d’appalto C8 – SmartOrder: Analisi multimodale per la creazione automatica di ordini}} \\[0.5em]
	
	\subsubsection{Riferimenti informativi}
	\underline{\href{https://www.math.unipd.it/~tullio/IS-1/2025/Dispense/T05.pdf}{Dispense ''Analisi dei requisiti'', relative alla quinta lezione del professor Tullio}} \\[0.5em]
	\underline{\href{https://ieeexplore.ieee.org/document/720574}{Specifica dei requisiti software secondo IEEE 830-1998}} \\[0.5em]
	\underline{\href{https://www.math.unipd.it/~tullio/IS-1/2009/Approfondimenti/ISO_12207-1995.pdf}{Standard ISO/IEC 12207:1995}} \\[0.5em]
    \underline{\href{https://nullpointersgroup.github.io/Documentazione/output/RTB/Documentazione_interna/Glossario.pdf}{Glossario}} \\[0.5em]