\section{Qualità di Prodotto}
La qualità del prodotto rappresenta la capacità del software di soddisfare in modo completo e affidabile i requisiti funzionali e non funzionali stabiliti.\\
Per garantire un monitoraggio continuo e oggettivo, il gruppo ha definito un insieme di metriche che consentono di valutare diversi aspetti del prodotto finale. Le metriche selezionate si ispirano ai principi degli standard internazionali per la qualità del software e permettono di misurare il livello qualitativo raggiunto.\\
Le categorie considerate sono: Funzionalità$^G$, Affidabilità, Usabilità, Efficienza$^G$ e Manutenibilità.

\subsection{Funzionalità}
La funzionalità$^G$ descrive la capacità del software di offrire tutte le funzioni richieste e di rispondere correttamente alle esigenze esplicitate nei requisiti. 
\begin{table}[h]
    \centering
    \renewcommand{\arraystretch}{1.15}
    \resizebox{\textwidth}{!}{
        \begin{tabular}{|>{\centering\arraybackslash}p{0.13\textwidth}|
                          >{\centering\arraybackslash}p{0.3\textwidth}|
                          >{\centering\arraybackslash}p{0.26\textwidth}|
                          >{\centering\arraybackslash}p{0.26\textwidth}|}
            \hline
            \rowcolor[gray]{0.9}
            \textbf{Metrica} & \textbf{Nome} & \textbf{Valore accettabile} & \textbf{Valore ottimo} \\
            \hline
            MQD\_01 & Requisiti obbligatori soddisfatti &  100\% & 100\% \\
            \hline
            MQD\_02 & Requisiti desiderabili soddisfatti  &  $\geq 30\%$ & 100\% \\
            \hline
            MQD\_03 & Requisiti opzionali soddisfatti & $\geq 0\%$ & $\geq 70\%$ \\
            \hline
        \end{tabular}
    }
    \vspace{-0.3cm}
    \caption{Funzionalità del prodotto}
\end{table}

\subsection{Affidabilità}
L’affidabilità rappresenta la capacità del prodotto software di operare in modo corretto e continuo nel tempo, riducendo al minimo guasti, malfunzionamenti o comportamenti imprevisti.
\begin{table}[h]
    \centering
    \renewcommand{\arraystretch}{1.15}
    \resizebox{\textwidth}{!}{
        \begin{tabular}{|>{\centering\arraybackslash}p{0.13\textwidth}|
                          >{\centering\arraybackslash}p{0.3\textwidth}|
                          >{\centering\arraybackslash}p{0.26\textwidth}|
                          >{\centering\arraybackslash}p{0.26\textwidth}|}
            \hline
            \rowcolor[gray]{0.9}
            \textbf{Metrica} & \textbf{Nome} & \textbf{Valore accettabile} & \textbf{Valore ottimo} \\
            \hline
            MQD\_04 & Code Coverage & $\geq 60\%$ & $\geq 80\%$ \\
            \hline
            MQD\_05 & Branch$^G$ Coverage  & $\geq 60\%$ & $\geq 80\%$ \\
            \hline
            MQD\_06 & Statement Coverage & $\geq 60\%$ & $\geq 80\%$ \\
            \hline
        \end{tabular}
    }
    \vspace{-0.3cm}
    \caption{Affidabilità del prodotto}
\end{table}

\subsection{Usabilità}
L’usabilità valuta quanto il software sia semplice da comprendere, apprendere e utilizzare per l’utente finale. 
\begin{table}[h]
    \centering
    \renewcommand{\arraystretch}{1.15}
    \resizebox{\textwidth}{!}{
        \begin{tabular}{|>{\centering\arraybackslash}p{0.13\textwidth}|
                          >{\centering\arraybackslash}p{0.3\textwidth}|
                          >{\centering\arraybackslash}p{0.26\textwidth}|
                          >{\centering\arraybackslash}p{0.26\textwidth}|}
            \hline
            \rowcolor[gray]{0.9}
            \textbf{Metrica} & \textbf{Nome} & \textbf{Valore accettabile} & \textbf{Valore ottimo} \\
            \hline
            MQD\_07 & Time on Task$^G$ & $\leq 60$ secondi & $\leq 30$ secondi \\
            \hline
        \end{tabular}
    }
    \vspace{-0.3cm}
    \caption{Usabilità del prodotto}
\end{table}

\subsection{Efficienza}
L’efficienza$^G$ indica la capacità del software di utilizzare in modo ottimale le risorse disponibili, garantendo prestazioni adeguate in termini di tempi di risposta, consumo di memoria, utilizzo della CPU e gestione del carico.
\begin{table}[h]
    \centering
    \renewcommand{\arraystretch}{1.15}
    \resizebox{\textwidth}{!}{
        \begin{tabular}{|>{\centering\arraybackslash}p{0.13\textwidth}|
                          >{\centering\arraybackslash}p{0.3\textwidth}|
                          >{\centering\arraybackslash}p{0.26\textwidth}|
                          >{\centering\arraybackslash}p{0.26\textwidth}|}
            \hline
            \rowcolor[gray]{0.9}
            \textbf{Metrica} & \textbf{Nome} & \textbf{Valore accettabile} & \textbf{Valore ottimo} \\
            \hline
            MQD\_08 & Response time & $\leq 2$ secondi & $\leq 1$ secondo \\
            \hline
        \end{tabular}
    }
    \vspace{-0.3cm}
    \caption{Efficienza del prodotto}
\end{table}

\subsection{Manutenibilità}
La manutenibilità esprime la facilità con cui il software può essere modificato, corretto, ampliato o migliorato nel tempo.
\begin{table}[h]
    \centering
    \renewcommand{\arraystretch}{1.15}
    \resizebox{\textwidth}{!}{
        \begin{tabular}{|>{\centering\arraybackslash}p{0.13\textwidth}|
                          >{\centering\arraybackslash}p{0.3\textwidth}|
                          >{\centering\arraybackslash}p{0.26\textwidth}|
                          >{\centering\arraybackslash}p{0.26\textwidth}|}
            \hline
            \rowcolor[gray]{0.9}
            \textbf{Metrica} & \textbf{Nome} & \textbf{Valore accettabile} & \textbf{Valore ottimo} \\
            \hline
            MQD\_09 & Code Smells per KLOC$^G$ & $\leq 15$ & $\leq 5$ \\
            \hline
            MQD\_10 & Coefficient of Coupling & $\leq 0.35$ & $\leq 0.2$ \\
            \hline
            MQD\_11 & Cyclomatic Complexity$^G$ & $\leq 20$ & $\leq 10$ \\
            \hline
        \end{tabular}
    }
    \vspace{-0.3cm}
    \caption{Manutenibilità del prodotto}
\end{table}