\section{Qualità di Processo}
La qualità del prodotto software dipende direttamente dai processi usati per svilupparlo.\\  
Misurare e monitorare questi processi (con metriche adatte) permette di trovare problemi prima che si trasformino in difetti nel software finale, risparmiando tempo e risorse.\\
In base allo standard ISO/IEC 12207:1995, le metriche di qualità di processo$^G$ possono essere suddivise nelle seguenti tre categorie:
\begin{itemize}[itemsep=0pt, parsep=1pt, label=$\scriptstyle\bullet$]
        \item \textbf{Processi primari};
        \item \textbf{Processi di supporto};
        \item \textbf{Processi organizzativi}.
\end{itemize}

\subsection{Processi primari}
Sono l’elemento centrale dello sviluppo$^G$ e generano valore diretto per il cliente.\\ 
Comprendono attività come l’acquisizione dei requisiti, la progettazione$^G$, l'imple-\newline mentazione, il testing e la manutenzione$^G$ del software.\\
Le metriche associate misurano l’efficacia$^G$ dell’intero ciclo di vita del prodotto.

\subsubsection{Fornitura} \vspace{-0.2cm}
\begin{table}[h]
    \centering
    \renewcommand{\arraystretch}{1.15}
    \resizebox{\textwidth}{!}{
        \begin{tabular}{|>{\centering\arraybackslash}p{0.13\textwidth}|
                          >{\centering\arraybackslash}p{0.3\textwidth}|
                          >{\centering\arraybackslash}p{0.26\textwidth}|
                          >{\centering\arraybackslash}p{0.26\textwidth}|}
            \hline
            \rowcolor[gray]{0.9}
            \textbf{Metrica} & \textbf{Nome} & \textbf{Valore accettabile} & \textbf{Valore ottimo} \\
            \hline
            MQC\_01 & Earned Value$^G$ & $\geq$ PV*0.8 & $\geq PV$ \\
            \hline
            MQC\_02 & Planned Value$^G$ & $\geq 0$ & $\leq$ BAC$^G$ \\
            \hline
            MQC\_03 & Actual Cost$^G$ & $\geq$ EV$^G$ e $\leq$ EV*1.1 & $\leq$ EV$^G$ \\
            \hline
            MQC\_04 & Cost Performance Index & $\geq 0.9$ & $\geq 1.0$ \\
            \hline
            MQC\_05 & Schedule Performance Index & $\geq 0.9$ & $\geq 1.0$ \\
            \hline
            MQC\_06 & Estimate At Completion$^G$ & entro $\pm 10\%$ del BAC$^G$ & entro $\pm 5\%$ del BAC$^G$ \\
            \hline
            MQC\_07 & Estimate To Complete & $\leq$ (BAC$^G$ - AC$^G$)*1.05 & $\leq$ (BAC$^G$ - AC$^G$) \\
            \hline
            MQC\_08 & Time Estimate At Completion$^G$ & entro $\pm 5\%$ del tempo pianificato & $\leq$ 2 settimane rispetto al tempo pianificato \\
            \hline
        \end{tabular}
    }
    \vspace{-0.3cm}
    \caption{Processi primari - Metriche del processo di fornitura}
\end{table}

\subsubsection{Sviluppo}\vspace{-0.2cm}
\begin{table}[H]
    \centering
    \renewcommand{\arraystretch}{1.15}
    \resizebox{\textwidth}{!}{
        \begin{tabular}{|>{\centering\arraybackslash}p{0.13\textwidth}|
                          >{\centering\arraybackslash}p{0.3\textwidth}|
                          >{\centering\arraybackslash}p{0.26\textwidth}|
                          >{\centering\arraybackslash}p{0.26\textwidth}|}
            \hline
            \rowcolor[gray]{0.9}
            \textbf{Metrica} & \textbf{Nome} & \textbf{Valore accettabile} & \textbf{Valore ottimo} \\
            \hline
            MQC\_09 & Completeness Issue$^G$ & $\geq 0.8$ & 1.0 \\
            \hline
        \end{tabular}
    }
    \vspace{-0.3cm}
    \caption{Processi primari - Metriche del processo di sviluppo}
\end{table}

\subsection{Processi di supporto}
Sostenengono i processi primari, assicurando che dispongano delle risorse e delle condizioni necessarie.\\
Rientrano in questa categoria processi come la gestione della configurazione, l'assicurazione della qualità, la verifica$^G$ e la convalida.\\  
Le relative metriche servono a valutare quanto bene questo \vr{supporto} funzioni e contribuisca al successo del progetto$^G$.

\subsubsection{Documentazione}\vspace{-0.2cm}
\begin{table}[h]
    \centering
    \renewcommand{\arraystretch}{1.15}
    \resizebox{\textwidth}{!}{
        \begin{tabular}{|>{\centering\arraybackslash}p{0.13\textwidth}|
                          >{\centering\arraybackslash}p{0.3\textwidth}|
                          >{\centering\arraybackslash}p{0.26\textwidth}|
                          >{\centering\arraybackslash}p{0.26\textwidth}|}
            \hline
            \rowcolor[gray]{0.9}
            \textbf{Metrica} & \textbf{Nome} & \textbf{Valore accettabile} & \textbf{Valore ottimo} \\
            \hline
            MQC\_10 & Indice di Gulpease & $\geq 50$ & $\geq 70$ \\
            \hline
        \end{tabular}
    }
    \vspace{-0.3cm}
    \caption{Processi di supporto - Metriche del processo di documentazione}
\end{table}

\subsubsection{Verifica}\vspace{-0.2cm}
\begin{table}[h]
    \centering
    \renewcommand{\arraystretch}{1.15}
    \resizebox{\textwidth}{!}{
        \begin{tabular}{|>{\centering\arraybackslash}p{0.13\textwidth}|
                          >{\centering\arraybackslash}p{0.3\textwidth}|
                          >{\centering\arraybackslash}p{0.26\textwidth}|
                          >{\centering\arraybackslash}p{0.26\textwidth}|}
            \hline
            \rowcolor[gray]{0.9}
            \textbf{Metrica} & \textbf{Nome} & \textbf{Valore accettabile} & \textbf{Valore ottimo} \\
            \hline
            MQC\_11 & Test$^G$ Success Rate & $\geq 80\%$ & 100\% \\
            \hline
        \end{tabular}
    }
    \vspace{-0.3cm}
    \caption{Processi di supporto - Metriche del processo di verifica}
\end{table}

\subsection{Processi Organizzativi}
Determinano la strategia e il quadro operativo generale. Comprendono la gestione, il miglioramento dei processi, la formazione e l'infrastruttura.\\
Le loro metriche valutano quanto l'organizzazione sappia gestire, imparare e adattarsi.\\
\begin{table}[h]
    \centering
    \renewcommand{\arraystretch}{1.15}
    \resizebox{\textwidth}{!}{
        \begin{tabular}{|>{\centering\arraybackslash}p{0.13\textwidth}|
                          >{\centering\arraybackslash}p{0.3\textwidth}|
                          >{\centering\arraybackslash}p{0.26\textwidth}|
                          >{\centering\arraybackslash}p{0.26\textwidth}|}
            \hline
            \rowcolor[gray]{0.9}
            \textbf{Metrica} & \textbf{Nome} & \textbf{Valore accettabile} & \textbf{Valore ottimo} \\
            \hline
            MQC\_12 & Quality Metrics Satisfied & $\geq 80\%$ & 100\% \\
            \hline
        \end{tabular}
    }
    \vspace{-0.3cm}
    \caption{Processi organizzativi}
\end{table}