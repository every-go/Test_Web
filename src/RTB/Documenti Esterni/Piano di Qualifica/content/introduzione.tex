\section{Introduzione}
	\subsection{Scopo del documento}
	Il presente documento ha l’obiettivo di definire in modo chiaro e sistematico come verrà garantita, monitorata e valutata la qualità all’interno del progetto$^G$ SmartOrder.\\
	Il documento stabilisce i criteri, le metriche e le procedure necessarie affinché il prodotto software e i processi che ne regolano la realizzazione rispettino gli standard qualitativi previsti.\\
	In particolare, il piano descrive:
	\begin{itemize}[itemsep=0pt, parsep=1pt, label=$\scriptstyle\bullet$]
        \item \textbf{Gli obiettivi di qualità} che il gruppo intende raggiungere, sia per il prodotto finale sia per le attività di processo$^G$;
        \item \textbf{Le metriche di valutazione} utilizzate per misurare in modo oggettivo l’avanzamento e la conformità del progetto$^G$ agli standard da noi definiti;
        \item \textbf{Le modalità di verifica$^G$ e controllo}, comprendenti le attività di testing, analisi e monitoraggio continuo;
        \item \textbf{Le strategie di miglioramento}, che consentono al gruppo di correggere eventuali criticità e mantenere un livello qualitativo costante nel tempo.
    \end{itemize}
	Il documento è soggetto ad aggiornamenti durante tutto il ciclo di vita del progetto$^G$, per recepire nuove informazioni, modifiche ai requisiti e risultati delle verifiche effettuate.\\
	Si configura quindi come uno strumento guida per assicurare che il prodotto soddisfi le aspettative e rispetti gli standard qualitativi stabiliti.

	\subsection{Glossario}
    Per assicurare una comprensione univoca dei termini usati in questo Piano di Qualifica$^G$, si fornisce un glossario dedicato.\\
	La nomenclatura utilizzata per segnalare che la definizione di una parola è contenuta nel glossario è la seguente: 
    \begin{center}
        termine$^{G}$
    \end{center}

    I termini sono ordinati alfabeticamente per facilitarne la consultazione e vengono inclusi sia termini tecnici che acronimi significativi.\\
    Il gruppo si impegna a visionare il Glossario periodicamente, per permettere la più completa comprensione di ogni tipo di documento pubblicato e per mantenere un allineamento semantico costante tra tutti i partecipanti al progetto$^G$. 


	\subsection{Riferimenti}
	\subsubsection{Riferimenti normativi}

	\begin{itemize}[itemsep=5pt, parsep=5pt, label=$\scriptstyle\bullet$]

    \item \textbf{Norme di Progetto$^G$, versione 1.0.0}\\
    \url{https://nullpointersgroup.github.io/Documentazione/output/RTB/Documenti\%20Interni/Norme\_di\_Progetto.pdf}\\[3pt]
    \textbf{Ultima consultazione: 5 Dicembre 2025}

    \item \textbf{Capitolato$^G$ C8 - Ergon Informatica Srl - SmartOrder}\\
    \url{https://www.math.unipd.it/~tullio/IS-1/2025/Progetto/C8.pdf}\\[3pt]
    \textbf{Ultima consultazione: 3 Dicembre 2025}

	\end{itemize}

	\subsubsection{Riferimenti informativi}
	\begin{itemize}[itemsep=5pt, parsep=5pt, label=$\scriptstyle\bullet$]

    \item \textbf{Standard ISO/IEC 9126}\\
     \url{https://en.wikipedia.org/wiki/ISO/IEC\_9126}\\[3pt]
    \textbf{Ultima consultazione: 3 Dicembre 2025}

    \item \textbf{Standard ISO/IEC/IEEE$^G$ 12207:1995}\\
     \url{https://www.math.unipd.it/~tullio/IS-1/2009/Approfondimenti/ISO\_12207-1995.pdf}\\[3pt]
    \textbf{Ultima consultazione: 3 Dicembre 2025}

    \item \textbf{Documentazione su qualità di prodotto}\\
     \url{https://www.math.unipd.it/~tullio/IS-1/2025/Dispense/T07.pdf}\\[3pt]
    \textbf{Ultima consultazione: 5 Dicembre 2025}

	\item \textbf{Documentazione su qualità di processo$^G$}\\
     \url{https://www.math.unipd.it/~tullio/IS-1/2025/Dispense/T08.pdf}\\[3pt]
    \textbf{Ultima consultazione: 3 Dicembre 2025}

	\item \textbf{Materiale su verifica$^G$ e validazione$^G$}\\
     \url{https://www.math.unipd.it/~tullio/IS-1/2025/Dispense/T09.pdf}\\[3pt]
    \textbf{Ultima consultazione: 5 Dicembre 2025}

	\item \textbf{Glossario, versione 1.0.0}\\
     \url{https://nullpointersgroup.github.io/Documentazione/output/RTB/Documenti\%20Interni/Glossario.pdf}\\[3pt]
    \textbf{Ultima consultazione: 5 Dicembre 2025}

	\end{itemize}