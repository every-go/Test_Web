% Configurazione
\documentclass{article}

\usepackage{titling} % Required for inserting the subtitle
\usepackage{graphicx} % Required for inserting images
\usepackage{tabularx} % Per l'ambiente tabularx (tabelle)
\usepackage{calc} % Sempre per le tabelle
\usepackage{tocloft}
\renewcommand{\cftsecleader}{\cftdotfill{\cftdotsep}}
\usepackage{xcolor}
\usepackage[
	colorlinks=true,
	linkcolor=blue,
	citecolor=blue,
	urlcolor=blue
]{hyperref}
\usepackage{colortbl} % Per colorare le celle delle tabelle
\usepackage{lipsum} % Per generare lorem ipsum
\usepackage[normalem]{ulem} % Per sottolineare il testo
\usepackage{array} % Per la visualizzazione fluttuante di array di domande e risposte
\usepackage{ragged2e} % Pacchetto necessario per \justifying che giustifica il testo di tabelle
\usepackage{placeins} % Impedisce a figure e tabelle di andare oltre le barriere definite
\usepackage{fancyhdr} % Intestazioni e piè di pagina
\usepackage{lastpage} % Per contare il numero totale di pagine
\usepackage{enumitem} 
\usepackage{amssymb} %per personalizzare gli elenchi
\usepackage[T1]{fontenc} %per {} come ""
\usepackage{xurl}
\usepackage{amsmath} % Per i simboli matematici
\usepackage{float}%per bloccare tabella sotto subsection

\newcommand{\ulhref}[2]{\href{#1}{\underline{#2}}} % Sottolineatura
\newcommand{\ulref}[1]{\uline{\ref{#1}}} % Sottolinea riferimenti a figure
\setlength{\parindent}{0pt} % Nessun rientro dei paragrafi
\newcommand{\vr}[1]{\textquotedblleft#1\textquotedblright}% per " "
\newcommand{\vrs}[1]{\textquoteleft#1\textquoteright}% per ' '
\definecolor{green}{HTML}{386257}

\graphicspath{ {immagini/} {../../../shared/images/} }

% --- Impostazioni intestazione e piè di pagina ---
\pagestyle{fancy}
\fancyhf{}
% Intestazione standard
\fancyhead[L]{NullPointers Group}
\fancyhead[R]{Piano di Qualifica}
\renewcommand{\headrulewidth}{0.4pt}
% Piè di pagina
\fancyfoot[C]{Pagina \thepage{} di \pageref*{LastPage}}
\renewcommand{\footrulewidth}{0.4pt}

% --- Ridefinizione dello stile plain per gli indici ---
\fancypagestyle{plain}{
  \fancyhf{}
  \fancyhead[L]{NullPointers Group}
  \fancyhead[R]{Piano di Qualifica}
  \fancyfoot[C]{Pagina \thepage{} di \pageref*{LastPage}}
  \renewcommand{\headrulewidth}{0.4pt}
  \renewcommand{\footrulewidth}{0.4pt}
}

% --- Solo piè di pagina per la prima pagina ---
\fancypagestyle{firstpage}{
  \fancyhf{}
  \fancyfoot[C]{Pagina \thepage{} di \pageref*{LastPage}}
  \renewcommand{\headrulewidth}{0pt}
  \renewcommand{\footrulewidth}{0.4pt}
}

% --- Creazione livello 4 (subsubsubsection) ---
\usepackage{titlesec}

\titleclass{\subsubsubsection}{straight}[\subsubsection]

\newcounter{subsubsubsection}[subsubsection]
\renewcommand{\thesubsubsubsection}{\thesubsubsection.\arabic{subsubsubsection}}

\titleformat{\subsubsubsection}
  {\normalfont\normalsize\bfseries}
  {\thesubsubsubsection}{0.5em}{}

\titlespacing*{\subsubsubsection}{0pt}{1ex}{1ex}

% --- Registrazione livello 4 per tocloft ---
\makeatletter
\newcommand{\l@subsubsubsection}{\@dottedtocline{4}{7.0em}{3.7em}}
\makeatother

% --- Toc depth ---
\setcounter{secnumdepth}{4}
\setcounter{tocdepth}{4}

%struttura
\begin{document}
\thispagestyle{firstpage} % Copertina con solo piè di pagina

%parte costante
\begin{minipage}{0.4\textwidth}
    \includegraphics[width=0.6\textwidth]{logo_unipd.png}
\end{minipage}
\begin{minipage}{0.55\textwidth}
    \textcolor{red}{\textbf{Università degli Studi di Padova}} \\
    \textcolor{red}{Laurea: Informatica} \\
    \textcolor{red}{Corso: Ingegneria del Software} \\
    \textcolor{red}{Anno Accademico: 2025/2026}
\end{minipage}

\begin{minipage}{0.4\textwidth}
    \includegraphics[width=0.6\textwidth]{logo_gruppo.jpeg}
\end{minipage}
\begin{minipage}{0.55\textwidth}
    \textbf{Gruppo: NullPointers Group} \\
    Email: \textsf{groupnullpointers@gmail.com}
\end{minipage}

\vspace{2cm}
{
		\centering
		\Huge\bfseries Norme di Progetto\par
		\vspace{1.0cm}
		\Large\bfseries \par
	}
\begin{center}
\begin{tabular}{r|l}
    Stato & In Approvazione \\[0.5em]
    Versione & 0.2.0 \\[0.5em]
    Data ultima modifica & 15/11/2025 \\[0.5em]
    Redattori & Lisa Casagrande \\
              & Matteo Mazzaretto \\[0.5em]
    Verificatori & Matteo Mazzaretto \\
                 & Tommaso Ceron \\[0.5em]
    Destinatari & Prof. Tullio Vardanega \\
    & Prof. Riccardo Cardin \\
    & Ergon Informatica Srl \\
    & NullPointers Group \\
\end{tabular}
\end{center}
\newpage
	
%variabile prima dell'indice
\section*{Registro delle modifiche}

\begin{table}[h]
	\centering
	\resizebox{\textwidth}{!}{
		\begin{tabular}{|c|c|c|c|c|}
			\hline
			\rowcolor[gray]{0.9}
			Vers & Data & Autore & Verificatore & Descrizione \\
			1.0.0 & 13-11 & L. Casagrande & L. Pieripolli & Approvazione documento \\
			\hline
			0.1.0 & 12-11 & L. Casagrande & M. Mazzaretto & Creazione e stesura documento \\
			\hline
	\end{tabular}}
\end{table}
\newpage


% Indice generale
\renewcommand{\contentsname}{Indice}
{
	\hypersetup{linkcolor=black}
	\tableofcontents
}


\newpage
% Indice delle tabelle
\renewcommand{\listtablename}{Elenco delle Tabelle}
{
	\hypersetup{linkcolor=black}
	\listoftables
}


\newpage
% Indice delle immagini
\renewcommand{\listfigurename}{Elenco delle Immagini}
{
	\hypersetup{linkcolor=black}
	\listoffigures
}

%parte variabile
\newpage
\section{Introduzione}
	\subsection{Scopo del documento}
	Il presente documento nasce per descrivere il Way of Working$^G$ adottato da \textbf{\vr{NullPointers Group}} durante lo svolgimento del progetto$^G$ SmartOrder.\\
	Lo standard di riferimento è l'ISO/IEC 12207:1995, il quale prevede tre tipologie di processi.
	\begin{itemize}
	    \setlength\itemsep{-0.1em}
	    \item \textbf{Processi primari:} processi fondamentali senza i quali un progetto$^G$ non può definirsi tale;
	    \item \textbf{Processi di Supporto:} processi che coadiuvano i processi primari nello svolgimento delle rispettive azioni;
	    \item \textbf{Processi organizzativi:} processi di carattere più generale che aiutano la realizzazione del progetto$^G$.
	\end{itemize}

	La stesura di questo documento è incrementale, cioè una stesura passo passo con modifiche, aggiunte e cancellazioni a seguito di miglioramenti del metodo di lavoro. I membri del gruppo si impegnano a visionare costantemente questo documento e a rispettare rigorosamente le regole definite in esso, per svolgere il progetto$^G$ in modo professionale, coerente ed uniforme.

\subsection{Scopo del prodotto}
	La gestione automatizzata degli ordini di acquisto in contesti multicanale rappresenta una sfida complessa per le aziende moderne, che devono affrontare la necessità di interpretare richieste provenienti da fonti eterogenee come email, chat, messaggi vocali e immagini.\\
	Il capitolato$^G$ numero C8 di Ergon Informatica propone di sviluppare una piattaforma intelligente in grado di analizzare input multimodali e convertirli automaticamente in ordini strutturati, pronti per l'inserimento nei sistemi gestionali aziendali.

	L'obiettivo che si è posto questo gruppo è realizzare un sistema basato su architettura a microservizi che integri tecniche avanzate di intelligenza artificiale$^G$, machine learning$^G$ e natural language processing$^G$, in grado di riconoscere le intenzioni del cliente, estrarre le informazioni rilevanti e validarle in maniera coerente con il catalogo prodotti aziendale. Questo approccio consentirà di ridurre drasticamente l'intervento umano nelle fasi ripetitive, migliorando al contempo l'efficienza$^G$ complessiva e la soddisfazione del cliente finale.

	Il progetto$^G$ SmartOrder si propone quindi di dimostrare come le tecnologie di intelligenza artificiale$^G$ possano essere applicate con successo a processi reali di business, trasformando un compito complesso e frammentato in un flusso lineare, automatizzato e scalabile. L'obiettivo è realizzare questo progetto$^G$ entro il 30 Aprile 2026 con un budget a disposizione di: Euro 11.440.
	
\subsection{Glossario}
    La realizzazione di un sistema software complesso come SmartOrder richiede, ancora prima della scrittura del codice, un'importante operazione di confronto, analisi e progettazione$^G$. Per supportare e facilitare il lavoro asincrono tra i membri del gruppo e garantire una comunicazione efficace con il committente$^G$, tutte le informazioni derivanti da questa attività saranno appositamente documentate in un glossario condiviso, utile per evitare ambiguità o incomprensioni riguardanti la nomenclatura adottata in tutti i documenti visionabili.

    In accordo con quanto stabilito nel \href{https://nullpointersgroup.github.io/Documentazione/output/RTB/Verbali\%20Interni/2025-11-06\_verbale\_interno.pdf}{verbale interno del 6 novembre 2025}, si è deciso 
	di adottare il glossario come strumento ufficiale per la standardizzazione della terminologia di progetto$^G$ e di assegnare la responsabilità della sua manutenzione$^G$ alla figura dell'Analista$^G$.

    La nomenclatura utilizzata per segnalare che la definizione di una parola è contenuta nel glossario è la seguente: 
    \begin{center}
        termine$^{G}$
    \end{center}

    I termini sono ordinati alfabeticamente per facilitarne la consultazione e vengono inclusi sia termini tecnici che acronimi significativi.

    Il gruppo si impegna a visionare il Glossario periodicamente, per permettere la più completa comprensione di ogni tipo di documento pubblicato e per mantenere un allineamento semantico costante tra tutti i partecipanti al progetto$^G$. 

    \subsection{Riferimenti}
	\subsubsection{Riferimenti normativi}

	\begin{itemize}[itemsep=5pt, parsep=5pt, label=$\scriptstyle\bullet$]

    \item \textbf{Capitolato$^G$ C8 - Ergon Informatica Srl - SmartOrder}\\
    \url{https://www.math.unipd.it/~tullio/IS-1/2025/Progetto/C8.pdf}\\[3pt]
    \textbf{Ultima consultazione: 30 Novembre 2025}

	\end{itemize}

	\subsubsection{Riferimenti informativi}
	\begin{itemize}[itemsep=5pt, parsep=5pt, label=$\scriptstyle\bullet$]

    \item \textbf{Standard ISO/IEC 9126}\\
     \url{https://en.wikipedia.org/wiki/ISO/IEC\_9126}\\[3pt]
    \textbf{Ultima consultazione: 13 Dicembre 2025}

    \item \textbf{Standard ISO/IEC/IEEE$^G$ 12207:1995}\\
     \url{https://www.math.unipd.it/~tullio/IS-1/2009/Approfondimenti/ISO\_12207-1995.pdf}\\[3pt]
    \textbf{Ultima consultazione: 12 Dicembre 2025}

	\item \textbf{Glossario, versione 1.0.0}\\
     \url{https://nullpointersgroup.github.io/Documentazione/output/RTB/Documenti\%20Interni/Glossario.pdf}\\[3pt]
    \textbf{Ultima consultazione: 13 Dicembre 2025}

	\end{itemize}


\newpage
\section{Qualità di Processo}
La qualità del prodotto software dipende direttamente dai processi usati per svilupparlo.\\  
Misurare e monitorare questi processi (con metriche adatte) permette di trovare problemi prima che si trasformino in difetti nel software finale, risparmiando tempo e risorse.\\
In base allo standard ISO/IEC 12207:1995, le metriche di qualità di processo$^G$ possono essere suddivise nelle seguenti tre categorie:
\begin{itemize}[itemsep=0pt, parsep=1pt, label=$\scriptstyle\bullet$]
        \item \textbf{Processi primari};
        \item \textbf{Processi di supporto};
        \item \textbf{Processi organizzativi}.
\end{itemize}

\subsection{Processi primari}
Sono l’elemento centrale dello sviluppo$^G$ e generano valore diretto per il cliente.\\ 
Comprendono attività come l’acquisizione dei requisiti, la progettazione$^G$, l'imple-\newline mentazione, il testing e la manutenzione$^G$ del software.\\
Le metriche associate misurano l’efficacia$^G$ dell’intero ciclo di vita del prodotto.

\subsubsection{Fornitura} \vspace{-0.2cm}
\begin{table}[h]
    \centering
    \renewcommand{\arraystretch}{1.15}
    \resizebox{\textwidth}{!}{
        \begin{tabular}{|>{\centering\arraybackslash}p{0.13\textwidth}|
                          >{\centering\arraybackslash}p{0.3\textwidth}|
                          >{\centering\arraybackslash}p{0.26\textwidth}|
                          >{\centering\arraybackslash}p{0.26\textwidth}|}
            \hline
            \rowcolor[gray]{0.9}
            \textbf{Metrica} & \textbf{Nome} & \textbf{Valore accettabile} & \textbf{Valore ottimo} \\
            \hline
            MQC\_01 & Earned Value$^G$ & $\geq$ PV*0.8 & $\geq PV$ \\
            \hline
            MQC\_02 & Planned Value$^G$ & $\geq 0$ & $\leq$ BAC$^G$ \\
            \hline
            MQC\_03 & Actual Cost$^G$ & $\geq$ EV$^G$ e $\leq$ EV*1.1 & $\leq$ EV$^G$ \\
            \hline
            MQC\_04 & Cost Performance Index & $\geq 0.9$ & $\geq 1.0$ \\
            \hline
            MQC\_05 & Schedule Performance Index & $\geq 0.9$ & $\geq 1.0$ \\
            \hline
            MQC\_06 & Estimate At Completion$^G$ & entro $\pm 10\%$ del BAC$^G$ & entro $\pm 5\%$ del BAC$^G$ \\
            \hline
            MQC\_07 & Estimate To Complete & $\leq$ (BAC$^G$ - AC$^G$)*1.05 & $\leq$ (BAC$^G$ - AC$^G$) \\
            \hline
            MQC\_08 & Time Estimate At Completion$^G$ & entro $\pm 5\%$ del tempo pianificato & $\leq$ 2 settimane rispetto al tempo pianificato \\
            \hline
        \end{tabular}
    }
    \vspace{-0.3cm}
    \caption{Processi primari - Metriche del processo di fornitura}
\end{table}

\subsubsection{Sviluppo}\vspace{-0.2cm}
\begin{table}[H]
    \centering
    \renewcommand{\arraystretch}{1.15}
    \resizebox{\textwidth}{!}{
        \begin{tabular}{|>{\centering\arraybackslash}p{0.13\textwidth}|
                          >{\centering\arraybackslash}p{0.3\textwidth}|
                          >{\centering\arraybackslash}p{0.26\textwidth}|
                          >{\centering\arraybackslash}p{0.26\textwidth}|}
            \hline
            \rowcolor[gray]{0.9}
            \textbf{Metrica} & \textbf{Nome} & \textbf{Valore accettabile} & \textbf{Valore ottimo} \\
            \hline
            MQC\_09 & Completeness Issue$^G$ & $\geq 0.8$ & 1.0 \\
            \hline
        \end{tabular}
    }
    \vspace{-0.3cm}
    \caption{Processi primari - Metriche del processo di sviluppo}
\end{table}

\subsection{Processi di supporto}
Sostenengono i processi primari, assicurando che dispongano delle risorse e delle condizioni necessarie.\\
Rientrano in questa categoria processi come la gestione della configurazione, l'assicurazione della qualità, la verifica$^G$ e la convalida.\\  
Le relative metriche servono a valutare quanto bene questo \vr{supporto} funzioni e contribuisca al successo del progetto$^G$.

\subsubsection{Documentazione}\vspace{-0.2cm}
\begin{table}[h]
    \centering
    \renewcommand{\arraystretch}{1.15}
    \resizebox{\textwidth}{!}{
        \begin{tabular}{|>{\centering\arraybackslash}p{0.13\textwidth}|
                          >{\centering\arraybackslash}p{0.3\textwidth}|
                          >{\centering\arraybackslash}p{0.26\textwidth}|
                          >{\centering\arraybackslash}p{0.26\textwidth}|}
            \hline
            \rowcolor[gray]{0.9}
            \textbf{Metrica} & \textbf{Nome} & \textbf{Valore accettabile} & \textbf{Valore ottimo} \\
            \hline
            MQC\_10 & Indice di Gulpease & $\geq 50$ & $\geq 70$ \\
            \hline
        \end{tabular}
    }
    \vspace{-0.3cm}
    \caption{Processi di supporto - Metriche del processo di documentazione}
\end{table}

\subsubsection{Verifica}\vspace{-0.2cm}
\begin{table}[h]
    \centering
    \renewcommand{\arraystretch}{1.15}
    \resizebox{\textwidth}{!}{
        \begin{tabular}{|>{\centering\arraybackslash}p{0.13\textwidth}|
                          >{\centering\arraybackslash}p{0.3\textwidth}|
                          >{\centering\arraybackslash}p{0.26\textwidth}|
                          >{\centering\arraybackslash}p{0.26\textwidth}|}
            \hline
            \rowcolor[gray]{0.9}
            \textbf{Metrica} & \textbf{Nome} & \textbf{Valore accettabile} & \textbf{Valore ottimo} \\
            \hline
            MQC\_11 & Test$^G$ Success Rate & $\geq 80\%$ & 100\% \\
            \hline
        \end{tabular}
    }
    \vspace{-0.3cm}
    \caption{Processi di supporto - Metriche del processo di verifica}
\end{table}

\subsection{Processi Organizzativi}
Determinano la strategia e il quadro operativo generale. Comprendono la gestione, il miglioramento dei processi, la formazione e l'infrastruttura.\\
Le loro metriche valutano quanto l'organizzazione sappia gestire, imparare e adattarsi.\\
\begin{table}[h]
    \centering
    \renewcommand{\arraystretch}{1.15}
    \resizebox{\textwidth}{!}{
        \begin{tabular}{|>{\centering\arraybackslash}p{0.13\textwidth}|
                          >{\centering\arraybackslash}p{0.3\textwidth}|
                          >{\centering\arraybackslash}p{0.26\textwidth}|
                          >{\centering\arraybackslash}p{0.26\textwidth}|}
            \hline
            \rowcolor[gray]{0.9}
            \textbf{Metrica} & \textbf{Nome} & \textbf{Valore accettabile} & \textbf{Valore ottimo} \\
            \hline
            MQC\_12 & Quality Metrics Satisfied & $\geq 80\%$ & 100\% \\
            \hline
        \end{tabular}
    }
    \vspace{-0.3cm}
    \caption{Processi organizzativi}
\end{table}

\newpage
\section{Qualità di Prodotto}
La qualità del prodotto rappresenta la capacità del software di soddisfare in modo completo e affidabile i requisiti funzionali e non funzionali stabiliti.\\
Per garantire un monitoraggio continuo e oggettivo, il gruppo ha definito un insieme di metriche che consentono di valutare diversi aspetti del prodotto finale. Le metriche selezionate si ispirano ai principi degli standard internazionali per la qualità del software e permettono di misurare il livello qualitativo raggiunto.\\
Le categorie considerate sono: Funzionalità$^G$, Affidabilità, Usabilità, Efficienza$^G$ e Manutenibilità.

\subsection{Funzionalità}
La funzionalità$^G$ descrive la capacità del software di offrire tutte le funzioni richieste e di rispondere correttamente alle esigenze esplicitate nei requisiti. 
\begin{table}[h]
    \centering
    \renewcommand{\arraystretch}{1.15}
    \resizebox{\textwidth}{!}{
        \begin{tabular}{|>{\centering\arraybackslash}p{0.13\textwidth}|
                          >{\centering\arraybackslash}p{0.3\textwidth}|
                          >{\centering\arraybackslash}p{0.26\textwidth}|
                          >{\centering\arraybackslash}p{0.26\textwidth}|}
            \hline
            \rowcolor[gray]{0.9}
            \textbf{Metrica} & \textbf{Nome} & \textbf{Valore accettabile} & \textbf{Valore ottimo} \\
            \hline
            MQD\_01 & Requisiti obbligatori soddisfatti &  100\% & 100\% \\
            \hline
            MQD\_02 & Requisiti desiderabili soddisfatti  &  $\geq 30\%$ & 100\% \\
            \hline
            MQD\_03 & Requisiti opzionali soddisfatti & $\geq 0\%$ & $\geq 70\%$ \\
            \hline
        \end{tabular}
    }
    \vspace{-0.3cm}
    \caption{Funzionalità del prodotto}
\end{table}

\subsection{Affidabilità}
L’affidabilità rappresenta la capacità del prodotto software di operare in modo corretto e continuo nel tempo, riducendo al minimo guasti, malfunzionamenti o comportamenti imprevisti.
\begin{table}[h]
    \centering
    \renewcommand{\arraystretch}{1.15}
    \resizebox{\textwidth}{!}{
        \begin{tabular}{|>{\centering\arraybackslash}p{0.13\textwidth}|
                          >{\centering\arraybackslash}p{0.3\textwidth}|
                          >{\centering\arraybackslash}p{0.26\textwidth}|
                          >{\centering\arraybackslash}p{0.26\textwidth}|}
            \hline
            \rowcolor[gray]{0.9}
            \textbf{Metrica} & \textbf{Nome} & \textbf{Valore accettabile} & \textbf{Valore ottimo} \\
            \hline
            MQD\_04 & Code Coverage & $\geq 60\%$ & $\geq 80\%$ \\
            \hline
            MQD\_05 & Branch$^G$ Coverage  & $\geq 60\%$ & $\geq 80\%$ \\
            \hline
            MQD\_06 & Statement Coverage & $\geq 60\%$ & $\geq 80\%$ \\
            \hline
        \end{tabular}
    }
    \vspace{-0.3cm}
    \caption{Affidabilità del prodotto}
\end{table}

\subsection{Usabilità}
L’usabilità valuta quanto il software sia semplice da comprendere, apprendere e utilizzare per l’utente finale. 
\begin{table}[h]
    \centering
    \renewcommand{\arraystretch}{1.15}
    \resizebox{\textwidth}{!}{
        \begin{tabular}{|>{\centering\arraybackslash}p{0.13\textwidth}|
                          >{\centering\arraybackslash}p{0.3\textwidth}|
                          >{\centering\arraybackslash}p{0.26\textwidth}|
                          >{\centering\arraybackslash}p{0.26\textwidth}|}
            \hline
            \rowcolor[gray]{0.9}
            \textbf{Metrica} & \textbf{Nome} & \textbf{Valore accettabile} & \textbf{Valore ottimo} \\
            \hline
            MQD\_07 & Time on Task$^G$ & $\leq 60$ secondi & $\leq 30$ secondi \\
            \hline
        \end{tabular}
    }
    \vspace{-0.3cm}
    \caption{Usabilità del prodotto}
\end{table}

\subsection{Efficienza}
L’efficienza$^G$ indica la capacità del software di utilizzare in modo ottimale le risorse disponibili, garantendo prestazioni adeguate in termini di tempi di risposta, consumo di memoria, utilizzo della CPU e gestione del carico.
\begin{table}[h]
    \centering
    \renewcommand{\arraystretch}{1.15}
    \resizebox{\textwidth}{!}{
        \begin{tabular}{|>{\centering\arraybackslash}p{0.13\textwidth}|
                          >{\centering\arraybackslash}p{0.3\textwidth}|
                          >{\centering\arraybackslash}p{0.26\textwidth}|
                          >{\centering\arraybackslash}p{0.26\textwidth}|}
            \hline
            \rowcolor[gray]{0.9}
            \textbf{Metrica} & \textbf{Nome} & \textbf{Valore accettabile} & \textbf{Valore ottimo} \\
            \hline
            MQD\_08 & Response time & $\leq 2$ secondi & $\leq 1$ secondo \\
            \hline
        \end{tabular}
    }
    \vspace{-0.3cm}
    \caption{Efficienza del prodotto}
\end{table}

\subsection{Manutenibilità}
La manutenibilità esprime la facilità con cui il software può essere modificato, corretto, ampliato o migliorato nel tempo.
\begin{table}[h]
    \centering
    \renewcommand{\arraystretch}{1.15}
    \resizebox{\textwidth}{!}{
        \begin{tabular}{|>{\centering\arraybackslash}p{0.13\textwidth}|
                          >{\centering\arraybackslash}p{0.3\textwidth}|
                          >{\centering\arraybackslash}p{0.26\textwidth}|
                          >{\centering\arraybackslash}p{0.26\textwidth}|}
            \hline
            \rowcolor[gray]{0.9}
            \textbf{Metrica} & \textbf{Nome} & \textbf{Valore accettabile} & \textbf{Valore ottimo} \\
            \hline
            MQD\_09 & Code Smells per KLOC$^G$ & $\leq 15$ & $\leq 5$ \\
            \hline
            MQD\_10 & Coefficient of Coupling & $\leq 0.35$ & $\leq 0.2$ \\
            \hline
            MQD\_11 & Cyclomatic Complexity$^G$ & $\leq 20$ & $\leq 10$ \\
            \hline
        \end{tabular}
    }
    \vspace{-0.3cm}
    \caption{Manutenibilità del prodotto}
\end{table}

\newpage
\input{content/strategieTesting}

\newpage
\input{content/cruscottoValutazione}

\end{document}
