\section{Decisioni e Azioni} 
\begin{table}[h]
	\centering
	\renewcommand{\arraystretch}{1.1} % Aumenta l'altezza delle righe del 50%
	\resizebox{\textwidth}{!}{
		\begin{tabular}{|p{0.13\textwidth}|p{0.65\textwidth}|}
			\hline
			\rowcolor[gray]{0.9}
			Codice & Descrizione \\
			\hline
			VE$^{G}$ 1.1 & I meeting verranno svolti all’occorrenza, con preavviso. Gli incontri online avverranno tramite Google Meet. \\ 
			\hline
			VE 1.2 & È stato scelto WhatsApp come canale di comunicazione. \\ 
			\hline
			VE 1.3 & Abbiamo deciso di proseguire con Git Flow e di mantenere il nostro metodo di suddivisione del lavoro in feature branch, che verranno eliminati dopo il merge. \\ 
			\hline
			VE 1.4 & Abbiamo deciso di adottare la licenza MIT per il repository. \\ 
			\hline
			VE 1.5 & Utilizzo di React per il front-end. \\ 
			\hline
			VE 1.6 & Utilizzo di Python per back-end/AI. \\ 
			\hline
			VE 1.7 & Utilizzo di PostgreSQL per i database. \\ 
			\hline
			VE 1.8 & Utilizzo di GPT di OpenAI per LLM. \\ 
			\hline
			VE 1.9 & Per la realizzazione di SmartOrder il gruppo ha deciso di preferire l'implementazione della trascrizione vocale all'analisi di immagini\\ 
			\hline
			VE 1.10 & Utilizzo di EasyOCR nel caso di implementazione della funzione di riconoscimento delle immagini. \\
			\hline
			VE 1.11 & Utilizzo di Draw.io per la creazione di UML. \\
			\hline
			VE 1.12 & Utilizzo di un database di test per le verifiche. \\
			\hline
			VE 1.13 & È necessaria la validazione dell'ordine: l'utente deve confermare l'ordine completo prima dell'invio. \\
			\hline
			VE 1.14 & Per prodotti ambigui il sistema consulta lo storico del cliente, in mancanza di esso propone opzioni invece di ``indovinare''. \\
			\hline
		\end{tabular}
	}
\end{table}